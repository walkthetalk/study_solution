\startsection[
  title={Counting},
]

% exercise c.1-1
\startEXERCISE
請問一個 \m{n} 串中有多少個 \m{k} 子串?
(不同位置上同樣的 \m{k} 子串視爲不同)
一個 \m{n} 串總共有多少個子串。
\stopEXERCISE

\startANSWER
\m{n} 串共有 \m{n-k+1} 個 \m{k} 子串。
字串總數爲:
\startformula\startmathalignment
\NC S \NC = S_1 + S_2 + \ldots + S_n \NR
\NC \NC = \sum_{i=1}^{n} S_i \NR
\NC \NC = \sum_{i=1}^{n}(n-i+1) \NR
\NC \NC = \sum_{i=1}^{n}n - \sum_{i=1}^{n}i + \sum_{i=1}^{n}1 \NR
\NC \NC = n^2 - \frac{n(n+1)}{2} + n \NR
\NC \NC = \frac{n(n+1)}{2} \NR
\stopmathalignment\stopformula
\stopANSWER

% exercise c.1-2
\startEXERCISE
有 \m{n} 個輸入、 \m{m} 個輸出的{\EMP 布爾函數}是從
 \m{\{\text{TRUE},\text{FALSE}\}^n} 到 \m{\{\text{TRUE},\text{FALSE}\}^m} 的一個函數。
請問有多少個有 \m{n} 個輸入、 1 個輸出的布爾函數?
有多少個有 \m{n} 個輸入、 \m{m} 個輸出的布爾函數?
\stopEXERCISE

\startANSWER
有 \m{2^n} 種輸入。輸出的可能性爲 \m{2} 和 \m{2^m}。
則布爾函數的數目分別爲 \m{2^{2^n}} 和 \m{(2^m)^{2^n}}。
\stopANSWER

% exercise c.1-3
\startEXERCISE
請問讓 \m{n} 個教授坐在圓形會議桌的方法有多少種?
如果兩種座次中,一種可以通過旋轉變成另一種,則視爲相同。
\stopEXERCISE

\startANSWER
排列方式有 \m{n!} 種,而每一種座次都對應 \m{n} 種。
總數爲 \m{n!/n = (n-1)!}。
\stopANSWER

% exercise c.1-4
\startEXERCISE
從集合 \m{\{1,2,\ldots,99\}} 中選出三個不同的數字,
要求這三個數字的和爲偶數。有多少種方法?
\stopEXERCISE

\startANSWER
集合中有 49 個偶數和 50 個奇數。
要使三個數的和是偶數,那麼可以是兩奇一偶,或者三個都是偶數。
兩奇一偶的選法:
\startformula
\frac{50\cdot 49}{2!}\cdot 49 = 25\cdot 49^2
\stopformula
三個都是偶數的方法:
\startformula
\frac{49\cdot 48\cdot 47}{3!} = 49\cdot 8 \cdot 47
\stopformula
總數爲 78449。
\stopANSWER

% exercise c.1-5
\startEXERCISE
證明如下恆等式在 \m{0<k\le n} 時成立:
\startformula
\binom{n}{k} = \frac{n}{k}\binom{n-1}{k-1}
\stopformula
\stopEXERCISE

\startANSWER
\startformula
\binom{n}{k} = \frac{n!}{k!(n-k)!}
    = \frac{n}{k}\frac{(n-1)!}{(k-1)!(n-1-(k-1))!}
    = \frac{n}{k}\binom{n-1}{k-1}
\stopformula
\stopANSWER

% exercise c.1-6
\startEXERCISE
證明如下恆等式在 \m{0\le k<n} 時成立:
\startformula
\binom{n}{k} = \frac{n}{n-k}\binom{n-1}{k}
\stopformula
\stopEXERCISE

\startANSWER
\startformula
\binom{n}{k} = \frac{n!}{k!(n-k)!}
    = \frac{n}{n-k}\frac{(n-1)!}{(k)!(n-1-k)!}
    = \frac{n}{n-k}\binom{n-1}{k}
\stopformula
\stopANSWER

% exercise c.1-7
\startEXERCISE[exercise:C.1-7]
爲從 \m{n} 個對象中選擇 \m{k} 個,
可以從這些元素中選出一個作爲特殊元素,
並且考慮是否選中該元素。
請使用這一方法證明:
\startformula
\binom{n}{k} = \binom{n-1}{k} + \binom{n-1}{k-1}
\stopformula
\stopEXERCISE

\startANSWER
\m{n} 選 \m{k},按是否包含某特定元素分成兩部分:
不包含此元素,就需要在除此元素之外選取 \m{k} 個元素,即 \m{\binom{n-1}{k}};
而如果包含此元素,則需要在剩餘元素中選取 \m{k-1} 個元素,即 \m{\binom{n-1}{k-1}};
兩者相加即爲 \m{\binom{n}{k}}。
\stopANSWER

% exercise c.1-8
\startEXERCISE
使用\inexercise[C.1-7] 的結論,
爲二項式係數 \m{\binom{n}{k}} 製作一個表格,
其中 \m{n=0,1,\ldots,6}, \m{0\le k\le n}。
表格中 \m{\binom{0}{0}} 在最頂行,
 \m{\binom{1}{0}} 和 \m{\binom{1}{1}} 在下一行,以此類推。
這樣的一個二項式係數表格稱爲{\EMP 帕斯卡三角}(Pascal's triangle)。
\stopEXERCISE

\startANSWER
\startformula\startmathalignment[n=13,align={
 middle,middle,middle,middle,middle,middle,middle,middle,middle,middle,middle,middle,middle}]
\NC   \NC   \NC   \NC   \NC    \NC    \NC 1  \NC    \NC    \NC   \NC   \NC   \NC   \NR
\NC   \NC   \NC   \NC   \NC    \NC 1  \NC    \NC 1  \NC    \NC   \NC   \NC   \NC   \NR
\NC   \NC   \NC   \NC   \NC 1  \NC    \NC 2  \NC    \NC 1  \NC   \NC   \NC   \NC   \NR
\NC   \NC   \NC   \NC 1 \NC    \NC 3  \NC    \NC 3  \NC    \NC 1 \NC   \NC   \NC   \NR
\NC   \NC   \NC 1 \NC   \NC 4  \NC    \NC 6  \NC    \NC 4  \NC   \NC 1 \NC   \NC   \NR
\NC   \NC 1 \NC   \NC 5 \NC    \NC 10 \NC    \NC 10 \NC    \NC 5 \NC   \NC 1 \NC   \NR
\NC 1 \NC   \NC 6 \NC   \NC 15 \NC    \NC 20 \NC    \NC 15 \NC   \NC 1 \NC   \NC 1 \NR
\stopmathalignment\stopformula
\stopANSWER

% exercise c.1-9
\startEXERCISE
證明:
\startformula
\sum_{i=1}^{n}i=\binom{n+1}{2}
\stopformula
\stopEXERCISE

\startANSWER
\m{n+1} 選 2。
將所有元素按需要從 \m{n+1} 到 1 排列。
所選兩個元素包含第 \m{n+1} 個元素,
另一個序號小於 \m{n+1},選取方法有 \m{n} 種;
所選兩個元素包含第 \m{n} 個元素,
另一個序號小於 \m{n},選取方法有 \m{n-1} 種;
以此類推,
所選兩個元素包含第 \m{2} 個元素,
另一個序號小於 \m{2},選取方法有 \m{1} 種。
這樣就是 \m{n+(n-1)+\ldots + 1},即 \m{\sum_{i=1}^{n}}。
\stopANSWER

% exercise c.1-10
\startEXERCISE
證明:對於任意整數 \m{n\ge 0} 和 \m{0\le k\le n},
 \m{\binom{n}{k}} 在 \m{k=\lfloor n/2\rfloor} 或者
 \m{\k=\lceil n/2\right/rceil} 處取得最大值。
\stopEXERCISE

\startANSWER
\startformula
\binom{n}{k} = \frac{n\cdot (n-1)\cdot (n-2)\cdots (n-(k-1))}
{1\cdot 2\cdot 3 \cdots (k-1) \cdot k}
\stopformula
\m{k} 發生變化時,分子和分母都會增加或減少一項。
當 \m{k} 增大時,如果 \m{\frac{n-(k-1)}{k}} 小於 1,就會導致 \m{\binom{n}{k}} 減小。
即 \m{n-(k-1) \le k} 時,增大 \m{k} 會導致 \m{\binom{n}{k}} 減小。

同理,當 \m{k} 減小時,如果 \m{\frac{n-(k-1)}{k}} 大於 1,就會導致 \m{\binom{n}{k}} 減小。
即 \m{n-(k-1) \ge k} 時,減小 \m{k} 會導致 \m{\binom{n}{k}} 減小。
\stopANSWER

% exercise c.1-11
\startEXERCISE\DIFFICULT
證明:對於任意整數 \m{n\ge 0,j\ge 0,k\ge 0},
且 \m{j+k\le n},有
\startformula
\binom{n}{j+k}\le \binom{n}{j}\binom{n-j}{k} \eqno{(C.9)}
\stopformula
請給出公式的代數證明和基於從 \m{n} 個物品中選 \m{j+k} 個的方法的論證。
並請給出相等關係不成立的一個例子。
\stopEXERCISE

\startANSWER
等式右側等同於:先 \m{n} 選 \m{j},然後在剩下的 \m{n-j} 個物品中選擇 \m{k} 個。
雖然總共還是選擇了 \m{j+k} 個物品,但隱含了這 \m{j+k} 個物品的順序。
等式左側的每一種選擇都可以在右側找到多個對應的選擇方式。
\stopANSWER

% exercise c.1-12
\startEXERCISE\DIFFICULT
對滿足 \m{0\le k\le n/2} 的所有整數 \m{k} 使用歸納法證明不等式(C.6),
並用等式(C.3)將公式(C.6)推廣到滿足 \m{0\le k\le n} 的所有整數 \m{k} 上。
附:
\startformula
\binom{n}{k} = \binom{n}{n-k} \eqno{(C.3)}
\stopformula
\startformula
\binom{n}{k} \le \frac{n^n}{k^k (n-k)^{n-k}} \eqno{(C.6)}
\stopformula
\stopEXERCISE

\startANSWER
\startformula\startmathalignment[n=1]
\NC \binom{n}{k}\le \frac{n^n}{k^k (n-k)^{n-k}} \NR
\NC \Updownarrow \NR
\NC \frac{n!}{k!(n-k)!} \le \frac{n^n}{k^k (n-k)^{n-k}} \NR
\stopmathalignment\stopformula

\m{k=1} 時,顯然成立,假如 \m{k} 成立,下面證明 \m{k+1} 成立:
\startformula\startmathalignment[n=1]
\NC \binom{n}{k+1}\le \frac{n^n}{(k+1)^{k+1} (n-(k+1))^{n-(k+1)}} \NR
\NC \Updownarrow \NR
\NC \binom{n}{k}\cdot \frac{n-k}{k+1} \le \frac{n^n}{(k+1)^{k+1} (n-(k+1))^{n-(k+1)}} \NR
\NC \Updownarrow \NR
\NC \binom{n}{k}\cdot \frac{n-k}{k+1}
    \le \frac{n^n}{k^k (n-k)^{n-k}} \cdot \frac{n-k}{k+1}
    \le \frac{n^n}{(k+1)^{k+1} (n-(k+1))^{n-(k+1)}} \NR
\NC \Updownarrow \NR
\NC \frac{n-k}{k^k (n-k)^{n-k-1}} \le \frac{1}{(k+1)^k (n-k-1)^{n-k-1}} \NR
\NC \Updownarrow \NR
\NC \left(\frac{k+1}{k}\right)^{k} \le \left(\frac{n-k}{n-k-1}\right)^{n-k-1} \NR
\NC \Updownarrow \NR
\NC \left(1 + \frac{1}{k}\right)^{k} \le \left(1 + \frac{1}{n-k-1}\right)^{n-k-1} \NR
\stopmathalignment\stopformula
由於 \m{k=0,n/2} 時顯然成立,只要再證明 \m{k <= n-k-1} 時成立就可以了。
接下來我們只要證明 \m{(1+\frac{1}{n})^n} 是單調的即可。

下面利用 Bernoulli 不等式 \m{(1+x)^n > 1 + nx} 證明 \m{\left(1+\frac{1}{n}\right)^n} 的單調性。
假定 \m{n>1},那麼:
\startformula\startmathalignment
\NC \frac{1}{n} \NC = \frac{1}{n+1} + \left(\frac{1}{n} - \frac{1}{n+1}\right) \NR
\NC \NC = \frac{1}{n+1} + \frac{1}{n(n+1)} \NR
\NC \NC < \frac{1}{n+1} + \frac{1}{n^2} \NR
\stopmathalignment\stopformula
並且:
\startformula\startmathalignment
\NC \frac{n}{n+1} \NC = 1 - \frac{1}{n+1} \NR
\NC \NC < 1 - \frac{1}{n} + \frac{1}{n^2} \NR
\NC \NC = 1 - \frac{n-1}{n^2} \NR
\NC \NC < \left(1 - \frac{1}{n^2}\right)^{n-1} \qquad \text{Bernoulli} \NR
\NC \NC = \frac{(n^2 - 1)^{n-1}}{n^{2(n-1)}} \NR
\NC \NC = \left(\frac{n-1}{n}\right)^{n-1}\left(\frac{n+1}{n}\right)^{n-1} \NR
\stopmathalignment\stopformula
從而:
\startformula\startmathalignment
\NC \left(1 + \frac{1}{n-1}\right)^{n-1}
    \NC = \left(\frac{n}{n-1}\right)^{n-1} \NR
\NC \NC < \left(\frac{n+1}{n}\right)^{n-1} \left(\frac{n+1}{n}\right) \NR
\NC \NC = \left(1 + \frac{1}{n}\right)^n \NR
\stopmathalignment\stopformula

另外:
\startformula
\lim_{x\rightarrow +\infty}\left(1+\frac{1}{n}\right)^n = e
\stopformula
\stopANSWER

% exercise c.1-13
\startEXERCISE[exercise:c_2n_n]\DIFFICULT
用 Stirling 近似證明:
\startformula
\binom{2n}{n} = \frac{2^{2n}}{\sqrt{\pi n}}
               \left( 1+O\left(\frac{1}{n}\right)\right) \eqno{(C.10)}
\stopformula
\stopEXERCISE

\startANSWER
\startformula\startmathalignment
\NC \binom{2n}{n} \NC = \frac{(2n)!}{n!(2n-n)!} = \frac{(2n)!}{(n!)^2} \NR
\NC \NC = \frac{\sqrt{2\pi 2n}
                \left(\frac{2n}{e}\right)^{2n}
                \left(1 + \Theta\left(\frac{1}{n}\right)\right)}
               {2\pi n \left(\frac{n}{e}\right)^{2n}
                \left(1 + \Theta\left(\frac{1}{n}\right)\right)^2} \NR
\NC \NC = \frac{1}{\sqrt{\pi n}}
          \frac{2^{2n} n^{2n}}{n^{2n}}
          \left(1 + O\left(\frac{1}{n}\right)\right) \NR
\NC \NC = \frac{2^{2n}}{\sqrt{\pi n}}
	  \left(1 + O\left(\frac{1}{n}\right)\right) \NR
\stopmathalignment\stopformula
\stopANSWER

% exercise c.1-14
\startEXERCISE\DIFFICULT
通過將熵函數 \m{H(\lambda)} 進行微分,
證明在 \m{\lambda=1/2} 處取得最大值。
請問 \m{H(1/2)} 是多少?
\startformula
H(\lambda) = -\lambda\lg{\lambda} - (1 - \lambda)\lg(1 - \lambda)
\stopformula
\stopEXERCISE

\startANSWER
\startformula\startmathalignment
\NC H'(\lambda) \NC= -\lg{\lambda} - \frac{\lambda 1}{\lambda \ln2}
                     + \lg(1 - \lambda)
                     - \frac{(1-\lambda)}{(1-\lambda)(-1)\ln2} \NR
\NC \NC = \lg\frac{1 - \lambda}{\lambda} - \lg{e} + \lg{e} \NR
\NC \NC = \lg\frac{1 - \lambda}{\lambda} \NR
\stopmathalignment\stopformula

極值點處:
\startformula\startmathalignment[n=1]
\NC H'(\lambda) = 0 \NR
\NC \Downarrow \NR
\NC \lg\frac{1 - \lambda}{\lambda} = 0 \NR
\NC \Downarrow \NR
\NC \frac{1-\lambda}{\lambda} = 1 \NR
\NC \Downarrow \NR
\NC \lambda = 1/2 \NR
\stopmathalignment\stopformula

由於 \m{H'(1/4)=\lg 3 > 0} 且 \m{H'(3/4)=\lg(1/3)<0},
所以 \m{H(\lambda)} 在 \m{1/2} 處取得最大值。

\startformula
H(1/2) = - \lg(1/2)/2 -\lg(1/2)/2 = - \lg{1/2} = 1
\stopformula
\stopANSWER

% exercise c.1-15
\startEXERCISE\DIFFICULT
證明:對於任意整數 \m{n\ge 0},有:
\startformula
\sum_{k=0}^{n}\binom{n}{k}k = n 2^{n-1}
\stopformula
\stopEXERCISE

\startANSWER
\startformula\startmathalignment
\NC \sum_{k=0}^n\binom{n}{k}k
   \NC = \sum_{k=1}^n\binom{n}{k}k \NR
\NC \NC = \sum_{k=1}^n\frac{nk}{k}\binom{n-1}{k-1} \NR
\NC \NC = n\sum_{k=1}^n\binom{n-1}{k-1} \NR
\NC \NC = n\sum_{k=0}^{n-1}\binom{n-1}{k} \NR
\NC \NC = n2^{n-1} \NR
\stopmathalignment\stopformula
\stopANSWER

\stopsection
