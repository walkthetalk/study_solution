\section{與 Direct3D 10 共享內存對象}

\subsection{簡介}

本擴展是為了使得在 OpenCL 和 Direct3D 10 間可以進行互操作。
其運作方式與\insection[clShareGl]和\insection[clEvtObjFromGlSync]中的 OpenGL 類似。
如果實作支持本擴展,
則 \cenum{CL_PLATFORM_EXTENSIONS} 或 \cenum{CL_DEVICE_EXTENSIONS} 中
應包含字串 \clext{cl_khr_d3d10_sharing},參見\reftab{cldevquery}。

\subsection{頭檔}

目前本擴展中的所有接口都由 \ccmm{cl_d3d10.h} 提供。

\subsection{新例程和新函式}

\startCLFUNC
cl_int clGetDeviceIDsFromD3D10KHR (
		cl_platform_id platform,
		cl_d3d10_device_source_khr d3d_device_source,
		void *d3d_object,
		cl_d3d10_device_set_khr d3d_device_set,
		cl_uint num_entries,
		cl_device_id *devices,
		cl_uint *num_devices)
cl_mem clCreateFromD3D10BufferKHR (
		cl_context context,
		cl_mem_flags flags,
		ID3D10Buffer *resource,
		cl_int *errcode_ret)
cl_mem clCreateFromD3D10Texture2DKHR (
		cl_context context,
		cl_mem_flags flags,
		ID3D10Texture2D *resource,
		UINT subresource,
		cl_int *errcode_ret)
cl_mem clCreateFromD3D10Texture3DKHR (
		cl_context context,
		cl_mem_flags flags,
		ID3D10Texture3D *resource,
		UINT subresource,
		cl_int *errcode_ret)
cl_int clEnqueueAcquireD3D10ObjectsKHR (
		cl_command_queue command_queue,
		cl_uint num_objects,
		const cl_mem *mem_objects,
		cl_uint num_events_in_wait_list,
		const cl_event *event_wait_list,
		cl_event *event)
cl_int clEnqueueReleaseD3D10ObjectsKHR (
		cl_command_queue command_queue,
		cl_uint num_objects,
		const cl_mem *mem_objects,
		cl_uint num_events_in_wait_list,
		const cl_event *event_wait_list,
		cl_event *event)
\stopCLFUNC

\subsection{新符記}

\clapi{clGetDeviceIDsFromD3D10KHR} 的參數 \carg{d3d_device_source} 的值可以是:
\startclc
CL_D3D10_DEVICE_KHR		0x4010
CL_D3D10_DXGI_ADAPTER_KHR	0x4011
\stopclc

\clapi{clGetDeviceIDsFromD3D10KHR} 的參數 \carg{d3d_device_set} 的值可以是:
\startclc
CL_PREFERRED_DEVICES_FOR_D3D10_KHR	0x4012
CL_ALL_DEVICES_FOR_D3D10_KHR		0x4013
\stopclc

\clapi{clCreateContext} 和 \clapi{clCreateContextFromType} 的
參數 \carg{properties} 可以是:
\startclc
CL_CONTEXT_D3D10_DEVICE_KHR	0x4014
\stopclc

\clapi{clGetContextInfo} 的參數 \carg{param_name} 可以是:
\startclc
CL_CONTEXT_D3D10_PREFER_SHARED_RESOURCES_KHR	0x402C
\stopclc

\clapi{clGetMemObjectInfo} 的參數 \carg{param_name} 可以是:
\startclc
CL_MEM_D3D10_RESOURCE_KHR	0x4015
\stopclc

\clapi{clGetImageInfo} 的參數 \carg{param_name} 可以是:
\startclc
CL_IMAGE_D3D10_SUBRESOURCE_KHR	0x4016
\stopclc

當 \carg{param_name} 是 \cenum{CL_EVENT_COMMAND_TYPE} 時,
 \clapi{clGetEentInfo} 的參數 \carg{param_alue} 可以是:
\startclc
CL_COMMAND_ACQUIRE_D3D10_OBJECTS_KHR	0x4017
CL_COMMAND_RELEASE_D3D10_OBJECTS_KHR	0x4018
\stopclc

如果指定的 Direct3D 10 \cnglo{device}與\cnglo{context}中的\cnglo{device}不兼容,
則 \clapi{clCreateContext} 和 \clapi{clCreateContextFromType} 會返回:
\startclc
CL_INVALID_D3D10_DEVICE_KHR		-1002
\stopclc

\clapi{clCreateFromD3D10BufferKHR} 的參數 \carg{resource} 必須是 Direct3D 10 \cnglo{bufobj},
而 \clapi{clCreateFromD3D10Texture2DKHR} 和 \clapi{clCreateFromD3D10Texture3DKHR} 的
參數 \carg{resource} 必須是 Direct3D 10 材質對象,
否則這些函式會返回:
\startclc
CL_INVALID_D3D10_RESOURCE_KHR		-1003
\stopclc

如果參數 \carg{mem_objects} 中的任一對象已經由 OpenCL 獲取了,
那麼 \clapi{clEnqueueAcquireD3D10ObjectsKHR} 會返回:
\startclc
CL_D3D10_RESOURCE_ALREADY_ACQUIRED_KHR	-1004
\stopclc

如果參數 \carg{mem_objects} 中的任一對象已經由 OpenCL 獲取了,
那麼 \clapi{clEnqueueReleaseD3D10ObjectsKHR} 會返回:
\startclc
CL_D3D10_RESOURCE_NOT_ACQUIRED_KHR	-1005
\stopclc

% Additions to Chapter 4 of the OpenCL 1.2 Specification
\subsection{對第 4 章的補充}

\insection[contexts]中,用下列內容取代 \clapi{clCreateContext} 後面對 \carg{properties} 的描述:
\startreplacepar
\carg{properties} 指向一個特性列,即 \ccmm{<特性名, 值>} 的陣列,以零終止。
如果此陣列中沒有某個特性,則使用其缺省值,參見\reftab{prptForclCreateContext}。
如果 \carg{properties} 是 \cmacro{NULL} 或者是空的(第一個值就是零),所有特性都使用缺省值。
\stopreplacepar

\reftab{prptCtx_D3D10}是對\reftab{prptForclCreateContext}的補充。

\placetable[here][tab:prptCtx_D3D10]
{創建上下文時所用特性}
{\input{chapter_optext/tbl/prpt_clCreateContext_d3d10.tex}}

下列內容作為對 \clapi{clCreateContext} 所返回的錯誤列的補充:
\startreplacepar
\startigBase
\item \cenum{CL_INVALID_D3D10_DEVICE_KHR},
如果屬性 \cenum{CL_CONTEXT_D3D10_DEVICE_KHR} 的值不是 \cmacro{NULL},
並且沒有指定一個有效的 Direct3D 10 \cnglo{device}(\carg{cl_device_ids})。

\item \cenum{CL_INVALID_OPERATION},
如果屬性 \cenum{CL_INVALID_D3D10_DEVICE_KHR} 的值不是 \cmacro{NULL},
以表明要與 Direct3D 10 進行互操作,但同時又指定了要與其他圖形 API 進行互操作。
\stopigBase
\stopreplacepar

\clapi{clCreateContextFromType} 所返回的錯誤列中也要添加上面所描述的兩個新錯誤。

\reftab{paramNameclGetContextInfo}要增加\reftab{prpt_clGetContextInfo_d3d10}中所列內容:

\placetable[here][tab:prpt_clGetContextInfo_d3d10]
{\clapi{clGetContextInfo} 所支持的屬性名}
{\input{chapter_optext/tbl/prpt_clGetContextInfo_d3d10.tex}}

% Additions to Chapter 5 of the OpenCL 1.2 Specification
\subsection{對第五章的補充}

為 \clapi{clGetMemObjectInfo} 所返回的錯誤碼添加以下內容:
\startreplacepar
\startigBase
\startitem
\cenum{CL_INVALID_D3D10_RESOURCE_KHR},
如果 \carg{param_name} 是 \cenum{CL_MEM_D3D10_RESOURCE_KHR},
並且 \carg{memobj} 不是由以下函式創建的:
\startigBase
\item \clapi{clCreateFromD3D10BufferKHR},
\item \clapi{clCreateFromD3D10Texture2DKHR},或
\item \clapi{clCreateFromD3D10Texture3DKHR}。
\stopigBase
\stopitem
\stopigBase
\stopreplacepar

將\reftab{meminfo_d3d10}中的內容添加到\reftab{meminfo}中。

\placetable[here][tab:meminfo_d3d10]
{\capi{clGetMemObjectInfo}所支持的\carg{param_names}}
{\input{chapter_optext/tbl/memobj_info_d3d10.tex}}

為 \clapi{clGetImageInfo} 所返回的錯誤碼添加以下內容:
\startreplacepar
\startigBase
\startitem
\cenum{CL_INVALID_D3D10_RESOURCE_KHR},
如果 \carg{param_name} 是 \cenum{CL_MEM_D3D10_SUBRESOURCE_KHR},
並且 \carg{image} 不是由以下函式創建的:
\startigBase
\item \clapi{clCreateFromD3D10Texture2DKHR},或
\item \clapi{clCreateFromD3D10Texture3DKHR}。
\stopigBase
\stopitem
\stopigBase
\stopreplacepar

將\reftab{clgetimginfo_d3d10}中的內容添加到\reftab{clgetimginfo}中。

\placetable[here][tab:clgetimginfo_d3d10]
{\capi{clGetImageInfo}所支持的\carg{param_names}}
{\input{chapter_optext/tbl/clgetimginfo_d3d10.tex}}

在\reftab{clGetEventInfo}中,
當 \carg{cl_event_info} = \cenum{CL_EVENT_COMMAND_TYPE} 時,
對其返回值的描述增加以下內容:
\startreplacepar
\startigBase
\item \cenum{CL_COMMAND_ACQUIRE_D3D10_OBJECTS_KHR}
\item \cenum{CL_COMMAND_RELEASE_D3D10_OBJECTS_KHR}
\stopigBase
\stopreplacepar

% Sharing Memory Objects with Direct3D 10 Resources
\subsection{與 Direct3D 10 資源共享內存對象}

\cnglo{app}可以使用本節所討論 OpenCL 函式將 Direct3D 10 資源當成 OpenCL \cnglo{memobj}使用。
這樣就可以在 OpenCL 和 Direct3D 10 之間有效的共享數據。
由 OpenCL API 所執行的\cnglo{kernel}可以通過讀寫\cnglo{memobj}來操控 Direct3D 10 資源。
可以由 Direct3D 10 材質資源創建 OpenCL \cnglo{imgobj};
也可以由 Direct3D 10 緩衝資源創建 OpenCL \cnglo{bufobj};
但是這樣做有個前提:當且僅當 OpenCL \cnglo{context}是在 Direct3D 10 設備上創建的。

\topclfunc{clGetDeviceIDsFromD3D10KHR}

\startCLFUNC
cl_int clGetDeviceIDsFromD3D10KHR(
		cl_platform_id platform,
		cl_d3d10_device_source_khr d3d_device_source,
		void *d3d_object,
		cl_d3d10_device_set_khr d3d_device_set,
		cl_uint num_entries,
		cl_device_id *devices,
		cl_uint *num_devices)
\stopCLFUNC

\carg{platform} 即 \clapi{clGetPlatformIDs} 所返回的\cnglo{platform} ID。

\carg{d3d_device_source} 指定了 \carg{d3d_object} 的類型,
且必須是\reftab{dxten_device_type} 中的一個。

\carg{d3d_object} 即所查詢 OpenCL \cnglo{device}所對應的對象。
其類型必須是\reftab{dxten_device_type} 中的一個。

\carg{d3d_device_set} 指定所返回的\cnglo{device}集,必須是\reftab{dxten_device_set} 中的值。

\carg{num_entries} 即 \ctype{cl_device_id} 條目(可以添加到 \carg{devices} 中)的數目,
如果 \carg{devices} 不是 \cmacro{NULL},則 \carg{num_entries} 必須大於零。

\carg{devices} 返回所找到的 OpenCL \cnglo{device}清單。
其中的 \ctype{cl_device_id} 可以用來識別一個特定的 OpenCL \cnglo{device}。
如果 \carg{devices} 是 \cmacro{NULL},則忽略此參數。
所返回的 OpenCL \cnglo{device}數量是 \carg{num_entries} 的值
和 \carg{d3d_object} 所對應 OpenCL \cnglo{device}的數量中較小的那個。

\carg{num_devices} 返回的是 \carg{d3d_object} 所對應的所有 OpenCL \cnglo{device}的數目。
如果 \carg{num_devices} 是 \cmacro{NULL},則忽略此參數。

如果執行成功,則 \clapi{clGetDeviceIDsFromD3D10KHR} 會返回 \cenum{CL_SUCCESS};
否則,返回下列錯誤碼之一:
\startigBase
\item \cenum{CL_INVALID_PLATFORM},如果 \carg{platform} 無效。

\item \cenum{CL_INVALID_VALUE},如果 \carg{d3d_device_source} 的值無效、
\carg{d3d_device_set} 的值無效、 \carg{num_entries} 爲零但 \carg{devices} 不是 \cmacro{NULL},
或者 \carg{num_devices} 和 \carg{devices} 都是 \cmacro{NULL}。

\item \cenum{CL_DEVICE_NOT_FOUND},
如果沒有找到任何與 \carg{d3d_object} 所對應的 OpenCL \cnglo{device}。
\stopigBase

\placetable[here][tab:dxten_device_type]
{\clapi{clGetDeviceIDsFromD3D10KHR} 所查詢的 OpenCL \cnglo{device}種類與對應對象的型別}
{\input{chapter_optext/tbl/dx10_device_type.tex}}

\placetable[here][tab:dxten_device_set]
{\clapi{clGetDeviceIDsFromD3D10KHR} 所查詢的 OpenCL \cnglo{device}集}
{\input{chapter_optext/tbl/dx10_device_set.tex}}

\todo{TODO:}

% Issues
\subsection{問題}

\todo{TODO:}

