\startsection[
  title={Proofs of shortest-paths properties},
]

%e24.5-1
\startEXERCISE
圖 24-2 中有兩棵最短路徑樹,請給出與其不同的另外兩棵最短路徑樹。
\stopEXERCISE

\startANSWER
\startcombination[nx=2]
{\externalfigure[output/e24_5_1-1]}{a}
{\externalfigure[output/e24_5_1-2]}{b}
\stopcombination
\stopANSWER

%e24.5-2
\startEXERCISE
\m{G=(V,E)} 是一個帶權重的有向圖,
權重函數爲 \m{\omega:E\rightarrow R}。
設 \m{s\in V} 爲某個源節點。
請舉出一個例子,使得圖 \m{G} 滿足下列條件:
對於每條邊 \m{(u,v)\in E},
存在一棵根節點爲 \m{s} 的包含邊 \m{(u,v)} 的最短路徑樹,
也包含一棵根節點爲 \m{s} 的不包含邊 \m{(u,v)} 的最短路徑樹。
\stopEXERCISE

\startANSWER
也許只有存在權重爲 0 的邊才行。

\externalfigure[output/e24_5_2-1]
\stopANSWER

%e24.5-3
\startEXERCISE
對引理 24.10 的證明進行改善,使其可以處理最短路徑權重爲 \m{\infty} 和 \m{-\infty} 的情況。
附引理 24.10:

(三角不等式)設 \m{G=(V,E)} 爲一個帶權重的有向圖,
其權重函數爲 \m{\omega: E\rightarrow R},源節點爲 \m{s}。
那麼對於所有的邊 \m{(u,v)\in E},有
\startformula
\delta(s,v) \le \delta(s,u) + \omega(u,v)
\stopformula
\stopEXERCISE

\startANSWER
如果從 \m{s} 無法到達 \m{v},但從 \m{s} 可以到達 \m{u},
則肯定無法從 \m{u} 到達 \m{v}。

\m{\infty + x = \infty}
\stopANSWER

%e24.5-4
\startEXERCISE
設 \m{G=(V,E)} 是一個帶權重的有向圖,
權重函數爲 \m{\omega:E\rightarrow R}。
設 \m{s\in V} 爲某個源節點。
調用 \ALGO{INITIALIZE-SINGLE-SOURCE(G,s)} 對其進行初始化。
證明:如果一系列鬆弛操作將 \m{s.\pi} 的值設爲一個非空值,
則圖 \m{G} 包含一個權重爲負值的環路。
\stopEXERCISE

\startANSWER
假定是遍歷節點 \m{u} 的時候設置 \m{s.\pi},
那麼 \m{\delta(s,u) + \delta(u,s) < 0},命題得證。
\stopANSWER

%e24.5-5
\startEXERCISE
設 \m{G=(V,E)} 是一個帶權重的有向圖,沒有負值環路。
設 \m{s\in V} 爲源節點。
對於節點 \m{v\in V-\{s\}},如果從源節點 \m{s} 可達,
我們允許 \m{v.\pi} 是節點 \m{v} 在任意一條最短路徑上的前驅;
如果不可達,則 \m{v.\pi} 爲 NIL。
請舉出一個圖例和一種 \m{\pi} 的賦值,
使得 \m{G_\pi} 中形成一條環路。
(根據引理 24.16,這樣的一種賦值不可能由一系列鬆弛操作生成。)
\stopEXERCISE

\startANSWER
\externalfigure[output/e24_5_5-1]
\stopANSWER

%e24.5-6
\startEXERCISE
設 \m{G=(V,E)} 爲一個帶權重的有向圖,
權重函數爲 \m{\omega: E\rightarrow R},
且不包含權重爲負值的環路。
設 \m{s\in V} 爲源節點,
假定圖 \m{G} 由 \ALGO{INITIALIZE-SINGLE-SOURCE(G,s)} 進行初始化。
證明:對於每個節點 \m{v\in V_\pi}, \m{G_\pi} 中存在一條從源節點 \m{s} 到節點 \m{v} 的路徑,
並且該性質在任何鬆弛操作序列中維持爲不變式。
\stopEXERCISE

\startANSWER
\TODO{略。}
\stopANSWER

%e24.5-7
\startEXERCISE
設 \m{G=(V,E)} 爲一個帶權重的有向圖,
且不包含權重爲負值的環路。
設 \m{s\in V} 爲源節點,
假定圖 \m{G} 由 \ALGO{INITIALIZE-SINGLE-SOURCE(G,s)} 進行初始化。
證明:對於所有節點 \m{v\in V_\pi},
存在一個由 \m{|V|-1} 個鬆弛步驟所組成的鬆弛序列來生成 \m{v.d=\delta(s,v)}。
\stopEXERCISE

\startANSWER
\TODO{略。}
\stopANSWER

%e24.5-8
\startEXERCISE
設 \m{G=(V,E)} 爲一個帶權重的有向圖,
且包含一個從源節點 \m{s} 可達的權重爲負值的環路。
請說明如何構造一個 \m{E} 的鬆弛操作的無線序列,
使得每一步鬆弛操作都能更新一個最短路徑的估值。
\stopEXERCISE

\startANSWER
\TODO{略。}
\stopANSWER

\stopsection
