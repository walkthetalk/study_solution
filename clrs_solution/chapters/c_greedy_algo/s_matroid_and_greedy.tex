\startsection[
  title={Matroids and greedy methods},
]

%e16.4-1
\startEXERCISE
證明:若 \m{S} 是任意一個有限集, \m{I_k} 是 \m{S} 的所有規模不超過 \m{k} 的子集的集合(\m{k\le |S|}),
則 \m{(S, I_k)} 是一個擬陣。
\stopEXERCISE

\startANSWER
\TODO{略。}
\stopANSWER

%e16.4-2
\startEXERCISE\DIFFICULT
給定某個域(如實數域)上的 \m{m\times n} 矩陣 \m{T},
證明: \m{(S, I)} 是一個擬陣,其中 \m{S} 是 \m{T} 的列的集合,
且 \m{A\in I} 當且僅當 \m{A} 中的列是線性無關的。
\stopEXERCISE

\startANSWER
\TODO{略。}
\stopANSWER

%e16.4-3
\startEXERCISE
證明:若 \m{(S, I)} 是一個擬陣,則 \m{(S, I')} 也是一個擬陣,其中
\startformula
I' = \{ \text{\m{A'}: \m{S-A'} 包含某些最大獨立子集 \m{A\in I}} \}
\stopformula
即 \m{(S,I')} 的最大獨立子集恰好是 \m{(S,I)} 的最大獨立子集的補集。
\stopEXERCISE

\startANSWER
\TODO{略。}
\stopANSWER

%e16.4-4
\startEXERCISE\DIFFICULT
令 \m{S} 是一個有限集, \m{S_1,S_2,\ldots,S_k} 是 \m{S} 的一個劃分,
這些集合都是非空且不相交的。
定義結構 \m{(S,I)} 滿足條件 \m{I=\{ A: |A\cap S_i|\le 1, i=1,2,\ldots,k \}}。
證明: \m{(S,I)} 是一個擬陣。
也就是說,與劃分中所有子集都最多有一個共同元素的集合 \m{A} 組成的集合構成了擬陣的獨立集。
\stopEXERCISE

\startANSWER
\TODO{略。}
\stopANSWER

%e16.4-5
\startEXERCISE
對於一個所需最優化解爲{\EMP 最小權重}最大獨立子集的加權擬陣問題,
如何將其權重函數進行轉換,
使其變爲標準的加權擬陣問題。
詳細論證你的轉換方法是正確的。
\stopEXERCISE

\startANSWER
\TODO{略。}
\stopANSWER

\stopsection
