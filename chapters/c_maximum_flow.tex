\startcomponent c_maximum_flow

\startchapter[
  title={Maximum Flow},
]

\startsection[
  title={Flow networks},
]

%e26.1-1
\startEXERCISE
證明:在一個流網絡中,
將一條邊分解爲兩條邊所得到的是一個等價的網絡。
更形式化地說,假定網絡 \m{G} 包含邊 \m{(u,v)},
我們以如下方式創建一個新的流網絡 \m{G'}:
創建一個新節點 \m{x},
用新的邊 \m{(u,x)} 和 \m{(x,v)} 來替換原來的邊 \m{(u,v)},
並設置 \m{c(u,x)=c(x,v)=c(u,v)}。
證明: \m{G'} 中的一個最大流與 \m{G} 中的一個最大流具有相同的值。
\stopEXERCISE

\startANSWER
\m{f(u,x)=f(x,v)}。
\stopANSWER

%e26.1-2
\startEXERCISE
將流的性值和定義推廣到多源點和多匯點的流問題上。
證明:在多源點多匯點流網絡中,
通過增加一個超級源點和超級匯點,可以所形成一個單源點單匯點流網絡,
新網絡與原網絡中的流是一一對應的。
\stopEXERCISE

\startANSWER
容量限制:對於所有 \m{u,v\in V},都有 \m{0\le f(u,v)\le c(u,v)};

流量守恆:對於所有 \m{u\in V-S-T},都有 \m{\sum_{v\in V}f(v,u)=\sum_{v\in V}f(u,v)}。
\stopANSWER

%e26.1-3
\startEXERCISE
假定流網絡 \m{G=(V,E)} 違反了如下假設:對於所有節點 \m{v\in V},
網絡必須包括一條路徑 \m{s\leadsto v\leadsto t}。
設節點 \m{u} 滿足:不存在路徑 \m{s\leadsto u\leadsto t}。
證明: \m{G} 中必然存在一個最大流 \m{f},
使得對於所有節點 \m{v\in V}, \m{f(u,v)=f(v,u)=0}。
\stopEXERCISE

\startANSWER
\m{u} 只能扇入或只能扇出,因此只能是既無扇入亦無扇出。
\stopANSWER

\stopsection

\startsection[
  title={The Ford-Fulkerson method},
]
\stopsection

\startsection[
  title={Maximum bipartite matching},
]
\stopsection

\startsection[
  title={Push-relabel algorithms},
]
\stopsection

\startsection[
  title={The relabel-to-front algorithm},
]
\stopsection

\startsubject[
  title={Problems},
]

\stopsubject%Problems

\stopchapter
\stopcomponent
