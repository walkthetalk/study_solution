\startsubject[
  title={Problems},
]

%p23-1
\startPROBLEM
(Second-best minimum spanning tree)
設 \m{G=(V,E)} 爲一連通無向圖,
其權重函數爲 \m{\omega:E\rightarrow R},
假定 \m{|E|\ge |V|} 並且所有的權重都互不相同。
我們定義一棵次優最小生成樹如下:
設 \m{\tau} 爲 \m{G} 的所有生成樹的集合,
 \m{T'} 爲 \m{G} 的一棵最小生成樹。
那麼{\EMP 次優最小生成樹}是生成樹 \m{T},
其滿足 \m{\omega(T)=\min_{T''\in \tau-\{T'\}}\{\omega(T'')\}}。
\startigBase[a]\startitem
證明:最小生成樹唯一,並不能保證次優最小生成樹唯一。
\stopitem\stopigBase

\startANSWER
\midaligned{
\startcombination[nx=3,distance=5em]
{\externalfigure[output/p23_1-1]}{最小生成樹}
{\externalfigure[output/p23_1-2]}{次優生成樹 1}
{\externalfigure[output/p23_1-3]}{次優生成樹 2}
\stopcombination
}
\stopANSWER

\startigBase[continue]\startitem
設 \m{T} 爲 \m{G} 的一棵最小生成樹。證明:
圖 \m{G} 中可以找到兩條邊 \m{(u,v)\in T} 和 \m{(x,y)\notin T},
使得 \m{T-\{(u,v)\}\cup\{(x,y)\}} 是 \m{G} 的一棵次優最小生成樹。
\stopitem\stopigBase

\startANSWER
由於任一生成樹均包含 \m{|V|-1} 條邊,
因此次優 MST 至少含有一條邊不在 MST 中。

如果次優 MST 中只有一條邊不在 MST 中,如 \m{(x,y)},
則次優 MST 與 MST 中邊的差異,只有 \m{(x,y)} 和另一條邊 \m{(u,v)}。
即次優 MST 中用 \m{(x,y)} 代替了 MST 中的 \m{(u,v)},其他邊都一樣。
這種情況下, \m{T-\{(u,v)\}\cup\{(x,y)\}} 是 \m{G} 的一棵次優 MST。

因此我們要證明替換 MST 中的兩條或多條邊,無法得到次優 MST。

令 \m{T} 爲 \m{G} 的 MST,
假設存在次優 MST \m{T'}, \m{T-T'} 中至少有兩條邊,
令 \m{T-T'} 中權重最小的邊爲 \m{(u,v)}。
如果將 \m{(u,v)} 加入 \m{T'},會形成環路 \m{c}。
這個環路包含 \m{T'-T} 中的某條邊 \m{(x,y)} (否則, \m{T} 會形成環路)。

我們要證明 \m{\omega(x,y)>\omega(u,v)}。
用反證法,先假設 \m{\omega(x,y)<\omega(u,v)}。
假設所有邊的權重都不同,因此不用考慮 \m{\omega(x,y)=\omega(u,v)} 的情況。
如果將 \m{(x,y)} 加入 \m{T},會形成環路 \m{c'},
此環路中必定包含 \m{T-T'} 中的某條邊 \m{(u',v')} (否則, \m{T'} 會形成環路)。
因此, \m{T''=T-\{(u',v')\}\cup\{(x,y)\}} 會形成生成樹,
同時必須滿足 \m{\omega(u',v')<\omega(x,y)},
否則 \m{T''} 的權重會小於 \m{T} 的權重, \m{T} 就不是 MST 了。
因此 \m{\omega(u',v')<\omega(x,y)<\omega(u,v)},
但這與假設不符: \m{(u,v)} 是 \m{T-T'} 中權重最小的邊。

由於 \m{(u,v)} 和 \m{(x,y)} 在同一個環路 \m{c} 上,
如果我們將 \m{(u,v)} 加入 \m{T'},
 \m{T'-\{(x,y)\}\cup\{(u,v)\}} 是一棵生成樹,且其權重小於 \m{\omega(T')}。
進一步, \m{T'} 與 \m{T} 不同(僅差一條邊)。
因此,我們找到了一棵生成樹,且其權重小於 \m{\omega(T')},但這棵樹不是 \m{T}。
因此, \m{T'} 不是次優 MST。
\stopANSWER

\startigBase[continue]\startitem
設 \m{T} 爲 \m{G} 的一棵 MST,
對於任意兩個節點 \m{u,v\in V},
設 \m{\max[u,v]} 表示樹 \m{T} 中從節點 \m{u} 到節點 \m{v} 的簡單路徑上權重最大的邊,
請給出一個 \m{O(V^2)} 時間複雜度的算法來計算 \m{\max[u,v]}。
\stopitem\stopigBase

\startANSWER
我們可以在 \m{O(V^2)} 時間內找到所有 \m{u,v\in V} 的 \m{\max[u,v]}。
用何種方法並不重要,廣度優先、深度優先,或者其他方式都行。

下面給出廣度優先和深度優先的僞碼。
但僞碼與第 22 章的僞碼有點區別,我們不需要計算 \m{d} 和 \m{f} 的值,
我們使用表 \m{\max} 本身來記錄是否訪問過一個節點。
當且僅當 \m{u=v} 或者還沒有訪問過節點 \m{v} 時,纔有 \m{\max[u,v]=NIL}。
注意,由於我們是訪問無向圖生成樹的邊,因此可以保證從 \m{u} 開始可以訪問到所有節點。
在節 22.3 中的 \ALGO{DFS} 中的“restart”就沒有必要了。
僞碼中假設每個節點的鄰接鏈表僅包含生成樹 \m{T} 中的邊。

\CLRSH{BFS-FILL-MAX(G, T, w)}
\startCLRS
let max be a new table with an entry max[u,v] for each u, v in G.V
for each vertex u in G.V
	for each vertex v in G.V
		max[u,v] = NIL
	Q = empty
	ENQUEUE(Q,u)
	while Q not empty
		x = DEQUEUE(Q)
		for each v in G.Adj[x]
			if max[u,v] == NIL and v != u
				if x == u or w(x,v) > max[u,x]
					max[u,v] = (x,v)
				else
					max[u,v] = max[u,x]
				ENQUEUE(Q,v)
return max
\stopCLRS

\CLRSH{DFS-FILL-MAX(G, T, w)}
\startCLRS
let max be a new table with an entry max[u,v] for each u, v in G.V
for each vertex u in G.V
	for each vertex v in G.V
		max[u,v] = NIL
	DFS-FILL-MAX-VISIT(G, u, u, max)
return max
\stopCLRS

\CLRSH{DFS-FILL-MAX-VISIT(G, u, x, max)}
\startCLRS
for each vertex v in G.Adj[x]
	if max[u,v] == NIL and v != u
		if x == u or w(x,v) > max[u,x]
			max[u,v] = w(x,v)
		else
			max[u,v] = max[u,x]
		DFS-FILL-MAX-VISIT(G, u, v, max)
\stopCLRS
\stopANSWER

\startigBase[continue]\startitem
給出一個有效算法來計算圖 \m{G} 的次優最小生成樹。
\stopitem\stopigBase

\startANSWER
在 b)中,我們已證明,只要替換最小生成樹中的一個邊就可以得到一棵次優 MST。
假如用邊 \m{(u,v)\notin T} 替換 \m{(x,y)\in T} 得到次優 MST \m{T'},
則 \m{\omega(T')=\omega(T)-\omega(x,y)+\omega(u,v)}。
如果確定了 \m{(u,v)},則要使 \m{\omega(T')} 最小,需要在 \m{T} 中找到
從 \m{u} 到 \m{v} 的唯一路徑上權值最大的邊。
如果我們已經計算除了 \m{T} 上的 \m{\max} 表(見 c)),
那麼我們就可以根據 \m{\max[u,v]} 直接找到這條邊了。
剩下唯一要做的就是找到邊 \m{(u,v)\notin T},使得 \m{\omega(\max[u,v])-\omega(u,v)} 最小。
因此,尋找次優 MST 的算法如下:

\startigNum[n]
\item 計算最小生成樹 \m{T},時間 \m{O(E+V\lg V)},
使用 Prim 算法(用 Fibonacci 堆實現優先隊列)。
由於 \m{|E|<|V|^2},因此運行時間爲 \m{O(V^2)}。

\item 用 MST 計算表 \m{\max},參見 c)。時間爲 \m{O(V^2)}。

\item 找到邊 \m{(u,v)\notin T},使得 \m{\omega(\max[u,v])-\omega(u,v)} 最小。
時間 \m{O(E)},即 \m{O(V^2)}。

\item 根據上一步結果, \m{T'=T-\{\max[u,v]\}\cup\{(u,v)\}} 即爲所求次優 MST。
\stopigNum

總時間爲 \m{O(V^2)}。
\stopANSWER

\stopPROBLEM

%p23-2
\startPROBLEM[problem:23-2]
(Minimum spanning tree in sparse graphs)
對於非常稀疏的圖 \m{G=(V,E)} 來說,
我們可以對 Prim 算法進行進一步改善,
改善後的時間將優於使用 Fibonacci 堆時的 \m{O(E+V\lg V)} 的運行時間。
改善所用的方法就是對圖 \m{G} 進行預處理來減少節點的數量,
然後在減少節點數量後的圖 \m{G} 上運行 Prim 算法。
具體來說,對於每個節點 \m{u},
我們選擇與節點 \m{u} 鄰接的邊中權重最小的 \m{(u,v)},
將其加入到正在構建的最小生成樹裏。
然後對所有選擇的邊進行收縮(請參閱\insection[appendix_graphs]的內容)。
不過,我們不是一條一條地收縮每條邊,
而是首先找出有哪些節點連接到了同一個新節點。
然後創建一個新的圖,
這個圖就如每次收縮這樣一條邊所得出的一樣,
但是我們通過“重新命名”來實現。
重新命名是根據每條邊的端點所在的節點集合來進行。
原始圖中的多條邊可能被重命名爲同樣的名。
在這種情況下,
重名的邊中只有一條邊留下,
這條邊對應原始邊中最小權重的邊。

在初始時,我們把將要構建的最小生成樹 \m{T} 設爲空,
對於每條邊 \m{(u,v)\in E},對其屬性進行如下初始化操作:
 \m{(u,v).orig = (u,v)}, \m{(u,v).c=\omega(u,v)}。
我們使用 \m{orig} 屬性來引用原始圖中與收縮後的圖的邊相關的邊。
屬性 \m{c} 記錄邊的權重,隨着邊的收縮,
我們根據上面選擇邊權重的方法來更新這個屬性。
下面的 \ALGO{MST-REDUCE} 算法以圖 \m{G} 和樹 \m{T} 作爲輸入,
返回一個收縮後的圖 \m{G'} 和更新後的屬性 \m{orig'} 與 \m{c'}。
該算法同時選出圖 \m{G} 中的邊來構成最小生成樹 \m{T}。

\CLRSH{MST-REDUCE(G,T)}
\startCLRS
for each v in G.V
	v.mark = FALSE
	MAKE-SET(v)
for each u in G.V
	if u.mark == FALSE
		choose v in G.Adj[u] such that (u,v).c is minimized
		UNION(u,v)
		T = T merge (u,v).orig
		u.mark = v.mark = TRUE
G'.V = {FIND-SET(v): v in G.V}
G'.E = empty
for each (x,y) in G.E
	u = FIND-SET(x)
	v = FIND-SET(y)
	if (u,v) notin G'.E
		G'.E = G'.E merge {(u,v)}
		(u,v).orig' = (x,y).orig
		(u,v).c' = (x,y).c
	else if (x,y).c < (u,v).c'
		(u,v).orig' = (x,y).orig
		(u,v).c' = (x,y).c
construct adjacency lists G': Adj for G'
return G' and T
\stopCLRS

\startigBase[a]\startitem
設 \m{T} 爲算法 \ALGO{MST-REDUCE} 所返回的邊的集合,
設 \m{A} 爲調用 \ALGO{MST-Prim(G',c',r)} 所生成的圖 \m{G'} 的最小生成樹,
這裏 \m{c'} 是 \m{G'.E} 中邊的權重屬性,
 \m{r} 是 \m{G'.V} 中的任意節點。
證明: \m{T\cup \{(x,y).orig': (x,y)\in A\}} 是圖 \m{G} 的一棵最小生成樹。
\stopitem\stopigBase

\startANSWER
\TODO{略。}
\stopANSWER

\startigBase[continue]\startitem
證明: \m{|G'.V|\le |V|/2}。
\stopitem\stopigBase

\startANSWER
\m{G'.V} 中每個節點至少包含 \m{G.V} 中的兩個節點。
\stopANSWER

\startigBase[continue]\startitem
請說明如何實現算法 \ALGO{MST-REDUCE},
才能讓其運行時間爲 \m{O(E)}。
(\hint 使用簡單的數據結構)
\stopitem\stopigBase

\startANSWER
用鄰接集合。
\stopANSWER

\startigBase[continue]\startitem
假定運行 \ALGO{MST-REDUCE} 算法 \m{k} 次,
使用前一次輸出的圖 \m{G'} 作爲下一次的輸入圖 \m{G},
並在 \m{T} 中將邊累積起來。
證明:算法運行 \m{k} 次的總時間爲 \m{O(kE)}。
\stopitem\stopigBase

\startANSWER
根據上一項可知。
\stopANSWER

\startigBase[continue]\startitem
假定在運行 \ALGO{MST-REDUCE} 算法 \m{k} 次後,
就如在本題的 d)部分那樣,
我們通過調用算法 \ALGO{MST-Prim(G',c',r)} 來運行 Prim 算法,
這裏圖 \m{G'} 是最後一個階段所返回的圖,
其權重屬性爲 \m{c'}, \m{r} 是 \m{G'.V} 中的任意節點。
請說明如何選擇 \m{k},才能使整體運行時間爲 \m{O(E\lg \lg V)}。
並證明所選擇的 \m{k} 使得總體漸進運行時間最短。
\stopitem\stopigBase

\startANSWER
Prim 算法運行時間爲 \m{O(E+V\lg V)}。
運行 \m{k} 次 \ALGO{MST-REDUCE} 後 \m{V'= V 2^{-k}}。
當 \m{k = \lg \lg V} 時,總運行時間 \m{O(kE) + O(E+V'\lg V')} 才爲 \m{O(E\lg \lg V)}。
\stopANSWER

\startigBase[continue]\startitem
要想使這種帶預處理的 Prim 算法的運行時間在漸進意義上由於不帶預處理的 Prim 算法,
 \m{|E|} 和 \m{|V|} 要滿足什麼關係?
\stopitem\stopigBase

\startANSWER
即 \m{O(E\lg\lg V) < O(E + V\lg V)},得 \m{E < \frac{V\lg V}{\lg\lg V}}。
\stopANSWER

\stopPROBLEM

%p23-3
\startPROBLEM
(Bottleneck spanning tree)
無向圖 \m{G} 的{\EMP 瓶頸生成樹} \m{T} 是 \m{G} 的一棵生成樹,
其最大邊的權重是 \m{G} 的所有生成樹中最小的。
我們將 \m{T} 中權重最大邊的權重值稱作瓶頸生成樹 \m{T} 的值。
\startigBase[a]\startitem
證明:最小生成樹是瓶頸生成樹。
\stopitem\stopigBase

\startANSWER
給 MST 添加一條邊必然形成一個環路,然後再去掉一個邊形成另一棵生成樹。
添加的邊的權重肯定大於環路中其他邊的權重。
新生成樹的值不可能小於 MST 的值。
\stopANSWER

本題的 a)部分顯示,找出一棵瓶頸生成樹並不比找出一棵最小生成樹更難。
在本題餘下的部分,我們來演示如何在線性時間內找到一棵瓶頸生成樹。

\startigBase[continue]\startitem
請給出一個線性時間的算法,
在給定圖 \m{G} 和整數 \m{b} 的情況下,
能夠判斷瓶頸生成樹的值是否最大不超過 \m{b}。
\stopitem\stopigBase

\startANSWER
用 \ALGO{DFS} 搜索,跳過權重大於 \m{b} 的邊,看最終是否可以遍歷所有節點。
(等效於刪除權重大於 \m{b} 的邊後,圖是否仍然連通)
\stopANSWER

\startigBase[continue]\startitem
使用本題 b)部分的算法,設計一個瓶頸生成樹問題的線性時間算法,
該算法將以 b)部分的算法作爲子程序。
(\hint 考慮使用一個子程序對邊的集合進行收縮,
就如同\refproblem{23-2} 中所描述的 \ALGO{MST-REDUCE} 算法一樣。)
\stopitem\stopigBase

\startANSWER
二分法搜索 \m{b}。
\stopANSWER
\stopPROBLEM

%p23-4
\startPROBLEM
(Alternative minimum-spanning-tree algorithms)
在本題中,我們給出三種不同算法的僞碼。
每種算法的輸入都是一個連通圖和一個權重函數,
返回值都是一個邊的集合 \m{T}。
對於每種算法,要麼證明 \m{T} 是一棵最小生成樹,
要麼證明 \m{T} 不是一棵最小生成樹。
同時給出每種算法的最有效的實現
(不管該算法是否能夠計算出最小生成樹)。

\startigBase[a]\startitem
\CLRSH{MAYBE-MST-A(G,w)}

\startCLRS
sort the edges into nonincreasing order of edge weights w
T = E
for each edge e, taken in nonincreasing order by weight
	if T - {e} is a connected graph
		T = T - {e}
return T
\stopCLRS
\stopitem\stopigBase

\startANSWER
是 MST。
\stopANSWER

\startigBase[continue]\startitem
\CLRSH{MAYBE-MST-B(G,w)}

\startCLRS
T = empty
for each edge e, taken in arbitrary order
	if T merge {e} has no cycles
		T = T merge {e}
return T
\stopCLRS
\stopitem\stopigBase

\startANSWER
不是 MST。
\stopANSWER

\startigBase[continue]\startitem
\CLRSH{MAYBE-MST-C(G,w)}

\startCLRS
T = empty
for each edge e, taken in arbitrary order
	T = T merge {e}
	if T has a cycle c
		let e' be a maximum-weight edge on c
		T = T - {e'}
return T
\stopCLRS
\stopitem\stopigBase

\startANSWER
是 MST。
\stopANSWER

\stopPROBLEM

\stopsubject%Problems
