\startsection[
  title={The maximum-subarray problem},
]

\startEXERCISE
當 \m{A} 中所有元素均爲負數時, \ALGO{FIND-MAXIMUM-SUBARRAY} 返回什麼?
\stopEXERCISE
\startANSWER
返回一個單元素數組,即最大負整數。
\stopANSWER

\startEXERCISE
編寫暴力求解方法的僞代碼來解決最大子數列問題,其運行時間應爲 \m{\Theta(n^2)}。
\stopEXERCISE
\startANSWER
\CLRSH{FIND-MAX-SUBARRAY(A, low, high)}
\startCLRS
left = 0
right = 0
sum = -∞
for i = low to high
	current-sum = 0
	for j = i to high
	current-sum += A[j]
	if sum < current-sum
		sum = current-sum
	left = i
	right = j
return (left, right, sum)
\stopCLRS
\stopANSWER

\startEXERCISE
在你的計算機上編碼實現用於解決最大子數列問題的暴力算法和遞迴算法。
請指出兩種算法性能交叉點處的問題規模 \m{n_0} ,即在此處遞迴算法將擊敗暴力算法?
然後,修改遞迴算法,當問題規模小於 \m{n_0} 時改用暴力算法。
修改後,性能交叉點有變化嗎?
\stopEXERCISE
\startANSWER
修改前 \m{n_0 = 30},修改算法後 \m{n_0 = 1}。
\stopANSWER

\startEXERCISE
假設修改最大子數列的定義,允許結果爲空,其和爲 \m{0},
需要如何修改現有算法才能使得結果可以爲空?
\stopEXERCISE
\startANSWER
需修改 \m{high = low + 1} 的分支,即當數組只有一個元素的情況;
若是這個元素小於 \m{0},則結果爲空,否則結果爲此元素。
\stopANSWER

\startEXERCISE
使用如下思想爲最大子數列問題設計一個非遞迴的、線性時間的算法。
從數列左邊界開始,從左至右處理,記錄到目前爲止已經處理過的最大子數列。
若已知 \m{A[1..j]}的最大子數列,基於如下性質將解擴展爲 \m{A[1..j+1]} 的最大子數列:
\m{A[1..j+1]} 的最大子數列要麼是 \m{A[1..j]} 的最大子數列,
要麼是某個子數列 \m{A[i..j+1]},其中 \m{1\leq i\leq j+1}。
在已知 \m{A[1..j]} 的最大子數列的情況下,可以在線性時間內找出形如 \m{A[i..j+1]} 的最大子數列。
\stopEXERCISE
\startANSWER
參考\SIMPLEURL{http://en.wikipedia.org/wiki/Maximum_subarray_problem}。

若允許空數列:
\startCLRS
MaxSoFar = 0
MaxEndingHere = 0
for i = 1 to N
	MaxEndingHere = max(0, MaxEndingHere + A[i])
	MaxSoFar = max(MaxSoFar, MaxEndingHere)
\stopCLRS

若不允許空數列:
\startCLRS
MaxSoFar = A[1]
MaxEndingHere = A[1]
for i = 2 to N
	MaxEndingHere = max(A[i], MaxEndingHere + A[i])
	MaxSoFar = max(MaxSoFar, MaxEndingHere)
\stopCLRS
\stopANSWER

\stopsection
