%
% author:	Ni Qingliang
% date:		2011-02-11
%
\startcomponent cpnarch
\product opencl-spec-zh
%\starttext

\chapter{OpenCL 架構}

{\ftEmp{OpenCL}}是一個開放的工業標準,可以為 CPU、 GPU 以及其他分散的計算設備
(這些設備被組織到同一\cnglo{platform}中)所組成的異構群進行編程。
他不只是一種語言。
OpenCL 是一個並行編程\cnglo{framework},
包括一種語言、 API、庫以及一個運行時系統來支持軟件開發。
例如,使用 OpenCL,程式員可以寫出一個能在 GPU 上執行的通用程式,
而不必將其算法映射到 3D 圖形 API (如 OpenGL 或 DirectX)上。

OpenCL 的目標是讓程式員可以寫出可移植並且高效的代碼。
這包括庫作者、中間件供應商,以及以性能為導向的應用程式員。
因此 OpenCL 提供了底層的硬件抽象和一個\cnglo{framework}來支持編程,
同時也暴露了底層硬件的許多細節。

我們將使用如下的模型體系來描述 OpenCL 背後的核心理念:
\startigBase
\item 平台模型
\item 內存模型
\item 執行模型
\item 編程模型
\stopigBase

\section{平台模型}

OpenCL 平台模型的定義可以查看\reffig{plfmodel}。
此模型中有一個\cngloemp{host},
並且有一個或多個 {\ftEmp{OpenCL}} \cnglo{device}與其相連。
每個 OpenCL \cnglo{device}可劃分成一個或多個\cnglo{computeunit}(CU),
每個\cnglo{computeunit}又可劃分成一個或多個\cnglo{prcele}(PE)。
\cnglo{device}上的計算是在\cnglo{prcele}中進行的。

\startreusableMPgraphic{plfMode}
u := 2mm;

ahangle := 30;
ahlength := u;

def picPE =
image(
fill unitsquare shifted (-0.5, -0.5) xscaled u yscaled 4u withcolor (153,153,0);
draw unitsquare shifted (-0.5, -0.5) xscaled u yscaled 4u;
)
enddef;

def picCU =
image(
fill unitsquare shifted (-0.5, -0.5) xscaled 10u yscaled 5u withcolor white;
draw unitsquare shifted (-0.5, -0.5) xscaled 10u yscaled 5u;
for i=-4, -2, 0, 4:
	draw picPE shifted (i*u, 0);
endfor;
for i=1, 2, 3:
	fill fullcircle scaled .5u shifted (i*u,0) withcolor (0,0,0);
endfor;
)
enddef;

def picCD =
image(
fill unitsquare shifted (-0.5, -0.5) xscaled 16u yscaled 10u withcolor white;
draw unitsquare shifted (-0.5, -0.5) xscaled 16u yscaled 10u;
for i=1,0,-1:
	draw picCU shifted (i*2u, i*2u);
endfor;
)
enddef;

def picHost =
image(
fill unitsquare shifted (-0.5, -0.5) scaled 5u withcolor white;
draw unitsquare shifted (-0.5, -0.5) scaled 5u withpen pencircle scaled 2;
label(btex \mplabel{\cnglo{host}} etex, (0,0));
)
enddef;

def picPLF =
image(
for i=2,0,-1,-2:
	draw picCD shifted (i*3u, i*2u);
	draw ((8u,0)--(10u,0)) withpen pencircle scaled 2 shifted (i*3u, i*2u);
endfor;

draw ((16u,4u)--(4u,-4u)) withpen pencircle scaled 2;
draw ((10u,0)--(15u,0)) withpen pencircle scaled 2;
draw picHost shifted (17.5u,0);

label(btex \mplabel{\cnglo{prcele}} etex, (0,0)) shifted (-19u,-3u);
draw ((-18u,-4u)--(-12u,-6u));

label(btex \mplabel{\cnglo{computeunit}} etex, (0,0)) shifted (-11u,-11u);
draw ((-11u,-10u)--(-6u,-8u));

label(btex \mplabel{計算設備} etex, (0,0)) shifted (7u,-10u);
draw ((4u,-9u)--(0u,-8u));
)
enddef;

draw picPLF;

%draw unitsquare scaled u;
\stopreusableMPgraphic

\placefigure[here][fig:plfmodel]
{平台模型}
{\reuseMPgraphic{plfMode}}

OpenCL \cnglo{app}會按照\cnglo{host}\cnglo{platform}的原生模型
在這個\cnglo{host}上運行。
\cnglo{host}上的 OpenCL \cnglo{app}提交\cngloemp{cmd}
給\cnglo{device}中的\cnglo{prcele}以執行計算任務。
\cnglo{computeunit}中的\cnglo{prcele}會作為 SIMD 單元(執行指令流的步伐一致)
或 SPMD 單元(每個 PE 維護自己的程式計數器)執行指令流。

\subsection{平台對多版本的支持}
OpenCL 的設計目標就是要支持同一\cnglo{platform}中多種具有不同能力的設備。
這些設備可以符合不同版本的 OpenCL 規範。
對於一個 OpenCL 系統,有三個重要的版本 ID 要考慮:
\cnglo{platform}版本、\cnglo{device}版本、
\cnglo{device}所支持的 OpenCL C 語言的版本。

\cnglo{platform}版本表明了所支持的 OpenCL 運行時的版本。
這包括可由\cnglo{host}用來與 OpenCL 運行時交互的所有 API,
像\cnglo{context}、\cnglo{memobj}、\cnglo{device}、\cnglo{cmdq}。

\cnglo{device}版本表明了\cnglo{device}的能力,
獨立於運行時和編譯器的版本,由 \capi{clGetDeviceInfo} 所返回的\cnglo{device}資訊來描述。
有很多特性都與\cnglo{device}版本有關,如資源限制和擴展功能。
所返回的版本號即為此\cnglo{device}所符合的 OpenCL 規範的最高版本號,
但不會高於\cnglo{platform}版本。

語言版本,可以讓開發人員據此知道此\cnglo{device}所支持的 OpenCL 編程語言具備哪些特性。
此版本會是所支持語言的最高版本。

OpenCL C 被設計為向後兼容的,
因此對於一個\cnglo{device}而言,只要支持語言的某一個版本,就可以說他符合標準。
如果某個\cnglo{device}支持多個語言版本,編譯器缺省使用最高的那個版本。
語言的版本不會高於\cnglo{platform}的版本,
但可能高於\cnglo{device}的版本(參見\insection[ctrlcveroption])。



\section{執行模型}
OpenCL \cnglo{program}的執行分為兩種情況:
在一個或多個 {\ftEmp{OpenCL}} \cngloemp{device}上執行\cngloemp{kernel};
在\cnglo{host}上執行\cngloemp{host}\cngloemp{program}。
\cnglo{host}\cnglo{program}為\cnglo{kernel}定義了\cnglo{context}
並管理\cnglo{kernel}的執行。

OpenCL 執行模型的核心就是\cnglo{kernel}是怎麼執行的。
\cnglo{host}提交\cnglo{kernel}時會定義一個索引空間。
\cnglo{kernel}的實體會在此空間中的所有點上執行。
\cnglo{kernel}的實體稱為\cngloemp{workitem},
通過在索引空間中的坐標來標識,這個坐標就是\cnglo{workitem}的\cnglo{glbid}。
所有\cnglo{workitem}都會執行相同的代碼,但是代碼的執行路徑和參與運算的數據可能會不同。

\cnglo{workitem}被組織到\cngloemp{workgrp}中。
\cnglo{workgrp}以更粗粒度對索引空間進行了分解。
\cnglo{workgrp}帶有一個唯一的 ID,他與\cnglo{workitem}所使用的索引空間具有同樣的維數。
\cnglo{workitem}具有一個\cnglo{locid},此 ID 在其所隸屬的\cnglo{workgrp}中是唯一的;
因此任一\cnglo{workitem}都可以通過其\cnglo{glbid}
或其\cnglo{locid}加\cnglo{workgrp} ID 來唯一標識。
同一\cnglo{workgrp}中的\cnglo{workitem}
會在同一\cnglo{computeunit}中的多個\cnglo{prcele}上並發執行。

在 OpenCL 中,索引空間又叫做 NDRange。
NDRange 是一個 N 維的索引空間,其中 N 可以是一、二或者三。
NDRange 由一個長度為 N 的整數陣列來定義,
他指定了索引空間各維度的寬度(起自偏移索引 F, F 缺省為 0)。
每個\cnglo{workitem}的\cnglo{glbid}和\cnglo{locid}都是 N 維元組。
\cnglo{glbid}的取值範圍從 F 開始,直到 F 加相應維度上的元素個數減一。

\cnglo{workgrp}的 ID 跟\cnglo{workitem}的\cnglo{glbid}差不多。
一個長度為 N 的陣列定義了每個維度上\cnglo{workgrp}的數目。
\cnglo{workitem}在所隸屬的\cnglo{workgrp}中有一個\cnglo{locid},
此 ID 中各維度的取值範圍為0到\cnglo{workgrp}在相應維度上的大小減一。
因此,\cnglo{workgrp}的 ID 加上其中一個\cnglo{locid}可以唯一確定一個\cnglo{workitem}。
有兩種途徑來標識一個\cnglo{workitem}:
根據全局索引,或根據\cnglo{workgrp}索引加一個局部索引。

接下來請看\reffig{indexspace}中的二維索引空間,
其中包括\cnglo{workitem}、其\cnglo{glbid}以及相應的 ID 元組:
\cnglo{workgrp} ID 和\cnglo{locid}。
\cnglo{workitem}的索引空間為 \math{(G_x, G_y)},
每個\cnglo{workgrp}的大小是 \math{(S_x, S_y)},
\cnglo{glbid}的偏移量是 \math{(F_x, F_y)}。
全局索引定義了一個 \math{G_x} 乘 \math{G_y} 的索引空間,
所能容納的\cnglo{workitem}總數是 \math{G_x} 和 \math{G_y} 的乘積。
局部索引定義了一個 \math{S_x} 乘 \math{S_y} 的索引空間,
一個\cnglo{workgrp}中所能容納\cnglo{workitem}的數目是 \math{S_x} 和 \math{S_y} 的乘積。
如果知道每個\cnglo{workgrp}的大小和\cnglo{workitem}的總數,
就能算出有多少\cnglo{workgrp}。
\cnglo{workgrp}是由一個二維的索引空間來唯一標識的。
\cnglo{workitem}可以用他的\cnglo{glbid} \math{(g_x, g_y)} 標識,
或用\cnglo{workgrp} ID \math{(w_x, w_y)}、
\cnglo{workgrp}的大小 \math{(S_x, S_y)} 和
在\cnglo{workgrp}中的\cnglo{locid} \math{(s_x, s_y)} 三項組合起來標識:
\startformula
(g_x, g_y) = (w_x * S_x + s_x + F_x, w_y * S_y + s_y + F_y)
\stopformula

\cnglo{workgrp}的數目可以這樣計算:
\startformula
(W_x, W_y) = (G_x / S_x, G_y / S_y)
\stopformula

給定\cnglo{glbid}和\cnglo{workgrp}大小,\cnglo{workitem}所屬\cnglo{workgrp}的 ID 為:
\startformula
(w_x, w_y) = ((g_x - s_x - F_x) / S_x, (g_y - s_y - F_y) / S_y)
\stopformula

\startreusableMPgraphic{NDRange}
u := 2.5mm;

ahangle := 30;
ahlength := u;

def drawSingleWrokGrp(expr xOff, yOff) =
begingroup
for i=0 upto 3:
	for j=0 upto 3:
		draw unitsquare scaled u shifted (i*u,j*u) shifted (xOff*u,yOff*u);
	endfor;
endfor;
endgroup
enddef;

% yOff < 0 is Bottom, or is Top
def drawEdgeDescTB(expr xOff, yOff, lblText) =
begingroup
path edge;
pair lblCenter;
drawdblarrow (xOff,yOff)..(-xOff,yOff);
edge := (xOff,yOff+0.5u)..(xOff,yOff-0.5u);
draw edge;
draw edge shifted (-xOff*2, 0);

if yOff > 0:
lblCenter := (0, yOff+1u);
else:
lblCenter := (0, yOff-1u);
fi
label(lblText, (0,0)) shifted lblCenter;
endgroup
enddef;

% xOff < 0 is Left, or is Right
def drawEdgeDescLR(expr xOff, yOff, lblText) =
begingroup
path edge;
pair lblCenter;
drawdblarrow (xOff,yOff)..(xOff,-yOff);
edge := (xOff+0.5u, yOff)..(xOff-0.5u, yOff);
draw edge;
draw edge shifted (0, -yOff*2);

if xOff > 0:
lblCenter := (xOff+1u, 0);
else:
lblCenter := (xOff-1u, 0);
fi
label(lblText, (0,0)) rotated 90 shifted lblCenter;
endgroup
enddef;

% draw group group
def drawWorkGrpOverview =
begingroup
for i=0 upto 2:
	for j=0 upto 2:
		drawSingleWrokGrp(-7+i*5,-7+j*5);
	endfor;
endfor;
draw unitsquare scaled 16u shifted(-8u,-8u);
% bottom arrow
drawEdgeDescTB(8u,-10u,btex \mplabel{\bf NDRange size \math{G_x}} etex);
% left arrow
drawEdgeDescLR(-10u, 8u, btex \mplabel{\bf NDRange size \math{G_y}} etex);
endgroup
enddef;

def drawWrokItem(expr xS, yS, lblContent) =
begingroup
draw unitsquare scaled 12u shifted (-6u, -6u) shifted (xS*7u,yS*7u);
label(lblContent, (0,0)) xscaled 0.5 shifted (xS*7u,yS*7u);
endgroup
enddef;

% draw workgrp detail
def drawWorkGrpDetail =
begingroup
drawWrokItem(-1, 1, btex \mplabel{%
{\bfa work-item}\\
\math{(w_x\cdot S_x + s_x + F_x, w_y\cdot S_y + s_y + F_y)}\\
\math{(S_x, S_y) = (0, 0)}} etex);

drawWrokItem(-1, -1, btex \mplabel{%
{\bfa work-item}\\
\math{(w_x\cdot S_x + s_x + F_x, w_y\cdot S_y + s_y + F_y)}\\
\math{(S_x, S_y) = (0, S_y-1)}} etex);

drawWrokItem(1, -1, btex \mplabel{%
{\bfa work-item}\\
\math{(w_x\cdot S_x + s_x + F_x, w_y\cdot S_y + s_y + F_y)}\\
\math{(S_x, S_y) = (S_x-1, S_y-1)}} etex);

drawWrokItem(1, 1, btex \mplabel{%
{\bfa work-item}\\
\math{(w_x\cdot S_x + s_x + F_x, w_y\cdot S_y + s_y + F_y)}\\
\math{(S_x, S_y) = (S_x-1, 0)}} etex);

label(btex \mplabel{\math{\cdots}} etex, (0,0)) scaled 2.5 rotated -45;
label(btex \mplabel{\math{\cdots}} etex, (0,0)) scaled 2.5 shifted (0,7u);
label(btex \mplabel{\math{\cdots}} etex, (0,0)) scaled 2.5 shifted (0,-7u);
label(btex \mplabel{\math{\cdots}} etex, (0,0)) scaled 2.5 rotated 90 shifted (-7u,0);
label(btex \mplabel{\math{\cdots}} etex, (0,0)) scaled 2.5 rotated 90 shifted (7u,0);

draw unitsquare scaled 30u shifted (-15u,-15u);

label(btex \mplabel{\bf work-group \math{(w_x, w_y)}} etex, (0,0)) shifted (0,14u);

drawEdgeDescTB(15u,16u,btex \mplabel{\bf work-group size \math{S_x}} etex);
drawEdgeDescLR(16u, 15u, btex \mplabel{\bf work-group size \math{S_y}} etex);
endgroup
enddef;

%%%%%%%%%%%%%%%%%%%%%%%%% start drawing
pair ovShift;
picture pOverview;
picture pDetail;

ovShift := (-25u,-3u);

drawWorkGrpOverview;
pOverview := currentpicture shifted ovShift;
currentpicture := nullpicture;

drawWorkGrpDetail;
pDetail := currentpicture;
currentpicture := nullpicture;

draw pOverview;
draw pDetail;

draw (ovShift-(2u,2u))..-(15u,15u) dashed evenly scaled 2 withcolor .3[white,black];
draw (ovShift+(2u,2u))..(15u,15u) dashed evenly scaled 2 withcolor .3[white,black];

draw (ovShift+(-2u,2u))..(-15u,15u) dashed evenly scaled 2 withcolor .3[white,black];
draw (ovShift-(-2u,2u))..-(-15u,15u) dashed evenly scaled 2 withcolor .3[white,black];
\stopreusableMPgraphic

\placefigure[here][fig:indexspace]
{NDRange 索引空间示例}
{\reuseMPgraphic{NDRange}}

很多編程模型都可以映射到這個執行模型上。
OpenCL 明確支持的有兩種:\cngloemp{dppm}和\cngloemp{tppm}。

% Execution Model: Context and Command Queues
\subsection[section:exemodel:contextandcmdq]{執行模型:上下文和命令隊列}

\cnglo{host}會為執行\cnglo{kernel}定義一個\cnglo{context}。
\cnglo{context}包括以下\cnglo{res}:
\startigNum
\item \cngloemp{device}:\cnglo{host}可以使用的 OpenCL \cnglo{device}集。
\item \cngloemp{kernel}:運行在 OpenCL \cnglo{device}上的 OpenCL 函式。
\item \cngloemp{programobj}:實現\cnglo{kernel}的\cnglo{program}源碼和執行體。
\item \cngloemp{memobj}:一組\cnglo{memobj},
對\cnglo{host}和 OpenCL \cnglo{device}可見。
\cnglo{memobj}包含一些值,\cnglo{kernel}實體可以在其上進行運算。
\stopigBase

\cnglo{host}使用 OpenCL API 中的函式來創建並操控\cnglo{context}。
\cnglo{host}會創建一個稱為\cnglo{cmdq}的數據結構
來協調\cnglo{device}上\cnglo{kernel}的執行。
\cnglo{host}還會將\cnglo{cmd}入隊,
這些\cnglo{cmd}將在\cnglo{context}中的\cnglo{device}上被調度。
這些\cnglo{cmd}包括:
\startigBase
\item {\ftEmp{\cnglo{kernel}執行命令}}:
在\cnglo{device}的\cnglo{prcele}上執行\cnglo{kernel}。

\item {\ftEmp{内存命令}}:讀寫\cnglo{memobj}或者在\cnglo{memobj}間傳輸數據,
或者從\cnglo{host}的位址空間中映射、解映射\cnglo{memobj}。

\item {\ftEmp{同步命令}}:限制命令的執行順序。
\stopigBase

\cnglo{cmdq}負責\cnglo{cmd}的調度,使其可以在\cnglo{device}上執行。
在\cnglo{host}和\cnglo{device}上,\cnglo{cmd}的執行是異步的。
\cnglo{cmd}的執行有兩種模式:
\startigBase
\item \cngloemp{inordexec}:
\cnglo{cmd}嚴格按照在\cnglo{cmdq}中出現的順序開始和結束執行。
換言之,前面的\cnglo{cmd}結束後,才能執行後面的\cnglo{cmd}。
這使隊列中\cnglo{cmd}的執行順序串行化。

\item \cngloemp{outordexec}:
按順序執行\cnglo{cmd},但後續\cnglo{cmd}執行前不必等待前面\cnglo{cmd}結束。
任何順序上的限制都是由程式員通過顯式的同步\cnglo{cmd}强加的。
\stopigBase

提交給隊列的\cnglo{kernel}執行命令和內存命令都會生成\cnglo{evtobj}。
這些\cnglo{evtobj}可以用來控制\cnglo{cmd}的執行順序、
協調\cnglo{cmd}在\cnglo{host}和\cnglo{device}間的運行。

一個\cnglo{context}中可以有多個隊列。
這些隊列並發運行、相互獨立,OpenCL 中沒有顯式的機制來對他們進行同步。

% Execution Model: Categories of Kernels
\subsection{執行模型:內核的種類}

OpenCL 執行模型支持兩種\cnglo{kernel}:
\startigBase
\item {\ftEmp{OpenCL \cnglo{kernel}}},
用 OpenCL C 編程語言編寫,並用 OpenCL C 編譯器編譯而成。
所有 OpenCL 的實作都支持 OpenCL \cnglo{kernel}。
實作也可能提供其他機制創建 OpenCL \cnglo{kernel}。

\item {\ftEmp{原生\cnglo{kernel}}},
通過\cnglo{host}函式指位器調用。
原生\cnglo{kernel}與 OpenCL \cnglo{kernel}一起入隊在\cnglo{device}上執行,
並共享\cnglo{memobj}。
例如,這些原生\cnglo{kernel}可以是\cnglo{app}代碼中定義的函式,也可以是從庫中導出的函式。
注意,在 OpenCL 中,
執行原生\cnglo{kernel}的能力是一個可選功能,原生\cnglo{kernel}的語義\cnglo{impdef}。
可以使用 OpenCL API 中的一些函式來查詢\cnglo{device}的能力
並確定\cnglo{device}是否具備某種能力。
\stopigBase



\section{內存模型}
\cnglo{workitem}在執行\cnglo{kernel}時可以存取四塊不同的\cnglo{memregion}:

\startigBase
\item \cngloemp{glbmem}:
所有\cnglo{workgrp}中的所有\cnglo{workitem}都可以對其進行讀寫。
\cnglo{workitem}可以讀寫此中\cnglo{memobj}的任意元素。
對\cnglo{glbmem}的讀寫可能會被緩存起來,這取決於\cnglo{device}的能力。

\item \cngloemp{constmem}:
\cnglo{glbmem}中的一塊區域,在\cnglo{kernel}的執行過程中保持不變。
\cnglo{host}負責對此中\cnglo{memobj}的分配和初始化。

\item \cngloemp{locmem}:
隸屬於某個\cnglo{workgrp}。
可以用來分配一些變量,這些變量由此\cnglo{workgrp}中的所有\cnglo{workitem}共享。
在 OpenCL \cnglo{device}上,可能會將其實現成一塊專用的\cnglo{memregion},
也可能將其映射到\cnglo{glbmem}中。

\item \cngloemp{prvmem}:
隸屬於某個\cnglo{workitem}。
一個\cnglo{workitem}的\cnglo{prvmem}中所定義的變量
對另外一個\cnglo{workitem}而言是不可見的。
\stopigBase

\reftab{memregion}列出了這些資訊:
\cnglo{kernel}或\cnglo{host}是否可以從某個\cnglo{memregion}中分配內存、
怎樣分配(靜態編譯時 vs. 動態運行時)
以及允許如何存取(即\cnglo{kernel}或\cnglo{host}是否可以對其進行讀寫)。

\placetable[here][tab:memregion]{內存區域——分配以及存取}{
\bTABLE
\setupTABLE[c][each][align={middle,lohi}]

\bTABLEhead
\bTR
\bTD \eTD
\bTD\cnglo{glbmem}\eTD \bTD\cnglo{constmem}\eTD
\bTD\cnglo{locmem}\eTD \bTD\cnglo{prvmem}\eTD
\eTR
\eTABLEhead

\bTABLEbody
\bTR
\bTD\cnglo{host}\eTD
\bTD 動態分配\par 可讀可寫\eTD \bTD 動態分配\par 可讀可寫\eTD
\bTD 動態分配\par 不可存取\eTD \bTD 不可分配\par 不可存取\eTD
\eTR

\bTR
\bTD\cnglo{kernel}\eTD
\bTD 不可分配\par 可讀可寫\eTD \bTD 靜態分配\par 只讀\eTD
\bTD 靜態分配\par 可讀可寫\eTD \bTD 靜態分配\par 可讀可寫\eTD
\eTR
\eTABLEbody

\eTABLE


}

\reffig{openclarch}描述了\cnglo{memregion}以及與\cnglo{platform}模型的關係。
圖中含有\cnglo{prcele}( PE )、\cnglo{computeunit}和\cnglo{device},
但是沒有畫出\cnglo{host}。


\startreusableMPgraphic{memArch}
picture picMain;
pair pairUR, pairLL, pairC;
pen penCmm;
color memColor;

u := 4mm;
v := 3.5mm;
ahangle := 30;
ahlength := .5v;
penCmm := pencircle scaled 2;
memColor := (1,0.93,0.98);

def drawGrid =
begingroup
fill fullcircle scaled u withcolor (0,255,255);

%right
for i=0 step u until (xpart pairUR):
	draw ((i, (ypart pairUR))--(i, (ypart pairLL))) withcolor 0.1[white, black];
	label(decimal(i/u), (0,0)) scaled 0.5 shifted (i, ypart pairUR) shifted (0,.5v);
endfor;
%left
for i=0 step -u until (xpart pairLL):
	draw ((i, (ypart pairUR))--(i, (ypart pairLL)))  withcolor 0.1[white, black];
	label(decimal(i/u), (0,0)) scaled 0.5  shifted (i, (ypart pairUR)) shifted (0,.5v);
endfor;

%top
for i=0 step 1 until ((ypart pairUR)/v):
	draw ((xpart pairLL), i*v)--((xpart pairUR), i*v) withcolor 0.1[white, black];
	label(decimal(i), (0,0)) shifted ((xpart pairLL),i*v) shifted (-.5u,0);
endfor;
%bottom
for i=0 step -1 until ((ypart pairLL)/v):
	draw ((xpart pairLL), i*v)--((xpart pairUR), i*v) withcolor 0.1[white, black];
	label(decimal(i), (0,0)) shifted ((xpart pairLL),i*v) shifted (-.5u,0);
endfor;
endgroup
enddef;

def pathPE =
unitsquare shifted(-0.5,-0.5) xscaled 4u yscaled v
enddef;
def pathPrvMem =
unitsquare shifted(-0.5,-0.5) xscaled 4u yscaled 2v
enddef;
def pathLocalMem =
unitsquare shifted(-0.5,-0.5) xscaled 4u yscaled 2v
enddef;
def pathGCMemCache =
unitsquare shifted(-0.5,-0.5) xscaled 26u yscaled 2v
enddef;
def pathGlbMem =
unitsquare shifted(-0.5,-0.5) xscaled 24u yscaled 2v
enddef;
def pathConstMem =
unitsquare shifted(-0.5,-0.5) xscaled 24u yscaled 2v
enddef;

def picPE(expr lbl) =
image(
fill pathPE withcolor (0.78,0.9,0.91);
draw pathPE;
label(lbl,(0,0));
)
enddef;

def picPrvMem(expr lbl) =
image(
fill pathPrvMem withcolor memColor;
draw pathPrvMem;
label(lbl,(0,0));
)
enddef;

def picLocalMem(expr lbl) =
image(
fill pathLocalMem withcolor memColor;
draw pathLocalMem;
label(lbl,(0,0));
)
enddef;

def picGlbMem =
image(
fill pathGlbMem withcolor memColor;
draw pathGlbMem;
label(btex \mplabel{\cnglo{glbmem}} etex,(0,0));
)
enddef;

def picConstMem =
image(
fill pathConstMem withcolor memColor;
draw pathConstMem;
label(btex \mplabel{\cnglo{constmem}} etex,(0,0));
)
enddef;

def picGCMemCache(expr lbl) =
image(
fill pathGCMemCache withcolor memColor;
draw pathGCMemCache;
label(lbl,(0,0));
)
enddef;

def picPeWithMem(expr lblPE, lblMem) =
image(
draw picPE(lblPE) shifted (0,-1.5v);
draw picPrvMem(lblMem) shifted (0,1v);
draw (0,0)--(0,-1v) withpen penCmm;
)
enddef;

def picDOTS =
image(
fill fullcircle scaled .5u;
fill fullcircle scaled .5u shifted (-.75u,0);
fill fullcircle scaled .5u shifted (.75u,0);
)
enddef;

def picCU(expr lbl) =
image(
	draw unitsquare shifted (-0.5,-0.5) xscaled 12u yscaled 6v;
	label(lbl, (0,0)) shifted (0,2.25v);
	draw picPeWithMem(btex \mplabel{\cnglo{prcele} 1} etex, btex \mplabel{\cnglo{prvmem} 1} etex) shifted (-3.5u,-0.5v);
	draw picDOTS;
	draw picPeWithMem(btex \mplabel{\cnglo{prcele} M} etex, btex \mplabel{\cnglo{prvmem} M} etex) shifted (3.5u,-0.5v);
)
enddef;

def picComputeDev =
image(
	draw unitsquare shifted (-0.5,-0.5) xscaled 28u yscaled 14v;
	label (btex \mplabel{計算設備} etex, (0,0)) scaled 1.2 shifted (0, 6v);
	draw picCU(btex \mplabel{\cnglo{computeunit} 1} etex) shifted (-7.5u, 2.5v);
	draw picLocalMem(btex \mplabel{\cnglo{locmem} 1} etex) shifted (-11u, -3v);
	drawdblarrow (-6u, -0.5v)--(-6u,-4.5v) withpen penCmm;
	drawdblarrow (-11u, -0.5v)--(-11u,-2v) withpen penCmm;
	draw picDOTS shifted (0,3v);
	draw picCU(btex \mplabel{\cnglo{computeunit} N} etex) shifted (7.5u, 2.5v);
	draw picLocalMem(btex \mplabel{\cnglo{locmem} N} etex) shifted (4u, -3v);
	drawdblarrow (9u, -0.5v)--(9u,-4.5v) withpen penCmm;
	drawdblarrow (4u, -0.5v)--(4u,-2v) withpen penCmm;
	draw picGCMemCache(btex \mplabel{\cnglo{glbmem}/\cnglo{constmem}數據緩存} etex) shifted (0,-5.5v);
)
enddef;

def picComputeDevMem =
image(
	draw unitsquare shifted (-0.5,-0.5) xscaled 26u yscaled 7v;
	label (btex \mplabel{\cnglo{computedevmem}} etex, (0,0)) scaled 1.2 shifted (0, -2.5v);
	draw picGlbMem shifted (0,2v);
	draw picConstMem shifted (0,-0.5v);
)
enddef;

picMain := image(
	draw picComputeDev shifted (0,5v);
	draw picComputeDevMem shifted (0,-6.5v);
	drawdblarrow (-7u,-1.5v)--(-7u,-3.5v) withpen penCmm;
	drawdblarrow (8u,-1.5v)--(8u,-6v) withpen penCmm;
);
pairUR := urcorner picMain;
pairLL := llcorner picMain;
pairC := center picMain;

%drawGrid;

draw picMain;
\stopreusableMPgraphic

\placefigure[here][fig:openclarch]
{OpenCL 設備架構的概念模型}
{\reuseMPgraphic{memArch}}

\cnglo{app}在\cnglo{host}上運行時,
使用 OpenCL API 在\cnglo{glbmem}中創建\cnglo{memobj},
並將內存命令(\refsection{exemodel:contextandcmdq}中有所描述)入隊以操作他們。

多數情況下,\cnglo{host}和 OpenCL \cnglo{device}的內存模型是相互獨立的。
其必然性主要在於 OpenCL 沒有囊括\cnglo{host}的定義。
然而,有時他們確實需要交互。
有兩種交互方式:顯式拷貝數據、將\cnglo{memobj}的部分区域映射和解映射。

為了顯式拷貝數據,\cnglo{host}會將一些\cnglo{cmd}插入隊列,
用來在\cnglo{memobj}和\cnglo{host}內存之間傳輸數據。
這些用於傳輸內存的命令可以是阻塞式的,也可以是非阻塞式的。
對於前者,一旦\cnglo{host}上相關內存資源可以被安全的重用,OpenCL 函式調用就會立刻返回。
而對於後者,一旦命令入隊,OpenCL 函式調用就會返回,而不管\cnglo{host}內存是否可以安全使用。

用映射、解映射的方法處理\cnglo{host}和 OpenCL \cnglo{memobj}的交互時,
\cnglo{host}可以將\cnglo{memobj}的某個區域映射到自己的位址空間中。
內存映射命令可能是阻塞的,也可能是非阻塞的。
一旦映射了\cnglo{memobj}的某個區域,\cnglo{host}就可以讀寫這塊區域。
當\cnglo{host}對這塊區域的存取(讀和/或寫)結束後,就會將其解映射。

% Memory Consistency
\subsection{內存一致性}
OpenCL 所使用的一致性內存模型比較寬鬆;
即,不保證不同\cnglo{workitem}所看到的內存狀態始終一致。

在\cnglo{workitem}內部,內存具有裝載、存儲的一致性。
在隸屬於同一\cnglo{workgrp}的\cnglo{workitem}之間,
\cnglo{locmem}在\cnglo{workgrpbarrier}上是一致的。
\cnglo{glbmem}亦是如此,
但對於執行同一\cnglo{kernel}的不同\cnglo{workgrp},則不保證其內存一致性。

對於已經入隊的\cnglo{cmd}所共享的\cnglo{memobj},其內存一致性由同步點來強制實施。


\section{編程模型}
OpenCL 執行模型支持\cnglo{dppm}和\cnglo{tppm},同時也支持這兩種模型的混合體。
對於 OpenCL 而言,驅動其設計的首要模型是數據並行。

% Data Parallel Programming Model
\subsection{數據並行編程模型}

\cnglo{dppm}依據同時應用到\cnglo{memobj}的多個元素上的指令序列來定義計算( computation )。
OpenCL 執行模型所關聯的索引空間定義了\cnglo{workitem},
以及數據怎樣映射到\cnglo{workitem}上。
在嚴格的數據並行模型中,\cnglo{workitem}和\cnglo{memobj}的元素間的映射關係為一對一,
\cnglo{kernel}可在這些元素上並行執行。
對於\cnglo{dppm},OpenCL 所實現的版本則比較寬鬆,不要求嚴格的一對一的映射。

OpenCL 所提供的\cnglo{dppm}是分級的。
有兩種方式來進行分級。
一種是顯式方式,程式員定義可以並行執行的\cnglo{workitem}的總數、
以及怎樣將這些\cnglo{workitem}劃分到\cnglo{workgrp}中。
另一種是隱式方式,程式員僅指定前者,後者由 OpenCL 實作來管理。

% Task Parallel Programming Model
\subsection{任務並行編程模型}
在 OpenCL 的\cnglo{tppm}中,\cnglo{kernel}的實體在執行時獨立於任何索引空間。
這在邏輯上等同於:在\cnglo{computeunit}上執行\cnglo{kernel}時,
相應\cnglo{workgrp}中只有一個\cnglo{workitem}。
這種模型下,用戶以如下方式表示並行:
\startigBase
\item 使用\cnglo{device}所實現的矢量數據型別;
\item 將多個任務入隊,和/或
\item 將多個原生\cnglo{kernel}入隊(他們是使用一個與 OpenCL 正交的編程模型開發的)。
\stopigBase

% Synchronization
\subsection{同步}
在 OpenCL 中,有兩方面的同步:
\startigBase
\item 隸屬同一\cnglo{workgrp}的\cnglo{workitem}之間;
\item 隸屬同一\cnglo{context}的\cnglo{cmdq}中的\cnglo{cmd}之間。
\stopigBase

前者是通過\cnglo{workgrpbarrier}實現的。
對於同一\cnglo{workgrp}中的所有\cnglo{workitem}來說,
任意一個要想越過\cnglo{barrier}繼續執行,
所有\cnglo{workitem}都必須先執行這個\cnglo{barrier}。
注意,在同一\cnglo{workgrp}中,
所有正在執行\cnglo{kernel}的\cnglo{workitem}必須都能執行到這個\cnglo{workgrpbarrier},
或者都不會去執行。
\cnglo{workgrp}之間沒有同步機制。

\cnglo{cmdq}中\cnglo{cmd}間的同步點是:
\startigBase
\item \cnglo{cmdqbarrier}。
他保證:所有之前排隊的\cnglo{cmd}都執行完畢,
並且他們對\cnglo{memobj}的所有更新,在後續\cnglo{cmd}開始執行前都是可見的。
他只能在隸屬同一\cnglo{cmdq}的\cnglo{cmd}間進行同步。

\item 等在一個事件上。
所有會將\cnglo{cmd}入隊的 OpenCL API 函式都會返回一個事件
(用來識別這個\cnglo{cmd}以及他所更新的\cnglo{memobj})。
如果某個後續\cnglo{cmd}正在等待那個事件,可以保證在其開始執行前,
可以見到對那些\cnglo{memobj}的所有更新。
\stopigBase


\section{內存對象}

\cnglo{memobj}分成兩類:\refglo{bufobj}和\refglo{imgobj}。
\refglo{bufobj}中所存儲的元素是一維的,
而\refglo{imgobj}則用來存儲二維或三維的材質、幀緩衝(frame-buffer)或圖像。

\refglo{bufobj}中的元素可以是標量數據型別(如 \ctype{int}、 \ctype{float})、
矢量數據型別或用戶自定義的結構體。
\refglo{imgobj}用來表示材質、幀緩衝等緩衝。
\cnglo{imgobj}中元素的格式必須從預定義格式中選取。
\cnglo{memobj}中至少要有一個元素。

\refglo{bufobj}和\refglo{imgobj}的根本區別是:
\startigBase
\item \refglo{bufobj}中的元素是順序存儲的,
\cnglo{kernel}在\cnglo{device}上運行時可以用指位器存取這些元素。
而\refglo{imgobj}中元素的存儲格式對用戶是透明的,不能通過指位器直接存取。
可以使用 OpenCL C 編程語言提供的內建函式來讀寫\cnglo{imgobj}。

\item 對於\refglo{bufobj},\cnglo{kernel}按其存儲格式存取其中的數據。
而對於\refglo{imgobj},其元素的存儲格式可能與\cnglo{kernel}中使用的數據格式不一樣。
\cnglo{kernel}中的圖像元素始終是四元矢量(每一元都可以是浮點型別或者帶符號/無符號整形)。
內建函式在讀寫圖像元素時會進行相應的格式轉換。
\stopigBase

\cnglo{memobj}是用 \ctype{cl_mem} 來表示的。
\cnglo{kernel}的輸入和輸出都是\cnglo{memobj}。


\section{OpenCL 框架}
OpenCL 框架中,
一個\cnglo{host}、加上不少於一個的 OpenCL \cnglo{device}
就可以構成一個異構並行計算機系統,並由\cnglo{app}來使用。
這個框架包含以下組件:
\startigBase
\item {\ftEmp{OpenCL 平台層}}:
\cnglo{host}\cnglo{program}可以發現 OpenCL \cnglo{device}及其能力,
也可以創建\cnglo{context}。

\item {\ftEmp{OpenCL 運行時}}:
一旦創建了\cnglo{context},\cnglo{host}\cnglo{program}就可以操控他。

\item {\ftEmp{OpenCL 編譯器}}:
OpenCL 編譯器可以創建含有 OpenCL \cnglo{kernel}的\cnglo{program}執行體。
他所實現的 OpenCL C 編程語言支持 ISO C99 的一個子集,並帶有並行擴展。
\stopigBase



\stopcomponent

