\startsection[
  title={The accounting method},
]

%e17.2-1
\startEXERCISE
假定對一個規模永遠不會超過 \m{k} 的棧執行一個操作序列。
執行 \m{k} 個操作後,我們複製整個棧進行備份。
通過爲不同的棧操作賦予合適的攤還代價,
證明: \m{n} 個棧操作(包括複製棧)的代價爲 \m{O(n)}。
\stopEXERCISE

\startANSWER
\input{tbl/tbl17.2-1}
\stopANSWER

%e17.2-2
\startEXERCISE
用覈算法重做\inexercise[power_cost]。
\stopEXERCISE

\startANSWER
\input{tbl/tbl17.2-2}
\stopANSWER

%e17.2-3
\startEXERCISE
假定我們不僅對計數器進行增 1 操作,
還會進行置 0 操作(即將所有位復位)。
設檢測或修改一位的時間爲 \m{\Theta(1)},
說明如何用一個位數列來實現計數器,
使得對一個初值爲 0 的計數器執行一個由任意 \m{n} 個 \ALGO{INCREMENT} 和 \ALGO{RESET} 操作組成的序列花費時間爲 \m{O(n)}。
(\hint 維護一個指針一直指向最高位的 1)
\stopEXERCISE

\startANSWER
\input{tbl/tbl17.2-3}
\stopANSWER

\stopsection
