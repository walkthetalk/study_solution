\startcomponent c_number_theoretic_algorithms

\startchapter[
  title={Number-Theoretic Algorithms},
  reference=chapter:number_theoretic,
]

\startformula\startmathalignment
\NC \text{自然數集} \NC \naturalnumbers \NR
\NC \text{整數集} \NC \integers \NR
\NC \text{有理數} \NC \rationals \NR
\NC \text{實數} \NC \reals \NR
\NC \text{復數} \NC \complexes \NR
\NC \text{素數} \NC \primes \NR
\stopmathalignment\stopformula

\startsection[
  title={Elementary number-theoretic notions},
]

%e31.1-1
\startEXERCISE
證明:如果 \m{a>b>0},且 \m{c=a+b},則 \m{c\mod a = b}。
\stopEXERCISE

\startANSWER
\startformula\startmathalignment
\NC c \mod a
       =\NC (a+b) \mod a \NR
\NC =\NC a \mod a + b \mod a \NR
\NC =\NC 0 + b \mod a \NR
\NC =\NC b \mod a \NR
\stopmathalignment\stopformula
由於 \m{a>b>0},因此 \m{b\mod a = b},所以 \m{c\mod a = b}。
\stopANSWER

%e31.1-2
\startEXERCISE
證明有無窮多個素數。(\hint 證明素數 \m{p_1}、 \m{p_2}、 \m{\ldots}、 \m{p_k} 都不能整除 \m{(p_1 p_2 \ldots p_k) + 1}。)
\stopEXERCISE

\startANSWER
假設素數個數是有限的,且由小至大依次爲 \m{p_1}、 \m{p_2}、 \m{\ldots}、 \m{p_k}。令 \m{m = p_1 p_2 \ldots p_k}, \m{n=m+1}:
\startigBase[n]
\startitem
如果 n 爲素數:則 \m{n} 大於 \m{p_1}、 \m{p_2}、 \m{\ldots}、 \m{p_k},從而 \m{p_k} 不是最大素數,矛盾;
\stopitem
\startitem
如果 n 爲合數:由於 m 和 n 的對大公約數爲 1,所以 n 不能被 \m{p_1}、 \m{p_2}、 \m{\ldots}、 \m{p_k} 整除,但是任一合數都可分解爲素數的乘積,所以分解得到的素數不屬於 \m{p_1}、 \m{p_2}、 \m{\ldots}、 \m{p_k},矛盾。
\stopitem
\stopigBase
即無論 n 是素數還是合數,都意味着在假設的有限個素數之外還存在其他素數。
所以有無窮多個素數。
\stopANSWER

%e31.1-3
\startEXERCISE
證明:如果 \m{a|b} 且 \m{b|c},則 \m{a|c}。
\stopEXERCISE

\startANSWER
\startformula\startmathalignment
\NC a|b \Rightarrow \NC b=q_1 a \NR
\NC b|c \Rightarrow \NC c=q_2 b \NR
\stopmathalignment\stopformula
因此 \m{c\mod a = (q_2 b)\mod a = (q_2 q_1 a)\mod a = 0},即 \m{a|c}。
\stopANSWER

%e31.1-4
\startEXERCISE
證明:如果 p 是素數,且 \m{0<k<p},則 \m{\gcd(k,p)=1}。
\stopEXERCISE

\startANSWER
令 \m{n=\gcd(k,p)},則 \m{n\le k < p},且 \m{n|p}。
由於 p 是素數,因此 \m{n|p} 意味着 \m{n=1} 或 \m{n=p}。
又由於 \m{n<p},因此 \m{n=1},即 \m{\gcd(k,p)=1}。
\stopANSWER

%e31.1-5
\startEXERCISE
證明推論 31.5。
(附{\EMP 推論 31.5}:
對於任意正整數 n、 a 和 b,如果 \m{n|ab} 且 \m{\gcd(a,n)=1},則 \m{n|b})。
\stopEXERCISE

\startANSWER
由於 \m{\gcd(a,n)=1},因此有 \m{ax+ny=1},所以 \m{abx+nby=b}。
模 n 得 \m{b\mod n = (abx+nby)\mod n = (abx\mod n)+(nby\mod n) = 0 + 0 = 0},所以 \m{n|b}。
\stopANSWER

%e31.1-6
\startEXERCISE
證明:如果 p 是素數,且 \m{0<k<p},則 \m{p|\binom{p}{k}}。
從而對於所有整數 a、 b 以及所有素數 p:
\startformula
(a+b)^p\equiv a^p + b^p \qquad (\mod p)
\stopformula
\stopEXERCISE

\startANSWER
\startformula\startmathalignment
\NC \binom{p}{k}
   =\NC \frac{p!}{(p-k)!k!} \NR
\NC=\NC \frac{p (p-1)!}{(p-k) (p-1-k)! k!} \NR
\NC=\NC \frac{\prod_{i=1}^{k}(p-i+1)}{\prod_{i=1}^{k}{i}} \NR
\NC=\NC p \frac{\prod_{i=2}^{k}(p-i+1)}{\prod_{i=2}^{k}{i}} \qquad \text{令} a = \prod_{i=2}^{k}(p-i+1)\text{, } b=\prod_{i=2}^{k}{i} \NR
\NC=\NC \frac{p a}{b} \NR
\stopmathalignment\stopformula
由於 p 是素數,且 \m{0<k<p},所以 \m{\gcd(p,b)=1}。而又由於 \m{\binom{p}{k}} 是整數,所以 \m{b|pa},由推論 31.5 可知 \m{b|a}。因此 \m{\frac{b}{a}} 是正整數,所以 \m{\binom{p}{k} \mod p = p \frac{a}{b} \mod p = 0},即 \m{p|\binom{p}{k}}。

\startformula\startmathalignment
\NC(a+b)^p
   =\NC \sum_{k=0}^{p}\binom{p}{k}a^k b^{p-k} \NR
\NC=\NC a^p + b^p + \sum_{k=1}^{p-1}\binom{p}{k}a^k b^{p-k} \NR
\stopmathalignment\stopformula
\stopANSWER

%e31.1-7
\startEXERCISE
證明:如果 a 和 b 是任意正整數,且 \m{a|b},那麼對於任意整數 x,都有 \m{(x\mod b)\mod a = x \mod a}。

證明:在同樣假設下,對於任意整數 x 和 y, \m{x\equiv y \qquad(\mod b)} 意味着 \m{x\equiv y \qquad(\mod a)}。
\stopEXERCISE

\startANSWER
令 \m{x=qb+r},且 \m{0<r<b},則:
\startformula\startmathalignment[n=1]
\NC (x\mod b)\mod a = r\mod a \NR
\NC x\mod a = (qb+r)\mod a = (qb)\mod a + r\mod a = 0 + r\mod a = r\mod a \NR
\stopmathalignment\stopformula

令 \m{x=q_x b + r}、 \m{y=q_y b + r},且 \m{0<r<b},則:
\startformula\startmathalignment[n=1]
\NC x\mod a = (q_x b + r)\mod a = (q_x b)\mod a + r\mod a = 0 + r\mod a = r\mod a \NR
\NC y\mod a = (q_y b + r)\mod a = (q_y b)\mod a + r\mod a = 0 + r\mod a = r\mod a \NR
\stopmathalignment\stopformula
\stopANSWER

%e31.1-8
\startEXERCISE
對於任一正整數 k,如果存在整數 a,使得 \m{a^k = n},則稱整數 n 是一個 {\EMP k 次冪}。進而,如果 \m{n>1} 是一個 k 次冪,且 \m{k>1},則稱 n 是{\EMP 非平凡冪}。
如何在關於 \m{\beta} 的多項式時間內確定給定的 \m{\beta} 位整數 n 是否是一個非平凡冪。
\stopEXERCISE

\startANSWER
\startformula
2^{\beta - 1} \le n = \sum_{i=0}^{\beta-1}a_i 2^i < 2^\beta
\stopformula
我們只需分析 n 是否是某個整數 a 的素數次冪,即只需分析 k 爲素數的情況。
即 n 是非平凡冪與下列描述等價:

存在整數 a,使得 \m{n=a^k},其中 k 爲素數。

由於 \m{a\ge 2},所以 \m{k<\beta},令小於 \m{\beta} 的素數集合爲 \m{\primes_\beta},則 \m{k\in \primes_\beta}。遍歷 \m{\primes_\beta} 中的元素,用二分查找搜索 a。
\stopANSWER

%e31.1-9
\startEXERCISE
證明公式 31.6 ~ 31.10。
\startformula\startmathalignment[
  n=3,
  align={left,left,right},]
\NC \gcd(a,b)  \NC = \gcd(b,a)     \NC (31.6) \NR
\NC \gcd(a,b)  \NC = \gcd(-a,b)    \NC (31.7) \NR
\NC \gcd(a,b)  \NC = \gcd(|a|,|b|) \NC (31.8) \NR
\NC \gcd(a,0)  \NC = |a|           \NC (31.9) \NR
\NC \gcd(a,ka) \NC = |a| \qquad \text{對任意 }k\in\integers \NC \qquad (31.10) \NR
\stopmathalignment\stopformula
\stopEXERCISE

\startANSWER
定理 31.2: \m{\gcd(a,b)} 即爲集合 \m{\{ax+by: x,y\in \integers\}} 中的最小正整數。
由於這個定義是對稱式的,所以 \m{\gcd(a,b)=\gcd(b,a)}。

又由於 x 和 y 是任意整數,所以 \m{\gcd(a,b)=\gcd(-a,b)=\gcd(|a|,|b|)}。

而 \m{\gcd(a,0)} 是集合 \m{\{ax: x\in\integers\}} 中的最小正整數,
即 \m{x=\pm 1} 的時候,乘 a 的結果是正整數或者 \m{|a|}。

\m{\gcd(a,ka)} 是集合 \m{\{ax+kay: x,y\in \integers\} = \{a(x+ky): x,y\in\integers\}} 中的最小正整數。此時 \m{x+ky=\pm 1},即 \m{a(x+ky)=|a|}。
\stopANSWER

%e31.1-10
\startEXERCISE
證明: \m{\gcd} 滿足結合律。即證明對於所有整數 a、 b 和 c,有:
\startformula
\gcd(a,\gcd(b,c)) = \gcd(\gcd(a,b),c)
\stopformula
\stopEXERCISE

\startANSWER
令 \m{\gcd(a,\gcd(b,c)) = A}, \m{\gcd(\gcd(a,b),c) = B},則:
\startformula\startmathalignment[n=1]
\NC A|a \qquad A|\gcd(b,c) \NR
\NC A|a \qquad A|b \qquad A|c \NR
\NC A|\gcd(a,b) \qquad A|c \NR
\NC A|\gcd(\gcd(a,b),c) \NR
\NC A|B \NR
\stopmathalignment\stopformula
同理 \m{B|A},所以 \m{A=B}。
\stopANSWER

%e31.1-11
\startEXERCISE\DIFFICULT
證明定理 31.8。(附{\EMP 定理 31.8}:唯一因子分解定理,合數 a 僅能以一種方式寫成如下乘積形式:
\startformula
a = p_1^{e_1}p_2^{e_2}\ldots p_r^{e_r}
\stopformula
其中 \m{p_i} 爲素數, \m{p_1<p_2<\ldots<p_r},且 \m{e_i} 爲正整數。)
\stopEXERCISE

\startANSWER
用反證法,先反證質數底相同,再反正冪相同。
\stopANSWER

%e31.1-12
\startEXERCISE
如何高效地計算 \m{\beta} 位整數除以短整數的商和餘數。
算法運行時間應爲 \m{\Theta(\beta^2)}。
\stopEXERCISE

\startANSWER
用移位、比較和加法進行運算
\stopANSWER

%e31.1-13
\startEXERCISE
給出一個高效算法,用於將給定的 \m{\beta} 位(二進制)整數轉換承拾進制。
證明:如果長度不大於 \m{\beta} 的整數乘法和除法所需時間爲 \m{M(\beta)},
則我們可以在時間 \m{\Theta(M(\beta)\lg\beta)} 內將二進制轉換成拾進制。
(\hint 分治策略,用遞迴獨立計算前段和後段)
\stopEXERCISE

\startANSWER
\CLRSH{BIN-TO-DEC(B, begin, end)}
\startCLRS
if begin == end
	return (B[begin], 1)

center = (begin + end) / 2
(Db, Rb) = BIN-TO-DEC(B, begin, center)
(Dt, Rt) = BIN-TO-DEC(B, center + 1, end)
return (Dt * Rb * 2 + Db, Rb * Rt)
\stopCLRS
\stopANSWER

\stopsection

\startsection[
  title={Greatest common divisor},
]

%e31.2-1
\startEXERCISE
證明:由式(31.11)和式(31.12)可推出式(31.13)。附:

\startformula\startmathalignment[n=3,align={right,left,right}]
\NC a = \NC p_1^{e_1} p_2^{e_2} \ldots p_r^{e_r} \NC (31.11) \NR
\NC b = \NC p_1^{f_1} p_2^{f_2} \ldots p_r^{f_r} \NC (31.12) \NR
\NC \gcd(a,b) = \NC p_1^{\min(e_1,f_1)} p_2^{\min(e_2,f_2)} \ldots p_r^{\min(e_r,f_r)} \NC \qquad (31.13) \NR
\stopmathalignment\stopformula
\stopEXERCISE

\startANSWER
令 \m{d=\gcd(a,b)},則 d 也可以寫成如下形式:
\startformula
d = p_1^{g_1} p_2^{g_2} \ldots p_r^{g_r}
\stopformula

由於 \m{d|a} 且 \m{d|b},則:
\startformula
p_i^{g_i} | a \qquad p_i^{g_i} | b
\stopformula
其中 \m{i=1,2,\ldots,r}。由於 \m{p_i} 爲素數,所以:
\startformula
p_i^{g_i} | p_i^{e_i} \qquad p_i^{g_i} | p_i^{f_i}
\stopformula

所以 \m{g_i\le e_i} 且 \m{g_i\le f_i},即 \m{g_i=\min(e_i,f_i)}。
\stopANSWER

%e31.2-2
\startEXERCISE
計算 \ALGO{EXTENDED-EUCLID(899, 493)} 的返回值 \m{(d, x, y)}。
\stopEXERCISE

\startANSWER

\startxtable[
    option=max,
    align={middle,lohi},
    split=yes,
    header=repeat,
    footer=repeat,
    offset=.25em,
]

% head
\startxtablehead[frame=off,bottomframe=on]
\startxrow[foregroundstyle=bold,]
  \xcell[rightframe=on]{}\processcommalist[\m{a}, \m{b}, \m{\left\lfloor a/b \right\rfloor}, \m{d}, \m{x}, \m{y}]{\xcell[align={middle}]}
\stopxrow
\stopxtablehead

% body
\startxtablebody[frame=off]
\startxrow \xcell[rightframe=on]{0}\processcommalist[899, 493,  1, 29, -6, 11]\xcell \stopxrow
\startxrow \xcell[rightframe=on]{1}\processcommalist[493, 406,  1, 29,  5, -6]\xcell \stopxrow
\startxrow \xcell[rightframe=on]{2}\processcommalist[406,  87,  4, 29, -1,  5]\xcell \stopxrow
\startxrow \xcell[rightframe=on]{3}\processcommalist[ 87,  58,  1, 29,  1, -1]\xcell \stopxrow
\startxrow \xcell[rightframe=on]{4}\processcommalist[ 58,  29,  2, 29,  0,  1]\xcell \stopxrow
\startxrow \xcell[rightframe=on]{5}\processcommalist[ 29,   0,  -, 29,  1,  0]\xcell \stopxrow
\stopxtablebody

\stopxtable

\stopANSWER

%e31.2-3
\startEXERCISE
證明:對於所有整數 a、 k 和 n,下式都成立:
\startformula
\gcd(a,n) = \gcd(a+kn,n)
\stopformula
\stopEXERCISE

\startANSWER
由 \ALGO{EUCLID} 計算最大公約數,
第一步 \m{a\mod n} 與 \m{(a+kn)\mod n} 的結果相同,
因此 \ALGO{EUCLID(a, n)} 和 \ALGO{EUCLID(a+kn, n)} 的結果相同。
\stopANSWER

%e31.2-4
\startEXERCISE
以迭代的形式重寫 \ALGO{EUCLID},要求只能使用常數內存(即,僅存儲常數個整數)。
\stopEXERCISE

\startANSWER
\CLRSH{ITER-EUCLID(a, b)}
\startCLRS
while b != 0
	tmp = a % b
	a = b
	b = tmp
return a
\stopCLRS
\stopANSWER

%e31.2-5
\startEXERCISE[exercise:euclid_time]
證明:如果 \m{a > b\ge 0},則 \ALGO{EUCLID(a, b)} 最多執行 \m{1 + \log_\phi b} 次遞歸調用。
將這個界改進爲 \m{1+\log_\phi (b/\gcd(a,b))}。
\stopEXERCISE

\startANSWER
利用 Fibonacci 序列,即 \m{F_0 = 0}, \m{F_1 = 1},且:
\startformula
F_k = F_{k-1} + F_{k-2} \qquad \text{對於} k\ge 2
\stopformula

當 k 比較大時,\m{F_{k}} 約爲 \m{\phi^k/\sqrt{5}}。
其中 \m{\phi} 爲黃金比例,值爲:
\startformula
\phi = \frac{1+\sqrt{5}}{2} = 1.61803\ldots
\stopformula

由 Lamè 定理可知,如果 \m{F_{k+1}>b},則 \ALGO{EUCLID(a,b)} 執行的遞迴調用少於 k 次。
 k 是滿足下式的最小整數:
\startformula
\frac{\phi_{k+1}}{\sqrt{5}} > b
\stopformula

即, k 是滿足下式的最小整數:
\startformula
k > \log_{\phi}b + \log_{\phi}\sqrt{5} - 1 = \log_{\phi}b + 1.67 - 1 = \log_{\phi}b + 0.67
\stopformula
解得 \m{k \le \left\lfloor \log_{\phi}b + \log_{\phi}\sqrt{5} - 1 + 0.5 \right\rfloor}:

因此, \ALGO{EUCLID(a, b)} 最多執行 \m{1 + \log_\phi b} 次遞歸調用。


令 \m{d = \gcd(a,b)},由 \ALGO{EUCLID} 的實現可知, \ALGO{EUCLID(a,b)} 的遞迴調用次數與 \ALGO{EUCLID(a/d, b/d)} 相同。
將上面的證明過程中的 \m{b} 替換爲 \m{b/d},即得改進後的界: \m{1+\log_\phi(b/\gcd(a,b))}。
\stopANSWER

%e31.2-6
\startEXERCISE
計算 \ALGO{EXTENDED-EUCLID(F_{k+1}, F_k)} 返回的值,並證明你的結論。
\stopEXERCISE

\startANSWER
令 \m{f(k) = 1 - 2 (k\mod 2)},即如果 k 爲偶數,則 \m{f(k)} 爲 1,否則爲 -1。
結果爲 \m{(1, f(k) F_k, -f(k) F_{k+1})}。

如果結論對 k 成立,則:
\startformula\startmathalignment
\NC EXTEND-EUCLID(F_{k+2}, F_{k+1})
    =\NC (d, -f(k) F_{k+1}, f(k) F_k - \left\lfloor\frac{F_{k+2}}{F_{k+1}}\right\rfloor (-f(k) F_{k+1})) \NR
\NC =\NC (d, -f(k) F_{k+1}, f(k) (F_k + \left\lfloor\frac{F_{k+2}}{F_{k+1}}\right\rfloor F_{k+1})) \NR
\NC =\NC (d, -f(k) F_{k+1}, f(k) F_{k+2}) \NR
\NC =\NC (d, f(k+1) F_{k+1}, -f(k+1) F_{k+2}) \NR
\stopmathalignment\stopformula
即結論對 \m{k+1} 也成立。
\stopANSWER

%31.2-7
\startEXERCISE
通過遞迴爲多於兩個的參數定義函數 \m{\gcd}: \m{\gcd(a_0,a_1,\ldots,a_n) = \gcd(a_0,\gcd(a_1,a_2,\ldots,a_n))}。
證明函數 \m{\gcd} 的返回值不依賴於參數的順序。
如何找到整數 \m{x_0,x_1,\ldots,x_n},使得:
\startformula
\gcd(a_0,a_1,\ldots,a_n) = a_0x_0 + a_1x_1 + \ldots + a_nx_n
\stopformula
並證明你的算法所執行的除法次數爲:
\startformula
O(n + \lg(\max(a_0,a_1,\ldots,a_n)))
\stopformula
\stopEXERCISE

\startANSWER
令 \m{d=\gcd(a_0,a_1,\ldots,a_n)}, \m{d'=\gcd(a_0,\gcd(a_1,a_2,\ldots,a_n))}:
因此 \m{d} 和 \m{d'} 均是 \m{a_0,a_1,\ldots,a_n} 的公約數。
由前至後得 \m{d \ge d'},而由後至前得 \m{d'\ge d},因此 \m{d' = d}。
即無論 \m{d'} 中的 \m{a_i} 順序如何,都有 \m{d'=d}。

根據最大公約數的定義,可以寫成如下形式:
\startformula
\gcd(a_0,a_1,\ldots,a_n) = a_0x_0 + a_1x_1 + a_2x_2 + \ldots + a_nx_n
\stopformula

以四個參數爲例:
\startformula\startmathalignment
\NC   \NC \gcd(a_0,a_1,a_2,a_3) \NR
\NC = \NC \gcd(a_0,\gcd(a_1,a_2,a_3)) \NR
\NC = \NC \gcd(a_0,\gcd(a_1,\gcd(a_2,a_3))) \NR
\stopmathalignment\stopformula
根據定理 31.2 可以將 \m{\gcd(a_2,a_3)} 寫爲 \m{a_2x_2 + a_3x_3}。所以:
\startformula\startmathalignment
\NC   \NC \gcd(a_0,a_1,a_2,a_3) \NR
\NC = \NC \gcd(a_0,\gcd(a_1,a_2,a_3)) \NR
\NC = \NC \gcd(a_0,\gcd(a_1, a_2x_2 + a_3x_3)) \NR
\stopmathalignment\stopformula
然後根據定理 31.2 將 \m{\gcd(a_1, a_2x_2 + a_3x_3)} 寫爲:
\startformula\startmathalignment
\NC   \NC \gcd(a_1, a_2x_2 + a_3x_3) \NR
\NC = \NC x_1a_1 + y_1(a_2x_2 + a_3x_3) \NR
\NC = \NC x_1a_1 + y_1x_2a_2 + y_1x_3a_3 \NR
\stopmathalignment\stopformula
因此:
\startformula\startmathalignment
\NC   \NC \gcd(a_0, a_1, a_2, a3) \NR
\NC = \NC \gcd(a_0, x_1a_1 + y_1x_2a_2 + y_1x_3a_3) \NR
\stopmathalignment\stopformula
從而:
\startformula\startmathalignment
\NC   \NC \gcd(a_0, a_1, a_2, a3) \NR
\NC = \NC x_0a_0 + y_2(x_1a_1 + y_1x_2a_2 + y_1x_3a_3) \NR
\NC = \NC x_0a_0 + y_2x_1a_1 + y_1y_2x_2a_2 + y_1y_2x_3a_3 \NR
\stopmathalignment\stopformula

由\refexercise{euclid_time} 可知: \m{1+\log_\phi (\min(a,b)/\gcd(a,b))}。
總的次數爲:
\startformula\startmathalignment[n=3]
\NC     \NC   \NC 1+\log_\phi\frac{\min(a_0,\gcd(a_1,a_2,\ldots,a_n))}{\gcd(a_0,\gcd(a_1,a_2,\ldots,a_n))} \NR
\NC     \NC + \NC 1+\log_\phi\frac{\min(a_1,\gcd(a_2,a_3,\ldots,a_n))}{\gcd(a_1,\gcd(a_2,a_3,\ldots,a_n))} \NR
\NC     \NC + \NC 1+\log_\phi\frac{\min(a_2,\gcd(a_3,a_4,\ldots,a_n))}{\gcd(a_1,\gcd(a_3,a_4,\ldots,a_n))} \NR
\NC     \NC + \NC \ldots \NR
\NC     \NC + \NC 1+\log_\phi\frac{\min(a_{n-1},a_n)}{\gcd(a_{n-1},a_n)} \NR
\NC \le \NC   \NC 1+\log_\phi\frac{\gcd(a_1,a_2,\ldots,a_n)}{\gcd(a_0,\gcd(a_1,a_2,\ldots,a_n))} \NR
\NC     \NC + \NC 1+\log_\phi\frac{\gcd(a_2,a_3,\ldots,a_n)}{\gcd(a_1,\gcd(a_2,a_3,\ldots,a_n))} \NR
\NC     \NC + \NC 1+\log_\phi\frac{\gcd(a_3,a_4,\ldots,a_n)}{\gcd(a_1,\gcd(a_3,a_4,\ldots,a_n))} \NR
\NC     \NC + \NC \ldots \NR
\NC     \NC + \NC 1+\log_\phi\frac{a_n}{\gcd(a_{n-1},a_n)} \NR
\NC =   \NC   \NC n-1 + \log_\phi \frac{a_n}{\gcd(a_0,a_1,\ldots,a_n)}\NR
\NC \le \NC   \NC n-1 + \log_\phi \max(a_0,a_1,\ldots,a_n) \NR
\stopmathalignment\stopformula
\stopANSWER

%31.2-8
\startEXERCISE
定義 \m{\lcm(a_1,a_2,\ldots,a_n)} 爲 n 個整數 \m{a_1,a_2,\ldots,a_n} 的{\EMP 最小公倍數(least common multiple)},
即,是所有 \m{a_i} 的倍數,且是最小的非負整數。
如何利用參數爲兩個的 \m{\gcd} 運算有效的計算 \m{\lcm(a_1,a_2,\ldots,a_n)}?
\stopEXERCISE

\startANSWER
令 \m{d = \gcd(a_0,a_1,\ldots,a_n)}:
\startformula\startmathalignment
\NC  \NC \lcm(a_0,a_1,\ldots,a_n) \NR
\NC =\NC d \frac{a_0}{d} \frac{a_1}{d} \ldots \frac{a_n}{d} \NR
\NC =\NC \frac{\prod_{i=0}^{n}a_i}{d^n} \NR
\stopmathalignment\stopformula
\stopANSWER

%e31.2-9
\startEXERCISE
證明:當且僅當 \m{\gcd(n_1n_2,n_3n_4) = \gcd(n_1n_3,n_2n_4) = 1} 時,
四個正整數 \m{n_1,n_2,n_3,n_4} 兩兩互素。

更一般地,當且僅當從 \m{n_i} 中導出的 \m{\lceil \lg{k}\rceil} 對整數互素時,
 \m{n_1,n_2,\ldots,n_k} 兩兩互素。
\stopEXERCISE

\startANSWER
如果 \m{n_1,n_2,n_3,n_4} 兩兩互素,則:
\startformula\startmathalignment[n=1]
\NC \gcd(n_1,n_3) = \gcd(n_2,n_3) = 1 \NR
\NC \gcd(n_1 n_2,n_3) = 1 \qquad \gcd(n_1 n_2,n_4) = 1 \qquad \text{(由定理 31.6 可得)} \NR
\NC \gcd(n_1 n_2, n_3 n_4) = \gcd(n_1 n_3, n_2 n_4) = 1 \qquad \text{(由定理 31.6 可得)} \NR
\stopmathalignment\stopformula

而如果 \m{\gcd(n_1 n_2, n_3 n_4) = \gcd(n_1 n_3, n_2 n_4) = 1},則存在整數 x、 y:
\startformula
n_1 n_3 x + n_2 n_4 y = 1
\stopformula
可以將其看作是 \m{n_1, n_2} 的線性組合,所以 \m{\gcd(n_1,n_2) = 1}。
同理可得其他,所以 \m{n_1,n_2,n_3,n_4} 兩兩互素。

同理可證更一般地情況。
\stopANSWER

\stopsection

\startsection[
  title={Modular arithmetic},
]

%e31.3-1
\startEXERCISE
畫出羣\m{(\integers_4, +_4)} 和羣 \m{(\integers_5^\ast, \cdot_5)} 的運算表。
通過找這兩個羣的元素間的一一對應關係 \alpha,
滿足 \m{a+b\equiv c (\mod 4)} 當且僅當 \m{\alpha(a)\cdot \alpha(b) \equiv \alpha(c) (\mod 5)},
來證明這兩個羣是同構的。
\stopEXERCISE

\startANSWER
先找到各自的單位元(Identify) \m{a_0},再找到 \m{a_1},使得 \m{a_1^{(2)} = a_0}。
以此類推,找到 \m{a_2} 和 \m{a_3}。

\startcolumns[n=3, rule=on]
\startformula\startmathalignment[n=3]
\NC +_4 \NC \rightarrow \NC \cdot_5 \NR
\NC 0 \NC \rightarrow \NC 1 \NR
\NC 1 \NC \rightarrow \NC 3 \NR
\NC 2 \NC \rightarrow \NC 4 \NR
\NC 3 \NC \rightarrow \NC 1 \NR
\stopmathalignment\stopformula
\column

\startxtable[
    option=max,
    align={middle,lohi},
    split=yes,
    header=repeat,
    footer=repeat,
    offset=.25em,
]

% head
\startxtablehead[frame=off,bottomframe=on]
\startxrow[foregroundstyle=bold,]
  \xcell[rightframe=on]{\m{+_4}}\processcommalist[0, 1, 2, 3]{\xcell[align={middle}]}
\stopxrow
\stopxtablehead

% body
\startxtablebody[frame=off]
\startxrow \xcell[rightframe=on]{0}\processcommalist[0,1,2,3]\xcell \stopxrow
\startxrow \xcell[rightframe=on]{1}\processcommalist[1,2,3,0]\xcell \stopxrow
\startxrow \xcell[rightframe=on]{2}\processcommalist[2,3,0,1]\xcell \stopxrow
\startxrow \xcell[rightframe=on]{3}\processcommalist[3,0,1,2]\xcell \stopxrow
\stopxtablebody

\stopxtable

\column
\startxtable[
    option=max,
    align={middle,lohi},
    split=yes,
    header=repeat,
    footer=repeat,
    offset=.25em,
]

% head
\startxtablehead[frame=off,bottomframe=on]
\startxrow[foregroundstyle=bold,]
  \xcell[rightframe=on]{\m{\cdot_5}}\processcommalist[1, 2, 3, 4]{\xcell[align={middle}]}
\stopxrow
\stopxtablehead

% body
\startxtablebody[frame=off]
\startxrow \xcell[rightframe=on]{1}\processcommalist[1,2,3,4]\xcell \stopxrow
\startxrow \xcell[rightframe=on]{2}\processcommalist[2,4,1,3]\xcell \stopxrow
\startxrow \xcell[rightframe=on]{3}\processcommalist[3,1,4,2]\xcell \stopxrow
\startxrow \xcell[rightframe=on]{4}\processcommalist[4,3,2,1]\xcell \stopxrow
\stopxtablebody

\stopxtable


\stopcolumns
\stopANSWER

%e31.3-2
\startEXERCISE
列出 \m{\integers_9} 和 \m{\integers_{13}^\ast} 的所有子羣。
\stopEXERCISE

\startANSWER
\startcolumns[n=2, rule=on]
\startformula\startmathalignment
\NC \langle 0 \rangle \NC= \{ 0 \} \NR
\NC \langle 1 \rangle \NC= \{ 0, 1, 2, 3, 4, 5, 6, 7, 8 \} \NR
\NC \langle 2 \rangle \NC= \{ 0, 1, 2, 3, 4, 5, 6, 7, 8 \} \NR
\NC \langle 3 \rangle \NC= \{ 0, 3, 6 \} \NR
\NC \langle 4 \rangle \NC= \{ 0, 1, 2, 3, 4, 5, 6, 7, 8 \} \NR
\NC \langle 5 \rangle \NC= \{ 0, 1, 2, 3, 4, 5, 6, 7, 8 \} \NR
\NC \langle 6 \rangle \NC= \{ 0, 3, 6 \} \NR
\NC \langle 7 \rangle \NC= \{ 0, 1, 2, 3, 4, 5, 6, 7, 8 \} \NR
\NC \langle 8 \rangle \NC= \{ 0, 1, 2, 3, 4, 5, 6, 7, 8 \} \NR
\stopmathalignment\stopformula
\column
\startformula\startmathalignment
\NC \langle 1 \rangle \NC= \{ 1 \} \NR
\NC \langle 2 \rangle \NC= \{ 1, 2, 3, 4, 5, 6, 7, 8, 9, 10, 11, 12 \} \NR
\NC \langle 3 \rangle \NC= \{ 1, 3, 9 \} \NR
\NC \langle 4 \rangle \NC= \{ 1, 3, 4, 9, 10, 12 \} \NR
\NC \langle 5 \rangle \NC= \{ 1, 5, 8, 12 \} \NR
\NC \langle 6 \rangle \NC= \{ 1, 2, 3, 4, 5, 6, 7, 8, 9, 10, 11, 12 \} \NR
\NC \langle 7 \rangle \NC= \{ 1, 2, 3, 4, 5, 6, 7, 8, 9, 10, 11, 12 \} \NR
\NC \langle 8 \rangle \NC= \{ 1, 5, 8, 12 \} \NR
\NC \langle 9 \rangle \NC= \{ 1, 3, 9 \} \NR
\NC \langle 10 \rangle \NC= \{ 1, 3, 4, 9, 10, 12 \} \NR
\NC \langle 11 \rangle \NC= \{ 1, 2, 3, 4, 5, 6, 7, 8, 9, 10, 11, 12 \} \NR
\NC \langle 12 \rangle \NC= \{ 1, 12 \} \NR
\stopmathalignment\stopformula
\stopcolumns
\stopANSWER

%e31.3-3
\startEXERCISE
證明定理 31.14。附:

定理 31.14 (一個有限羣的非空封閉子集是一個子羣):如果 \m{(S,\oplus)} 是一個有限羣,
 \m{S'} 是 \m{S} 的任意一個非空子集並滿足對所有 \m{a,b\in S'},都有 \m{a\oplus b\in S'},
則 \m{(S',\oplus)} 是 \m{(S,\oplus)} 的一個子羣。
\stopEXERCISE

\startANSWER
只需證明 \m{S'} 是一個羣即可。
\startigBase[n]
\item {\EMP 封閉性}:假設已經滿足;
\item {\EMP 單位元}:\TODO{};
\item {\EMP 結合律}:繼承自 \m{S},也滿足;
\item {\EMP 逆元}:\TODO{}。
\stopigBase
\stopANSWER

%e31.3-4
\startEXERCISE
證明:如果 p 是素數且 e 是正整數,則:
\startformula
\phi(p^e) = p^{e-1}(p-1)
\stopformula
\stopEXERCISE

\startANSWER
\startformula
\phi(p^e) = p^e (1-\frac{1}{p}) = p^{e-1}(p-1)
\stopformula
\stopANSWER

%e31.3-5
\startEXERCISE
證明:對任意 \m{n>1} 和任意 \m{a\in\integers_n^\ast},
由 \m{f_a(x)=ax \mod n} 所定義的函數 \m{f_a: \integers_n^\ast \rightarrow \integers_n^\ast} 是 \m{\integers_n^\ast} 的一個置換。
\stopEXERCISE

\startANSWER
對於任一 a 和 n,每一個 x 都有一個唯一的 y 與其對應,也就是 \m{f(x)} 是一個雙射函數。
\stopANSWER

\stopsection

\startsection[
  title={Solving modular linear equations},
]

%e31.4-1
\startEXERCISE
找出方程 \m{35x\equiv 10 (\mod 50)} 的所有解。
\stopEXERCISE

\startANSWER
\startformula
d = 5 \qquad x' = 3 \qquad y' = -2 \qquad x_0 = 6
\stopformula
解爲 6、 16、 26、 36、 46。
\stopANSWER

%e31.4-2
\startEXERCISE
證明:只要 \m{\gcd(a,n)=1},方程 \m{ax\equiv ay (\mod n)} 就意味着 \m{x\equiv y (\mod n)}。
通過一個 \m{\gcd(a,n)>1} 情況下的反例,
證明條件 \m{\gcd(a,n)=1} 是必要的。
\stopEXERCISE

\startANSWER
由於 \m{ax \equiv ay (\mod n)},所以 \m{n|a(x-y)}。

而由於 \m{\gcd(a,n)=1},而根據推論 31.5 得 \m{n|(x-y)},所以 \m{x\equiv y (\mod n)}。

而如果 \m{\gcd(a,n)>1},令 \m{a=12, x=6, y=5, n=3},滿足 \m{ax\equiv ay (\mod n)},
但是不滿足 \m{x\equiv y (\mod n)}。
\stopANSWER

%e31.4-3
\startEXERCISE
考察下列過程 \ALGO{MODULAR-LINEAR-EQUIATION-SOLVER} 的第三行的修改:

\CLRSH{MODULAR-LINEAR-EQUATION-SOLVER(a, b, n)}
\startCLRS
(d, x', y') = EXTENDED-EUCLID(a, n)
if d | b
	x0 = x'(b/d) mod (n/d)
	for i = 0 to d - 1
		print (x0 + i(n/d)) mod n
else print "no solutions"
\stopCLRS
能否正確進行?並解釋原因。
\stopEXERCISE

\startANSWER
能,因爲 \m{x'(b/d) \mod (n/d)} 和 \m{x'(b/d) \mod n} 對 \m{n/d} 同餘。
\stopANSWER

%e31.4-4
\startEXERCISE[exercise:model_zero]\DIFFICULT
令 p 是一個素數,且 \m{f(x) \equiv f_0 + f_1 x + \ldots + f_t x^t (\mod p)} 是一個 t 次多項式,
係數 \m{f_i} 是從 \m{\integers_p} 得到的。
如果 \m{f(a) \equiv 0 (\mod p)},則將 \m{a\in \integers_p} 稱爲 f 的{\EMP 零元}。
證明:如果 a 是 f 的一個零元,則對某個 \m{t-1} 次的多項式 \m{g(x)},
有 \m{f(x)\equiv (x-a)g(x) (\mod p)}。
通過對 t 進行歸納來證明:如果 p 是素數, t 次多項式 \m{f(x)} 對模 p 至多有 t 個不同的零元。
\stopEXERCISE

\startANSWER
\startformula\startmathalignment[n=6]
\NC \NC     f(x)    \NC - (x-a)g(x) \NC \NC \NC \NR
\NC=\NC     f_0     \NC + f_1 x   \NC + f_2 x^2   \NC + \ldots + f_t x^t           \NC \NR
\NC \NC - (x-a)(g_0 \NC + g_1 x   \NC + g_2 x^2   \NC + \ldots + g_{t-1}x^{t-1})   \NC \NR
\NC=\NC     f_0     \NC + f_1 x   \NC + f_2 x^2   \NC + \ldots + f_{t-1}x^{t-1}    \NC + f_t x^t \NR
\NC \NC             \NC - g_0 x   \NC - g_1 x^2   \NC - \ldots - g_{t-2}x^{t-1}    \NC - g_{t-1} x^t \NR
\NC \NC     + g_0 a \NC + g_1 a x \NC + g_2 a x^2 \NC + \ldots + g_{t-1} a x^{t-1} \NC \NR
\NC=\NC (f_0 + g_0 a) \NC + (f_1 - g_0 + g_1 a)x \NC + (f_2 - g_1 + g_2 a)x^2 \NC + \ldots + (f_{t-1} - g_{t-2} + g_{t-1}a)x^{t-1} \NC + (f_t - g_{t-1})x^t \NR
\stopmathalignment\stopformula

按順序求解 \m{g_i}:
\startformula\startmathalignment[align={left,left}]
\NC g_0 \quad \NC (f_0 + g_0 a) \mod p = 0 \NR
\NC g_1 \NC (f_1 - g_0 + g_1 a) \mod p = 0 \NR
\NC \vdots \NC \vdots \NR
\NC g_{t-1} \NC (f_{t-1} - g_{t-2} + g_{t-1} a) \mod p = 0 \NR
\NC g_t \NC (f_t - g_{t-1}) \mod p = 0 \NR
\stopmathalignment\stopformula

由 \m{f(x)\equiv(x-a)g(x) (\mod p)} 類推:
\startformula
f(x) \equiv (x - a_0)(x - a_1) \ldots (x - a_t) (\mod p)
\stopformula
從而 \m{a_0,a_1,\ldots,a_n} 中的任何一個都是零元。
如果他們兩兩均不相等且爲整數,即最多有 n 個零元。
\stopANSWER

\stopsection

\startsection[
  title={The Chinese remainder theorem},
]

%31.5-1
\startEXERCISE
找出所有解,使得 \m{x\equiv 4 (\mod 5)} 和 \m{x\equiv 5 (\mod 11)} 同時成立。
\stopEXERCISE

\startANSWER
\startformula
x \equiv 49 (\mod 55)
\stopformula

\startxtable[
    option=max,
    align={middle,lohi},
    split=yes,
    header=repeat,
    footer=repeat,
    offset=.25em,
]

% head
\startxtablehead[frame=off,bottomframe=on]
\startxrow[foregroundstyle=bold,]
  \xcell[rightframe=on]{}\processcommalist[ 0, 1, 2, 3, 4, 5, 6, 7, 8, 9,10]\xcell
\stopxrow
\stopxtablehead

% body
\startxtablebody[frame=off]
\startxrow \xcell[rightframe=on]{0}\processcommalist[ 0,45,35,25,15, 5,50,40,30,20,10]\xcell \stopxrow
\startxrow \xcell[rightframe=on]{1}\processcommalist[11, 1,46,36,26,16, 6,51,41,31,21]\xcell \stopxrow
\startxrow \xcell[rightframe=on]{2}\processcommalist[22,12, 2,47,37,27,17, 7,52,42,32]\xcell \stopxrow
\startxrow \xcell[rightframe=on]{3}\processcommalist[33,23,13, 3,48,38,28,18, 8,53,43]\xcell \stopxrow
\startxrow \xcell[rightframe=on]{4}\processcommalist[44,34,24,14, 4,{\bfa 49},39,29,19, 9,54]\xcell \stopxrow
\stopxtablebody

\stopxtable

\stopANSWER

%31.5-2
\startEXERCISE
找出被 9、 8、 7 除時,餘數分別爲 1、 2、 3 的所有整數 x。
\stopEXERCISE

\startANSWER
由於 \m{\lcm(9,8,7) = 504},則:
\startformula
x \equiv 10 (\mod 504)
\stopformula
\stopANSWER

%e1.5-3
\startEXERCISE
論證:在定理 31.27 的定義下,如果 \m{\gcd(a,n)=1},則:
\startformula
(a^{-1} \mod n) \leftrightarrow
  ((a_1^{-1} \mod n_1),(a_2^{-1} \mod n_2),\ldots,(a_k^{-1} \mod n_k))
\stopformula
\stopEXERCISE

\startANSWER
\TODO{需要證明}
\stopANSWER

%31.5-4
\startEXERCISE
在定理 31.27 的定義下,證明:
對於任意的多項式 f,方程 \m{f(x)\equiv 0 (\mod n)} 的根的個數
等於 \m{f(x)\equiv 0 (\mod n_1), f(x)\equiv 0 (\mod n_2), \ldots, f(x) \equiv 0 (\mod n_k)} 中每個方程根的個數的積。
\stopEXERCISE

\startANSWER
\TODO{需要證明}
\stopANSWER

\stopsection

\startsection[
  title={Powers of an element},
]
\stopsection

\startsection[
  title={The RSA public-key cryptosystem},
]
\stopsection

\startsection[
  title={Primality testing},
]
\stopsection

\startsection[
  title={Integer factorization},
]
\stopsection

\startsubject[
  title={Problems},
]

\stopsubject%Problems

\stopchapter
\stopcomponent
