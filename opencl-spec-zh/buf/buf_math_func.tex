% acos
\startbuffer[funcproto:acos]
gentype acos(gentype)
\stopbuffer
\startbuffer[funcdesc:acos]
反餘弦函數。
\stopbuffer

% acosh
\startbuffer[funcproto:acosh]
gentype acosh(gentype)
\stopbuffer
\startbuffer[funcdesc:acosh]
反雙曲餘弦函數。
\stopbuffer

% acospi
\startbuffer[funcproto:acospi]
gentype acospi(gentype x)
\stopbuffer
\startbuffer[funcdesc:acospi]
計算 \math{acos(\marg{x}) / \pi}。
\stopbuffer

%% asin-related
% asin
\startbuffer[funcproto:asin]
gentype asin(gentype)
\stopbuffer
\startbuffer[funcdesc:asin]
反正弦函數。
\stopbuffer

% asinh
\startbuffer[funcproto:asinh]
gentype asinh(gentype)
\stopbuffer
\startbuffer[funcdesc:asinh]
反雙曲正弦函數。
\stopbuffer

% asinpi
\startbuffer[funcproto:asinpi]
gentype asinpi(gentype x)
\stopbuffer
\startbuffer[funcdesc:asinpi]
計算 \math{asin(\marg{x}) / \pi}。
\stopbuffer

%% atan-related
% atan
\startbuffer[funcproto:atan]
gentype atan (gentype y_over_x)
\stopbuffer
\startbuffer[funcdesc:atan]
反正切函數。
\stopbuffer

% atan2
\startbuffer[funcproto:atan2]
gentype atan2 (gentype y,
		gentype x)
\stopbuffer
\startbuffer[funcdesc:atan2]
\math{\marg{y} / \marg{x}} 的反正切。
\stopbuffer

% atanh
\startbuffer[funcproto:atanh]
gentype atanh (gentype)
\stopbuffer
\startbuffer[funcdesc:atanh]
反雙曲正切函數。
\stopbuffer

% atanpi
\startbuffer[funcproto:atanpi]
gentype atanpi (gentype x)
\stopbuffer
\startbuffer[funcdesc:atanpi]
計算 \math{atan(\marg{x}) / \pi}。
\stopbuffer

% atan2pi
\startbuffer[funcproto:atan2pi]
gentype atan2pi (gentype y,
		gentype x)
\stopbuffer
\startbuffer[funcdesc:atan2pi]
計算 \math{atan2(\marg{y},\marg{x}) / \pi}。
\stopbuffer

% cbrt
\startbuffer[funcproto:cbrt]
gentype cbrt(gentype)
\stopbuffer
\startbuffer[funcdesc:cbrt]
計算立方根。
\stopbuffer

% ceil
\startbuffer[funcproto:ceil]
gentype ceil(gentype)
\stopbuffer
\startbuffer[funcdesc:ceil]
向正無窮捨入成整數值。
\stopbuffer

% copysign
\startbuffer[funcproto:copysign]
gentype copysign(gentype x,
		 gentype y)
\stopbuffer
\startbuffer[funcdesc:copysign]
將 \carg{x} 的符號改成 \carg{y} 的,並將其返回。
\stopbuffer

% cos
\startbuffer[funcproto:cos]
gentype cos(gentype)
\stopbuffer
\startbuffer[funcdesc:cos]
計算餘弦。
\stopbuffer

% cosh
\startbuffer[funcproto:cosh]
gentype cosh(gentype)
\stopbuffer
\startbuffer[funcdesc:cosh]
計算雙曲餘弦。
\stopbuffer

% cospi
\startbuffer[funcproto:cospi]
gentype cospi(gentype x)
\stopbuffer
\startbuffer[funcdesc:cospi]
計算 \math{cos(\pi \marg{x})}。
\stopbuffer

% erfc
\startbuffer[funcproto:erfc]
gentype erfc(gentype x)
\stopbuffer
\startbuffer[funcdesc:erfc]
餘補誤差函數
 \math{1 - \mapiemp{erf}(\marg{x}) = \frac{2}{\sqrt{\pi}} \intop^{\infty}_{\marg{x}}e^{-\theta^2}d\theta}。
\stopbuffer

% erf
\startbuffer[funcproto:erf]
gentype erf(gentype x)
\stopbuffer
\startbuffer[funcdesc:erf]
誤差函數,表示正態分布的積分
 \math{\frac{2}{\sqrt{\pi}} \intop^{\infty}_{\marg{x}}e^{-\theta^2}d\theta}。
\stopbuffer

% exp
\startbuffer[funcproto:exp]
gentype exp(gentype x)
\stopbuffer
\startbuffer[funcdesc:exp]
計算 \math{e} 的 \carg{x} 次冪 \math{e^{\marg{x}}}。
\stopbuffer

% exp
\startbuffer[funcproto:exp]
gentype exp(gentype x)
\stopbuffer
\startbuffer[funcdesc:exp]
計算 \math{e} 的 \carg{x} 次冪 \math{e^{\marg{x}}}。
\stopbuffer

% exp2
\startbuffer[funcproto:exp2]
gentype exp2(gentype)
\stopbuffer
\startbuffer[funcdesc:exp2]
底數為 \math{2} 的冪。
\stopbuffer

% exp10
\startbuffer[funcproto:exp10]
gentype exp10(gentype)
\stopbuffer
\startbuffer[funcdesc:exp10]
底數為 \math{10} 的冪。
\stopbuffer

% expm1
\startbuffer[funcproto:expm1]
gentype expm1(gentype x)
\stopbuffer
\startbuffer[funcdesc:expm1]
計算 \math{e^{\marg{x}}-1.0}。
\stopbuffer

% fabs
\startbuffer[funcproto:fabs]
gentype fabs(gentype)
\stopbuffer
\startbuffer[funcdesc:fabs]
計算浮點數的絕對值。
\stopbuffer

% fdim
\startbuffer[funcproto:fdim]
gentype fdim(gentype x,
	     gentype y)
\stopbuffer
\startbuffer[funcdesc:fdim]
如果 \math{\marg{x} < \marg{y}},返回 \math{\marg{x} - \marg{y}};否則返回 \math{+0}。
\stopbuffer

% floor
\startbuffer[funcproto:floor]
gentype floor(gentype)
\stopbuffer
\startbuffer[funcdesc:floor]
向負無窮捨入成整數值。
\stopbuffer

% fma
\startbuffer[funcproto:fma]
gentype fma(gentype a,
	    gentype b,
	    gentype c)
\stopbuffer
\startbuffer[funcdesc:fma]
返回 \math{\marg{a} \times \marg{b} + \marg{c}},其中乘法具有無限精度即不會進行捨入,但會對加法進行正確地捨入。
邊界條件下的行為遵循 IEEE 754-2008 標準。
\stopbuffer

% fmax
\startbuffer[funcproto:fmax]
gentype fmax (gentype x, gentype y)
gentypef fmax (gentypef x, float y)
gentyped fmax (gentyped x, double y)
\stopbuffer
\startbuffer[funcdesc:fmax]
如果 \math{\marg{x} < \marg{y}},則返回 \carg{y},否則返回 \carg{x}。
如果一個引數是 NaN,則返回另一個引數;
如果兩個引數都是 NaN,則返回 NaN。
\stopbuffer

% fmaxh
\startbuffer[funcproto:fmaxh]
gentype fmax (gentype x, gentype y)
gentype fmax (gentype x, half y)
\stopbuffer
\startbuffer[funcdesc:fmaxh]
如果 \math{\marg{x} < \marg{y}},則返回 \carg{y},否則返回 \carg{x}。
如果一個引數是 NaN,則返回另一個引數;
如果兩個引數都是 NaN,則返回 NaN。
\stopbuffer

% fmin
\startbuffer[funcproto:fmin]
gentype fmin (gentype x, gentype y)
gentypef fmin (gentypef x, float y)
gentyped fmin (gentyped x, double y)
\stopbuffer
\startbuffer[funcdesc:fmin]
如果 \math{\marg{y} < \marg{x}},則返回 \carg{y},否則返回 \carg{x}。
如果一個引數是 NaN,則返回另一個引數;
如果兩個引數都是 NaN,則返回 NaN\footnote{
在處理 sNaN (signaling NaN)時, \capi{fmin} 和 \capi{fmax} 的行為遵守 C99 中的定義,
但可能與 IEEE 754-2008 中定義的 \capi{minNum} 和 \capi{maxNum} 處理方式不同。
特別是,可能將 sNaN 當成 qNaN (quiet NaN)。}。
\stopbuffer

% fminh
\startbuffer[funcproto:fminh]
gentype fmin (gentype x, gentype y)
gentype fmin (gentype x, half y)
\stopbuffer
\startbuffer[funcdesc:fminh]
如果 \math{\marg{y} < \marg{x}},則返回 \carg{y},否則返回 \carg{x}。
如果一個引數是 NaN,則返回另一個引數;
如果兩個引數都是 NaN,則返回 NaN。
\stopbuffer

% fmod
\startbuffer[funcproto:fmod]
gentype fmod (gentyped x, gentype y)
\stopbuffer
\startbuffer[funcdesc:fmod]
模。返回 \math{\marg{x} - \marg{y} \times \mapiemp{trunc}(\marg{x} / \marg{y})}。
\stopbuffer

% fract
\startbuffer[funcproto:fract]
gentype fract (gentype x,
	__global gentype *iptr)
gentype fract (gentype x,
	__local gentype *iptr)
gentype fract (gentype x,
	__private gentype *iptr)
\stopbuffer
\startbuffer[funcdesc:fract]
返回 \math{\mapiemp{fmin} (\marg{x} - \mapiemp{floor} ( \marg{x} ), \mcmm{0x1.fffffep - 1f})}。
而 \math{\mapiemp{floor}(\marg{x})} 將在 \carg{iptr} 中返回\footnote{
此處的 \math{\mapiemp{min}()} 是為了避免 \capi{fract}(-small) 返回 1.0。
有了 \math{\mapiemp{min}()},這種情況就會返回小於 1.0 的最大正浮點數。}。
\stopbuffer

% frexp half
\startbuffer[funcproto:frexph]
halfn frexp (halfn x,
	__global intn *exp)
halfn frexp (halfn x,
	__local intn *exp)
halfn frexp (halfn x,
	__private intn *exp)
half frexp (half x,
	__global int *exp)
half frexp (half x,
	__local int *exp)
half frexp (half x,
	__private int *exp)
\stopbuffer
\startbuffer[funcdesc:frexph]
從 \carg{x} 中分離出尾數和指數。
返回的尾數(記為 m)型別為 \ctype{half},值為 0 或者屬於 \math{[1/2, 1)}。
\carg{x} 的每個組件都等於 \math{m \times 2^{\marg{exp}}}。
\stopbuffer

% frexp float
\startbuffer[funcproto:frexpf]
floatn frexp (floatn x,
	__global intn *exp)
floatn frexp (floatn x,
	__local intn *exp)
floatn frexp (floatn x,
	__private intn *exp)
float frexp (float x,
	__global int *exp)
float frexp (float x,
	__local int *exp)
float frexp (float x,
	__private int *exp)
\stopbuffer
\startbuffer[funcdesc:frexpf]
從 \carg{x} 中分離出尾數和指數。
返回的尾數(記為 m)型別為 \ctype{float},值為 0 或者屬於 \math{[1/2, 1)}。
\carg{x} 的每個組件都等於 \math{m \times 2^{\marg{exp}}}。
\stopbuffer

% frexp double
\startbuffer[funcproto:frexpd]
doublen frexp (doublen x,
	__global intn *exp)
doublen frexp (doublen x,
	__local intn *exp)
doublen frexp (doublen x,
	__private intn *exp)
double frexp (double x,
	__global int *exp)
double frexp (double x,
	__local int *exp)
double frexp (double x,
	__private int *exp)
\stopbuffer
\startbuffer[funcdesc:frexpd]
從 \carg{x} 中分離出尾數和指數。
返回的尾數(記為 m)型別為 \ctype{double},值為 0 或者屬於 \math{[1/2, 1)}。
\carg{x} 的每個組件都等於 \math{m \times 2^{\marg{exp}}}。
\stopbuffer

% hypot
\startbuffer[funcproto:hypot]
gentype hypot (gentype x, gentype y)
\stopbuffer
\startbuffer[funcdesc:hypot]
計算 \math{\sqrt{\marg{x}^2 + \marg{y}^2}},不會有過分的上溢或下溢。
\stopbuffer

% ilogb
\startbuffer[funcproto:ilogb]
intn ilogb (floatn x)
int ilogb (float x)
intn ilogb (doublen x)
int ilogb (double x)
\stopbuffer
\startbuffer[funcdesc:ilogb]
返回整形對數。
\stopbuffer

% ilogbh
\startbuffer[funcproto:ilogbh]
intn ilogb (halfn x)
int ilogb (half x)
\stopbuffer
\startbuffer[funcdesc:ilogbh]
返回整形對數。
\stopbuffer

% ldexp
\startbuffer[funcproto:ldexp]
floatn ldexp (floatn x, intn k)
floatn ldexp (floatn x, int k)
float ldexp (float x, int k)
doublen ldexp (doublen x, intn k)
doublen ldexp (doublen x, int k)
double ldexp (double x, int k)
\stopbuffer
\startbuffer[funcdesc:ldexp]
返回 \math{\marg{x} \times 2^{\marg{k}}}。
\stopbuffer

% ldexph
\startbuffer[funcproto:ldexph]
halfn ldexp (halfn x, intn k)
halfn ldexp (halfn x, int k)
half ldexp (half x, int k)
\stopbuffer
\startbuffer[funcdesc:ldexph]
返回 \math{\marg{x} \times 2^{\marg{k}}}。
\stopbuffer

% lgamma
\startbuffer[funcproto:lgamma]
gentype lgamma (gentype x)
floatn lgamma_r (floatn x,
	__global intn *signp)
floatn lgamma_r (floatn x,
	__local intn *signp)
floatn lgamma_r (floatn x,
	__private intn *signp)
float lgamma_r (float x,
	__global int *signp)
float lgamma_r (float x,
	__local int *signp)
float lgamma_r (float x,
	__private int *signp)
doublen lgamma_r (doublen x,
	__global intn *signp)
doublen lgamma_r (doublen x,
	__local intn *signp)
doublen lgamma_r (doublen x,
	__private intn *signp)
double lgamma_r (double x,
	__global int *signp)
double lgamma_r (double x,
	__local int *signp)
double lgamma_r (double x,
	__private int *signp)
\stopbuffer
\startbuffer[funcdesc:lgamma]
返回伽馬函數絕對值的自然對數。
\capi{lgamma_r} 的引數 \carg{signp} 中會返回伽馬函數的符號。
 \math{ln|\Gamma(\marg{x})|}
\stopbuffer

% lgammah
\startbuffer[funcproto:lgammah]
gentype lgamma (gentype x)
halfn lgamma_r (halfn x,
	__global intn *signp)
halfn lgamma_r (halfn x,
	__local intn *signp)
halfn lgamma_r (halfn x,
	__private intn *signp)
half lgamma_r (half x,
	__global int *signp)
half lgamma_r (half x,
	__local int *signp)
half lgamma_r (half x,
	__private int *signp)
\stopbuffer
\startbuffer[funcdesc:lgammah]
返回伽馬函數絕對值的自然對數。
\capi{lgamma_r} 的引數 \carg{signp} 中會返回伽馬函數的符號。
 \math{ln|\Gamma(\marg{x})|}
\stopbuffer

% log
\startbuffer[funcproto:log]
gentype log (gentype)
\stopbuffer
\startbuffer[funcdesc:log]
計算自然對數。
\stopbuffer

% log2
\startbuffer[funcproto:log2]
gentype log2 (gentype)
\stopbuffer
\startbuffer[funcdesc:log2]
計算以 2 為底的對數。
\stopbuffer

% log10
\startbuffer[funcproto:log10]
gentype log10 (gentype)
\stopbuffer
\startbuffer[funcdesc:log10]
計算以 10 為底的對數。
\stopbuffer

% log1p
\startbuffer[funcproto:log1p]
gentype log1p (gentype x)
\stopbuffer
\startbuffer[funcdesc:log1p]
計算 \math{log_e(1.0 + \marg{x})}。
\stopbuffer

% logb
\startbuffer[funcproto:logb]
gentype logb (gentype x)
\stopbuffer
\startbuffer[funcdesc:logb]
\math{log_r|\marg{x}|}的整數部分。
\stopbuffer

% mad
\startbuffer[funcproto:mad]
gentype mad (gentype a,
	gentype b,
	gentype c)
\stopbuffer
\startbuffer[funcdesc:mad]
\capi{mad} 逼近 \math{\marg{a} \times \marg{b} + \marg{c}}。
至於 \math{\marg{a} \times \marg{b}} 是否要捨入、怎樣捨入,
以及如何處理超常(supernormal)或次常(subnormal)的乘法都沒有定義。
在那些速度比準確更重要的地方,就可以使用 \capi{mad}\footnote{
需要提醒用戶的是,對於一些情況,
如 \math{\mapiemp{mad}(\marg{a}, \marg{b}, -\marg{a}\times \marg{b})},
\capi{mad} 定義的非常寬鬆,
以至於當 \carg{a} 和 \carg{b} 為某些特定值時,返回任何值都有可能}。
\stopbuffer

% maxmag
\startbuffer[funcproto:maxmag]
gentype maxmag (gentype x, gentype y)
\stopbuffer
\startbuffer[funcdesc:maxmag]
如果 \math{|\marg{x}| > |\marg{y}|},則返回 \carg{x};
如果 \math{|\marg{y}| > |\marg{x}|},則返回 \carg{y};
否則返回 \math{\capi{fmax}(x,y)}。
\stopbuffer

% minmag
\startbuffer[funcproto:minmag]
gentype minmag (gentype x, gentype y)
\stopbuffer
\startbuffer[funcdesc:minmag]
如果 \math{|\marg{x}| < |\marg{y}|},則返回 \carg{x};
如果 \math{|\marg{y}| < |\marg{x}|},則返回 \carg{y};
否則返回 \math{\capi{fmin}(x,y)}。
\stopbuffer

% modf
\startbuffer[funcproto:modf]
gentype modf (gentype x,
	__global gentype *iptr)
gentype modf (gentype x,
	__local gentype *iptr)
gentype modf (gentype x,
	__private gentype *iptr)
\stopbuffer
\startbuffer[funcdesc:modf]
分解浮點數,將引數 \carg{x} 分解為整數部分和小數部分,
兩部分的符號都與 \carg{x} 相同。
整數部分存儲在 \carg{iptr} 所指對象中。
\stopbuffer

% nan
\startbuffer[funcproto:nan]
floatn nan (uintn nancode)
float nan (uint nancode)
doublen nan (ulongn nancode)
double nan (ulong nancode)
\stopbuffer
\startbuffer[funcdesc:nan]
返回 quiet NaN。
其中將 \carg{nancode} 放在尾數的位置。
\stopbuffer

% nanh
\startbuffer[funcproto:nanh]
halfn nan (ushortn nancode)
half nan (ushort nancode)
\stopbuffer
\startbuffer[funcdesc:nanh]
返回 quiet NaN。
其中將 \carg{nancode} 放在尾數的位置。
\stopbuffer

% nextafter
\startbuffer[funcproto:nextafter]
gentype nextafter (gentype x,
		gentype y)
\stopbuffer
\startbuffer[funcdesc:nextafter]
返回 \carg{x} 在往 \carg{y} 的方向上下一個可表示的單精度浮點值。
因此,如果 \carg{y} 小於 \carg{x},
則返回小於 \carg{x} 的最大的可表示的浮點數。
\stopbuffer

% pow
\startbuffer[funcproto:pow]
gentype pow (gentype x, gentype y)
\stopbuffer
\startbuffer[funcdesc:pow]
計算 \math{\marg{x}^{\marg{y}}}。
\stopbuffer

% pown
\startbuffer[funcproto:pown]
floatn pown (floatn x, intn y)
float pown (float x, int y)
doublen pown (doublen x, intn y)
double pown (double x, int y)
\stopbuffer
\startbuffer[funcdesc:pown]
計算 \math{\marg{x}^{\marg{y}}},其中 \carg{y} 是整數。
\stopbuffer

% pownh
\startbuffer[funcproto:pownh]
halfn pown (halfn x, intn y)
half pown (half x, int y)
\stopbuffer
\startbuffer[funcdesc:pownh]
計算 \math{\marg{x}^{\marg{y}}},其中 \carg{y} 是整數。
\stopbuffer

% powr
\startbuffer[funcproto:powr]
gentype powr (gentype x,
		gentype y)
\stopbuffer
\startbuffer[funcdesc:powr]
計算 \math{\marg{x}^{\marg{y}}},其中 \math{\marg{x}\geq 0}。
\stopbuffer

% remainder
\startbuffer[funcproto:remainder]
gentype remainder (gentype x,
		gentype y)
\stopbuffer
\startbuffer[funcdesc:remainder]
返回值記為 \math{r},則 \math{r = \marg{x} - n \times \marg{y}}。
其中 \math{n} 是最接近精確值 \math{\marg{x} / \marg{y}} 的整數。
如果有兩個這樣的整數,則選擇偶數作為 \math{n}。
如果 \math{r} 是零,則符號與 \carg{x} 一樣。
\stopbuffer

% remquoh
\startbuffer[funcproto:remquoh]
halfn remquo (halfn x,
	halfn y,
	__global intn *quo)
halfn remquo (halfn x,
	halfn y,
	__local intn *quo)
halfn remquo (halfn x,
	halfn y,
	__private intn *quo)
half remquo (half x,
	half y,
	__global int *quo)
half remquo (half x,
	half y,
	__local int *quo)
half remquo (half x,
	half y,
	__private int *quo)
\stopbuffer
\startbuffer[funcdesc:remquoh]
返回值記為 \math{r},則 \math{r = \marg{x} - k \times \marg{y}}。
其中 \math{k} 是最接近精確值 \math{\marg{x} / \marg{y}} 的整數。
如果有兩個這樣的整數,則選擇偶數作為 \math{k}。
如果 \math{r} 是零,則符號與 \carg{x} 一樣。
\math{r} 跟 \capi{remainder} 返回的值一樣。
區別就是 \capi{remquo} 還會計算整數商 \math{\marg{x} / \marg{y}} 的最低七位,
連帶 \math{\marg{x} / \marg{y}} 的符號一同存儲在 \carg{quo} 所指對象中。
\stopbuffer

% remquof
\startbuffer[funcproto:remquof]
floatn remquo (floatn x,
	floatn y,
	__global intn *quo)
floatn remquo (floatn x,
	floatn y,
	__local intn *quo)
floatn remquo (floatn x,
	floatn y,
	__private intn *quo)
float remquo (float x,
	float y,
	__global int *quo)
float remquo (float x,
	float y,
	__local int *quo)
float remquo (float x,
	float y,
	__private int *quo)
\stopbuffer
\startbuffer[funcdesc:remquof]
返回值記為 \math{r},則 \math{r = \marg{x} - k \times \marg{y}}。
其中 \math{k} 是最接近精確值 \math{\marg{x} / \marg{y}} 的整數。
如果有兩個這樣的整數,則選擇偶數作為 \math{k}。
如果 \math{r} 是零,則符號與 \carg{x} 一樣。
\math{r} 跟 \capi{remainder} 返回的值一樣。
區別就是 \capi{remquo} 還會計算整數商 \math{\marg{x} / \marg{y}} 的最低七位,
連帶 \math{\marg{x} / \marg{y}} 的符號一同存儲在 \carg{quo} 所指對象中。
\stopbuffer

% remquod
\startbuffer[funcproto:remquod]
doublen remquo (doublen x,
	doublen y,
	__global intn *quo)
doublen remquo (doublen x,
	doublen y,
	__local intn *quo)
doublen remquo (doublen x,
	doublen y,
	__private intn *quo)
double remquo (double x,
	double y,
	__global int *quo)
double remquo (double x,
	double y,
	__local int *quo)
double remquo (double x,
	double y,
	__private int *quo)
\stopbuffer
\startbuffer[funcdesc:remquod]
返回值記為 \math{r},則 \math{r = \marg{x} - k \times \marg{y}}。
其中 \math{k} 是最接近精確值 \math{\marg{x} / \marg{y}} 的整數。
如果有兩個這樣的整數,則選擇偶數作為 \math{k}。
如果 \math{r} 是零,則符號與 \carg{x} 一樣。
\math{r} 跟 \capi{remainder} 返回的值一樣。
區別就是 \capi{remquo} 還會計算整數商 \math{\marg{x} / \marg{y}} 的最低七位,
連帶 \math{\marg{x} / \marg{y}} 的符號一同存儲在 \carg{quo} 所指對象中。
\stopbuffer

% rint
\startbuffer[funcproto:rint]
gentype rint (gentype)
\stopbuffer
\startbuffer[funcdesc:rint]
捨入為最近偶數,雖然值為整數,但用的是浮點格式。
關於捨入模式請參考\insection[roundingMode]。
\stopbuffer

% rootn
\startbuffer[funcproto:rootn]
floatn rootn (floatn x, intn y)
float rootn (float x, int y)

doublen rootn (doublen x, intn y)
doublen rootn (double x, int y)
\stopbuffer
\startbuffer[funcdesc:rootn]
計算 \math{\marg{x}^{1/\marg{y}}}。
\stopbuffer

% rootnh
\startbuffer[funcproto:rootnh]
halfn rootn (halfn x, intn y)
half rootn (half x, int y)
\stopbuffer
\startbuffer[funcdesc:rootnh]
計算 \math{\marg{x}^{1/\marg{y}}}。
\stopbuffer

% round
\startbuffer[funcproto:round]
gentype round (gentype x)
\stopbuffer
\startbuffer[funcdesc:round]
返回整數值,從零往外捨入,無視當前捨入方向。
\stopbuffer

% rsqrt
\startbuffer[funcproto:rsqrt]
gentype rsqrt (gentype)
\stopbuffer
\startbuffer[funcdesc:rsqrt]
計算 \math{ 1 / \sqrt{\marg{x}}}。
\stopbuffer

% sin
\startbuffer[funcproto:sin]
gentype sin (gentype)
\stopbuffer
\startbuffer[funcdesc:sin]
計算正弦。
\stopbuffer

% sincos
\startbuffer[funcproto:sincos]
gentype sincos (gentype x,
	__global gentype *cosval)
gentype sincos (gentype x,
	__local gentype *cosval)
gentype sincos (gentype x,
	__private gentype *cosval)
\stopbuffer
\startbuffer[funcdesc:sincos]
計算 \carg{x} 的正弦和餘弦。
返回的是正弦,餘弦放到 \carg{cosval} 中。
\stopbuffer

% sinh
\startbuffer[funcproto:sinh]
gentype sinh (gentype)
\stopbuffer
\startbuffer[funcdesc:sinh]
計算雙曲正弦。
\stopbuffer

% sinpi
\startbuffer[funcproto:sinpi]
gentype sinpi (gentype x)
\stopbuffer
\startbuffer[funcdesc:sinpi]
計算 \math{\mapiemp{sin}(\pi \marg{x})}。
\stopbuffer

% sqrt
\startbuffer[funcproto:sqrt]
gentype sqrt (gentype)
\stopbuffer
\startbuffer[funcdesc:sqrt]
計算平方根。
\stopbuffer

% tan
\startbuffer[funcproto:tan]
gentype tan (gentype)
\stopbuffer
\startbuffer[funcdesc:tan]
計算正切。
\stopbuffer

% tanh
\startbuffer[funcproto:tanh]
gentype tanh (gentype)
\stopbuffer
\startbuffer[funcdesc:tanh]
計算雙曲正切。
\stopbuffer

% tanpi
\startbuffer[funcproto:tanpi]
gentype tanpi (gentype)
\stopbuffer
\startbuffer[funcdesc:tanpi]
計算 \math{\mapiemp{tan}(\pi \marg{x})}。
\stopbuffer

% tgamma
\startbuffer[funcproto:tgamma]
gentype tgamma (gentype)
\stopbuffer
\startbuffer[funcdesc:tgamma]
計算 \math{\Gamma(x)}。
\stopbuffer

% trunc
\startbuffer[funcproto:trunc]
gentype trunc (gentype)
\stopbuffer
\startbuffer[funcdesc:trunc]
向零捨入成整數值。
\stopbuffer
