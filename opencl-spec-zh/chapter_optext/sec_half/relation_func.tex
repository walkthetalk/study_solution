\subsection[section:relationFunc]{關係函式}

對\reftab{svRelationalFunc}中所列內建關係函式作了擴充,
引數可以為 \cldt{half} 和 \cldt[n]{half},
其中 \ccmmsuffix{n} 可以是 2、 3、 4、 8、 16,
參見\reftab{relationalFuncHalf}。

關係算子和相等算子(<、 <=、 >、 >=、 !=、 ==)也可用於矢量型別 \cldt[n]{half},
所產生的結果為 \cldt[n]{short},參見\insection[operator]。

對於標量型別的引數,如果所指定的關係為 {\ftRef{false}},則下列函式(參見\reftab{svRelationalFunc})會返回 0,否則返回 1:
\startigBase[indentnext=no]
\item \capi{isequal}、 \capi{isnotequal}、
\item \capi{isgreater}、 \capi{isgreaterequal}、
\item \capi{isless}、 \capi{islessequal}、
\item \capi{islessgreater}、
\item \capi{isfinite}、 \capi{isinf}、
\item \capi{isnan}、 \capi{isnormal}、
\item \capi{isordered}、 \capi{isunordered} 和
\item \capi{signbit}。
\stopigBase
而對於矢量型別的引數,如果所指定的關係為 {\ftRef{false}},則返回 0,
否則返回 -1 (即所有位都是 1)。

如果任一引數為 NaN,則下列關係函式返回 0:
\startigBase[indentnext=no]
\item \capi{isequal}、
\item \capi{isgreater}、 \capi{isgreaterequal}、
\item \capi{isless}、 \capi{islessequal} 和
\item \capi{islessgreater}。
\stopigBase
如果引數為標量,則當任一引數為 NaN 時, \capi{isnotequal} 返回 1;
而如果引數為矢量,則當任一引數為 NaN 時, \capi{isnotequal} 返回 -1。

\placetable[here,split][tab:relationalFuncHalf]
{內建關係函式}
{\input{chapter_optext/tbl/relational_func_half.tex}}
