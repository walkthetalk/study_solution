\startsection[
  title={Proof of the master theorem},
]

%e4.6-1
\startEXERCISE
證明 $\sum_{j=0}^{\lfloor \log_b n\rfloor}(\log_b n - j)^k = \Omega(\log_b^{k_1}n)$。
\stopEXERCISE
\startANSWER
令 $f(n)=\sum_{j=0}^{\lfloor \log_b n\rfloor}(\log_b n - j)^k$,
則如果 $f(n)\ge c \log_b^{k+1}n$,那麼:
\startformula
f(bn) = \sum_{j=0}^{\lfloor \log_b n\rfloor + 1}(\log_b n -j + 1)^k
\stopformula
\stopANSWER

\startEXERCISE \DIFFICULT
對 \m{b} 時正整數而非任意實數的情況,給出公式(4.27)中 \m{n_j} 的表達式,要求既簡單又準確。
附公式(4.27):
\startformula
 n_j = \startmathcases
   \NC n \NC 若 \m{j = 0} \NR
   \NC \left\lceil n_{j-1}/b \right\rceil \NC 若 \m{j > 0} \NR
\stopmathcases
\stopformula
\stopEXERCISE
\startANSWER
若 \m{n} 爲正整數,則解爲:
\startformula
n_j = \left\lceil n / b^j \right\rceil
\stopformula

\simpleurl{http://math.stackexchange.com/questions/509862/closed-form-expression-for-n-j-defined-by-n-j-lceil-n-j-1-b-rceil} 上有如下證明:

若 \m{a} 和 \m{b} 均爲正整數, \m{x} 爲任意實數,則有:
\startformula
\left\lceil \frac{x}{a} \right\rceil - 1
< \frac{x}{a}
\le \left\lceil \frac{x}{a} \right\rceil
\stopformula

兩邊同除以 \m{b},則:
\startformula
\frac{1}{b}(\left\lceil \frac{x}{a} \right\rceil - 1)
< \frac{x}{ab}
\le \frac{1}{b} \left\lceil \frac{x}{a} \right\rceil
\stopformula

令 \m{P = \left\lceil \frac{x}{a} \right\rceil},記 \m{P = q \cdot b - r},
其中 \m{0 \le r < b}, \m{r}、 \m{b} 均爲整數。
則由 \m{P > (q - 1) \cdot b} 可知 \m{P - 1 \ge (q - 1) \cdot b},因此:
\startformula
q - 1 \le \frac{P-1}{b} < \frac{x}{ab} \le \frac{P}{b} \le q
\stopformula

由此可知:
\startformula
\left\lceil \frac{x}{ab} \right\rceil = \left\lceil \frac{1}{b} \left\lceil \frac{x}{a} \right\rceil \right\rceil
\stopformula

顯然,推廣後可以執行任意次除法,也可以將 \m{\left\lceil \right\rceil} 換乘 \m{\lfloor \rfloor},
但是不能混合使用。因此
\startformula
n_j = \left\lceil \frac{n_{j-1}}{b} \right\rceil
    = \left\lceil \frac{1}{b} \left\lceil \frac{n}{b^{j-1}} \right\rceil \right\rceil
    = \left\lceil \frac{n}{b^j} \right\rceil
\stopformula

如果 \m{b} 是任意正實數,則結論不成立,
因爲無法由 \m{P > (q-1)\cdot b} 得出 \m{P-1 \ge (q-1) \cdot b}。
\stopANSWER

\startEXERCISE[exercise:master_method_generalized] \DIFFICULT
證明:若 \m{f(n) = \Theta(n^{\log_b{a}}\lg^k{n})},其中 \m{k\ge 0},
則主遞迴式的解爲 \m{T(n) = \Theta(n^{\log_b{a}}\lg^{k+1}n )}。
簡單起見,假定 \m{n} 是 \m{b} 的冪。(\m{T(n) = aT(n/b)+f(n)})
\stopEXERCISE
\startANSWER
\startformula\startmathalignment[n=2]
\NC g(n) \NC= \sum_{j=0}^{\log_b{n}-1}a^jf(n/b^j) \NR
\NC f(n/b^j) \NC= \Theta\Big((n/b^j)^{\log_b{a}}\lg^k(n/b^j)\Big) \NR
\NC g(n) \NC= \Theta\Big(\sum_{j=0}^{\log_b{n}-1}a^j\big(\frac{n}{b^j}\big)^{\log_b{a}}\lg^k\big(\frac{n}{b^j}\big)\Big) = \Theta(A) \NR
\NC A \NC= \sum_{j=0}^{\log_b{n}-1}a^j\big(\frac{n}{b^j}\big)^{\log_b{a}}\lg^k\frac{n}{b^j} \NR
\NC   \NC= n^{\log_b{a}}\sum_{j=0}^{\log_b{n}-1}\Big(\frac{a}{b^{\log_b{a}}}\Big)^j\lg^k\frac{n}{b^j} \NR
\NC   \NC= n^{\log_b{a}}\sum_{j=0}^{\log_b{n}-1}\lg^k\frac{n}{b^j} \NR
\NC   \NC= n^{\log_b{a}}B \NR
\NC \lg^k\frac{n}{d} \NC= (\lg{n} - \lg{d})^k = \lg^k{n} + o(\lg^k{n}) \NR
\NC B \NC= \sum_{j=0}^{\log_b{n}-1}\lg^k\frac{n}{b^j} \NR
\NC   \NC= \sum_{j=0}^{\log_b{n}-1}\Big(\lg^k{n} - o(\lg^k{n})\Big) \NR
\NC   \NC= \log_b{n}\lg^k{n} + \log_b{n} \cdot o(\lg^k{n}) \NR
\NC   \NC= \Theta(\log_b{n}\lg^k{n}) \NR
\NC   \NC= \Theta(\lg^{k+1}{n}) \NR
\NC g(n) \NC= \Theta(A) = \Theta(n^{\log_b{a}}B) = \Theta(n^{\log_b{a}}\lg^{k+1}{n}) \NR
\stopmathalignment\stopformula
\stopANSWER

\startEXERCISE \DIFFICULT
證明:主定理中第三種情況被過分強調了,因爲對於某個常數 \m{c<1},
正則條件 \m{af(n/b)\le cf(n)} 成立本身就意味着存在常數 \m{\epsilon > 0},
使得 \m{f(n) = \Omega(n^{\log_b{a} + \epsilon})}。
\stopEXERCISE
\startANSWER
證明如下,參見\simpleurl{http://math.stackexchange.com/questions/510897/why-does-afn-b-cfn-for-c-1-imply-that-fn-omegan-log-ba-ep}:
\startformula\startmathalignment[n=1]
\NC a f(n/b) \le cf(n) \NR
\NC \alpha f(n/b) \le f(n) \quad \alpha = a/c \NR
\NC \alpha f(n) \le f(nb) \NR
\NC \alpha^i f(1) \le f(b^i) \NR
\NC n = b^i \Rightarrow i = \log_{b}n \Rightarrow f(n) \ge \alpha^{\log_b{n}}f(1) = n^{\log_{b}\alpha} \NR
\NC \alpha > a \Rightarrow \alpha = a + d \quad (c < 1, d > 0) \NR
\NC \Rightarrow f(n) = n^{\log_b{a} + log_b{d}} = n^{\log_b{a}+\epsilon} \quad (\epsilon = \log_{b}d) \NR
\stopmathalignment\stopformula
\stopANSWER

\stopsection
