\subsection{創建圖像對象}

\topclfunc{clCreateImage}

\startCLFUNC
cl_mem clCreateImage (cl_context context,
		cl_mem_flags flags,
		const cl_image_format *image_format,
		const cl_image_desc *image_desc,
		void *host_ptr,
		cl_int *errcode_ret)
\stopCLFUNC

此函式可用來創建 1D 圖像、 1D 圖像緩衝(image buffer)、 1D 圖像陣列(image array)、
2D 圖像、 2D 圖像陣列以及 3D \cnglo{imgobj}。

\carg{context} 即將要創建的\cnglo{imgobj}所處的 OpenCL \cnglo{context}。

\carg{flags} 是位欄,
用來指明如何分配以及怎樣使用將要創建的\cnglo{imgobj},參見\reftab{clmemflags}。

對於所有類型的\cnglo{imgobj},除了 \cenum{CL_MEM_OBJECT_IMAGE1D_BUFFER},
如果 \carg{flags} 的值為 0,則使用缺省值 \cenum{CL_MEM_READ_WRITE}。

對於類型為 \cenum{CL_MEM_OBJECT_IMAGE1D_BUFFER} 的圖像對象,
如果 \carg{flags} 中沒有設置 \cenum{CL_MEM_READ_WRITE}、 \cenum{CL_MEM_READ_ONLY}
或 \cenum{CL_MEM_WRITE_ONLY},則會從 \carg{buffer} 中繼承這些屬性。
而 \carg{flags} 中不能設置 \cenum{CL_MEM_USE_HOST_PTR}、
\cenum{CL_MEM_ALLOC_HOST_PTR} 和 \cenum{CL_MEM_COPY_HOST_PTR},
這些也會由 \carg{buffer} 繼承。
即使 \carg{buffer} 的內存存取限定符中有 \cenum{CL_MEM_COPY_HOST_PTR},
也並不意味着創建子\cnglo{bufobj}時會有額外的拷貝。
如果 \carg{flags} 中沒有設置 \cenum{CL_MEM_HOST_WRITE_ONLY}、
\cenum{CL_MEM_HOST_READ_ONLY} 或 \cenum{CL_MEM_HOST_NO_ACCESS},
則會從 \carg{buffer} 中繼承這些屬性。

\carg{image_format} 指明圖像的格式。參見\insection[imgFmtDsc]。

\carg{image_desc} 指明圖像的類型以及維數。參見\insection[imgDsc]。

\carg{host_ptr} 指向圖像數據(可能已由\cnglo{app}分配好)。
\reftab{hostPtrLimit}中列出了 \carg{host_ptr} 所指緩衝大小的一些限制。

\placetable[here][tab:hostPtrLimit]
{\carg{host_ptr} 的限制}
{\input{chapter_rt/tbl/tbl_imgtypesize.tex}}

對於 3D 圖像 或 2D 圖像陣列,
\carg{host_ptr} 所指圖像數據分別按相鄰的 2D 圖像平面或 2D 圖像線性序列進行存儲。
每個 2D 圖像都是相鄰掃描列(scanline)的線性序列。
每個掃描列都是相鄰圖像元素的線性序列。

對於 2D 圖像,
\carg{host_ptr} 所指圖像數據按相鄰掃描列的線性序列進行存儲。
每個掃描列都是相鄰圖像元素的線性序列。

對於 1D 圖像陣列, \carg{host_ptr} 所指圖像數據按相鄰 1D 圖像的線性序列進行存儲。
每個 1D 圖像或 1D 圖像緩衝就是一個掃描列,是相鄰圖像元素的線性序列。

\carg{errcode_ret} 會返回相應的錯誤碼。
如果 \carg{errcode_ret} 是 \cmacro{NULL},不會返回錯誤碼。

如果成功創建了\cnglo{imgobj}, \capi{clCreateImage} 會將其返回,
並將 \carg{errcode_ret} 置為 \cenum{CL_SUCCESS}。
否則,返回 \cmacro{NULL} 並將 \carg{errcode_ret} 置為下列錯誤碼之一:
\startigBase
\item \cenum{CL_INVALID_CONTEXT},如果 \carg{context} 無效。

\item \cenum{CL_INVALID_VALUE},如果 \carg{flags} 的值無效。

\item \cenum{CL_INVALID_IMAGE_FORMAT_DESCRIPTOR},
如果 \carg{image_format} 中的值無效,
或者 \carg{image_format} 是 \cmacro{NULL}。

\item \cenum{CL_INVALID_IMAGE_DESCRIPTOR},
如果 \carg{image_desc} 中的值無效,
或者 \carg{image_desc} 是 \cmacro{NULL}。

\item \cenum{CL_INVALID_IMAGE_SIZE},
如果 \carg{image_desc} 中圖像任一維度的值
超過了 \carg{context} 中任一\cnglo{device}在相應維度上的最大值
(參見\reftab{cldevquery})。

\item \cenum{CL_INVALID_HOST_PTR},
如果 \carg{image_desc} 中 \carg{host_ptr} 是 \cmacro{NULL},
但 \carg{flags} 中設置了 \cenum{CL_MEM_USE_HOST_PTR} 或 \cenum{CL_MEM_COPY_HOST_PTR};
或者 \carg{host_ptr} 不是 \cmacro{NULL},
但 \carg{flags} 中設置了 \cenum{CL_MEM_COPY_HOST_PTR} 或 \cenum{CL_MEM_USE_HOST_PTR}。

\item \cenum{CL_INVALID_VALUE},如果要創建的是 1D 圖像緩衝,
為其指定 \cenum{CL_MEM_WRITE_ONLY} 的同時
\carg{flags} 中設置了 \cenum{CL_MEM_READ_WRITE} 或 \cenum{CL_MEM_READ_ONLY};
或者為其指定 \cenum{CL_MEM_READ_ONLY} 的同時
\carg{flags} 中設置了 \cenum{CL_MEM_READ_WRITE} 或 \cenum{CL_MEM_WRITE_ONLY};
或者 \carg{flags} 中設置了 \cenum{CL_MEM_USE_HOST_PTR}、 \cenum{CL_MEM_ALLOC_HOST_PTR}、
或 \cenum{CL_MEM_COPY_HOST_PTR}。

\item \cenum{CL_INVALID_VALUE},如果要創建的是 1D 圖像緩衝,
為其指定 \cenum{CL_MEM_HOST_WRITE_ONLY} 的同時
\carg{flags} 中設置了 \cenum{CL_MEM_HOST_READ_ONLY};
或者為其指定 \cenum{CL_MEM_HOST_READ_ONLY} 的同時
\carg{flags} 中設置了 \cenum{CL_MEM_HOST_WRITE_ONLY};
或者為其指定 \cenum{CL_MEM_HOST_NO_ACCESS} 的同時
\carg{flags} 中設置了 \cenum{CL_MEM_HOST_READ_ONLY} 或 \cenum{CL_MEM_HOST_WRITE_ONLY}。

\item \cenum{CL_IMAGE_FORMAT_NOT_SUPPORTED},如果 \carg{image_format} 的值不受支持。

\item \cenum{CL_MEM_OBJECT_ALLOCATION_FAILURE},如果為\cnglo{imgobj}分配內存失敗。

\item \cenum{CL_INVALID_OPERATION},
如果 \carg{context} 中的所有\cnglo{device}都不支持圖像
(即\reftab{cldevquery}中的 \cenum{CL_DEVICE_IMAGE_SUPPORT} 是 \cenum{CL_FALSE} )。

\item \cenum{CL_OUT_OF_RESOURCES},如果\scdevfailres。

\item \cenum{CL_OUT_OF_HOST_MEMORY},如果\schostfailres。
\stopigBase

\input{chapter_rt/sec_imgobj/subsec_create/img_fmt_dsc.tex}
\input{chapter_rt/sec_imgobj/subsec_create/img_dsc.tex}

