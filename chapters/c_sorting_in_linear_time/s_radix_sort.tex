\startsection[
  title={Radix sort},
]

\startEXERCISE
參照圖 8-3 的方法,說明 \ALGO{RADIX-SORT} 在下列英文單詞上的操作過程:
 COW、 DOG、 SEA、 RUG、 ROW、 MOB、 BOX、 TAB、 BAR、 EAR、 TAR、 DIG、 BIG、 TEA、 NOW、 FOX。
\stopEXERCISE

\startANSWER
\leavevmode
\externalfigure[output/e8_2_2-1]
\externalfigure[output/e8_2_2-2]
\externalfigure[output/e8_2_2-3]
\externalfigure[output/e8_2_2-4]
\stopANSWER

\startEXERCISE
下面的排序算法中哪些是穩定的:插入排序、歸並排序、堆排序和快速排序?
給出一個能使任何配序算法都穩定的方法。
你所給出的方法帶來的額外時間和空間開銷是多少?
\stopEXERCISE

\startANSWER
插入排序、歸並排序是穩定的; 堆排序和快速排序是不穩定的。

我們可以用額外空間存儲每個元素的索引,索引和元素本身一起移動,
比較時,先比較元素本身,如果元素本身相同,則比較索引,
這樣相當於保證所有元素互異,這樣無論用哪種排序算法都能保證穩定。
所需的額外空間爲 \m{\Theta(n)}。
所需的額外時間爲當於算法本身時間復雜度具有同樣的數量級。
\stopANSWER

\startEXERCISE
利用歸納法來證明基數排序是正確的。
在你所給出的證明中,
在哪裏需要假設所用的底層配序算法是穩定的?
\stopEXERCISE

\startANSWER
不變式:

{\EMP 在 for 循環開始,數列已經對最低 \m{i-1} 位排好序。}

{\EMP 初始化:}數列對最低的 0 位已經排好序;

{\EMP 保持:}假設數列已經對最低 \m{i-1} 位排好序,
在對第 \m{i} 位排序後,數列就對最低 \m{i} 位都排好序了。
顯然第 \m{i} 位本身的排序是沒問題的,問題是當第 \m{i} 位相同時,
由於排序算法是穩定的且最低 \m{i-1} 位已經排好序,所以最低 \m{i} 位均是排好序的。

{\EMP 終止:} \m{i=d+1} 時循環終止。根據不變式,所有 \m{d} 位都是排好序的。

在{\EMP 保持}時需要假設算法是穩定的。
\stopANSWER

\startEXERCISE
說明如何在 \m{O(n)} 時間內,對 0 到 \m{n^3-1} 區間內的 \m{n} 個整數進行排序。
\stopEXERCISE

\startANSWER
將數列元素以 \m{n} 進制表示,則有 3 位。
根據{\EMP 引理 8.3},所需時間爲 \m{\Theta(d(n+k))=\Theta(3(n+n))=\Theta(6n)=\Theta(n)}。
\stopANSWER

\startEXERCISE\DIFFICULT
在本節給出的第一個卡片排序算法中,
爲排序 \m{d} 位十進制數,
在最壞情況下需要多少輪排序?
在最壞情況下,操作員需要記錄多少堆卡片?
\stopEXERCISE

\startANSWER
算法執行過程爲(以三進制爲例):
\startCLRS
- 按第一位分組
  - 將第一位爲 0 的分組
    - 將第二位爲 0 的分組
      。。。
    - 將第二位爲 1 的分組
      。。。
    - 將第二位爲 2 的分組
      。。。
  - 將第一位爲 1 的分組
    。。。
  - 將第一位爲 2 的分組
    。。。
\stopCLRS

可以看出算法爲指數級的。需要時間爲 \m{\Theta(k^d)}。
空間則爲 \m{\Theta(nk)}。
\stopANSWER

\stopsection
