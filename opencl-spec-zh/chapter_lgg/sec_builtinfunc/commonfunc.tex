\subsection[section:commonFunc]{公共函式}

\reftab{svCommonFunc}中列出了內建的公共函式。
這些函式都是按組件逐一運算的,其中的描述也是針對單個組件的。
泛型 \ctype{gentype} 表明函式的引數可以是:
\startigBase[indentnext=no]
\item \ctype{float}、
\item \ctype{float{2|3|4|8|16}}、
\item \ctype{double}、
\item \ctype{double{2|3|4|8|16}}。
\stopigBase
泛型 \ctype{gentypef} 表明函式的引數可以是:
\startigBase[indentnext=no]
\item \ctype{float}、
\item \ctype{float{2|3|4|8|16}}。
\stopigBase
泛型 \ctype{gentyped} 表明函式的引數可以是:
\startigBase[indentnext=no]
\item \ctype{double}、
\item \ctype{double{2|3|4|8|16}}。
\stopigBase

內建的公共函式實現時用的捨入模式是捨入為最近偶數。

\startnotepar
可以使用化簡(如 \capi{mad} 或 \capi{fma})來實現 \capi{mix} 和 \capi{smoothstep}。
\stopnotepar

\placetable[here][tab:svCommonFunc]
{引數既可為標量整數,也可為矢量整數的內建公共函式}
{\startCLFD

\clFD{clamp}
\clFD{degrees}
\clFD{max}
\clFD{min}
\clFD{mix}
\clFD{radians}
\clFD{step}
\clFD{smoothstep}
\clFD{sign}

\stopCLFD
}
