\startsection[
  title={The master method for solving recurrences},
]

%e4.5-1
\startEXERCISE
用主定理求下列遞迴式的漸進緊確界。
\startigBase[a]
\item $T(n) = 2T(n/4) + 1$
\item $T(n) = 2T(n/4) + \sqrt{n}$
\item $T(n) = 2T(n/4) + n$
\item $T(n) = 2T(n/4) + n^2$
\stopigBase
\stopEXERCISE
\startANSWER
\startigBase[a]
\item case 1: $b=4,a=2,f(n)=1,\epsilon=1>0$, $\Theta(n^{\log_4{2}}) = \Theta(\sqrt{n})$
\item case 2: $b=4,a=2,f(n)=\sqrt{n},k=0\ge 0$, $\Theta(n^{\log_4{2}}\lg{n}) = \Theta(\sqrt{n}\lg{n})$
\item case 3: $b=4,a=2,f(n)=n,\epsilon=2>0$, $\Theta(n)$
\item case 3: $b=4,a=2,f(n)=n^2,\epsilon=14>0$, $\Theta(n^2)$
\stopigBase
\stopANSWER

%e4.5-2
\startEXERCISE
Caesar 教授想設計一個算法用於矩陣乘法,
想讓其時間複雜度優於 Strassen 算法。
他的算法使用分治策略,
將每個矩陣分解爲 $n/4 \times n/4$ 個子矩陣,
分解和合並步驟花費時間爲 $\Theta(n^2)$。
他需要確定,這個算法需要創建多少個子問題,
才能擊敗 Strassen 算法。
如果他的算法創建 $a$ 個規模爲 $n/4$ 的子問題,
則描述運行時間 $T(n)$ 的遞迴式爲 $T(n) = a T(n/4) + \Theta(n^2)$。
這個算法要想漸進優於 Strassen 算法, $a$ 的最大整數值應爲多少?
\stopEXERCISE
\startANSWER
當 $a < 16$ 時適用於主定理的第一種情況。
此時,此算法 $T(n) = \Theta(n^{\log_4{a}})$。
要想更快,需 $\log_4{a} < \log_{2}7$,其中 $a < 49$。
因此, $a$ 的最大整數值爲 $15$。
\stopANSWER

%e4.5-3
\startEXERCISE
用主定理證明:二分查找遞迴式 $T(n) = T(n/2) + \Theta(1)$ 的解爲 $T(n) = \Theta(\lg n)$。
(關於二分查找參見\inexercise[bin_search])
\stopEXERCISE
\startANSWER
適用主定理第二種情況:
\startsplitformula\startmathalignment
\NC a =\NC 1\NR
\NC b =\NC 2\NR
\NC f(n) =\NC \Theta(1) = \Theta(n^{\log_b^a}\lg^{k}n)\qquad \text{其中 $k=0$} \NR
\NC T(n) =\NC \Theta(\lg n) \NR
\stopmathalignment\stopsplitformula
\stopANSWER

%e4.5-4
\startEXERCISE
現有函式 $f(n)=\lg n$。
試分析:
儘管 $f(n/2)<f(n)$,
任意常數 $c<1$ 都無法滿足正則條件 $af(n/b)\le cf(n)$,
其中 $a=1,b=2$。
更進一步,任意 $\epsilon>0$,
也無法滿足第三種情況中的條件 $f(n)=\Omega(n^{\log_b{a+\epsilon}})$。
\stopEXERCISE
\startANSWER
\startformula
af(n/b)=f(n/2)=\lg(n/2)=\lg n - 1 = f(n)-1\le cf(n)
\stopformula
如果 $c<1$,則 $(1-c)f(n)\le 1$,即 $f(n)\le \frac{1}{1-c}$,
而無論 $c$ 有多小,總能找到 $n_0$,當 $n>n_0$ 時,使得 $f(n)>\frac{1}{1-c}$。
因此 $af(n/b)\le cf(n)$ 無法成立。

\startsplitformula\startmathalignment
\NC f(n) \NC =\Omega(n^{\log_b{a+\epsilon}}) \NR
\NC \NC \ge c n^{\lg\epsilon} \NR
\stopmathalignment\stopsplitformula
令 $k=\log_b{a+\epsilon}>0$,即 $\lg n \ge c n^k$。
 $cn^k$ 增長速度要快於 $\lg n$。
\stopANSWER

%e4.5-5
\startEXERCISE
有函式 $f(n)=2^{\lceil \lg n\rceil}$,
選擇適當的常數 $a,b,\epsilon$,
可以使其滿足主定理第三種情況中所有條件(正則條件除外)。
\stopEXERCISE
\startANSWER
令 $2^{k-1}< n\le 2^{k}$,則 $f(n)=2^k$,
要想使得 $f(n)=\Omega(n^{\log_b{a+\epsilon}})$,
需要滿足 $n^{\log_b{a+\epsilon}} < n\le f(n)$,
即 $a+\epsilon < 0$,
可以找到 $a>0,b>1,\epsilon>0$ 滿足此式。
\stopANSWER

\stopsection
