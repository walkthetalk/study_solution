\startsubject[
  title={Problems},
]

%pD-1
\startPROBLEM
(Vandermonde matrix)
給定數值 \m{x_0,x_1,\ldots,x_{n-1}},
證明 Vandermonde 矩陣的行列式
\startformula
V(x_0,x_1,\ldots,x_n)=
\left[\startmatrix
\NC 1 \NC x_0 \NC x_0^2 \NC \cdots \NC x_0^{n-1} \NR
\NC 1 \NC x_1 \NC x_1^2 \NC \cdots \NC x_1^{n-1} \NR
\NC \vdots \NC \vdots \NC \vdots \NC \ddots \NC \vdots \NR
\NC 1 \NC x_{n-1} \NC x_{n-1}^2 \NC \cdots \NC x_{n-1}^{n-1} \NR
\stopmatrix\right]
\stopformula
是
\startformula
\det(V(x_0,x_1,\ldots,x_{n-1}) = \prod_{0\le j < k\le n-1} (x_k - x_j)
\stopformula
(\hint 對於 \m{i=n-1,n-2,\ldots,1},
將第 \m{i} 列乘以 \m{-x_0} 以後加到第 \m{i+1} 列上,
然後使用歸納法。)
\stopPROBLEM

\startANSWER
\TODO{略。}
\stopANSWER

%pD-2
\startPROBLEM
(Permutations defined by matrix-vector multiplication over GF(2))
利用 \m{GF(2)} 上的矩陣乘法可以定義一類集合 \m{S_n=\{0,1,2,\ldots,2^n-1]\}} 中整數的排列。
對於 \m{S_n} 中每個整數,可以將他的二進制表示形式看作一個 \m{n} 位向量
\startformula
\left[\startmatrix
\NC x_0 \NR
\NC x_1 \NR
\NC x_2 \NR
\NC \vdots \NR
\NC x_{n-1} \NR
\stopmatrix\right]
\stopformula
其中 \m{\sum_{i=0}^{n-1} x_i 2^i}。
如果 \m{A} 是一個元素均爲 0 或 1 的 \m{n\times n} 矩陣,
則我們可以定義一個排列。
該排列將 \m{S_n} 中的每一個值 \m{x} 映射到一個數上,
該數的二進制表示形式爲矩陣——向量積 \m{Ax}。
這裏,我們按照 \m{GF(2)} 執行所有算數運算:
所有的值爲 0 或 1,
並且除特例 \m{1+1=0} 外,
其他常規加法、乘法規則均適用。
讀者可以認爲 \m{GF(2)} 算數運算除了只使用最低有效位,
其他均與常規整數算術運算一致。

例如,對於 \m{S_2=\{0,1,2,3\}},矩陣
\startformula
A=\left[\startmatrix
\NC 1 \NC 0 \NR
\NC 1 \NC 1 \NR
\stopmatrix\right]
\stopformula
定義了如下排列 \m{\pi_A: \pi_A(0)=0,\pi_A(1)=3,\pi_A(2)=2,\pi_A(3)=1}。
下面解釋 \m{\pi_A(3)=1} 的理由,在 \m{GF(2)} 中
\startformula
\pi_A(3)=
\left[\startmatrix
\NC 1 \NC 0 \NR
\NC 1 \NC 1 \NR
\stopmatrix\right]
\left[\startmatrix
\NC 1 \NC 1 \NR
\stopmatrix\right]
\left[\startmatrix
\NC 1\cdot 1 + 0\cdot 1 \NR
\NC 1\cdot 1 + 1\cdot 1 \NR
\stopmatrix\right]
\left[\startmatrix
\NC 1 \NR
\NC 0 \NR
\stopmatrix\right]
\stopformula
就是 1 的二進制表示。

我們繼續在 \m{GF(2)} 上討論本問題,
並且所有矩陣和向量的元素均爲 0 或 1。
定義 0-1 矩陣(勻速均爲 0 或 1 的矩陣)在 \m{GF(2)} 上的秩與普通矩陣一致,
但是所有的決定線性相關的算術運算都按 \m{GF(2)} 進行。
定義 \m{n\times n} 0-1 矩陣 \m{A} 的取值範圍爲
\startformula
R(A) = \{y: y=Ax,x\in S_n\}
\stopformula
這樣, \m{R(A)} 是 \m{S_n} 中一類數的集合,
這類數可以由將 \m{S_n} 中每個值 \m{x} 乘以 \m{A} 得到。
\stopPROBLEM

\stopsubject%Problems
