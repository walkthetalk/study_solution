\startsection[
  title={Multiplying square matrices},
]

%4.1-1
\startEXERCISE
\ALGO{MATRIX-MULTIPLY-RECURSIVE} 用於計算兩個大小均爲 $n\times n$ 的矩陣的乘積,
將其進行推廣,使得 $n$ 不必是 $2$ 的冪。
用遞歸的方式給出其運行時間。
並分析最壞運行時間爲 $\Theta(n^3)$。
\stopEXERCISE

\startANSWER
可以將其補全,使得 $n$ 爲 $2$ 的冪。
最壞運行時間依舊是 $\Theta(n^3)$。
\stopANSWER

%4.1-2
\startEXERCISE
怎樣快速計算兩個大小分別爲 $k n\times n$ ($kn$ 行, $n$ 列)
和 $n\times kn$ 的矩陣的乘積?
其中 $k\ge 1$,
可以調用 \ALGO{MATRIX-MULTIPLY-RECURSIVE}。
如果兩個矩陣大小改爲 $n\times kn$ 和 $kn\times n$,
又該如何計算?
那一個漸進更快一些?快多少?
\stopEXERCISE

\startANSWER
將矩陣劃分爲大小均爲 $n\times n$ 的子陣。

前者爲:$T(kn,n) = (2k-1)T(n,n)$。

後者爲:$T(n,kn) = k T(n,n)$。

後者更快一些,運行時間約爲前者的一半。
\stopANSWER

%4.1-3
\startEXERCISE
在 \ALGO{MATRIX-MULTIPLY-RECURSIVE} 中,
設想一下,我們不是通過計算索引劃分矩陣,
而是將 $A,B,C$ 中相應元素複製到大小爲 $n/2\times n/2$ 的矩陣
$A_{11},A_{12},A_{21},A_{22}$、 $B_{11},B_{12},B_{21},B{22}$
以及 $C_{11},C_{12},C_{21},C_{22}$ 中。
待遞迴結束後,再將 $C_{11},C_{12},C_{21},C_{22}$ 中的結果
複製到 $C$ 中的相應位置。
則 (4.9)中的遞迴式要怎麼改?結果是什麼?
附遞迴式(4.9):
\startformula
T(n) = 8T(n/2) + \Theta(1)
\stopformula
\stopEXERCISE

\startANSWER
\startformula\startmathalignment
\NC T(n) \NC = 8T(n/2) + \Theta(n^2) \NR
\NC \NC = \Theta(n^3) \NR
\stopmathalignment\stopformula
\stopANSWER

%4.1-4
\startEXERCISE
要計算兩個矩陣 $A$ 和 $B$ 之和,
其大小均爲 $n\times n$,
可以將其都劃分爲四個 $n/2\times n/2$ 子陣,
然後遞歸對相對應的子陣求和,
試寫出矩陣求和分治算法 \ALGO{MATRIX-ADD-RECURSIVE} 的僞碼。
假設劃分子陣、計算索引所用時間爲 $\Theta(1)$。
寫出其最壞運行時間的遞迴式,並求解。
如果劃分時不是計算索引,
而是改爲用時 $\Theta(n^2)$ 的複製,
會發生什麼?
\stopEXERCISE

\startANSWER
\CLRSH{MATRIX-ADD-RECURSIVE(A,B,C,n)}
\startCLRSCODE
if n \le 1
	return A+B
\ALGO{MATRIX-ADD-RECURSIVE(A_{11},B_{11},C_{11},n/2)}
\ALGO{MATRIX-ADD-RECURSIVE(A_{12},B_{12},C_{12},n/2)}
\ALGO{MATRIX-ADD-RECURSIVE(A_{21},B_{21},C_{21},n/2)}
\ALGO{MATRIX-ADD-RECURSIVE(A_{22},B_{22},C_{22},n/2)}
\stopCLRSCODE

\startformula\startmathalignment
\NC T(n) \NC = 4T(n/2) + \Theta(1) \NR
\NC \NC = \Theta(n^2) \NR
\stopmathalignment\stopformula
\stopANSWER

\stopsection
