\section[section:getFuncPtr]{獲取 OpenCL API 擴展函式指位器}

\topclfunc{clGetExtensionFunctionAddressForPlatform}

\startCLFUNC
void* clGetExtensionFunctionAddressForPlatform (
			cl_platform_id platform,
			const char *funcname)
\stopCLFUNC

\startnotepar
由於沒有任何方式來限定對\cnglo{device}的查詢,
對於此擴展在這個\cnglo{platform}中不同\cnglo{device}上的所有實作而言,
所返回的函式指位器必須都能正常運作。
如果\cnglo{device}不支持此擴展,則調用擴展中的函式時,其行為\cnglo{undef}。
\stopnotepar

此函式會返回給定 \carg{platform} 的擴展函式 \carg{funcname} 的位址。
需要將所返回的指位器轉型成對應的函式指位器型別,
此擴展函式在對應的擴展規範和頭檔中定義。
如果返回的是 \cmacro{NULL},則表明實作中沒有所指定的函式,或者 \carg{platform} 無效。
即使返回的不是 \cmacro{NULL},也不保證 \carg{platform} 真正支持此擴展函式。
要想確定 OpenCL 實作是否支持某個擴展,
\cnglo{app}必須用下列兩種方式之一進行查詢:
\startclc
clGetPlatformInfo(platform, CL_PLATFORM_EXTENSIONS, ... )
clGetDeviceInfo(device, CL_DEVICE_EXTENSIONS, ... )
\stopclc

對於 OpenCL 的核心函式(非擴展函式),不能使用此函式進行查詢。
而對於那些能用此函式進行查詢的函式,
實作也可以選擇由實現那些函式的目標庫靜態導入那些函式。
然而,\cnglo{app}要想可移植,就不能依賴這種行為。

對於所有會增加 API 引入點的擴展,都必須聲明 \ccmm{typedef} 的函式指位器型別。
這些 \ccmm{typedef} 是擴展接口所必需的一部分,在相應頭檔中提供
(如果是 OpenCL 擴展,則為 \ccmm{cl_ext.h};
而如果是 OpenCL / OpenGL 共享擴展,則為 \ccmm{cl_gl_ext.h})。

所有會影響\cnglo{host} API 的擴展都必須遵循下列約定:
\startclc[indentnext=no]
#ifndef extension_name
#define extension_name		1

// all data typedefs, token #defines, prototypes, and
// function pointer typedefs for this extension

// function pointer typedefs must use the
// following naming convention
typedef CL_API_ENTRY /BTEX{\ftRef{return type}}/ETEX
		(CL_API_CALL */BTEX{\ftRef{clextension_func_nameTAG_fn}}/ETEX)(...);

#endif // extension_name
\stopclc
其中 \ccmm{TAG} 可以是 \ccmm{KHR}、 \ccmm{EXT} 或 \ccmm{vendor-specific}。

例如,擴展 \clext{cl_khr_gl_sharing} 在 \ccmm{cl_gl_ext.h} 中增加了下列代碼:
\startclc
#ifndef cl_khr_gl_sharing
#define cl_khr_gl_sharing	1

// all data typedefs, token #defines, prototypes, and
// function pointer typedefs for this extension
#define CL_INVALID_GL_SHAREGROUP_REFERENCE_KHR	-1000
#define CL_CURRENT_DEVICE_FOR_GL_CONTEXT_KHR	0x2006
#define CL_DEVICES_FOR_GL_CONTEXT_KHR		0x2007
#define CL_GL_CONTEXT_KHR			0x2008
#define CL_EGL_DISPLAY_KHR			0x2009
#define CL_GLX_DISPLAY_KHR			0x200A
#define CL_WGL_HDC_KHR				0x200B
#define CL_CGL_SHAREGROUP_KHR			0x200C

// function pointer typedefs must use the
// following naming convention
typedef CL_API_ENTRY cl_int
	(CL_API_CALL *clGetGLContextInfoKHR_fn)(
		const cl_context_properties * /* properties */,
		cl_gl_context_info /* param_name */,
		size_t /* param_value_size */,
		void * /* param_value */,
		size_t * /*param_value_size_ret*/);

#endif // cl_khr_gl_sharing
\stopclc
