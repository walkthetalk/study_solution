% query device
\section{查詢設備}
%%%%%%%%%%%%%%%%%%%%%%%%%%%%%%%%%%%%%%%%%%%%%clGetDeviceIDs
\topclfunc{clGetDeviceIDs}

\startCLFUNC
cl_int clGetDeviceIDs(
		cl_platform_id platform,
		cl_device_type device_type,
		cl_uint num_entries,
		cl_device_id *devices,
		cl_uint *num_devices)
\stopCLFUNC

函式 \capi{clGetDeviceIDs}\footnote{%
\capi{clGetDeviceIDs} 可能返回 \carg{platform} 中
所有與 \carg{device_type} 匹配的真正的物理設備,也可能只是其中一個子集。}
可用來獲取一個\cnglo{platform}上所有可用\cnglo{device}的清單。

\carg{platform} 即 \capi{clGetPlatformIDs} 所返回的\cnglo{platform} ID,
也可能是 \cmacro{NULL}。
如果是 \cmacro{NULL},則其行為\cnglo{impdef}。

\carg{device_type} 是位欄( bitfield ),用來識別 OpenCL \cnglo{device}的類型。
可以用來查詢某種 OpenCL \cnglo{device},也可以查詢所有的。
請參考\reftab{cldevctgr}。

\placetable[here][tab:cldevctgr]
{OpenCL 設備種類清單}
{\startED[\ctype{cl_device_type}]

\clED{CL_DEVICE_TYPE_CPU}{
  \cnglo{host}處理器。OpenCL 實作運行其上,是單核或多核 CPU。
}

\clED{CL_DEVICE_TYPE_GPU}{
  GPU。這意味着此\cnglo{device}也可以用來加速 3D API (如 OpenGL 或 DirectX)。
}

\clED{CL_DEVICE_TYPE_ACCELERATOR}{
  OpenCL 專用加速器(如 IBM CELL Blade)。
  這些設備通過外圍設備互聯總線(如 PCIe)與\cnglo{host}處理器通信。
}

\clED{CL_DEVICE_TYPE_CUSTOM}{
  一種專用加速器,但是不支持用 OpenCL C 編寫的\cnglo{program}。
}

\clED{CL_DEVICE_TYPE_DEFAULT}{
  系統中默認的 OpenCL \cnglo{device}。
  不能是 \cenum{CL_DEVICE_TYPE_CUSTOM} 類型的設備。
}

\clED{CL_DEVICE_TYPE_ALL}{
  系統中所有可用的 OpenCL \cnglo{device}。
  不包括 \cenum{CL_DEVICE_TYPE_CUSTOM} 類型的設備。
}

\stopED

}

\carg{num_entries} 是 \carg{devices}中所能容納 \ctype{cl_device} 表項的數目。
如果 \carg{devices} 不是 \cmacro{NULL},則 \carg{num_entries} 必須大於零。

\carg{devices} 用來返回所找到的 OpenCL \cnglo{device}。
\carg{devices} 中返回的 \ctype{cl_device_id} 用來識別一個 OpenCL \cnglo{device}。
如果參數 \carg{devices} 是 \cmacro{NULL},則忽略。
所返回的 OpenCL \cnglo{device}數目
為 \carg{num_entries} 或符合 \ctype{device_type} 的\cnglo{device}数目,
取二者中較小的那個。

\carg{num_devices} 返回符合 \ctype{device_type} 的所有 OpenCL \cnglo{device}的數目。
如果 \carg{num_devices} 是 \cmacro{NULL},則忽略。

如果執行成功,則 \capi{clGetDeviceIDs} 會返回 \cenum{CL_SUCCESS}。
否則,返回下列錯誤碼之一:
\startigBase
\item \cenum{CL_INVALID_PLATFORM},如果 \carg{platform} 無效。
\item \cenum{CL_INVALID_DEVICE_TYPE},如果 \carg{device_type} 無效。

\item \cenum{CL_INVALID_VALUE},
如果 \carg{num_entries} 等於零且 \carg{devices} 不是 \cmacro{NULL},
或者 \carg{num_devices} 和 \carg{devices} 都是 \cmacro{NULL}。

\item \cenum{CL_DEVICE_NOT_FOUND},
如果沒有找到任何符合 \carg{device_type} 的 OpenCL \cnglo{device}。

\item \cenum{CL_OUT_OF_RESOURCES},如果\scdevfailres。
\item \cenum{CL_OUT_OF_HOST_MEMORY},如果\schostfailres。
\stopigBase

對於 \capi{clGetDeviceIDs} 所返回的 OpenCL \cnglo{device},
\cnglo{app}可以查詢其能力,從而決定使用哪些\cnglo{device}。

\topclfunc{clGetDeviceInfo}

\startCLFUNC
cl_int clGetDeviceInfo(
		cl_device_id device,
		cl_device_info param_name,
		size_t param_value_size,
		void *param_value,
		size_t *param_value_size_ret)
\stopCLFUNC

函数 \capi{clGetDeviceInfo} 可用來獲取 OpenCL \cnglo{device}的特定資訊,
如\reftab{cldevquery}所示。

\carg{device} 是 \capi{clGetDeviceIDs} 所返回的\cnglo{device},
或者由 \capi{clCreateSubDevices} 創建的\cnglo{subdev}。
如果是\cnglo{subdev},則返回\cnglo{subdev}的資訊。

\carg{param_name} 是枚舉常量,用來指明要查詢什麼資訊,參見\reftab{cldevquery}。

\carg{param_value} 是指位器,所指內存用來存儲 \carg{param_name} 所對應的值
(參見\reftab{cldevquery})。
如果 \carg{param_value} 是 \cmacro{NULL},則忽略。

\carg{param_value_size} 就是 \carg{param_value} 所指內存塊的字節數。
其值必須 >= \reftab{cldevquery}中所列返回型別的大小。

\carg{param_value_size_ret} 返回 \carg{param_value} 所對應數據的實際大小。
如果 \carg{param_value_size_ret} 是 \cmacro{NULL},則忽略。

\placetable[here,split][tab:cldevquery]
{OpenCL 設備查詢}
{\input{chapter_plf/tbl/tbl_dev_query.tex}}

無論對於\cnglo{rootdev}(即 \capi{clGetDeviceIDs} 所返回的\cnglo{device})
還是由他創建的\cnglo{subdev},
\reftab{cldevquery}中所列查詢都會返回相同的資訊,不過下列查詢例外:
\startigBase
\item \cenum{CL_DEVICE_GLOBAL_MEM_CACHE_SIZE}
\item \cenum{CL_DEVICE_BUILT_IN_KERNELS}
\item \cenum{CL_DEVICE_PARENT_DEVICE}
\item \cenum{CL_DEVICE_PARTITION_TYPE}
\item \cenum{CL_DEVICE_REFERENCE_COUNT}
\stopigBase

如果執行成功,\capi{clGetDeviceInfo} 會返回 \cenum{CL_SUCCESS};
否則,返回下列錯誤碼之一:
\startigBase
\item \cenum{CL_INVALID_DEVICE}——如果 \carg{device} 無效。

\item \cenum{CL_INVALID_VALUE}——如果不支持 \carg{param_name};
或者 \carg{param_value_size} 的值 < \reftab{cldevquery}中所列返回型別的大小,
並且 \carg{param_value} 不是 \cmacro{NULL};
或者 \carg{param_name} 指的是某個擴展,但是此\cnglo{device}不支持此擴展。

\item \cenum{CL_OUT_OF_RESOURCES}——如果\scdevfailres。
\item \cenum{CL_OUT_OF_HOST_MEMORY}——如果\schostfailres。
\stopigBase

