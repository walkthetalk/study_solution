\startETD[cl_device_partition_property][partition value]

\clETD{CL_DEVICE_PARTITION_EQUALLY}{unsigned int}{
將一個大的\cnglo{device}集合分割成許多小的\cnglo{device}集合,
每個都含有 \math{n} 個\cnglo{computeunit}。
\math{n} 伴隨此屬性一起傳遞。
如果不能均勻平分 \cenum{CL_DEVICE_PARTITION_MAX_COMPUTE_UNITS},
則不會使用剩下的\cnglo{computeunit}。
}

\clETD{CL_DEVICE_PARTITION_BY_COUNTS}{unsigned int}{
此屬性後面跟着的是\cnglo{computeunit}數目清單,
以 \cenum{CL_DEVICE_PARTITION_BY_COUNTS_LIST_END} 終止。
對於清單中每個非零的 \math{m},
都會創建一個具有 \math{m} 個\cnglo{computeunit}的\cnglo{subdev}。
\cenum{CL_DEVICE_PARTITION_BY_COUNTS_LIST_END}的值是 0。
清單中非零值的數目不能超過 \cenum{CL_DEVICE_PARTITION_MAX_SUB_DEVICES}。
\cnglo{computeunit}的總數不能超過 \cenum{CL_DEVICE_PARTITION_MAX_COMPUTE_UNITS}。
}

\clETD{CL_DEVICE_PARTITION_BY_AFFINITY_DOMAIN}{cl_device_affinity_domain}{
將\cnglo{device}分割成許多小的\cnglo{device}集合,每個集合中包含一個或多個\cnglo{computeunit}。
他們共享部分緩存體系。
跟隨此屬性的值從下列值中選取:
\startigBase
\item \cenum{CL_DEVICE_AFFINITY_DOMAIN_NUMA},
\cnglo{subdev}中的\cnglo{computeunit}之間共享一個 NUMA 節點。

\item \cenum{CL_DEVICE_AFFINITY_DOMAIN_L4_CACHE},
\cnglo{subdev}中的\cnglo{computeunit}之間共享一個 4 級數據緩存。

\item \cenum{CL_DEVICE_AFFINITY_DOMAIN_L3_CACHE},
\cnglo{subdev}中的\cnglo{computeunit}之間共享一個 3 級數據緩存。

\item \cenum{CL_DEVICE_AFFINITY_DOMAIN_L2_CACHE},
\cnglo{subdev}中的\cnglo{computeunit}之間共享一個 2 級數據緩存。

\item \cenum{CL_DEVICE_AFFINITY_DOMAIN_L1_CACHE},
\cnglo{subdev}中的\cnglo{computeunit}之間共享一個 1 級數據緩存。

\item \cenum{CL_DEVICE_AFFINITY_DOMAIN_NEXT_PARTITIONABLE},
將\cnglo{device}按下一個可再分的相似域進行分割。
實作會判定此\cnglo{device}或者\cnglo{subdev}可以怎樣進一步細分,
按 NUMA、L4、L3、L2、L1 的順序,取最前面的那個。
對於所劃分成的\cnglo{subdev},其中的\cnglo{computeunit}間共享這一級的內存子系統。
\stopigBase

用戶可以通過對\cnglo{subdev}調用 \capi{clGetDeviceInfo}
(\cenum{CL_DEVICE_PARTITION_TYPE})來確定發生了什麼。
}

\stopETD
