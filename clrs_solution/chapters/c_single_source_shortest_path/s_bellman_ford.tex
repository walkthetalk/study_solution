\startsection[
  title={The Bellman-Ford algorithm},
  reference=section:24.1,
]

%e24.1-1
\startEXERCISE
在圖 24-4 上運行 Bellman-Ford 算法,
使用節點 \m{z} 作爲源節點。
在每一遍鬆弛過程中,以圖中相同的次序對每條邊進行鬆弛,
給出每遍鬆弛操作後的 \m{d} 值和 \m{\pi} 值。
然後,把邊 \m{(z,x)} 的權重改爲 4,
再次運行該算法,這次使用 \m{s} 作爲源節點。
\stopEXERCISE

\startANSWER
從 \m{z} 點開始:

\startcombination[nx=3,ny=2]
{\externalfigure[output/e24_1_1-1]}{}
{\externalfigure[output/e24_1_1-2]}{}
{\externalfigure[output/e24_1_1-3]}{}
{\externalfigure[output/e24_1_1-4]}{}
{\externalfigure[output/e24_1_1-5]}{}
\stopcombination

改了 \m{(z,x)} 的權重後,會出現總權重爲負值的環路。

\externalfigure[output/e24_1_1-10]
\stopANSWER

%e24.1-2
\startEXERCISE
證明推論 24.3。附推論 24.3:

設 \m{G=(V,E)} 是一帶權重的源節點爲 \m{s} 的有向圖,
其權重函數爲 \m{\omega: E\rightarrow R}。
假定圖 \m{G} 不包含從 \m{s} 可以到達的權重爲負值的環路,
則對於所有節點 \m{v\in V},
存在一條從源節點 \m{s} 到節點 \m{v} 的路徑
當且僅當 \ALGO{BELLMAN-FORD} 算法終止時有 \m{v.d < \infty}。
\stopEXERCISE

\startANSWER
首先,假設 \m{G} 中存在 \m{s} 到 \m{v} 的路徑,
證明算法終止時  \m{v.d < \infty}。
根據引理 24.2 可知 \m{v.d=\delta(s,v)},滿足 \m{v.d < \infty}。

現在,假定算法 \ALGO{BELLMAN-FORD} 終止時 \m{v.d<\infty}。
而根據引理 24.2,可知 \m{v.d=\infty} 或者 \m{v.d=\delta(s,v)}。
根據假設可知 \m{v.d=\delta(s,v)}。
因此 \m{G} 中存在從 \m{s} 到 \m{v} 的路徑。

附引理 24.2:

設 \m{G=(V,E)} 爲一個帶權重的源節點爲 \m{s} 的有向圖,
其權重函數爲 \m{\omega: E\rightarrow R}。
假定圖 \m{G} 不包含從 \m{s} 可以到達的權重爲負值的環路。
那麼算法 \ALGO{BELLMAN-FORD} 的第 2~4 行的 {\EMP for} 循環
執行了 \m{|V|-1} 次之後,
對於所有從源節點 \m{s} 可以到達的節點 \m{v},
我們有 \m{v.d=\delta(s,v)}。
\stopANSWER

%e24.1-3
\startEXERCISE
給定有向圖 \m{G=(V,E)} 帶權重且沒有權重爲負值的環路,
對於所有節點 \m{v\in V},
從源節點 \m{s} 到節點 \m{v} 之間的最短路徑中,
包含邊的條數的最大值爲 \m{m}。
(這裏,判斷最短路徑的根據是權重,不是邊的條數。)
請修改算法 \ALGO{BELLMAN-FORD},
讓其可以在 \m{m+1} 遍鬆弛操作後終止,
即使事先不知道 \m{m} 的值。
\stopEXERCISE

\startANSWER
如果一次遍歷過程中沒有任何節點需要鬆弛操作,則終止循環。
\stopANSWER

%e24.1-4
\startEXERCISE[exercise:24.1-4]
修改算法 \ALGO{BELLMAN-FORD},使其對於所有節點 \m{v} 來說,
如果從源節點 \m{s} 到節點 \m{v} 的一條路徑上存在權重爲負值的環路,
則將 \m{v.d} 的值設置爲 \m{-\infty}。
\stopEXERCISE

\startANSWER
\CLRSH{BELLMAN-FORD-B(G,w,s)}
\startCLRS
INITIALIZE-SINGLE-SOURCE(G,s)
for i = 1 to |G.V| - 1
	for each edge (u,v) in G.E
		RELAX(u,v,w)
for each edge (u,v) in G.E
	if v.d > u.d + w(u,v)
		v.d = -infty
		let t = v.pi
		while t != NIL and t.d != -infty
			t.d = -infty
			t = t.pi
for each edge (u,v) in G.E
	if v.d > u.d + w(u,v)
		return FALSE
return TRUE
\stopCLRS
\stopANSWER

%e24.1-5
\startEXERCISE\DIFFICULT
設 \m{G=(V,E)} 爲一個帶權重的有向圖,
其權重函數爲 \m{\omega: E\rightarrow R}。
請給出一個時間複雜度爲 \m{O(VE)} 的算法,
對於每個節點 \m{v\in V},
計算出數值 \m{\delta*(v)=\min_{u\in V}\{\delta(u,v)\}}。
\stopEXERCISE

\startANSWER
\CLRSH{RELAX-MOD(u,v,w)}
\startCLRS
min = w(u,v) + (u.d < 0 ? u.d : 0)
if v.d > min
	v.d = min
\stopCLRS

\ALGO{BELLMAN-FORD} 的時間複雜度不變,仍爲 \m{O(VE)}。
\stopANSWER

%e24.1-6
\startEXERCISE\DIFFICULT
設 \m{G=(V,E)} 爲一個帶權重的有向圖,
且圖中存在權重爲負值的環路。
請給出一個有效的算法來列出所有屬於該環路上的節點。
並證明算法的正確性。
\stopEXERCISE

\startANSWER
利用\inexercise[24.1-4] 的代碼。
\stopANSWER

\stopsection
