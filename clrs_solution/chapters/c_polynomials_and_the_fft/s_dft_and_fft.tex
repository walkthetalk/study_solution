\startsection[
  title={The DFT and FFT},
  reference=section:dft_fft,
]

%e30.2-1
\startEXERCISE
證明推論 30.4。
\stopEXERCISE

\startANSWER
\startformula\startmathalignment
\NC \omega_n^{n/2} \NC = e^{\frac{2\pi i}{n} * \frac{n}{2}}
    = e^{\pi i} = \cos(\pi) + i\sin(\pi) = -1 \NR
\NC \omega_2 \NC = e^{\frac{2\pi i}{2}} = e^{\pi i} = -1 \NR
\stopmathalignment\stopformula
所以 \m{\omega_{n}^{n/2} = \omega_2 = -1}。
\stopANSWER

%e30.2-2
\startEXERCISE
計算向量 \m{(0,1,2,3)} 的 \ALGO{DFT}。
\stopEXERCISE

\startANSWER
\TODO{略。}
\stopANSWER

%e30.2-3
\startEXERCISE
採用運行時間爲 \m{\Theta(n\lg n)} 的方案完成\inexercise[30.1-1]。
\stopEXERCISE

\startANSWER
\TODO{略。}
\stopANSWER

%e30.2-4
\startEXERCISE
寫出僞碼,在 \m{\Theta(n\lg n)} 時間內計算出 \m{\DFT_{n}^{-1}}。
\stopEXERCISE

\startANSWER
\TODO{略。}
\stopANSWER

%e30.2-5
\startEXERCISE
請把 \ALGO{FFT} 推廣到 \m{n} 是 3 的冪的情形,
寫出運行時間的遞迴式並求解。
\stopEXERCISE

\startANSWER
\m{T(n) = 3T(n/3) + O(n) = O(n\lg n)}。
\stopANSWER

%e30.2-6
\startEXERCISE[exercise:30.2-6]\DIFFICULT
假設我們不是在複數域上執行 \m{n} 個元素的 \ALGO{FFT} (其中 \m{n} 爲偶數),
而在整數模 \m{m} 生成的環 \m{\integers_m} 上執行 \ALGO{FFT},
其中 \m{m=2^{m/2} + 1},且 \m{t} 是任意正整數。
在模 \m{m} 的意義下,用 \m{\omega=2^t} 代替 \m{\omega_n} 作爲主 \m{n} 次單位根。
證明:在該系統中, \ALGO{DFT} 與逆 \ALGO{DFT} 的定義是完備的。
\stopEXERCISE

\startANSWER
\TODO{略。}
\stopANSWER

%e30.2-7
\startEXERCISE
給定一組值 \m{z_0,z_1,\ldots,z_{n-1}} (可以重複),
說明如何求出僅以 \m{z_0,z_1,\ldots,z_{n-1}} (可以重複)爲
零點的一個次數界爲 \m{n+1} 的多項式 \m{P(x)} 的係數。
你給出的過程運行時間應爲 \m{O(n\lg^{2}n)}。
(\hint 當且僅當 \m{P(x)} 是 \m{(x - z_j)} 的倍數時,
多項式 \m{P(x)} 在 \m{z_j} 處值爲 0。)
\stopEXERCISE

\startANSWER
\TODO{略。}
\stopANSWER

%e30.2-8
\startEXERCISE\DIFFICULT
一個向量 \m{a=(a_0,a_1,\ldots,a_{n-1})} 的{\EMP 線性調頻變換}(chirp transform)
是向量 \m{y=(y_0,y_1,\ldots,y_{n-1})},其中 \m{y_k = \sum_{j=0}^{n-1}a_j z^{kj}},
其中 \m{z} 是任意複數。
因此,通過取 \m{z=\omega_n}, \ALGO{DFT} 是線性調頻變換的一種特殊情形。
對任意複數 \m{z},
請說明如何在 \m{O(n\lg n)} 時間內求出線性調頻變換的值。
(\hint 利用等式
\startformula
y_k = z^{k^2/2}\sum_{j=0}^{n-1}(a_j z^{j^2/2})(z^{-(k-j)^2/2})
\stopformula
可以把線性調頻變換看作一個卷積。)
\stopEXERCISE

\startANSWER
令 \m{z = b\omega},其中 \m{\omega = \cos u + i\sin u}, \m{b} 爲實數。
則:
\startformula
y_k = \sum_{j=0}^{n-1}a_j (b\omega)^{kj}
\stopformula
\startformula
y_k = b^{k} \omega^{\frac{j^2}{2}}
	\sum_{j=0}^{n-1}(a_j b^j \omega^{\frac{j^2}{2}})(\omega^{\frac{(k-j)^2}{2}})
\stopformula
其中 \m{b^0,b^1,\ldots,b^{n-1}} 可以經過 \m{O(n)} 次循環求出,
 \m{\omega^{n}} 則可以通過如下歐拉函數公式在 \m{O(1)} 時間內算出:
\startformula
(\cos u + i\sin u)^n = \cos nu + i\sin nu
\stopformula

令:
\startformula
g(j) = a_j b^j \omega^{\frac{j^2}{2}}
\stopformula
\startformula
h(k-j)=\omega^{\frac{(k-j)^2}{2}}
\stopformula
\startformula
t_k = b^k \omega^{\frac{k^2}{2}}
\stopformula
則:
\startformula
y_k = t_k \sum_{j=0}^{n-1}g(j)*h(k-j)
\stopformula
多項式 \m{g} 和 \m{h} 相乘得到卷積 \m{g\otimes h}。
\stopANSWER
\stopsection
