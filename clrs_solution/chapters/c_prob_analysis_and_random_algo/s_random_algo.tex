\startsection[
  title={Randomized algorithms},
]

%5.3-1
\startEXERCISE
Marceau 教授不同意引理 5.4 證明中所用的循環不變式。
第一次迭代之前循環不變式是否成立?他對此提出了質疑。
他的理由是,我們也可以宣稱一個空數列不包含 0-排列,
從而在第一次迭代之前使得循環不變式失效。
請重寫過程 \ALGO{RANDOMLY-PERMUTE},
使得在第一次迭代之前子數列非空的情况下,
循環不變式依舊適用。
並据此证明引理 5.4。

\CLRSH{RANDOMLY-PERMUTE(A,n)}

\startCLRSCODE
for i = 1 to n
	// swap A[i] with A[RANDOM(i,n)]
\stopCLRSCODE
\stopEXERCISE
\startANSWER
變通一下,其實我們可以在循環之前,隨機選擇一個元素來替代第一個。
這樣不變式對於 1-排列是有效的。
\stopANSWER

\startEXERCISE
Kelp 教授決定寫一個過程來隨機產生除恆等排列(identity permutation)外的任意排列。
他提出了如下過程:

\CLRSH{PERMUTE-WITHOUT-IDENTITY(A)}
\startCLRS
n = A.length
for i = 1 to n - 1
	swap A[i] with A[RANDOM(i + 1, n)]
\stopCLRS
這段代碼實現了 Kelp 教授的意圖嗎?
\stopEXERCISE
\startANSWER
不能。因爲這個過程改變了所有元素的位置。
我們無法得到恆等排列,但同時也無法使得任何一個元素處於原有位置。
\stopANSWER

\startEXERCISE
假設我們不是將元素 \m{A[i]} 與子數列 \m{A[i..n]} 中的一個隨機元素交換,
而是將他與數列任意位置上的隨機元素交換:

\CLRSH{PERMUTE-WITH-ALL(A)}
\startCLRS
n = A.length
for i = 1 to n
	swap A[i] with A[RANDOM(1, n)]
\stopCLRS
這段代碼會產生一個均勻隨機排列嗎?爲什麼?
\stopEXERCISE
\startANSWER
不會。直覺上來講,有 \m{n^n} 種不同方式,但只有 \m{n!} 種組合。
由於 \m{n!} 無法整除 \m{n^n},不可能是均勻分布
(爲什麼無法整除?這就是直覺。對於 \m{n>2}, \m{n-1} 可以整除 \m{n!},但卻不能整除 \m{n^n})。
參見 \simpleurl{http://blog.codinghorror.com/the-danger-of-naivete/}。
\stopANSWER

\startEXERCISE
Armstrong 教授建議用下面的過程來產生一個均勻隨機排列:

\CLRSH{PERMUTE-BY-CYCLIC(A)}
\startCLRS
n = A.length
let B[1..n] be a new array
offset = RANDOM(1, n)
for i = 1 to n
	dest = i + offset
	if dest > n
		dest = dest - n
	B[dest] = A[i]
return B
\stopCLRS

請說明每個元素 \m{A[i]} 出現在 \m{B} 中任何特定位置的概率是 \m{1/n}。
然後通過說明排列結果不是均勻隨機排列,證明 Armstrong 教授錯了。
\stopEXERCISE
\startANSWER
如果 \m{j \equiv \text{offset} + i \pmod{n}},則 \m{A[i]} 就會出現在 \m{B[j]} 處,其概率爲 \m{1/n}。
但他不能生成所有排列,他所生成的排列都是數列 \m{A} 循環右移得到。
\stopANSWER

\startEXERCISE \DIFFICULT
證明:在過程 \ALGO{PERMUTE-BY-SORTING} 的數列 \m{P} 中,所有元素都唯一的概率至少是 \m{1 - 1/n}。
\stopEXERCISE
\startANSWER
令 \m{\Pr\{j\}} 爲索引爲 \m{j} 的元素唯一的概率。
如果有 \m{n^3} 個元素,則 \m{\Pr\{j\} = 1 - \frac{j-1}{n^3}}。
\startformula\startmathalignment
\NC \Pr\{1 \cap 2 \cap 3 \cap \ldots\}
       \NC= \Pr\{1\} \cdot \Pr\{2 | 1\} \cdot \Pr\{3 | 1 \cap 2\} \cdots \NR
\NC    \NC= 1 (1 - \frac{1}{n^3})
            (1 - \frac{2}{n^3})
            (1 - \frac{3}{n^3})
            \cdots \NR
\NC    \NC\ge 1 (1 - \frac{n}{n^3})
            (1 - \frac{n}{n^3})
            (1 - \frac{n}{n^3})
            \cdots \NR
\NC    \NC\ge (1 - \frac{1}{n^2})^n \NR
\NC    \NC\ge 1 - \frac{1}{n} \NR
\stopmathalignment\stopformula
你問最後一步如何推導出來的? \m{(1-x)^n \ge 1 - nx}。
\stopANSWER

\startEXERCISE
請解釋如何實現算法 \ALGO{PERMUTE-BY-SORTING},
以處理兩個或更多優先級相同的情形。
也就是說,即使有兩個或更多優先級相同,你的算法也應該產生一個均勻隨機排列。
\stopEXERCISE
\startANSWER
愚蠢的算法需要用愚蠢的方式解決。只要重新生成優先級重試就可以了。
\stopANSWER

\startEXERCISE
假設我們希望創建集合 \m{\{1,2,3,\ldots,n\}} 的一個{\EMP 隨機樣本(random sample)},
即一個具有 \m{m} 個元素的子集 \m{S},其中 \m{0\le m\le n},
使得以同樣的概率創建每個 \m{m} 子集。
一種方法是對 \m{i = 1,2,\ldots,n} 設 \m{A[i] = i},
調用 \ALGO{RANDOMIZE-IN-PLACE(A)},
然後取最前面的 \m{m} 個數列元素。
這種方法會調用 \m{n} 次過程 \ALGO{RANDOM}。
如果 \m{n} 比 \m{m} 大的多,
只調用 \ALGO{RANDOM} 少量次數,就能創建一個隨機樣本。
請說明下面遞迴過程返回 \m{\{1,2,3,\ldots,n\}} 的一個隨機 \m{m} 子集 \m{S},
其中每個 \m{m} 子集是等概率的,
然後只調用 \m{m} 次 \ALGO{RANDOM}。

\CLRSH{RANDOM-SAMPLE(m,n)}
\startCLRS
if m == 0
	return ∅
else S = RANDOM-SAMPLE(m - 1, n - 1)
	i = RANDOM(1, n)
	if i ∈ S
		S = S ∪ {n}
	else S = S ∪ {i}
	return S
\stopCLRS
\stopEXERCISE
\startANSWER
每種組合的概率應爲 \m{1/\binom{n}{m}}。
我們來證明下面的命題:

\ALGO{RANDOM-SAMPLE(m,n)} 返回均勻分布的組合。

在 \m{m} 上進行歸納。 \m{m} 爲 \m{1} 或 \m{0} 時顯然是成立的。
現在假定 \m{m-1} 時命題成立,看看 \m{m} 會發生什麼。

遞迴調用 \m{m - 1} 返回的樣本時均勻分布的。
然後有兩種情況,新的 \m{m} 子集是否包含 \m{n}。

如果包含 \m{n} (概率: \m{m/n}),則其概率爲:
\startformula
\frac{m}{n}\frac{1}{\binom{n-1}{m-1}} = 1/\binom{n}{m}
\stopformula

如果不包含 \m{n} (概率: \m{(n-m)/n}),則其概率爲:
\startformula
\frac{n-m}{n}\frac{1}{\binom{n-1}{m}} = 1/\binom{n}{m}
\stopformula
\stopANSWER

\stopsection
