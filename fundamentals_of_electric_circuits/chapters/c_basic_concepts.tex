\startcomponent c_basic_concepts
\startchapter[
  title={Basic Concepts},
]

% 1.1
\startQ
How much charge is represented by these number of electrons?
\startIG
\item $\munit{6.482e17}$
\item $\munit{1.24=e18}$
\item $\munit{2.46=e19}$
\item $\munit{1.628e20}$
\stopIG
\stopQ

\startA
\startIG
\item $q = (\munit{1.602e-19 coulomb}) \times \munit{6.482e17} = \munit{_0.10384 coulomb}$
\item $q = (\munit{1.602e-19 coulomb}) \times \munit{1.24=e18} = \munit{_0.19865 coulomb}$
\item $q = (\munit{1.602e-19 coulomb}) \times \munit{2.46=e19} = \munit{_3.941== coulomb}$
\item $q = (\munit{1.602e-19 coulomb}) \times \munit{1.628e20} = \munit{26.08=== coulomb}$
\stopIG
\stopA

% 1.2
\startQ
Determine the current flowing through an element if the charge flow is given by
\startIG
\item $q(t) = (3t+8)               \munit[l]{milli coulomb}$
\item $q(t) = (8t^2 + 4t − 2)      \munit[l]{coulomb}$
\item $q(t) = (3e^{−t} − 5e^{−2t}) \munit[l]{nano coulomb}$
\item $q(t) = 10 \sin 120π t       \munit[l]{pico coulomb}$
\item $q(t) = 20e^{−4t} \cos 50t   \munit[l]{micro coulomb}$
\stopIG
\stopQ

\startA
\startIG
\item $i = \frac{d q}{d t} = 3                      \munit[l]{milli ampere}$
\item $i = \frac{d q}{d t} = (16t + 4)              \munit[l]{ampere}$
\item $i = \frac{d q}{d t} = (-3e^{−t} + 10e^{−2t}) \munit[l]{nano ampere}$
\item $i = \frac{d q}{d t} = 1200π \cos 120π t      \munit[l]{pico ampere}$
\item $i = \frac{d q}{d t} = -(80 \cos 50t + 1000 \sin 50 t) e^{−4t} \munit[l]{micro ampere}$
\stopIG
\stopA

% 1.3
\startQ
Find the charge $q(t)$ flowing through a device if the current is:
\startIG
\item $i(t) = \munit{3 ampere}, q(0) = \munit{1 coulomb}$
\item $i(t) = (2t + 5) \munit[l]{milli ampere}, q(0) = 0$
\item $i(t) = 20 \cos (10t + π / 6) \munit[l]{micro ampere}, q(0) = \munit{2 micro coulomb}$
\item $i(t) = 10e^{−30t} \sin 40t \munit[l]{ampere}, q(0) = 0$
\stopIG
\stopQ

\startA
\startIG
\item $q(t) = \int_0 i(t) + q(0) = (3t+1) \munit[l]{coulomb}$
\item $q(t) = \int_0 i(t) + q(0) = (t^2 + 5t) \munit[l]{milli coulomb}$
\item $q(t) = \int_0 i(t) + q(0) = (2 \sin (10t + π / 6) + 1) \munit[l]{micro coulomb}$
\item $q(t) = \int_0 i(t) + q(0) = (-e^{-30t}(0.16\cos 40t + 0.12\sin 40t) + 0.16) \munit[l]{coulomb}$
\stopIG
\stopA

% 1.4
\startQ
A total charge of $\munit{300 coulomb}$ flows past a given cross section of a conductor in 30 seconds.
What is the value of the current?
\stopQ

\startA
$\munit[r]{300 coulomb} / \munit[l]{30 second} = \munit{10 ampere}$
\stopA

% 1.5
\startQ
Determine the total charge transferred over the time interval
of $0 \le t \le \munit{10 second}$ when $i(t) = \frac{1}{2}t \munit[l]{ampere}$.
\stopQ

\startA
\startformula
q(t) = \int_0^{10}\frac{1}{2}t = \left.\frac{t^2}{4}\right|_0^{10} = 25
\stopformula
\stopA

% 1.6
\startQ[Q:1.6]
The charge entering a certain element is shown in \reffig{1.23}. Find the current at:
\startIG
\item $t = \munit{_1 milli second}$
\item $t = \munit{_6 milli second}$
\item $t = \munit{10 milli second}$
\stopIG
\placefigure[here][fig:1.23]{for \refq{1.6}}{
\startMPcode
	defcoord(10,2,15,15,1.5);
	defdim.x("t(ms)")(2,4,6,8,10,12);
	defdim.y("q(t)(mC)")(30);
	draw coordsys;
	drawemp((0,0)--(2,30)--(8,30)--(12,0));
	drawaux((2,0)--(2,30));
	drawaux((8,0)--(8,30));
\stopMPcode
}
\stopQ

\startA
\startIG
\item $i(\munit{t}) = \frac{dq}{dt} = \frac{d(15t)}{dt}= 15, i(1) = 15 \munit[l]{coulomb}$
\item $i(\munit{t}) = \frac{dq}{dt} = \frac{d(30)}{dt} = 0, i(6) = 0 \munit[l]{coulomb}$
\item $i(\munit{t}) = \frac{dq}{dt} = \frac{d(90-7.5t)}{dt} = -7.5, i(10) = -7.5 \munit[l]{coulomb}$
\stopIG
\stopA

% 1.7
\startQ[Q:1.7]
The charge flowing in a wire is plotted in \reffig{1.24}. Sketch the corresponding current.
\placefigure[here][fig:1.24]{for \refq{1.7}}{
\startMPcode
	defcoord(20,2,15,15,1.5);
	defdim.x("t(s)")(1,2,3,4);
	defdim.y("q(C)")(-10,10);
	draw coordsys;
	drawemp((0,0)--(1,10)--(2,-10)--(3,-10)--(4,0));
\stopMPcode
}
\stopQ

\startA
\startMPcode
	defcoord(20,2,15,15,1.5);
	defdim.x("t(s)")(1,2,3,4);
	setloc.x("lrt")(1,2);
	defdim.y("q(C)")(-20,-10,10);
	draw coordsys;
	drawemp((0,10)--(1,10)--(1,-20)--(2,-20)--(2,0)--(3,0)--(3,10)--(4,10));
\stopMPcode
\stopA

% 1.8
\startQ[Q:1.8]
 The current flowing past a point in a device is shown in \reffig{1.25}.
 Calculate the total charge through the point.
\placefigure[here][fig:1.25]{for \refq{1.8}}{
\startMPcode
	defcoord(30,3,15,15,1.5);
	defdim.x("t(ms)")(1,2);
	defdim.y("i(mA)")(10);
	draw coordsys;
	drawemp((0,0)--(1,10)--(2,10)--(2,0));
\stopMPcode
}
\stopQ

\startA
\startformula
q = \int i dt = \frac{10\times 1}{2} + 10\times 1 = \munit{15 milli coulomb}
\stopformula
\stopA

% 1.9
\startQ[Q:1.9]
The current through an element is shown in \reffig{1.26}.
Determine the total charge that passed through the element at:
\startcolumns[n=3]
\startIG
\item $t=\munit{1 second}$
\item $t=\munit{3 second}$
\item $t=\munit{5 second}$
\stopIG
\stopcolumns
\placefigure[here][fig:1.26]{for \refq{1.9}}{
\startMPcode
	defcoord(15,3,15,15,1.5);
	defdim.x("t(s)")(1,2,3,4,5);
	defdim.y("i(A)")(5,10);
	draw coordsys;
	drawemp((0,10)--(1,10)--(2,5)--(4,5)--(5,0));
\stopMPcode
}
\stopQ

\startA
\startIG
\item $q=\int_0^1 i dt = \int_0^1 10 dt = \munit{10 coulomb}$
\item $q=\int_0^3 i dt
        = \int_0^1 i dt + \int_1^2 i dt         + \int_2^3 i dt
        =\int_0^1 10 dt + \int_1^2 (15 - 5t) dt + \int_2^3 5 dt
        =10             + 7.5                   + 5
        =\munit{22.5 coulomb}$
\item $q=\int_0^5 i dt
        =\int_0^3 i dt + \int_3^4 i dt + \int_4^5 i dt
        = 22.5         + 5             + 2.5
        =\munit{30 coulomb}$
\stopIG
\stopA

% 1.10
\startQ
A lightning bolt with \munit{10 kilo ampere} strikes an object for \munit{15 micro second}.
How much charge is deposited on the object?
\stopQ

\startA
$q = (\munit{10e3 ampere}) \times (\munit{15e-6 second})
   = \munit{150e-3 coulomb} = \munit{0.15 coulomb}$
\stopA

% 1.11
\startQ
A rechargeable flashlight battery is capable of delivering
\munit{90 milli ampere} for about \munit{12 hour}.
How much charge can it release at that rate?
If its terminal voltage is \munit{1.5 volt},
how much energy can the battery deliver?
\stopQ
\startA
$q = (\munit{90e-3 ampere}) \times (12 \times 3600 \munit{hour})
   = \munit{3888 coulomb}$

$E = pt = ivt = qv = \munit{3888 coulomb} \times \munit{1.5 volt} = \munit{5832 joule}$
\stopA

% 1.12
\startQ
If the current flowing through an element is given by
\startformula
i(t)=\startmathcases
\NC 3t \munit[l]{ampere}, \MC 0 \le t < \munit{6 second} \NR
\NC \munit{18 ampere},    \MC 6 \le t < \munit{10 second} \NR
\NC \munit{-12 ampere},   \MC 10\le t < \munit{15 second} \NR
\NC 0, \MC t \ge \munit{15 second} \NR
\stopmathcases
\stopformula
Plot the charge stored in the element over $0 < t < \munit{20 second}$.
\stopQ
\startA
\startMPcode
	defcoord(5,2,15,15,1.5);
	defdim.x("(t) s")(6,10,15);
	defdim.y("i(t) A")(-12,18);
	draw coordsys;
	drawemp((0,0)--(6,18)--(10,18)--(10,-12)--(15,-12)--(15,0)--(17,0));
\stopMPcode
\stopA

\stopchapter
\stopcomponent
