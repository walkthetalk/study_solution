\section{64 位原子函式}

下列兩個可選擴展實現了 \cqlf{__global} 和 \cqlf{__local} 內存中的 64 位
帶符號和無符號整數上的原子運算:
\startigBase
\item \clext{cl_khr_int64_base_atomics}
\item \clext{cl_khr_int64_extended_atomics}
\stopigBase

\cnglo{app}中要想使用這些擴展,
必須在 OpenCL \cnglo{program}源碼中包含下列編譯指示之一:
\startclc
#pragma OPENCL EXTENSION cl_khr_int64_base_atomics : enable
#pragma OPENCL EXTENSION cl_khr_int64_extended_atomics : enable
\stopclc

\reftab{atomic64_base}中列出了擴展 \clext{cl_khr_int64_base_atomics} 所支持的原子函式,
其中所有函式都在一個原子事務內實施。

\placetable[here,split][tab:atomic64_base]
{擴展 \clext{cl_khr_int64_base_atomics} 的內建原子函式}
{% atomic_add
\startbuffer[funcproto:atomic64_add]
long atomic_add (
	volatile __global long *p,
	long val)
long atomic_add (
	volatile __local long *p,
	long val)

ulong atomic_add (
	volatile __global ulong *p,
	ulong val)
ulong atomic_add (
	volatile __local ulong *p,
	ulong val)
\stopbuffer
\startbuffer[funcdesc:atomic64_add]
讀取 \carg{p} 所指向的 64 位值(記為 \math{old})。
計算 \math{(old + \marg{val})} 並將結果存儲到 \carg{p} 所指位置中。
此函式返回 \math{old}。
\stopbuffer

% atomic_sub
\startbuffer[funcproto:atomic64_sub]
long atomic_sub (
	volatile __global long *p,
	long val)
long atomic_sub (
	volatile __local long *p,
	long val)

ulong atomic_sub (
	volatile __global ulong *p,
	ulong val)
ulong atomic_sub (
	volatile __local ulong *p,
	ulong val)
\stopbuffer
\startbuffer[funcdesc:atomic64_sub]
讀取 \carg{p} 所指向的 64 位值(記為 \math{old})。
計算 \math{(old - \marg{val})} 並將結果存儲到 \carg{p} 所指位置中。
此函式返回 \math{old}。
\stopbuffer

% atomic_xchg
\startbuffer[funcproto:atomic64_xchg]
long atomic_xchg (
	volatile __global long *p,
	long val)
long atomic_xchg (
	volatile __local long *p,
	long val)

ulong atomic_xchg (
	volatile __global ulong *p,
	ulong val)
ulong atomic_xchg (
	volatile __local ulong *p,
	ulong val)
\stopbuffer
\startbuffer[funcdesc:atomic64_xchg]
將位置 \carg{p} 中所存儲的值 \math{old} 和 \carg{val} 中的新值相互交換。
返回 \math{old}。
\stopbuffer

% atomic_inc
\startbuffer[funcproto:atomic64_inc]
long atomic_inc (volatile __global long *p)
long atomic_inc (volatile __local long *p)

ulong atomic_inc (
	volatile __global ulong *p)
ulong atomic_inc (
	volatile __local ulong *p)
\stopbuffer
\startbuffer[funcdesc:atomic64_inc]
讀取 \carg{p} 所指向的 64 位值(記為 \math{old})。
計算 \math{(old+1)} 並將結果存儲到 \carg{p} 所指位置中。
此函式返回 \math{old}。
\stopbuffer

% atomic_dec
\startbuffer[funcproto:atomic64_dec]
long atomic_dec (volatile __global long *p)
long atomic_dec (volatile __local long *p)

ulong atomic_dec (
	volatile __global ulong *p)
ulong atomic_dec (
	volatile __local ulong *p)
\stopbuffer
\startbuffer[funcdesc:atomic64_dec]
讀取 \carg{p} 所指向的 64 位值(記為 \math{old})。
計算 \math{(old-1)} 並將結果存儲到 \carg{p} 所指位置中。
此函式返回 \math{old}。
\stopbuffer

% atomic_cmpchg
\startbuffer[funcproto:atomic64_cmpxchg]
long atomic_cmpxchg (
	volatile __global long *p,
	long cmp, long val)
long atomic_cmpxchg (
	volatile __local long *p,
	long cmp,
	long val)

ulong atomic_cmpxchg (
	volatile __global ulong *p,
	ulong cmp,
	ulong val)
ulong atomic_cmpxchg (
	volatile __local ulong *p,
	ulong cmp,
	ulong val)
\stopbuffer
\startbuffer[funcdesc:atomic64_cmpxchg]
讀取 \carg{p} 所指向的 64 位值(記為 \math{old})。
計算 \math{(old == cmp) ? val : old} 並將結果存儲到 \carg{p} 所指位置中。
此函式返回 \math{old}。
\stopbuffer


% begin table
\startCLFD
\clFD{atomic64_add}
\clFD{atomic64_sub}
\clFD{atomic64_xchg}
\clFD{atomic64_inc}
\clFD{atomic64_dec}
\clFD{atomic64_cmpxchg}
\stopCLFD
}

\reftab{atomic64_ext}中列出了擴展 \clext{cl_khr_int64_extended_atomics} 所支持的原子函式,
其中所有函式都在一個原子事務內實施。

\placetable[here,split][tab:atomic64_ext]
{擴展 \clext{cl_khr_int64_extended_atomics} 的內建原子函式}
{% atomic_min
\startbuffer[funcproto:atomic64_min]
long atomic_min (
	volatile __global long *p,
	long val)
long atomic_min (
	volatile __local long *p,
	long val)

ulong atomic_min (
	volatile __global ulong *p,
	ulong val)
ulong atomic_min (
	volatile __local ulong *p,
	ulong val)
\stopbuffer
\startbuffer[funcdesc:atomic64_min]
讀取 \carg{p} 所指向的 64 位值(記為 \math{old})。
計算 \math{\mapiemp{min}(old, \marg{val})} 並將結果存儲到 \carg{p} 所指位置中。
此函式返回 \math{old}。
\stopbuffer

% atomic_max
\startbuffer[funcproto:atomic64_max]
long atomic_max (
	volatile __global long *p,
	long val)
long atomic_max (
	volatile __local long *p,
	long val)

ulong atomic_max (
	volatile __global ulong *p,
	ulong val)
ulong atomic_max (
	volatile __local ulong *p,
	ulong val)
\stopbuffer
\startbuffer[funcdesc:atomic64_max]
讀取 \carg{p} 所指向的 64 位值(記為 \math{old})。
計算 \math{\mapiemp{max}(old, \marg{val})} 並將結果存儲到 \carg{p} 所指位置中。
此函式返回 \math{old}。
\stopbuffer

% atomic_and
\startbuffer[funcproto:atomic64_and]
long atomic_and (
	volatile __global long *p,
	long val)
long atomic_and (
	volatile __local long *p,
	long val)

ulong atomic_and (
	volatile __global ulong *p,
	ulong val)
ulong atomic_and (
	volatile __local ulong *p,
	ulong val)
\stopbuffer
\startbuffer[funcdesc:atomic64_and]
讀取 \carg{p} 所指向的 64 位值(記為 \math{old})。
計算 \math{(old \mcmm{&} \marg{val})} 並將結果存儲到 \carg{p} 所指位置中。
此函式返回 \math{old}。
\stopbuffer

% atomic_or
\startbuffer[funcproto:atomic64_or]
long atomic_or (
	volatile __global long *p,
	long val)
long atomic_or (
	volatile __local long *p,
	long val)

ulong atomic_or (
	volatile __global ulong *p,
	ulong val)
ulong atomic_or (
	volatile __local ulong *p,
	ulong val)
\stopbuffer
\startbuffer[funcdesc:atomic64_or]
讀取 \carg{p} 所指向的 64 位值(記為 \math{old})。
計算 \math{(old \mcmm{|} \marg{val})} 並將結果存儲到 \carg{p} 所指位置中。
此函式返回 \math{old}。
\stopbuffer

% atomic_xor
\startbuffer[funcproto:atomic64_xor]
long atomic_xor (
	volatile __global long *p,
	long val)
long atomic_xor (
	volatile __local long *p,
	long val)

ulong atomic_xor (
	volatile __global ulong *p,
	ulong val)
ulong atomic_xor (
	volatile __local ulong *p,
	ulong val)
\stopbuffer
\startbuffer[funcdesc:atomic64_xor]
讀取 \carg{p} 所指向的 64 位值(記為 \math{old})。
計算 \math{(old \mcmm{^} \marg{val})} 並將結果存儲到 \carg{p} 所指位置中。
此函式返回 \math{old}。
\stopbuffer


% begin table
\startCLFD
\clFD{atomic64_min}
\clFD{atomic64_max}
\clFD{atomic64_and}
\clFD{atomic64_or}
\clFD{atomic64_xor}
\stopCLFD
}

對於執行這些原子函式的\cnglo{device}而言,這些事務都是原子的。
而對於在多個\cnglo{device}上執行的\cnglo{kernel}而言,
如果這些原子運算是在同一內存位置上實施的,則不保證其原子性。

\startnotepar
64 位整數和 32 位整數(包括 \ctype{float})上的原子運算相互之間也是原子的。
\stopnotepar
