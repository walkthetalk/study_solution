\startsection[
  title={Flow networks},
  reference=section:flow_networks,
]

%e26.1-1
\startEXERCISE
證明:在一個流網絡中,
將一條邊分解爲兩條邊所得到的是一個等價的網絡。
更形式化地說,假定網絡 \m{G} 包含邊 \m{(u,v)},
我們以如下方式創建一個新的流網絡 \m{G'}:
創建一個新節點 \m{x},
用新的邊 \m{(u,x)} 和 \m{(x,v)} 來替換原來的邊 \m{(u,v)},
並設置 \m{c(u,x)=c(x,v)=c(u,v)}。
證明: \m{G'} 中的一個最大流與 \m{G} 中的一個最大流具有相同的值。
\stopEXERCISE

\startANSWER
\m{f(u,x)=f(x,v)}。
\stopANSWER

%e26.1-2
\startEXERCISE
將流的性值和定義推廣到多源點和多匯點的流問題上。
證明:在多源點多匯點流網絡中,
通過增加一個超級源點和超級匯點,可以所形成一個單源點單匯點流網絡,
新網絡與原網絡中的流是一一對應的。
\stopEXERCISE

\startANSWER
容量限制:對於所有 \m{u,v\in V},都有 \m{0\le f(u,v)\le c(u,v)};

流量守恆:對於所有 \m{u\in V-S-T},都有 \m{\sum_{v\in V}f(v,u)=\sum_{v\in V}f(u,v)}。
\stopANSWER

%e26.1-3
\startEXERCISE
假定流網絡 \m{G=(V,E)} 違反了如下假設:對於所有節點 \m{v\in V},
網絡必須包括一條路徑 \m{s\leadsto v\leadsto t}。
設節點 \m{u} 滿足:不存在路徑 \m{s\leadsto u\leadsto t}。
證明: \m{G} 中必然存在一個最大流 \m{f},
使得對於所有節點 \m{v\in V}, \m{f(u,v)=f(v,u)=0}。
\stopEXERCISE

\startANSWER
\m{u} 只能扇入或只能扇出,因此只能是既無扇入亦無扇出。
\stopANSWER

%e26.1-4
\startEXERCISE
設 \m{f} 爲網絡中的一個流, \m{\alpha} 爲一實數,
將 \m{\alpha f} 稱爲{\EMP 標量流積(scalar flow product)},
是從 \m{V\times V} 到 \m{R} 的函數,定義如下:
\startformula
(\alpha f)(u,v) = \alpha \cdot f(u,v)
\stopformula
證明:網絡中的流形成一個{\EMP 凸集(convex set)}。
也就是說,證明:如果 \m{f_1} 和 \m{f_2} 爲兩個流,
則 \m{\alpha f_1 + (1-\alpha)f_2} 也是一個流,這裏 \m{0\le \alpha \le 1}。
\stopEXERCISE

\startANSWER
\m{0\le \alpha f_1 + (1-\alpha)f_2 \le \max(f_1,f_2)}。
\stopANSWER

%e26.1-5
\startEXERCISE
將最大流問題表述爲一個線性規劃問題。
\stopEXERCISE

\startANSWER
\startformula\startmathalignment
\NC \max \NC (\sum_{v\in V}f(s,v) - \sum_{v\in V}f(v,s)) \NR
\NC s.t. \NC 0\le f(u,v) \le c(u,v) \NR
\NC \NC \sum_{v\in V}f(v,u) - \sum_{v\in V}f(u,v) = 0 \NR
\stopmathalignment\stopformula
\stopANSWER

%e26.1-6
\startEXERCISE
Adam 教授有兩個兒子,可不幸的是,他們互相討厭對方。
隨着時間的推移,問題變得如此嚴重,
他們不僅不願意一起到學校,
而且都拒絕走對方當天所走過的街區。
兩個孩子並不在意自己走的路徑與對方所走的路徑在街角交叉。
幸運的是,教授的房子和學校都位於街角。
但除此之外,教授不能肯定是否可以在滿足上述條件的情況下
把兩個小孩送到同一所學校。
教授有一份小鎮的地圖,試說明如何將這個問題轉換爲一個最大流問題,
以便決定是否可以將孩子送到同一所學校。
\stopEXERCISE

\startANSWER
每條路的容量都是 1,最大流必須大於 1。
\stopANSWER

%e26.1-7
\startEXERCISE
假定除邊的容量外,流網絡還有{\EMP 節點容量}。
即對於每個節點 \m{v},有一個極限值 \m{l(v)},
這是可以流經此節點的最大流量。
請說明如何將一個帶有節點容量的流網絡 \m{G=(V,E)} 轉換
爲一個等價的但沒有節點容量的流網絡 \m{G'=(V',E')},
使得 \m{G'} 中的最大流與 \m{G} 中的最大流取值相同。
圖 \m{G'} 裏有多少個節點和多少條邊?
\stopEXERCISE

\startANSWER
將每個節點 \m{v} 轉換成一條邊 \m{v,v‘},其容量爲 \m{l(v)}。
 \m{V' = 2V}, \m{E'=V+E}。
\stopANSWER

\stopsection
