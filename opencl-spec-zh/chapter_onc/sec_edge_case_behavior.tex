% Edge Case Behavior
\section[section:edgeCaseBehavior]{邊界條件下的行為}

數學函式(\insection[mathFunc])在邊界條件下的行為
要符合《ISO/IEC 9899:TC2》(通常稱作 C99, TC2)中的{\ftRef{節 F.9}} 和{\ftRef{節 G.6}},
\insection[addReqBeyondC99]中所列之處除外。

% Additional Requirements Beyond C99 TC2
\subsection[section:addReqBeyondC99]{C99 TC2 之外的附加要求}

如果函式會返回 NaN,且有多個算元為 NaN,則應當返回其中一個 NaN 算元。
如果是 sNaN,會返回 NaN 算元的函式可能會使其消音。
非 sNaN 轉換後應當還是非 sNaN。
sNaN 轉換後應當是 NaN,但可能被轉換成非 sNaN。
而 NaN 的凈荷位和正負號如何轉換則未定義。

函式 \capi{half_<funcname>} 所起作用與不帶前綴 \capi{half_} 的同名函式一樣。
他們必須符合同一邊界條件需求(參見《C99, TC2》中的{\ftRef{節 F.9}} 和{\ftRef{節 G.6}})。
對於其他情況,除了明確提到的地方,准許單精度函式的最大誤差為 8192 ulp
(以單精度結果來衡量),儘管鼓勵更高的精確度。

對於那些明確規定了其結果的值(參見《C99, TC2》中的{\ftRef{節 F.9}},
或者\insection[addReqBeyondC99]以及後面的\insection[ftzmECB])
(如: \ccmm{ceil(-1 < x < 0)} 會返回 \ccmm{-0}),
捨入誤差(\insection[relativeError])或刷新行為(\insection[ftzmECB])的容差不會起作用。
那些值肯定產生規定的答案,不會是其他的。
使用 \math{\pm} 的地方,正負號會保留下來。
例如, \math{sin(\pm{0}) = \pm{0}} 的意思就是:
\math{sin(+0) = +0} 以及 \math{sin(-0) = -0}。

% func description
\startEnglishWithoutWhiteSpace
\setupnarrower[left=2em,right=0em]
\startnarrower[left,right]
\input{chapter_onc/misc/funcdesc.tex}
\stopnarrower
\stopEnglishWithoutWhiteSpace

% Changes to C99 TC2 Behavior
\subsection{對 C99 TC2 的行為做出的改變}

\capi{modf} 的作用如同如下實作:
\startclc[indentnext=no]
gentype modf(gentype value, gentype *iptr)
{
	*iptr = trunc(value);
	return copysign(isinf( value ) ? 0.0 : value – *iptr, value);
}
\stopclc
\capi{rint} 始終捨入為最近偶數,即使其調用者處於其他捨入模式下。

% Edge Case Behavior in Flush To Zero Mode
\subsection[section:ftzmECB]{Flush-To-Zero 模式中,邊界條件下的行為}

如果去規格化數被刷成了零,則函式可能返回下列結果之一:
\startigNum
\item[item:nftz]任何符合 non-flush-to-zero 模式的結果;

\item 如果捨入前,\initem[nftz]所給出的結果是次規格化數,則可能會將其刷成零;

\item[item:nfcr] 如果有一個或多個次規格化算元被刷成了零,則結果為符合此函式的任意未刷新的值。

\item 如果捨入前,\initem[nfcr]的結果是次規格化數,則可能將其刷成零。
\stopigNum

在上述任一情況中,如果某個算元或結果被刷成了零,零的正負號未定義。

如果次規格化數被刷成了零,
則\cnglo{device}可能採用下列方式來處理 \capi{nextafter} 的邊界情況,
以取代\insection[addReqBeyondC99]中的那些:

\startEnglishWithoutWhiteSpace
\setupnarrower[left=2em,right=0em]
\startnarrower[left,right]
\clapi{nextafter} ( +smallest normal, y < +smallest normal ) = +0.\par
\clapi{nextafter} ( -smallest normal, y > -smallest normal ) = -0.\par
\clapi{nextafter} ( -0, y > 0 ) returns smallest positive normal value.\par
\clapi{nextafter} ( +0, y < 0 ) returns smallest negative normal value.\par
\stopnarrower
\stopEnglishWithoutWhiteSpace

清晰起見,將次規格化數或去規格化數定義為位於區間 \ccmm{0 < x < TYPE_MIN}
和 \ccmm{-TYPE_MIN < x < -0} 內的一組可表示的數。
他們不包括 \math{\pm{0}}。
如果在捨入前,一個非零數規格化後,其以 2 為底的指數小於 \ccmm{(TYPE_MIN_EXP - 1)},
就說他是次規格化數。\footnote{%
此處要用相應浮點型別常量來替換 \cenum{TYPE_MIN} 和 \cenum{TYPE_MIN_EXP},
如對於 \ctype{float} 這兩個常量就應該是 \cenum{FLT_MIN} 和 \cenum{FLT_MIN_EXP}。}

