\startsubject[
  title={Problems},
]

%p28-1
\startPROBLEM[problem:tridiag_linear_equation]
(Tridiagonal systems of linear equations)
考察三對角矩陣:
\startformula
A=\left[\startmatrix
\NC 1 \NC -1\NC 0 \NC 0 \NC 0 \NR
\NC -1 \NC 2\NC -1 \NC 0 \NC 0 \NR
\NC 0 \NC -1\NC 2 \NC -1 \NC 0 \NR
\NC 0 \NC 0 \NC -1 \NC 2 \NC -1 \NR
\NC 0 \NC 0 \NC 0 \NC -1 \NC 2 \NR
\stopmatrix\right]
\stopformula

\startigBase[a]\startitem
求矩陣 \m{A} 的 LU 分解。
\stopitem\stopigBase

\startANSWER
\TODO{略。}
\stopANSWER

\startigBase[continue]\startitem
通過正向替換與反響替換求解方程 \m{Ax=\left(\startmatrix
\NC 1 \NC 1 \NC 1 \NC 1 \NC 1 \NR
\stopmatrix\right)^T}。
\stopitem\stopigBase

\startANSWER
\TODO{略。}
\stopANSWER

\startigBase[continue]\startitem
求 \m{A} 的逆矩陣。
\stopitem\stopigBase

\startANSWER
\TODO{略。}
\stopANSWER

\startigBase[continue]\startitem
請說明對任意的 \m{n\times n} 對稱正定三對角矩陣 \m{A} 和任意 \m{n} 維向量 \m{b},
如何通過運用一個 LU 分解可在 \m{O(n)} 時間內求解方程 \m{Ax=b}。
論證在最壞情況下,任何基於求 \m{A^{-1}} 的方法在漸進意義下要花費更多時間。
\stopitem\stopigBase

\startANSWER
\TODO{略。}
\stopANSWER

\startigBase[continue]\startitem
請說明對任意 \m{n\times n} 非奇異的三對角矩陣 \m{A} 和任意 \m{n} 維向量 \m{b},
如何運用一個 LUP 分解在 \m{O(n)} 時間內求解方程 \m{Ax=b}。
\stopitem\stopigBase

\startANSWER
\TODO{略。}
\stopANSWER
\stopPROBLEM

%28-2
\startPROBLEM
(Splines)
把一組點插值到一條曲線中的一種實用方法是採用{\EMP 三次樣條}(cubic spline)。
已知 \m{n+1} 個點值對組成的集合 \m{\{(x_i,y_i): i=0,1,\ldots,n\}},
其中 \m{x_0<x_1<\ldots<x_n}。
我們希望擬合出這些點的分段三次曲線(樣條) \m{f(x)}。
也就是說,曲線 \m{f(x)} 由 \m{n} 個三次多項式組成:
\startformula
f_i(x) = a_i + b_i x + c_i x^2 + d_i x^3
\stopformula
其中 \m{i = 0,1,\ldots,n-1},
如果 \m{x} 落在區間 \m{x_i\le x\le x_{i+1}} 中,
那麼曲線的值由 \m{f(x) = f_i(x-x_i)} 給出。
把三次多項是“粘和”在一起的點 \m{x_i} 稱爲{\EMP 結}(knot)。
簡單起見,假定對 \m{i=0,1,\ldots,n},有 \m{x_i = i}。

爲了保證 \m{f(x)} 的連續性,我們要求當 \m{i=0,1,\ldots,n-1} 時,
\startformula\startmathalignment
\NC f(x_i) \NC = f_i(0) = y_i \NR
\NC f(x_{i+1}) \NC = f_i(1) = y_{i+1} \NR
\stopmathalignment\stopformula
爲保證 \m{f(x)} 足夠光滑,我們還要求當 \m{i=0,1,\ldots,n-2} 時,
在每個結的一階導數是連續的:
\startformula
f'(x_{i+1}) = f'_i(1) = f'_{i+1}(0)
\stopformula

\startigBase[a]\startitem
假定當 \m{i=0,1,\ldots,n} 時,
我們不僅知道點值對 \m{\{(x_i,y_i)\}},
而且知道每個結的一階導數 \m{D_i = f'(x_i)}。
請用值 \m{y_i}、 \m{y_{i+1}}、 \m{D_i} 和 \m{D_{i+1}} 來
表示每個係數 \m{a_i}、 \m{b_i}、 \m{c_i} 和 \m{d_i}。
(記住 \m{x_i = i})
根據點值對和一階導數計算除 \m{4n} 個係數需要多少時間?
\stopitem\stopigBase

\startANSWER
\TODO{略。}
\stopANSWER

如何選擇 \m{f(x)} 在每個結的一階導數仍然是個問題。
一種方法是要求當 \m{i=0,1,\ldots,n-2} 時,
二階導數在每個結處連續:
\startformula
f''(x_{i+1}) = f''_i(1) = f''_{i+1}(0)
\stopformula
在第一個結和最後一個結,
假設 \m{f''(x_0) = f''(0) = 0} 以及 \m{f''(x_n) = f''_{n-1}(1) = 0};
這些假設使 \m{f(x)} 成爲一個{\EMP 自然}三次樣條。

\startigBase[continue]\startitem
利用二階導數的連續性限制,說明當 \m{i=1,2,\ldots,n-1} 時,
\startformula
D_{i-1} + 4D_i + D_{i+1} = 3(y_{i+1} - y_{i-1}) \eqno{28.21}
\stopformula
\stopitem\stopigBase

\startANSWER
\TODO{略。}
\stopANSWER

\startigBase[continue]\startitem
請說明:
\startformula\startmathalignment
\NC 2D_0 + D_1 \NC = 3(y_1 - y_0) \NR
\NC D_{n-1} + 2D_n \NC = 3(y_n - y_{n-1}) \NR
\stopmathalignment\stopformula
\stopitem\stopigBase

\startANSWER
\TODO{略。}
\stopANSWER

\startigBase[continue]\startitem
重寫等式(28.21)~(28.23)爲
包含未知量向量 \m{D=\langle D_0,D_1,\ldots,D_n\rangle} 的矩陣方程。
你所給的方程中矩陣具有什麼性質?
\stopitem\stopigBase

\startANSWER
\TODO{略。}
\stopANSWER

\startigBase[continue]\startitem
論證:運用自然三次樣條可以在 \m{O(n)} 時間內對一組 \m{n+l} 個點值對組成的集合對
進行插值(參見\refproblem{tridiag_linear_equation})。
\stopitem\stopigBase

\startANSWER
\TODO{略。}
\stopANSWER

\startigBase[continue]\startitem
請說明當 \m{x_i} 不一定等於 \m{i} 時,
如何確定出一個自然三次樣條對一組 \m{n+l} 個點 \m{(x_i,y_i)}
 (滿足 \m{x_0<x_1<\ldots<x_n})進行插值。
你必須求解什麼楊的矩陣方程?
你所給出的算法運行速度有多快?
\stopitem\stopigBase

\startANSWER
\TODO{略。}
\stopANSWER
\stopPROBLEM

\stopsubject%Problems
