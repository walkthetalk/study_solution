\startsection[
  title={Implementing pointers and objects},
]

%e10.3-1
\startEXERCISE
畫出序列 \m{\langle 13,4,8,19,5,11 \rangle} 在多維數列表示的雙向鏈表中的存儲方式。
同時畫出在單數列表示的雙向鏈表中的存儲方式。
\stopEXERCISE

\startANSWER
\externalfigure[output/e10_3_1-1]

\externalfigure[output/e10_3_1-2]
\stopANSWER

%e10.3-2
\startEXERCISE
爲同構對象的單數列表示法寫出過程 \ALGO{ALLOCATE-OBJECT} 和 \ALGO{FREE-OBJECT}。
\stopEXERCISE

\startANSWER
\CLRSH{ALLOCATE-OBJECT}
\startCLRS
if free = NIL
	error "out of space"
else
	x = free
	free = A[free + 1]
	return x
\stopCLRS

\CLRSH{FREE-OBJECT}
\startCLRS
A[x+1] = free
free = x
\stopCLRS
\stopANSWER

%e10.3-3
\startEXERCISE
爲什麼在 \ALGO{ALLOCATE-OBJECT} 和 \ALGO{FREE-OBJECT} 的實現中,
我們不需要設置或重置對象的 \m{prev}?
\stopEXERCISE

\startANSWER
因爲我們不使用 \m{prev}。
對空閒鏈表而言,單鏈就足夠了。
甚至連 \m{key} 都不需要。
\stopANSWER

%e10.3-4
\startEXERCISE[exercise:compact_list]
我們往往希望雙向鏈表的所有元素在存儲器中保持緊湊,
例如,在多數列表示法中佔用前 \m{m} 個下標位置。
(在頁式虛擬內存的計算環境下就是這樣的。)
說明如何實現 \ALGO{ALLOCATE-OBJECT} 和 \ALGO{FREE-OBJECT} 才能達到這種效果。
假設在鏈表外部沒有指針指向鏈表內元素。
(\hint 用棧的數列實現方式)
\stopEXERCISE

\startANSWER
我們可以在數列的起始位置開始分配元素。
當釋放元素時,除非是棧的最後一個,我們都需要將其後面的所有元素左移,
並更新所有指向他們的的下標。
需要線性時間。
\stopANSWER

%e10.3-5
\startEXERCISE
令 \m{L} 爲雙向鏈表,長度爲 \m{n},存儲在長度爲 \m{m} 的 \m{key}、 \m{prev} 和 \m{next} 三個數列中。
這些數列由 \ALGO{ALLOCATE-OBJECT} 和 \ALGO{FREE-OBJECT} 通過空閒雙向鏈表 \m{F} 進行管理,
假設 \m{L} 中有 \m{n} 個元素, \m{F} 中有 \m{m-n} 個元素。
寫出過程 \m{COMPACTIFY-LIST(L, F)},
參數爲鏈表 \m{L} 和空閒鏈表 \m{F},
移動 \m{L} 中的元素,使其佔用數列中前 \m{n} 個位置,
並調整空閒鏈表 \m{F},使其佔用數列中後 \m{m-n} 個位置。
此過程的運行時間應爲 \m{\Theta(n)},
且使用固定數量的額外存儲空間。
並證明其正確性。
\stopEXERCISE

\startEXERCISE
利用 \m{prev} 來存儲一個特定值來標記此元素屬於空閒鏈表。
\startigBase[n]
\item 從左至右掃描數列的前 \m{n} 個元素,找到第一個空閒元素;
\item 從頭至尾掃描 \m{L} 找到第一個不在數列中前 \m{n} 個位置中的元素;
\item 將這兩個元素互換;
\item 重複以上步驟,直至掃描完畢。
\stopigBase
\stopEXERCISE

\stopsection
