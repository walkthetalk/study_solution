\startcomponent cpnoptext
\product opencl-spec-zh

\chapter[chapter:optExt]{可選擴展}

%OpenCL 1.2 所支持的可選特性在《OpenCL 1.2 擴展規範》中有所描述。
本章列出了 OpenCL 1.2 所支持的可選特性。
一些 OpenCL \cnglo{device} 可能支持這些可選擴展。
對於符合 OpenCL 的實作而言,不要求支持這些可選擴展,
但仍然希望這些可選擴展能夠獲取廣泛的支持。
將來修訂 OpenCL 規範時,可能將這些可選擴展中所定義的功能移入必需的特性集中。
下面簡單描述了這些擴展的定義。

OpenCL 工作組所批准的 OpenCL 擴展均遵守寫列命名約定:
\startigBase
\item 每個擴展都有一個唯一的{\ftRef{名字字串}},形如“\clext{cl_khr_<名字>}”。
如果實作支持此擴展,
則此字串會出現在\reftab{cldevquery}中的 \cenum{CL_PLATFORM_EXTENSIONS} 或 \cenum{CL_DEVICE_EXTENSIONS} 中。

\item 對於擴展中所定義的所有 API 函式,其名字均形如:\clapi{cl<函式名>KHR}。

\item 對於擴展中所定義的所有枚舉,其名字均形如:\clapi{CL_<枚舉名>_KHR}。
\stopigBase

後續修訂 OpenCL 時, OpenCL 工作組所批准的 OpenCL 擴展都可能被{\ftRef{提升}}為必需的核心特性。
一旦如此,就會將相應的擴展規範合併到核心規範中。
而對於其中的函式和枚舉,則會移除其後綴 {\ftEmp{KHR}}。
對於相應的 OpenCL 實作而言,
仍然要在 \cenum{CL_PLATFORM_EXTENSIONS} 或 \cenum{CL_DEVICE_EXTENSIONS} 中導出此擴展名,
而且也要支持帶有後綴 {\ftEmp{KHR}} 的函式和枚舉,
以方便在不同的 OpenCL 版本間進行遷移。

對於供應商自定義的擴展,則要遵循下列命名約定:
\startigBase
\item 每個擴展都有一個唯一的{\ftRef{名字字串}},形如“\clext{cl_<供應商的名字>_<名字>}”。
如果實作支持此擴展,
則此字串會出現在\reftab{cldevquery}中的 \cenum{CL_PLATFORM_EXTENSIONS} 或 \cenum{CL_DEVICE_EXTENSIONS} 中。

\item 對於擴展中所定義的所有 API 函式,其名字均形如:\clapi{cl<函式名><供應商的名字>}。

\item 對於擴展中所定義的所有枚舉,其名字均形如:\clapi{CL_<枚舉名>_<供應商的名字>}。
\stopigBase

\section{可選擴展的編譯指示}

編譯指示 \cdrtemp{#pragma OPENCL EXTENSION} 控制着 OpenCL 編譯器關於擴展的行為。
此指示定義如下:
\startclc[indentnext=no]
#pragma OPENCL EXTENSION /BTEX\cref{extension_name}/ETEX : /BTEX\cref{behavior}/ETEX
#pragma OPENCL EXTENSION all : /BTEX\cref{behavior}/ETEX
\stopclc
其中 \cref{extension_name} 就是擴展的名字,
形如 \clext{cl_khr_<名字>}(OpenCL 工作組所批准的擴展)或者 \clext{cl_<供應商的名字>_<名字>}(供應商的擴展)。
而符記 \cemp{all} 則意味着在編譯器所支持的所有擴展上應用此行為。
可以將 \cref{behavior} 設置為\reftab{behaviorDsc}中的值。

\placetable[here][tab:behaviorDsc]
{可選擴展的行為}
{\startbuffer[enExtDsc]
其行為遵守擴展 \cref{extension_name} 的規定。

如果指定了 \cemp{all},或者不支持 \cref{extension_name},
則會在 \cdrtemp{#pragma OPENCL EXTENSION} 上報告錯誤。
\stopbuffer

\startbuffer[disExtDsc]
其行為(包括發出錯誤、警告)如同語言定義中沒有擴展 \cref{extension_name} 那樣。

如果指定了 \cemp{all},則其行為必須回退到所編譯語言的無擴展核心版本。

如果不支持 \cref{extension_name},
則會在 \cdrtemp{#pragma OPENCL EXTENSION} 上發出警告。
\stopbuffer

\startCLOD[behavior][描述]
\clOD{enable}{\getbuffer[enExtDsc]}
\clOD{disable}{\getbuffer[disExtDsc]}
\stopCLOD
}

就設定每個擴展的行為而言,
編譯指示 \cdrtemp{#pragma OPENCL EXTENSION} 是一種簡單、底層的機制。
但是他並沒有定義任何策略,比如哪種組合比較恰當;
這些必須在其他地方定義。
在設定每個擴展的行為時,指示的順序非常重要。
後出現的指示會覆蓋早出現的。
而變體 \cemp{all} 會設定所有擴展的行為,將覆蓋前面出現的所有擴展指示,
但 \cref{behavior} 只能設置為 \cemp{disable}。

編譯器的初始狀態相當於如下指示:
\startclc[indentnext=no]
#pragma OPENCL EXTENSION all : disable
\stopclc
告訴編譯器必須按照此規格來報告所有錯誤和警告,忽略所有擴展。

每個擴展,只要會影響 OpenCL 語言的語義、文法或者為語言增加了內建函式,
都必須創建一個預處理 \ccmm{#define} 來匹配擴展名。
當且僅當實作支持此擴展時,這個 \ckey{#define} 才可用。

例:

如果一個擴展增加了擴展字串“\clext{cl_khr_3d_image_writes}”,
那麼同時應當增加相應的預處理 \cemp{#define}。
現在\cnglo{kernel}就可以像這樣來使用這個預處理 \ccmm{#define}:
\startclc
#ifdef cl_khr_3d_image_writes
	// do something using the extension
#else
	// do something else or #error!
#endif
\stopclc

\section[section:getFuncPtr]{獲取 OpenCL API 擴展函式指位器}

\topclfunc{clGetExtensionFunctionAddressForPlatform}

\startCLFUNC
void* clGetExtensionFunctionAddressForPlatform (
			cl_platform_id platform,
			const char *funcname)
\stopCLFUNC

\startnotepar
由於沒有任何方式來限定對\cnglo{device}的查詢,
對於此擴展在這個\cnglo{platform}中不同\cnglo{device}上的所有實作而言,
所返回的函式指位器必須都能正常運作。
如果\cnglo{device}不支持此擴展,則調用擴展中的函式時,其行為\cnglo{undef}。
\stopnotepar

此函式會返回給定 \carg{platform} 的擴展函式 \carg{funcname} 的位址。
需要將所返回的指位器轉型成對應的函式指位器型別,
此擴展函式在對應的擴展規範和頭檔中定義。
如果返回的是 \cmacro{NULL},則表明實作中沒有所指定的函式,或者 \carg{platform} 無效。
即使返回的不是 \cmacro{NULL},也不保證 \carg{platform} 真正支持此擴展函式。
要想確定 OpenCL 實作是否支持某個擴展,
\cnglo{app}必須用下列兩種方式之一進行查詢:
\startclc
clGetPlatformInfo(platform, CL_PLATFORM_EXTENSIONS, ... )
clGetDeviceInfo(device, CL_DEVICE_EXTENSIONS, ... )
\stopclc

對於 OpenCL 的核心函式(非擴展函式),不能使用此函式進行查詢。
而對於那些能用此函式進行查詢的函式,
實作也可以選擇由實現那些函式的目標庫靜態導入那些函式。
然而,\cnglo{app}要想可移植,就不能依賴這種行為。

對於所有會增加 API 引入點的擴展,都必須聲明 \ccmm{typedef} 的函式指位器型別。
這些 \ccmm{typedef} 是擴展接口所必需的一部分,在相應頭檔中提供
(如果是 OpenCL 擴展,則為 \ccmm{cl_ext.h};
而如果是 OpenCL / OpenGL 共享擴展,則為 \ccmm{cl_gl_ext.h})。

所有會影響\cnglo{host} API 的擴展都必須遵循下列約定:
\startclc[indentnext=no]
#ifndef extension_name
#define extension_name		1

// all data typedefs, token #defines, prototypes, and
// function pointer typedefs for this extension

// function pointer typedefs must use the
// following naming convention
typedef CL_API_ENTRY /BTEX{\ftRef{return type}}/ETEX
		(CL_API_CALL */BTEX{\ftRef{clextension_func_nameTAG_fn}}/ETEX)(...);

#endif // extension_name
\stopclc
其中 \ccmm{TAG} 可以是 \ccmm{KHR}、 \ccmm{EXT} 或 \ccmm{vendor-specific}。

例如,擴展 \clext{cl_khr_gl_sharing} 在 \ccmm{cl_gl_ext.h} 中增加了下列代碼:
\startclc
#ifndef cl_khr_gl_sharing
#define cl_khr_gl_sharing	1

// all data typedefs, token #defines, prototypes, and
// function pointer typedefs for this extension
#define CL_INVALID_GL_SHAREGROUP_REFERENCE_KHR	-1000
#define CL_CURRENT_DEVICE_FOR_GL_CONTEXT_KHR	0x2006
#define CL_DEVICES_FOR_GL_CONTEXT_KHR		0x2007
#define CL_GL_CONTEXT_KHR			0x2008
#define CL_EGL_DISPLAY_KHR			0x2009
#define CL_GLX_DISPLAY_KHR			0x200A
#define CL_WGL_HDC_KHR				0x200B
#define CL_CGL_SHAREGROUP_KHR			0x200C

// function pointer typedefs must use the
// following naming convention
typedef CL_API_ENTRY cl_int
	(CL_API_CALL *clGetGLContextInfoKHR_fn)(
		const cl_context_properties * /* properties */,
		cl_gl_context_info /* param_name */,
		size_t /* param_value_size */,
		void * /* param_value */,
		size_t * /*param_value_size_ret*/);

#endif // cl_khr_gl_sharing
\stopclc

\section{64 位原子函式}

下列兩個可選擴展實現了 \cqlf{__global} 和 \cqlf{__local} 內存中的 64 位
帶符號和無符號整數上的原子運算:
\startigBase
\item \clext{cl_khr_int64_base_atomics}
\item \clext{cl_khr_int64_extended_atomics}
\stopigBase

\cnglo{app}中要想使用這些擴展,
必須在 OpenCL \cnglo{program}源碼中包含下列編譯指示之一:
\startclc
#pragma OPENCL EXTENSION cl_khr_int64_base_atomics : enable
#pragma OPENCL EXTENSION cl_khr_int64_extended_atomics : enable
\stopclc

\reftab{atomic64_base}中列出了擴展 \clext{cl_khr_int64_base_atomics} 所支持的原子函式,
其中所有函式都在一個原子事務內實施。

\placetable[here,split][tab:atomic64_base]
{擴展 \clext{cl_khr_int64_base_atomics} 的內建原子函式}
{% atomic_add
\startbuffer[funcproto:atomic64_add]
long atomic_add (
	volatile __global long *p,
	long val)
long atomic_add (
	volatile __local long *p,
	long val)

ulong atomic_add (
	volatile __global ulong *p,
	ulong val)
ulong atomic_add (
	volatile __local ulong *p,
	ulong val)
\stopbuffer
\startbuffer[funcdesc:atomic64_add]
讀取 \carg{p} 所指向的 64 位值(記為 \math{old})。
計算 \math{(old + \marg{val})} 並將結果存儲到 \carg{p} 所指位置中。
此函式返回 \math{old}。
\stopbuffer

% atomic_sub
\startbuffer[funcproto:atomic64_sub]
long atomic_sub (
	volatile __global long *p,
	long val)
long atomic_sub (
	volatile __local long *p,
	long val)

ulong atomic_sub (
	volatile __global ulong *p,
	ulong val)
ulong atomic_sub (
	volatile __local ulong *p,
	ulong val)
\stopbuffer
\startbuffer[funcdesc:atomic64_sub]
讀取 \carg{p} 所指向的 64 位值(記為 \math{old})。
計算 \math{(old - \marg{val})} 並將結果存儲到 \carg{p} 所指位置中。
此函式返回 \math{old}。
\stopbuffer

% atomic_xchg
\startbuffer[funcproto:atomic64_xchg]
long atomic_xchg (
	volatile __global long *p,
	long val)
long atomic_xchg (
	volatile __local long *p,
	long val)

ulong atomic_xchg (
	volatile __global ulong *p,
	ulong val)
ulong atomic_xchg (
	volatile __local ulong *p,
	ulong val)
\stopbuffer
\startbuffer[funcdesc:atomic64_xchg]
將位置 \carg{p} 中所存儲的值 \math{old} 和 \carg{val} 中的新值相互交換。
返回 \math{old}。
\stopbuffer

% atomic_inc
\startbuffer[funcproto:atomic64_inc]
long atomic_inc (volatile __global long *p)
long atomic_inc (volatile __local long *p)

ulong atomic_inc (
	volatile __global ulong *p)
ulong atomic_inc (
	volatile __local ulong *p)
\stopbuffer
\startbuffer[funcdesc:atomic64_inc]
讀取 \carg{p} 所指向的 64 位值(記為 \math{old})。
計算 \math{(old+1)} 並將結果存儲到 \carg{p} 所指位置中。
此函式返回 \math{old}。
\stopbuffer

% atomic_dec
\startbuffer[funcproto:atomic64_dec]
long atomic_dec (volatile __global long *p)
long atomic_dec (volatile __local long *p)

ulong atomic_dec (
	volatile __global ulong *p)
ulong atomic_dec (
	volatile __local ulong *p)
\stopbuffer
\startbuffer[funcdesc:atomic64_dec]
讀取 \carg{p} 所指向的 64 位值(記為 \math{old})。
計算 \math{(old-1)} 並將結果存儲到 \carg{p} 所指位置中。
此函式返回 \math{old}。
\stopbuffer

% atomic_cmpchg
\startbuffer[funcproto:atomic64_cmpxchg]
long atomic_cmpxchg (
	volatile __global long *p,
	long cmp, long val)
long atomic_cmpxchg (
	volatile __local long *p,
	long cmp,
	long val)

ulong atomic_cmpxchg (
	volatile __global ulong *p,
	ulong cmp,
	ulong val)
ulong atomic_cmpxchg (
	volatile __local ulong *p,
	ulong cmp,
	ulong val)
\stopbuffer
\startbuffer[funcdesc:atomic64_cmpxchg]
讀取 \carg{p} 所指向的 64 位值(記為 \math{old})。
計算 \math{(old == cmp) ? val : old} 並將結果存儲到 \carg{p} 所指位置中。
此函式返回 \math{old}。
\stopbuffer


% begin table
\startCLFD
\clFD{atomic64_add}
\clFD{atomic64_sub}
\clFD{atomic64_xchg}
\clFD{atomic64_inc}
\clFD{atomic64_dec}
\clFD{atomic64_cmpxchg}
\stopCLFD
}

\reftab{atomic64_ext}中列出了擴展 \clext{cl_khr_int64_extended_atomics} 所支持的原子函式,
其中所有函式都在一個原子事務內實施。

\placetable[here,split][tab:atomic64_ext]
{擴展 \clext{cl_khr_int64_extended_atomics} 的內建原子函式}
{% atomic_min
\startbuffer[funcproto:atomic64_min]
long atomic_min (
	volatile __global long *p,
	long val)
long atomic_min (
	volatile __local long *p,
	long val)

ulong atomic_min (
	volatile __global ulong *p,
	ulong val)
ulong atomic_min (
	volatile __local ulong *p,
	ulong val)
\stopbuffer
\startbuffer[funcdesc:atomic64_min]
讀取 \carg{p} 所指向的 64 位值(記為 \math{old})。
計算 \math{\mapiemp{min}(old, \marg{val})} 並將結果存儲到 \carg{p} 所指位置中。
此函式返回 \math{old}。
\stopbuffer

% atomic_max
\startbuffer[funcproto:atomic64_max]
long atomic_max (
	volatile __global long *p,
	long val)
long atomic_max (
	volatile __local long *p,
	long val)

ulong atomic_max (
	volatile __global ulong *p,
	ulong val)
ulong atomic_max (
	volatile __local ulong *p,
	ulong val)
\stopbuffer
\startbuffer[funcdesc:atomic64_max]
讀取 \carg{p} 所指向的 64 位值(記為 \math{old})。
計算 \math{\mapiemp{max}(old, \marg{val})} 並將結果存儲到 \carg{p} 所指位置中。
此函式返回 \math{old}。
\stopbuffer

% atomic_and
\startbuffer[funcproto:atomic64_and]
long atomic_and (
	volatile __global long *p,
	long val)
long atomic_and (
	volatile __local long *p,
	long val)

ulong atomic_and (
	volatile __global ulong *p,
	ulong val)
ulong atomic_and (
	volatile __local ulong *p,
	ulong val)
\stopbuffer
\startbuffer[funcdesc:atomic64_and]
讀取 \carg{p} 所指向的 64 位值(記為 \math{old})。
計算 \math{(old \mcmm{&} \marg{val})} 並將結果存儲到 \carg{p} 所指位置中。
此函式返回 \math{old}。
\stopbuffer

% atomic_or
\startbuffer[funcproto:atomic64_or]
long atomic_or (
	volatile __global long *p,
	long val)
long atomic_or (
	volatile __local long *p,
	long val)

ulong atomic_or (
	volatile __global ulong *p,
	ulong val)
ulong atomic_or (
	volatile __local ulong *p,
	ulong val)
\stopbuffer
\startbuffer[funcdesc:atomic64_or]
讀取 \carg{p} 所指向的 64 位值(記為 \math{old})。
計算 \math{(old \mcmm{|} \marg{val})} 並將結果存儲到 \carg{p} 所指位置中。
此函式返回 \math{old}。
\stopbuffer

% atomic_xor
\startbuffer[funcproto:atomic64_xor]
long atomic_xor (
	volatile __global long *p,
	long val)
long atomic_xor (
	volatile __local long *p,
	long val)

ulong atomic_xor (
	volatile __global ulong *p,
	ulong val)
ulong atomic_xor (
	volatile __local ulong *p,
	ulong val)
\stopbuffer
\startbuffer[funcdesc:atomic64_xor]
讀取 \carg{p} 所指向的 64 位值(記為 \math{old})。
計算 \math{(old \mcmm{^} \marg{val})} 並將結果存儲到 \carg{p} 所指位置中。
此函式返回 \math{old}。
\stopbuffer


% begin table
\startCLFD
\clFD{atomic64_min}
\clFD{atomic64_max}
\clFD{atomic64_and}
\clFD{atomic64_or}
\clFD{atomic64_xor}
\stopCLFD
}

對於執行這些原子函式的\cnglo{device}而言,這些事務都是原子的。
而對於在多個\cnglo{device}上執行的\cnglo{kernel}而言,
如果這些原子運算是在同一內存位置上實施的,則不保證其原子性。

\startnotepar
64 位整數和 32 位整數(包括 \ctype{float})上的原子運算相互之間也是原子的。
\stopnotepar

\section{寫入 3D 圖像對象}

OpenCL 支持\cnglo{kernel}讀寫 2D \cnglo{imgobj}。
同一\cnglo{kernel}不能對同一 2D \cnglo{imgobj}既讀又寫。
OpenCL 也支持\cnglo{kernel}讀取 3D \cnglo{imgobj},
但是不支持寫入 3D \cnglo{imgobj},除非實現了擴展 \clext{cl_khr_3d_image_writes}。
同一\cnglo{kernel}不能對同一 3D \cnglo{imgobj}既讀又寫。

\cnglo{app}要想使用此擴展寫入 3D \cnglo{imgobj},
需要在 OpenCL \cnglo{program}源碼中包含下列編譯指示:
\startclc
#pragma OPENCL EXTENSION cl_khr_3d_image_writes : enable
\stopclc

\reftab{write_3d_image}中列出了擴展 \clext{cl_khr_3d_image_writes} 所實現的內建函式。

\placetable[here][tab:write_3d_image]
{擴展 \clext{cl_khr_3d_image_writes} 所實現的內建函式}
{% atomic_add
\startbuffer[funcproto:write_image_3d]
void write_imagef (image3d_t image,
		int4 coord,
		float4 color)

void write_imagei (image3d_t image,
		int4 coord,
		int4 color)

void write_imageui (image3d_t image,
		int4 coord,
		uint4 color)
\stopbuffer
\startbuffer[funcdesc:write_image_3d]
將 \carg{color} 的值寫入 3D \cnglo{imgobj} \carg{image}中坐標 \math{(x,y,z)} 處。
寫入前會對顏色值進行恰當的數據格式轉換。
會將 \carg{coord.x}、 \carg{coord.y} 和 \carg{coord.z} 視為非歸一化坐標,
且其值必須分別位於區間 \math{0\cdots\mvar{圖像寬度}-1}、 \math{0\cdots\mvar{圖像高度}-1} 和 \math{0\cdots\mvar{圖像深度}-1} 內。

對於 \capi{write_imagef},
創建\cnglo{imgobj}時所用的 \carg{image_channel_data_type} 必須是預定義壓縮過的格式
或者 \cenum{CL_SNORM_INT8}、 \cenum{CL_UNORM_INT8}、 \cenum{CL_SNORM_INT16}、
 \cenum{CL_UNORM_INT16}、 \cenum{CL_HALF_FLOAT} 或 \cenum{CL_FLOAT}。
會將通道數據由浮點值轉換成存儲數據所用的實際數據格式。

對於 \capi{write_imagei} 而言,
創建\cnglo{imgobj}時所用的 \carg{image_channel_data_type} 必須是下列值之一:
\startigBase
\item \cenum{CL_SIGNED_INT8}
\item \cenum{CL_SIGNED_INT16}
\item \cenum{CL_SIGNED_INT32}
\stopigBase

對於 \capi{write_imageui} 而言,
創建\cnglo{imgobj}時所用的 \carg{image_channel_data_type} 必須是下列值之一:
\startigBase
\item \cenum{CL_UNSIGNED_INT8}
\item \cenum{CL_UNSIGNED_INT16}
\item \cenum{CL_UNSIGNED_INT32}
\stopigBase

如果創建\cnglo{imgobj}時所用的 \carg{image_channel_data_type} 不再上述所列範圍內,
或者坐標 \math{(x, y, z)} 不在 \math{(0 \cdots \mvar{圖像寬度}-1, 0 \cdots \mvar{圖像高度}-1, 0 \cdots \mvar{圖像深度}-1)} 範圍內,
則 \capi{write_imagef}、 \capi{write_imagei} 和 \capi{write_imageui} 的行為\cnglo{undef}。
\stopbuffer

% begin table
\startCLFD
\clFD{write_image_3d}
\stopCLFD
}

\section{半精度浮點數}

此擴展增加了對 \cldt{half} 標量和矢量型別的支持,
可以 \cldt{half} 作為內建型別進行算術運算、轉換等。
\cnglo{app}要想使用型別 \cldt{half} 和 \cldt[n]{half},
必須包含編譯指示 \cemp{#pragma OPENCL EXTENSION cl_khr_fp16 : enable}。

\reftab{builtInScalarDataTypes}和\reftab{builtInVectorDataTypes}中所列內建標量、矢量數據型別又做了如下擴充:

\placetable[here][tab:half_type_dsc]
{\cldt{half} 相關數據型別}
{\startCLOD[型別][描述]

\clOD{\cldt{half2}}{2 組件半精度浮點矢量。}

\clOD{\cldt{half3}}{3 組件半精度浮點矢量。}

\clOD{\cldt{half4}}{4 組件半精度浮點矢量。}

\clOD{\cldt{half8}}{8 組件半精度浮點矢量。}

\clOD{\cldt{half16}}{16 組件半精度浮點矢量。}

\stopCLOD
}

在 OpenCL API(以及頭檔)中,內建矢量數據型別 \cldt[n]{half} 被聲明為其他型別,
以更好的為\cnglo{app}所用。
\reftab{bihalf2appdt}中列出了 OpenCL C 編程語言中
所定義的內建矢量數據型別 \cldt[n]{half} 與\cnglo{app}所用型別間的對應關係。

\placetable[here][tab:bihalf2appdt]
{內建矢量數據型別與應用程式所用型別的對應關係}
{\startCLOO[OpenCL 語言中的型別][\cnglo{app}所用 API 中的型別]

\clOO{\cldt{half2}}{\cldt{cl_half2}}
\clOO{\cldt{half3}}{\cldt{cl_half3}}
\clOO{\cldt{half4}}{\cldt{cl_half4}}
\clOO{\cldt{half8}}{\cldt{cl_half8}}
\clOO{\cldt{half16}}{\cldt{cl_half16}}

\stopCLOO


}

\refsection{operator}中所描述的關係、相等、邏輯以及邏輯單元算子
均可用於 \cldt{half} 標量和 \cldt[n]{half} 矢量型別,
所產生的結果分別為標量 \cldt{int} 和矢量 \cldt[n]{short}。

可以為浮點常值添加後綴 \ccmm{h} 或 \ccmm{H},
以表明此常值型別為 \cldt{half}。

\subsection{轉換}

現在,\refsection{implicityConversion}中的隱式轉換規則也適用於 \cldt{half} 標量和 \cldt[n]{half} 矢量數據型別。

\refsection{explicitCast}中的顯式轉型也做了擴充,
適用於 \cldt{half} 標量數據型別和 \cldt[n]{half} 矢量數據型別。

\refsection{explicitConversion}中所描述的顯式轉換函式也做了擴充,
適用於 \cldt{half} 標量數據型別和 \cldt[n]{half} 矢量數據型別。

\refsection{as_typen}中所描述的用於重釋型別的函式 \clapi[n]{as_type} 也做了擴充,
允許在 \cldt[n]{short}、 \cldt[n]{ushort} 和 \cldt[n]{half} 標量、矢量數據型別間進行無需轉換的轉型。

\subsection{數學函式}

對\reftab{svMathFunc}中所列內建數學函式作了擴充,
函式引數和返回值也可以是 \cldt{half} 和 \cldt[n]{half},
參見\reftab{svMathFuncHalf}。
現在, \cldt{gentype} 也包含 \cldt{half} 和 \cldt[n]{half},
其中 \ccmmsuffix{n} 可以是 2、 3、 4、 8、 16。

對於函式的任一特定用法,所有引數以及返回值的實際型別必須相同。

\placetable[here,split][tab:svMathFuncHalf]
{標量和矢量引數內建數學函式表}
{\startCLFD

\clFD{acos}
\clFD{acosh}
\clFD{acospi}
\clFD{asin}
\clFD{asinh}
\clFD{asinpi}
\clFD{atan}
\clFD{atan2}
\clFD{atanh}
\clFD{atanpi}
\clFD{atan2pi}
\clFD{cbrt}
\clFD{ceil}
\clFD{copysign}
\clFD{cos}
\clFD{cosh}
\clFD{cospi}
\clFD{erfc}
\clFD{erf}
\clFD{exp}
\clFD{exp2}
\clFD{exp10}
\clFD{expm1}
\clFD{fabs}
\clFD{fdim}
\clFD{floor}
\clFD{fma}
\clFD{fmaxh}
\clFD{fminh}
\clFD{fmod}
\clFD{fract}
\clFD{frexph}
\clFD{hypot}
\clFD{ilogbh}
\clFD{ldexph}
\clFD{lgammah}
\clFD{log}
\clFD{log2}
\clFD{log10}
\clFD{log1p}
\clFD{logb}
\clFD{mad}
\clFD{maxmag}
\clFD{minmag}
\clFD{modf}
\clFD{nanh}
\clFD{nextafter}
\clFD{pow}
\clFD{pownh}
\clFD{powr}
\clFD{remainder}
\clFD{remquoh}
\clFD{rint}
\clFD{rootnh}
\clFD{rsqrt}
\clFD{sin}
\clFD{sincos}
\clFD{sinh}
\clFD{sinpi}
\clFD{sqrt}
\clFD{tan}
\clFD{tanh}
\clFD{tanpi}
\clFD{tgamma}
\clFD{trunc}

\stopCLFD
}

巨集 \cmacroemp{FP_FAST_FMAF} 用來指明對於半精度浮點數,
 \capi{fma} 函式族是否比直接編碼更快。
如果定義了此巨集,則表明對算元為 \ctype{float} 的乘、加運算,
函式 \capi{fma} 一般跟直接編碼一樣快,或者更快。

下列巨集必須使用指定的值。
可以在預處理指示 \ccmm{#if} 中使用這些常量算式。
\startclc
#define HALF_DIG		3
#define HALF_MANT_DIG		11
#define HALF_MAX_10_EXP		+4
#define HALF_MAX_EXP		+16
#define HALF_MIN_10_EXP		-4
#define HALF_MIN_EXP		-13
#define HALF_RADIX		2
#define HALF_MAX		0x1.ffcp15h
#define HALF_MIN		0x1.0p-14h
#define HALF_EPSILON		0x1.0p-10f
\stopclc

\reftab{tblHalfMacroAndApp}中給出了上面所列巨集與\cnglo{app}所用的巨集名字之間的對應關係。

\placetable[here][tab:tblHalfMacroAndApp]
{半精度浮點巨集與應用程式所用巨集的對應關係}
{\startCLOO[OpenCL 語言中的巨集][\cnglo{app}所用的巨集]

\clMMH{DIG}
\clMMH{MANT_DIG}
\clMMH{MAX_10_EXP}
\clMMH{MAX_EXP}
\clMMH{MIN_10_EXP}
\clMMH{MIN_EXP}
\clMMH{RADIX}
\clMMH{MAX}
\clMMH{MIN}
\clMMH{EPSILSON}

\stopCLOO
}

除此之外,還有一些常量可用,如\reftab{tblHalfMacroConst}所示。
他們的型別都是 \ctype{float},在 \ctype{float} 型別的精度內是準確的。

\placetable[here][tab:tblHalfMacroConst]
{半精度浮點常量}
{\startCLOO[常量][描述]

\clCM{M_E_H}{e}
\clCM{M_LOG2E_H}{log_{2}e}
\clCM{M_LOG10E_H}{log_{10}e}
\clCM{M_LN2_H}{log_{e}2}
\clCM{M_LN10_H}{log_{e}10}
\clCM{M_PI_H}{\pi}
\clCM{M_PI_2_H}{\pi/2}
\clCM{M_PI_4_H}{\pi/4}
\clCM{M_1_PI_H}{1/\pi}
\clCM{M_2_PI_H}{2/\pi}
\clCM{M_2_SQRTPI_H}{2/\sqrt{\pi}}
\clCM{M_SQRT2_H}{\sqrt{2}}
\clCM{M_SQRT1_2_H}{1/\sqrt{2}}

\stopCLOO
}

\subsection{公共函式}

對\reftab{svCommonFunc}中所列的內建公共函式作了擴充,
函式引數和返回值也可以是 \cldt{half} 和 \cldt[n]{half},
參見\reftab{svCommonFuncHalf}。
現在, \cldt{gentype} 也包含 \cldt{half} 和 \cldt[n]{half},
其中 \ccmmsuffix{n} 可以是 2、 3、 4、 8、 16。

\startnotepar
可以使用化簡(如 \capi{mad} 或 \capi{fma})來實現 \capi{mix} 和 \capi{smoothstep}。
\stopnotepar

\placetable[here][tab:svCommonFuncHalf]
{內建公共函式}
{\startCLFD

\clFD{clamp_half}
\clFD{degrees}
\clFD{max_half}
\clFD{min_half}
\clFD{mix_half}
\clFD{radians}
\clFD{step_half}
\clFD{smoothstep_half}
\clFD{sign}

\stopCLFD
}

\subsection[section:geomtricFunc]{幾何函式}

對\reftab{svGeometricFunc}中所列的內建幾何函式作了擴充,
函式引數和返回值也可以是 \cldt{half} 和 \cldt[n]{half},
參見\reftab{svGeometricFuncHalf}。
現在, \cldt{gentype} 也包含 \cldt{half} 和 \cldt[n]{half},
其中 \ccmmsuffix{n} 可以是 2、 3、 4。

\startnotepar
可以使用化簡(如 \capi{mad} 或 \capi{fma})來實現幾何函式。
\stopnotepar

\placetable[here][tab:svGeometricFuncHalf]
{內建幾何函式}
{\startCLFD

\clFD{cross_half}
\clFD{dot_half}
\clFD{distance_half}
\clFD{length_half}
\clFD{normalize_half}

\stopCLFD
}

\subsection[section:relationFunc]{關係函式}

對\reftab{svRelationalFunc}中所列內建關係函式作了擴充,
引數可以為 \cldt{half} 和 \cldt[n]{half},
其中 \ccmmsuffix{n} 可以是 2、 3、 4、 8、 16,
參見\reftab{relationalFuncHalf}。

關係算子和相等算子(<、 <=、 >、 >=、 !=、 ==)也可用於矢量型別 \cldt[n]{half},
所產生的結果為 \cldt[n]{short},參見\insection[operator]。

對於標量型別的引數,如果所指定的關係為 {\ftRef{false}},則下列函式(參見\reftab{svRelationalFunc})會返回 0,否則返回 1:
\startigBase[indentnext=no]
\item \capi{isequal}、 \capi{isnotequal}、
\item \capi{isgreater}、 \capi{isgreaterequal}、
\item \capi{isless}、 \capi{islessequal}、
\item \capi{islessgreater}、
\item \capi{isfinite}、 \capi{isinf}、
\item \capi{isnan}、 \capi{isnormal}、
\item \capi{isordered}、 \capi{isunordered} 和
\item \capi{signbit}。
\stopigBase
而對於矢量型別的引數,如果所指定的關係為 {\ftRef{false}},則返回 0,
否則返回 -1 (即所有位都是 1)。

如果任一引數為 NaN,則下列關係函式返回 0:
\startigBase[indentnext=no]
\item \capi{isequal}、
\item \capi{isgreater}、 \capi{isgreaterequal}、
\item \capi{isless}、 \capi{islessequal} 和
\item \capi{islessgreater}。
\stopigBase
如果引數為標量,則當任一引數為 NaN 時, \capi{isnotequal} 返回 1;
而如果引數為矢量,則當任一引數為 NaN 時, \capi{isnotequal} 返回 -1。

\placetable[here,split][tab:relationalFuncHalf]
{內建關係函式}
{\startCLFD

\clFD{isequal_half}
\clFD{isnotequal_half}
\clFD{isgreater_half}
\clFD{isgreaterequal_half}
\clFD{isless_half}
\clFD{islessequal_half}
\clFD{islessgreater_half}
\clFD{isfinite_half}
\clFD{isinf_half}
\clFD{isnan_half}
\clFD{isnormal_half}
\clFD{isordered_half}
\clFD{isunordered_half}
\clFD{signbit_half}
\clFD{bitselect_half}
\clFD{select_half}

\stopCLFD
}

\subsection[section:vectorLsFuncHalf]{矢量數據裝載和存儲函式}

對\reftab{vectorLsFunc}中所列的矢量數據裝載(\clapi[n]{vload})和存儲(\clapi[n]{vstore})函式作了擴充,
可以讀寫 \cldt{half} 標量和矢量值,
參見\reftab{vectorLsFuncHalf}。

對泛型 \cldt{gentype} 也作了擴充,包含 \cldt{half}。
而對泛型 \cldt[n]{gentype} 也作了擴充,包含了 \cldt[n]{half},
其中 \ccmmsuffix{n} 為 2、 3、 4、 8 或 16。

\startnotepar
\capi{vload3} 和 \capi{vstore3}
均由位址 \math{(\marg{p} + (\marg{offset}\times 3))} 讀寫矢量組件 \ccmm{x}、 \ccmm{y}、 \ccmm{z}。
\stopnotepar

\placetable[here,split][tab:vectorLsFuncHalf]
{矢量數據裝載、存儲函式表}
{\startCLFD
\clFD{vloadn}
\clFD{vstoren}
\stopCLFD
}

\subsection[section:asyncCopyPrefetch]{在全局內存和局部內存間的異步拷貝以及預取}

OpenCL C 編程語言實現了\reftab{asyncCopyPrefetch}中所列函式,
可在\cnglo{glbmem}和\cnglo{locmem}間進行異步拷貝,
以及從\cnglo{glbmem}中預取(prefetch)。

對泛型 \ctype{gentype} 作了擴充,
包含 \cldt{half} 和 \cldt[n]{half},
其中 \ccmmsuffix{n} 可以是 2、 3、 4、 8、 16。

\placetable[here,split][tab:asyncCopyPrefetch]
{內建異步拷貝和預取函式}
{\startCLFD
\clFD{async_work_group_copy}
\clFD{async_work_group_strided_copy}
\clFD{wait_group_events}
\clFD{prefetch}
\stopCLFD
}\


% Image Read and Write Functions
\subsection[section:imgRwFunc]{圖像讀寫函式}

對\reftab{imgReadFunc}、\reftab{imgReadWithoutSamplerFunc}
和\reftab{imgWriteFunc}中所列的圖像讀寫函式做了擴充,
以支持型別為 \cldt{half} 的圖像顏色值。

\placetable[here,split][tab:imgReadFuncHalf]
{內建圖像讀取函式}
{\startCLFD
\clFD{read_imageh_2d}
\clFD{read_imageh_3d}
\clFD{read_imageh_2da}
\clFD{read_imageh_1d}
\clFD{read_imageh_1da}
\stopCLFD

}

\placetable[here,split][tab:imgReadWithoutSamplerFuncHalf]
{內建無採樣器圖像讀取函式}
{\startCLFD
\clFD{read_imageh_2d_s}
\clFD{read_imageh_3d_s}
\clFD{read_imageh_2da_s}
\clFD{read_imageh_1d_s}
\clFD{read_imageh_1da_s}
\stopCLFD

}

\placetable[here,split][tab:imgWriteFuncHalf]
{內建圖像寫入函式}
{\startCLFD
\clFD{write_imageh_2d}
\clFD{write_imageh_2da}
\clFD{write_imageh_1d}
\clFD{write_imageh_1da}
\clFD{write_imageh_3d}
\stopCLFD

}


\subsection{IEEE754 符合性}

\reftab{half_query}中的表項作為對\reftab{cldevquery}的補充,
對於支持半精度浮點數的 OpenCL \cnglo{device},
\cnglo{app}可以用 \clapi{clGetDeviceInfo} 來查詢其組態資訊。

\placetable[here][tab:half_query]
{\cldt{half} 相關查詢}
{\startETD[cl_device_info][返回型別]

\clETD{CL_DEVICE_HALF_FP_CONFIG}{cl_device_fp_config}{
描述 OpenCL \cnglo{device}的半精度浮點能力。
此位欄包含下列值:
\startigBase
\item \cenum{CL_FP_DENORM},支持去規格化數。
\item \cenum{CL_FP_INF_NAN},支持 INF 和 NaN。
\item \cenum{CL_FP_ROUND_TO_NEAREST},支持捨入為最近偶數。
\item \cenum{CL_FP_ROUND_TO_ZERO},支持向零捨入。
\item \cenum{CL_FP_ROUND_TO_INF},支持向正負無窮捨入。
\item \cenum{CL_FP_FMA},支持 IEEE754-2008 中的積和熔加運算。
\item \cenum{CL_FP_SOFT_FLOAT},軟件實現了基本的浮點運算(像加、減、乘)。
\stopigBase

此擴展要求半精度浮點能力至少為 \cenum{CL_FP_INF_NAN}
 以及 \cenum{CL_FP_ROUND_TO_ZERO} 或 \cenum{CL_FP_ROUND_TO_NEAREST}。
}

\stopETD

}


% Relative Error as ULPs
\subsection[section:relativeErrorHalf]{相對誤差即 ULP}

本節中,我們將討論相對誤差(定義為 \ccmm{ulp},即 units in the last place,
浮點數間的最小間隔)的最大值。
對於半精度浮點數,
如果支持 \cenum{CL_FP_ROUND_TO_NEAREST},則缺省捨入模式為捨入為最近偶數;
否則缺省捨入模式為向零捨入。
而對於半精度浮點運算,如加、減、乘、積和熔加,則要求用缺省捨入模式正確捨入。

轉換為半精度浮點格式時,
如果指定了捨入模式 \ccmm{convert_},則用此模式進行捨入;
否則用缺省捨入模式進行捨入,或者 C 風格的轉型。

而將 \cldt{half} 轉換為整數格式時,
如果指定了捨入模式 \ccmm{convert_},則用此模式進行捨入;
否則向零捨入,或者使用 C 風格的轉型。

由 \cldt{half} 轉換為浮點格式時都是無損的。

\reftab{hpMathUlp}描述的是半精度浮點算術運算的最小精度,以 ULP 為單位。
計算 ULP 值時所參考的是無限精確的結果。
其中 0 ulp 表示相應函式無需捨入。

\placetable[here,split][tab:hpMathUlp]
{半精度內建數學函式的 ULP 值}
{\startCLFA[函式][最小精度—— ULP 值]

\clFAM{x+y}{正確捨入}
\clFAM{x-y}{正確捨入}
\clFAM{x*y}{正確捨入}
\clFAM{1.0/y}{正確捨入}
\clFAM{x/y}{正確捨入}

\clFAA{acos}{<= 2 ulp}
\clFAA{acospi}{<= 2 ulp}
\clFAA{asin}{<= 2 ulp}
\clFAA{asinpi}{<= 2 ulp}
\clFAA{atan}{<= 2 ulp}
\clFAA{atan2}{<= 2 ulp}
\clFAA{atanpi}{<= 2 ulp}
\clFAA{atan2pi}{<= 2 ulp}
\clFAA{acosh}{<= 2 ulp}
\clFAA{asinh}{<= 2 ulp}
\clFAA{atanh}{<= 2 ulp}
\clFAA{cbrt}{<= 2 ulp}
\clFAA{ceil}{正確捨入}
\clFAA{copysign}{0 ulp}
\clFAA{cos}{<= 2 ulp}
\clFAA{cosh}{<= 2 ulp}
\clFAA{cospi}{<= 2 ulp}
\clFAA{erfc}{<= 4 ulp}
\clFAA{erf}{<= 4 ulp}
\clFAA{exp}{<= 2 ulp}
\clFAA{exp2}{<= 2 ulp}
\clFAA{exp10}{<= 2 ulp}
\clFAA{expm1}{<= 2 ulp}
\clFAA{fabs}{0 ulp}
\clFAA{fdim}{正確捨入}
\clFAA{floor}{正確捨入}
\clFAA{fma}{正確捨入}
\clFAA{fmax}{0 ulp}
\clFAA{fmin}{0 ulp}
\clFAA{fmod}{0 ulp}
\clFAA{fract}{正確捨入}
\clFAA{frexp}{0 ulp}
\clFAA{hypot}{<= 2 ulp}
\clFAA{ilogb}{0 ulp}
\clFAA{ldexp}{正確捨入}
\clFAA{log}{<= 2 ulp}
\clFAA{log2}{<= 2 ulp}
\clFAA{log10}{<= 2 ulp}
\clFAA{log1p}{<= 2 ulp}
\clFAA{logb}{0 ulp}
\clFAA{mad}{所允許的任何值(無窮 ulp)}
\clFAA{maxmag}{0 ulp}
\clFAA{minmag}{0 ulp}
\clFAA{modf}{0 ulp}
\clFAA{nan}{0 ulp}
\clFAA{nextafter}{0 ulp}
\clFAA{pow(x, y)}{<= 4 ulp}
\clFAA{pown(x, y)}{<= 4 ulp}
\clFAA{powr(x, y)}{<= 4 ulp}
\clFAA{remainder}{0 ulp}
\clFAA{remquo}{0 ulp}
\clFAA{rint}{正確捨入}
\clFAA{rootn}{<= 4 ulp}
\clFAA{round}{正確捨入}
\clFAA{rsqrt}{<= 1 ulp}
\clFAA{sin}{<= 2 ulp}
\clFAA{sincos}{正弦值和餘弦值都是 <= 2 ulp}
\clFAA{sinh}{<= 2 ulp}
\clFAA{sinpi}{<= 2 ulp}
\clFAA{sqrt}{正確捨入}
\clFAA{tan}{<= 2 ulp}
\clFAA{tanh}{<= 2 ulp}
\clFAA{tanpi}{<= 2 ulp}
\clFAA{tgamma}{<= 4 ulp}
\clFAA{trunc}{正確捨入}

\stopCLFA

}

\startnotepar
在 \cldt{half} 標量或矢量數據型別上實施運算時,
實作可能將 \cldt{half} 值轉換為 \cldt{float} 值,
並在 \cldt{float} 值上實施運算。
這種情況下,實作僅將 \cldt{half} 標量或矢量數據型別作為存儲格式。
\stopnotepar




\section{由 GL 上下文或共享組創建 CL 上下文}

% Overview
\subsection{概覽}

\refsection{clShareGl}中定義了如何與 OpenGL 實作中的材質和緩衝對象共享數據,
但對於如何在 OpenCL \cnglo{context}和 OpenGL \cnglo{context}或共享組間建立聯繫,
則沒有定義。
此擴展為創建 OpenCL \cnglo{context}的例程定義了一個可選特性,
可以將 GL \cnglo{context}或共享組對象與新建的 OpenCL \cnglo{context}關聯起來。
如果實作支持此擴展,則\reftab{cldevquery}中所描述的 \cenum{CL_PLATFORM_EXTENSIONS} 或
\cenum{CL_DEVICE_EXTENSIONS} 中將包含字串 \clext{cl_khr_gl_sharing}。

此擴展要求 OpenGL 實作支持\cnglo{bufobj},以及與 OpenCL 共享材質和\cnglo{bufobj}圖像。

% New Procedures and Functions
\subsection{新例程和新函式}

\startclc
cl_intclGetGLContextInfoKHR (
		const cl_context_properties *properties,
		cl_gl_context_info param_name,
		size_t param_value_size,
		void *param_value,
		size_t *param_value_size_ret);
\stopclc

% New Tokens
\subsection{新的符記}

如果 \carg{properties} 中所指定的 OpenGL \cnglo{context}或共享組對象句柄無效,
則 \clapi{clCreateContext}、 \clapi{clCreateContextFromType} 和 \clapi{clGetGLContextInfoKHR} 會返回:
\startclc
CL_INVALID_GL_SHAREGROUP_REFERENCE_KHR		-1000
\stopclc

\clapi{clGetGLContextInfoKHR} 的引數 \carg{param_name} 接受下列值:
\startclc
CL_CURRENT_DEVICE_FOR_GL_CONTEXT_KHR		0x2006
CL_DEVICES_FOR_GL_CONTEXT_KHR			0x2007
\stopclc

\clapi{clCreateContext} 和 \clapi{clCreateContextFromType} 的引數 \carg{properties} 接受下列特性名:
\startclc
CL_GL_CONTEXT_KHR	0x2008
CL_EGL_DISPLAY_KHR	0x2009
CL_GLX_DISPLAY_KHR	0x200A
CL_WGL_HDC_KHR		0x200B
CL_CGL_SHAREGROUP_KHR	0x200C
\stopclc

%  Additions to Chapter 4 of the OpenCL 1.2 Specification
\subsection{對第 4 章的補充}

\refsection{contexts}中,用下列內容取代 \clapi{clCreateContext} 後面對 \carg{properties} 的描述:

\carg{properties} 指向一個特性列,即 \ccmm{<特性名, 值>} 的陣列,此陣列已排好序,以零終止。
如果此陣列中沒有某個特性,則使用其缺省值,參見\reftab{prptForclCreateContext}。
如果 \carg{properties} 是 \cmacro{NULL} 或者是空的(第一個值就是零),所有特性都使用缺省值。

\refsection{clShareGl}中定義了一些特性,
用來控制如何與 OpenGL 緩衝、材質和渲染緩衝對象共享 OpenCL \cnglo{memobj}。

可以設置下列特性來識別 OpenGL \cnglo{context},當然,
這取決於一些特定\cnglo{platform}的 API (用來將 OpenGL \cnglo{context}綁定到視窗系統上):
\startigBase
\item 如果支持 CGL\footnote{CGL 是 Mac OS X 的 OpenGL 接口。} 綁定 API,
應當將特性 \cenum{CL_CGL_SHAREGROUP_KHR} 設置為一個 CGLShareGroup 句柄,
指向一個 CGL 共享組對象。

\item 如果支持 EGL\footnote{%
EGL 是 Khronos 渲染 API (如 OpenGL ES 或 OpenVG)和底層原生平台視窗系統之間的接口。%
} 綁定 API,
應當將特性 \cenum{CL_GL_CONTEXT_KHR} 設置為一個 EGLContext 句柄,
指向一個 OpenGL ES 或 OpenGL \cnglo{context},
而將特性 \cenum{CL_EGL_DISPLAY_KHR} 設置為一個 EGLDisplay 句柄,
指向用於創建這個 OpenGL ES 或 OpenGL \cnglo{context}的顯示屏。

\item 如果支持 GLX\footnote{GLX 是 X11 的 OpenGL 接口。} 綁定 API,
應當將特性 \cenum{CL_GL_CONTEXT_KHR} 設置為一個 GLXContext 句柄,
指向一個 OpenGL \cnglo{context},
而將特性 \cenum{CL_GLX_DISPLAY_KHR} 設置為一個 Display 句柄,
指向用於創建這個 OpenGL \cnglo{context}的 X 視窗系統顯示屏。

\item 如果支持 WGL\footnote{WGL 是 Microsoft Windows 的 OpenGL 接口。} 綁定 API,
應當將特性 \cenum{CL_GL_CONTEXT_KHR} 設置為一個 HGLRC 句柄,
指向一個 OpenGL \cnglo{context},
而將特性 \cenum{CL_WGL_HDC_KHR} 設置為一個 HDC 句柄,
指向用於創建這個 OpenGL \cnglo{context}的顯示屏。
\stopigBase

如果是在這樣的\cnglo{context}中創建的\cnglo{memobj},
那麼他可以被指定的 OpenGL 或 OpenGL ES \cnglo{context}
(也包括此\cnglo{context}的共享列中其他 OpenGL \cnglo{context},參見 GLX 1.4 和 EGL 1.4 規範,以及 Microsoft Windows 上 OpenGL 實作 WGL 的文檔)、
或者 CGL 共享組所共享。

如果特性列中沒有指定 OpenGL 或 OpenGL ES \cnglo{context}或者共享組,
那麼就不能共享\cnglo{memobj},
並且調用\refsection{clShareGl}中的\cnglo{cmd}時
會導致錯誤 \cenum{CL_INVALID_GL_SHAREGROUP_REFERENCE_KHR}。

OpenCL / OpenGL 間的共享不支持屬性 \cenum{CL_CONTEXT_INTEROP_USER_SYNC}
 (參見\reftab{prptForclCreateContext})。
如果創建帶有 OpenCL / OpenGL 共享的\cnglo{context}時指定了此屬性,則會返回錯誤。

\reftab{prptForclCreateContextPF}是對\reftab{prptForclCreateContext}的補充。

\placetable[here][tab:prptForclCreateContextPF]
{創建上下文時所用特性}
{\startETD[cl_context_properties][屬性值]

\clETD{CL_GL_CONTEXT_KHR}{OpenGL \cnglo{context}句柄}{
OpenCL \cnglo{context}所關聯的 OpenGL \cnglo{context}。
缺省值為 \cenumemp{0}。
}

\clETD{CL_CGL_SHAREGROUP_KHR}{OpenGL 共享組句柄}{
OpenCL \cnglo{context}所關聯的 CGL 共享組。
缺省值為 \cenumemp{0}。
}

\clETD{CL_EGL_DISPLAY_KHR}{EGLDisplay 句柄}{
OpenGL \cnglo{context}所對應的 EGLDisplay。
缺省值為 \cenumemp{EGL_NO_DISPLAY}。
}

\clETD{CL_GLX_DISPLAY_KHR}{X 句柄}{
OpenGL \cnglo{context}所對應的 X Display。
缺省值為 \cenumemp{None}。
}

\clETD{CL_WGL_HDC_KHR}{HDC 句柄}{
OpenGL \cnglo{context}所對應的 HDC。
缺省值為 \cenumemp{0}。
}

\stopETD

}

下列內容取代 \clapi{clCreateContext} 所返回的錯誤列中的第一個:
\startreplacepar
如果\cnglo{context}由下列任一方式指定:
\startigBase[indentnext=no]
\item 通過設置特性 \cenum{CL_GL_CONTEXT_KHR} 和 \cenum{CL_EGL_DISPLAY_KHR} 為
基於 EGL 的 OpenGL ES 或 OpenGL 實作指定了一個\cnglo{context}。

\item 通過設置特性 \cenum{CL_GL_CONTEXT_KHR} 和 \cenum{CL_GLX_DISPLAY_KHR} 為
基於 GLX 的 OpenGL 實作指定了一個\cnglo{context}。

\item 通過設置特性 \cenum{CL_GL_CONTEXT_KHR} 和 \cenum{CL_WGL_HDC_KHR} 為
基於 WGL 的 OpenGL 實作指定了一個\cnglo{context}。
\stopigBase
並且滿足下列任一條件:
\startigBase[indentnext=no]
\item 所指定的 display 和\cnglo{context}特性不能標識一個有效的 OpenGL 或 OpenGL ES \cnglo{context}。

\item 所指定的\cnglo{context}不支持\cnglo{bufobj}和渲染緩衝對象。

\item 所指定的\cnglo{context}與所創建的 OpenCL \cnglo{context}不兼容。
例如,位於物理上不同的位址空間內,如另一個硬件設備上;或者由於實作的局限不支持與 OpenCL 共享數據。
\stopigBase
則 \carg{errcode_ret} 會返回 \cenum{CL_INVALID_GL_SHAREGROUP_REFERENCE_KHR}。

如果通過設置特性 \cenum{CL_CGL_SHAREGROUP_KHR} 為基於 CGL 的 OpenGL 實作指定了一個共享組,
但是所指定的共享組不能標識一個有效的 CGL 共享組對象,
那麼 \carg{errcode_ret} 會返回 \cenum{CL_INVALID_GL_SHAREGROUP_REFERENCE_KHR}。

如果按上面所描述的那樣指定了一個\cnglo{context},並且滿足下列任一條件:
\startigBase[indentnext=no]
\item 為 CGL、 EGL、 GLX 或 WGL 其中之一指定了一個\cnglo{context}或共享組對象,
但是 OpenGL 實作不支持視窗系統綁定 API。

\item 為 \cenum{CL_CGL_SHAREGROUP_KHR}、 \cenum{CL_EGL_DISPLAY_KHR}、
 \cenum{CL_GLX_DISPLAY_KHR} 以及 \cenum{CL_WGL_HDC_KHR} 中一個以上的特性設置了非缺省值。

\item 為特性 \cenum{CL_CGL_SHAREGROUP_KHR} 和 \cenum{CL_GL_CONTEXT_KHR} 都設置了非缺省值。

\item 引數 \carg{devices} 中任一\cnglo{device}不支持 OpenCL 對象與 OpenGL 對象共享數據存儲,
參見\refsection{clShareGl}。
\stopigBase
則 \carg{errcode_ret} 會返回 \cenum{CL_INVALID_OPERATION}。

如果 \carg{properties} 中任一特性名無效
或者 \carg{properties} 中有特性 \cenum{CL_CONTEXT_INTEROP_USER_SYNC},
則 \carg{errcode_ret} 會返回 \cenum{CL_INVALID_PROPERTY}。
\stopreplacepar

下列內容取代 \clapi{clCreateContextFromType} 中對 \carg{properties} 的描述:
\startreplacepar
\carg{properties} 指向一個特性列,
其格式以及有效內容與 \clapi{clCreateContext} 的引數 \carg{properties} 相同。
\stopreplacepar

用上面所描述的兩個新錯誤取代 \clapi{clCreateContextFromType} 的錯誤列中的第一個。

% Additions to section 9.7 of the OpenCL 1.2 Extension Specification
\subsection{對節 9.7 的補充}

下列內容作為{\ftRef{節 9.7.7}}:
\startreplacepar
可以查詢 OpenGL \cnglo{context}對應的 OpenCL \cnglo{device}。
不一定有這樣的\cnglo{device}
(例如, OpenGL \cnglo{context}所在 GPU 可能不支持 OpenCL \cnglo{cmdq},
但卻支持共享的 CL / GL 對象),即使有,也可能隨時間發生變化。
如果存在這樣的\cnglo{device},
在此\cnglo{device}所對應的\cnglo{cmdq}上兼并和釋放共享的 CL / GL 對象
可能會比在 OpenCL \cnglo{context}可用的其他\cnglo{device}所對應的\cnglo{cmdq}上要快一些。
用以下函式查詢當前所對應的\cnglo{device}:

\topclfunc{clGetGLContextInfoKHR}

\startCLFUNC
cl_int clGetGLContextInfoKHR (
		const cl_context_properties *properties,
		cl_gl_context_info param_name,
		size_t param_value_size,
		void *param_value,
		size_t *param_value_size_ret)
\stopCLFUNC

\carg{properties} 指向一個特性列,
其格式和有效內容與 \clapi{clCreateContext} 的引數 \carg{properties} 相同。
\carg{properties} 必須能夠識別唯一一個有效的 GL \cnglo{context}或 \cnglo{glsharegrp}對象。

\carg{param_name} 是一個常量,指定所要查詢的 GL \cnglo{context}資訊,
有效值參見\reftab{ctxprop}。

\placetable[here][tab:ctxprop]
{可以用 \clapi{clGetGLContextInfoKHR} 查詢的 GL 上下文資訊}
{\startETD[cl_gl_context_info][返回型別]

\clETD{CL_CURRENT_DEVICE_FOR_GL_CONTEXT_KHR}{cl_device_id}{
返回與指定 OpenGL \cnglo{context}目前所關聯的 CL \cnglo{device}。
}

\clETD{CL_DEVICES_FOR_GL_CONTEXT_KHR}{cl_device_id[]}{
返回與指定 OpenGL \cnglo{context}所關聯的所有 CL \cnglo{device}。
}

\stopETD

}

\carg{param_value} 所指內存用來存儲查詢結果,參見\reftab{ctxprop}。
如果其值為 \cmacro{NULL},則忽略。

\carg{param_value_size} 為 \carg{param_value} 所指內存的字節數。
其值必須大於等於\reftab{ctxprop}中返回型別的大小。

\carg{param_value_size_ret} 返回查詢結果的實際字節數。
如果其值為 \cmacro{NULL},則忽略。

如果執行成功, \clapi{clGetGLContextInfoKHR} 會返回 \cenum{CL_SUCCESS}。
如果 \carg{param_name} 沒有對應的\cnglo{device},
調用不會失敗,但是 \carg{param_value_size_ret} 的值將會是零。

如果\cnglo{context}由下列任一方式指定:
\startigBase[indentnext=no]
\item 通過設置特性 \cenum{CL_GL_CONTEXT_KHR} 和 \cenum{CL_EGL_DISPLAY_KHR} 為
基於 EGL 的 OpenGL ES 或 OpenGL 實作指定了一個\cnglo{context}。

\item 通過設置特性 \cenum{CL_GL_CONTEXT_KHR} 和 \cenum{CL_GLX_DISPLAY_KHR} 為
基於 GLX 的 OpenGL 實作指定了一個\cnglo{context}。

\item 通過設置特性 \cenum{CL_GL_CONTEXT_KHR} 和 \cenum{CL_WGL_HDC_KHR} 為
基於 WGL 的 OpenGL 實作指定了一個\cnglo{context}。
\stopigBase
並且滿足下列任一條件:
\startigBase[indentnext=no]
\item 所指定的 display 和\cnglo{context}特性不能標識一個有效的 OpenGL 或 OpenGL ES \cnglo{context}。

\item 所指定的\cnglo{context}不支持\cnglo{bufobj}和渲染緩衝對象。

\item 所指定的\cnglo{context}與所創建的 OpenCL \cnglo{context}不兼容。
例如,位於物理上不同的位址空間內,如另一個硬件設備上;或者由於實作的局限不支持與 OpenCL 共享數據。
\stopigBase
則 \clapi{clGetGLContextInfoKHR} 會返回 \cenum{CL_INVALID_GL_SHAREGROUP_REFERENCE_KHR}。

如果通過設置特性 \cenum{CL_CGL_SHAREGROUP_KHR} 為基於 CGL 的 OpenGL 實作指定了一個共享組,
但是所指定的共享組不能標識一個有效的 CGL 共享組對象,
那麼 \clapi{clGetGLContextInfoKHR} 會返回 \cenum{CL_INVALID_GL_SHAREGROUP_REFERENCE_KHR}。

如果按上面所描述的那樣指定了一個\cnglo{context},並且滿足下列任一條件:
\startigBase[indentnext=no]
\item 為 CGL、 EGL、 GLX 或 WGL 其中之一指定了一個\cnglo{context}或共享組對象,
但是 OpenGL 實作不支持視窗系統綁定 API。

\item 為 \cenum{CL_CGL_SHAREGROUP_KHR}、 \cenum{CL_EGL_DISPLAY_KHR}、
 \cenum{CL_GLX_DISPLAY_KHR} 以及 \cenum{CL_WGL_HDC_KHR} 中一個以上的特性設置了非缺省值。

\item 為特性 \cenum{CL_CGL_SHAREGROUP_KHR} 和 \cenum{CL_GL_CONTEXT_KHR} 都設置了非缺省值。

\item 引數 \carg{devices} 中任一\cnglo{device}不支持 OpenCL 對象與 OpenGL 對象共享數據存儲,
參見\refsection{clShareGl}。
\stopigBase
則 \clapi{clGetGLContextInfoKHR} 會返回 \cenum{CL_INVALID_OPERATION}。

如果 \carg{properties} 中任一特性名無效,
則 \carg{errcode_ret} 會返回 \cenum{CL_INVALID_VALUE}。
\problem{maybe should CL_INVALID_PROPERTIES}

另外,如果 \carg{param_name} 無效(參見\reftab{ctxprop}),
或者 \carg{param_value_size} 的值小於\reftab{ctxprop}中返回型別的大小,
並且 \carg{param_value} 不是 \cmacro{NULL},
則 \clapi{clGetGLContextInfoKHR} 會返回 \cenum{CL_INVALID_VALUE}。
如果\schostfailres,則返回 \cenum{CL_OUT_OF_RESOURCES}。
如果\scdevfailres,則返回 \cenum{CL_OUT_OF_HOST_MEMORY}。

\stopreplacepar

% Issues
\subsection{問題}

\startQUESTION
創建所關聯的 OpenCL \cnglo{context}時,如何識別 OpenGL \cnglo{context}?
\stopQUESTION
\startANSWER
已解決:使用特性對 \ccmm{(display, context handle)} 來
識別任一 OpenGL 或 OpenGL ES \cnglo{context}
(此\cnglo{context}與某一視窗系統綁定層,即 EGL、 GLX 或 WGL 相關);
用共享組句柄來識別 CGL 共享組。
如果指定了一個\cnglo{context},對於調用 \clapi{clCreateContext} 的線程而言,
\cnglo{context}不必是最近的。

以前所建議的另一種途徑是:
使用布爾特性 \cenum{CL_USE_GL_CONTEXT_KHR} 創建的\cnglo{context}
可以與當前所綁定的 OpenGL \cnglo{context}相關聯。
這種方式仍然可以實現為一個單獨的擴展,
在一些特定情況下,如與所綁定的 GL \cnglo{context}在同一線程中執行時,
保留 / 釋放的行為可能更加高效。
\stopANSWER

\startQUESTION
特性列的格式應當是什麼樣的?
\stopQUESTION
\startANSWER
經過大量討論後,我們認為特性列格式可以是一組{\ftRef{<特性名、值>}}元組,以零終止。
此特性列作為 \ccmm{cl_context_properties *properties} 傳遞,
其中 \cldt{cl_context_properties} 在 \ccmm{cl.h} 中 typedef 為 \cldt{intptr_t}。

這樣可以將\cnglo{host} API 中的所有標量整數、指針、以及句柄值都編碼進引數列中,
與 EGL 特性列的結構體、型別類似。
同時也允許特性列為 \cmacro{NULL}。
同樣與 EGL 一樣,特性列中所沒有的特性都使用缺省值。

就 EGL、 GLX 和 WGL 的經驗而言,特性列是一種足夠靈活、通用的機制,
完全可以滿足一些調用(像創建\cnglo{context})的管理需求。
當然,他還不完全通用,如對浮點、非標量特性值還不能直接編碼,
對於這些情況建議使用其他途徑,
如不透明的特性列,使用 getter / setter 方法、或者帶有變長數組的結構體。
\stopANSWER

\startQUESTION
對於所關聯的 OpenGL 或 OpenCL \cnglo{context}而言,
如果他正在使用所關聯的另一個\cnglo{context}的資源,可是那個\cnglo{context}卻被銷毀了,
這時其行為如何?
\stopQUESTION
\startANSWER
已解決:如\refsection{clShareGl}所言,
創建 OpenCL 對象時,會在對應的 GL 對象的數據存儲中放置一個址參器。
對應的 GL 名可能會被刪除,但只要有 CL 對象引用他,數據存儲本身會一直存在。
然而,如果銷毀了與 CL \cnglo{context}對應的共享組中的所有 GL \cnglo{context},
則使用對應的 CL 對象時所導致的行為\cnglo{impdef},直至並且包括\cnglo{program}終止。
\stopANSWER

\startQUESTION
怎樣與 D3D 共享?
\stopQUESTION
\startANSWER
在 D3D 和 OpenCL 間共享時,應當使用同樣的特性列機制,
儘管參數明顯不同,並將其作為一個相似的並行 OpenCL 擴展使用。
由於還不清楚是否可以創建一個 CL \cnglo{context}同時共享 GL 和 D3D 對象,
因此在那個擴展和本擴展中間可能需要進行交互。
\stopANSWER

\startQUESTION
什麼情況下,創建\cnglo{context}時會由於共享而失敗?
\stopQUESTION
\startANSWER
已解決:前面已經列出了幾個跨平台的失敗條件
(GL \cnglo{context}或 CGL 共享組不存在,
 GL \cnglo{context}不支持\refsection{clShareGl}中接口所要求類型的 GL 對象,
 GL \cnglo{context}的實作不允許共享),
但由於具體實作的原因可能還會導致其他問題,
這些情況也要加入此擴展。
 OpenCL 和 OpenGL 間的共享要求在驅動內部構件的級別進行整合。
\stopANSWER

\startQUESTION
應當將 \clapi{clEnqueueAcquire/ReleaseGLObjects} 放置到什麼樣的\cnglo{cmdq}中?
\stopQUESTION
\startANSWER
已解決:所有\cnglo{cmdq}。
創建\cnglo{context}時就會實施此局限。
如果創建\cnglo{context}時所傳遞的\cnglo{device}中任何一個不支持共享的 CL/GL 對象,
那麼創建\cnglo{context}就會失敗,並返回錯誤 \cenum{CL_INVALID_OPERATION}。
\stopANSWER

\startQUESTION
\cnglo{app}如何確定 Acquire/Release 放置到了哪個\cnglo{cmdq}中?
\stopQUESTION
\startANSWER
已解決: \clapi{clGetGLContextInfoKHR} 會返回
當前與特定 GL \cnglo{context}所對應的 CL \cnglo{device}
(典型的就是 GL \cnglo{context}正運行其上的顯示屏),
或者特定\cnglo{context}可能運行其上的所有 CL \cnglo{device}
(在多線程 / “虛擬屏幕”環境中可能有用)。
並沒有只是簡單地將此\cnglo{cmd}放到\refsection{clShareGl}中,
因為他依賴於本擴展中所介紹的用於指定 GL \cnglo{context}的特性列。
\stopANSWER

\startQUESTION
查詢 \cenum{CL_DEVICES_FOR_GL_CONTEXT_KHR} 是什麼意思?
\stopQUESTION
\startANSWER
已解決:就是曾經與特定 GL \cnglo{context}相關聯的所有 CL \cnglo{device}。
在一些\cnglo{platform}上,如 MacOS X,
“虛擬屏幕”的概念允許多個 GPU 跑在同一個虛擬顯示屏上。
其他視窗系統上也可能會實現類似的概念,
如透明異構多線程 X 伺服器。
因此此查詢的確切意義需要根據所使用的綁定層 API 來解釋。
\stopANSWER

\todo{與 1.0 相比的變動}


\section[section:clShareGl]{與 OpenGL/OpenGL ES 緩衝、材質和渲染緩衝對象共享內存對象}

本節中的 OpenCL 函式允許\cnglo{app}將 OpenGL 緩衝、材質和渲染緩衝對象用作 OpenCL \cnglo{memobj}。
這樣可以在 OpenCL 和 OpenGL 間高效的共享數據。
 OpenCL API 所執行的\cnglo{kernel}即可讀寫\cnglo{memobj},也可讀寫 OpenGL 對象。

OpenCL \cnglo{imgobj}可能是由 OpenGL 材質或渲染緩衝對象創建的。
OopenCL \cnglo{bufobj}可能是由 OpenGL \cnglo{bufobj}創建的。

當且僅當 OpenCL \cnglo{context}是由 OpenGL 共享組對象或\cnglo{context}創建的,
才能由 OpenGL 對象創建 OpenCL \cnglo{memobj}。
OpenGL 共享組和\cnglo{context}都是由特定平台 API 創建的,
如 EGL、 CGL、 WGL 以及 GLX。
在 MacOS X 上, OpenCL \cnglo{context}可能是
用 OpenCL \cnglo{platform}擴展 \clext{cl_apple_gl_sharing} 由 OpenGL 共享組對象創建的。
在其他\cnglo{platform}上,包括 Microsoft Windows、 Linux/Unix 等等,
 OpenCL \cnglo{context}可能是
用 Khronos \cnglo{platform}擴展 \clext{cl_khr_gl_sharing} 由 OpenGL \cnglo{context}創建的。
請參考對應的 OpenCL 實作的\cnglo{platform}文檔,
或者訪問 \from[khronosRegistryCL] 以獲取更多資訊。

對於 GL 共享組對象、或 GL \cnglo{context}
(由其創建的 CL \cnglo{context})所關聯的共享組中
所定義的任何 OpenGL 對象,只要支持,就可以被共享,
但是有個例外,就是缺省的 OpenGL 對象(即命名為零的對象)不能被共享。

\subsection{共享對象的生命週期}

只要對應的 GL 對象沒有被刪除,
則由其創建的 OpenCL \cnglo{memobj}(下文中稱為“共享的 CL/GL 對象”)就會一直有效。
如果通過 GL API (如 \capi{glDeleteBuffers}、 \capi{glDeleteTextures} 或 \capi{glDeleteRenderbuffers})刪除了 GL 對象,
則後續使用 CL \cnglo{bufobj}或\cnglo{imgobj}時會導致\cnglo{undef}的行為,
包括但不限於 CL 錯誤和數據訛誤,但不會使\cnglo{program}終止。

CL \cnglo{context}和對應的\cnglo{cmdq}依賴於 GL 共享組對象、
或 GL \cnglo{context}(由其創建的 CL \cnglo{context})所關聯的共享組的存在。
如果銷毀了 GL 共享組對象或者共享組中的所有 GL \cnglo{context},
則使用 CL \cnglo{context}或\cnglo{cmdq}會導致\cnglo{undef}的行為,
包括\cnglo{program}終止。
在銷毀對應的 GL 共享組或\cnglo{context}之前,
\cnglo{app}應當先銷毀 CL \cnglo{cmdq}和 CL \cnglo{context}。


\subsection{由 GL 緩衝對象創建 CL 緩衝對象}

\topclfunc{clCreateFromGLBuffer}

\startCLFUNC
cl_mem clCreateFromGLBuffer (
			cl_context context,
			cl_mem_flags flags,
			GLuint bufobj,
			cl_int *errcode_ret)
\stopCLFUNC

此函式可以由 OpenGL 緩衝對象創建 OpenCL 緩衝對象。

\carg{context} 是一個由 OpenGL \cnglo{context}創建的 OpenCL \cnglo{context}。

\carg{flags} 是位欄,用於指定用法資訊。參見\reftab{clmemflags}。
此處只能使用 \cenum{CL_MEM_READ_ONLY}、 \cenum{CL_MEM_WRITE_ONLY} 和 \cenum{CL_MEM_READ_WRITE}。

\carg{bufobj} 是 GL \cnglo{bufobj}的名字。
其數據存儲必須在之前已經通過調用 \capi{glBufferData} 創建好了,
但其內容可以不必初始化。
數據存儲的大小將用來確定 CL \cnglo{bufobj}的大小。

\carg{errcode_ret} 會返回相應的錯誤碼,見下文。
如果他是 \cmacro{NULL},則不會返回錯誤碼。

如果成功創建了\cnglo{bufobj},則 \clapi{clCreateFromGLBuffer} 會將其返回,
並將 \carg{errcode_ret} 置為 \cenum{CL_SUCCESS}。
否則,返回 \cmacro{NULL},並將 \carg{errcode_ret} 置為下列錯誤碼之一:
\startigBase
\item \cenum{CL_INVALID_CONTEXT},
如果 \carg{context} 無效,或者不是由 GL \cnglo{context}創建的。

\item \cenum{CL_INVALID_VALUE},如果 \carg{flags} 的值無效。

\item \cenum{CL_INVALID_GL_OBJECT},
如果 \carg{bufobj} 不是 GL \cnglo{bufobj},
或者是 GL \cnglo{bufobj},但沒有數據存儲,或者其大小是 0。

\item \cenum{CL_OUT_OF_RESOURCES},如果\scdevfailres。

\item \cenum{CL_OUT_OF_HOST_MEMORY},如果\schostfailres。
\stopigBase

調用 \clapi{clCreateFromGLBuffer} 時,
會將 GL \cnglo{bufobj}數據存儲的大小用作所返回\cnglo{bufobj}的大小。
如果存在對應的 CL \cnglo{bufobj},
但是通過 GL API (如 \capi{glBufferData})修改了 GL \cnglo{bufobj}的狀態,
則後續使用 CL \cnglo{bufobj}時會導致\cnglo{undef}的行為。

可使用函式 \clapi{clRetainMemObject} 和 \clapi{clReleaseMemObject} 來
\cnglo{retain}和\cnglo{release}此\cnglo{bufobj}。

用 \clapi{clCreateFromGLBuffer} 創建的 CL \cnglo{bufobj}可用來創建 CL 1D \cnglo{imgobj}。


\subsection{由 GL 材質創建 CL 圖像對象}

\topclfunc{clCreateFromGLTexture}

\startCLFUNC
cl_mem clCreateFromGLTexture (
			cl_context context,
			cl_mem_flags flags,
			GLenum texture_target,
			GLint miplevel,
			GLuint texture,
			cl_int *errcode_ret)
\stopCLFUNC

此函式可以:
\startigBase
\item 由 OpenGL 2D 材質對象、或 OpenGL cubemap 材質對象的某一面創建 OpenCL 2D \cnglo{imgobj}。

\item 由 OpenGL 2D 材質陣列對象創建 OpenCL 2D 圖像陣列對象。

\item 由 OpenGL 1D 材質對象創建 OpenCL 1D 圖像對象。

\item 由 OpenGL 材質緩衝對象創建 OpenCL 1D 圖像緩衝對象。

\item 由 OpenGL 1D 材質陣列對象創建 OpenCL 1D 圖像陣列對象。

\item 由 OpenGL 3D 材質對象創建 OpenCL 3D 圖像對象。
\stopigBase

\carg{context} 是一個由 OpenGL \cnglo{context}創建的 OpenCL \cnglo{context}。

\carg{flags} 是位欄,用於指定用法資訊。參見\reftab{clmemflags}。
此處只能使用 \cenum{CL_MEM_READ_ONLY}、 \cenum{CL_MEM_WRITE_ONLY} 和 \cenum{CL_MEM_READ_WRITE}。

\carg{texture_target} 必須是下列值之一:
\startigBase[indentnext=no]
\item \cenum{GL_TEXTURE_1D}
\item \cenum{GL_TEXTURE_1D_ARRAY}
\item \cenum{GL_TEXTURE_BUFFER}
\item \cenum{GL_TEXTURE_2D}
\item \cenum{GL_TEXTURE_2D_ARRAY}
\item \cenum{GL_TEXTURE_3D}
\item \cenum{GL_TEXTURE_CUBE_MAP_POSITIVE_X}
\item \cenum{GL_TEXTURE_CUBE_MAP_POSITIVE_Y}
\item \cenum{GL_TEXTURE_CUBE_MAP_POSITIVE_Z}
\item \cenum{GL_TEXTURE_CUBE_MAP_NEGATIVE_X}
\item \cenum{GL_TEXTURE_CUBE_MAP_NEGATIVE_Y}
\item \cenum{GL_TEXTURE_CUBE_MAP_NEGATIVE_Z}
\item \cenum{GL_TEXTURE_RECTANGLE}\footnote{%
要求 OpenGL 3.1。
如果支持 OpenGL 擴展 \clext{GL_ARG_texture_rectangle},
也可以是 \cenum{GL_TEXTURE_RECTANGLE_ARG}。}
\stopigBase
\carg{texture_target} 僅用來定義 \carg{texture} 的圖像類型。
此參數不會引用所綁定的 GL 材質對象。

\carg{miplevel} 即所用的 mipmap 級別\footnote{%
如果 \carg{miplevel} 的值 > 0,則實作可能返回 \cenum{CL_INVALID_OPERATION}。}。
如果 \carg{texture_target} 是 \cenum{GL_TEXTURE_BUFFER},
則 \carg{miplevel} 必須是 0。

\carg{texture} 是一個 GL 1D、 2D、 3D、 1D 陣列、 2D 陣列、 cubemap、 矩形或緩衝材質對象。
這個材質對象(按照材質上 OpenGL 規則的完整性)必須是完整的材質。
OpenGL 為 \carg{miplevel} 的 \carg{texture} 所定義的格式和維度
會被用來創建 OpenCL \cnglo{imgobj}。
僅當 GL 材質對象的內部格式可以映射到
\reftab{imgChannelOrder}和\reftab{imgChannelDataType}中所列通道次序和數據型別時,
才可以用來創建 OpenCL \cnglo{imgobj}。

\carg{errcode_ret} 會返回相應的錯誤碼,見下文。
如果他是 \cmacro{NULL},則不會返回錯誤碼。

如果成功創建了\cnglo{imgobj},則 \clapi{clCreateFromGLTexture} 會將其返回,
並將 \carg{errcode_ret} 置為 \cenum{CL_SUCCESS}。
否則,返回 \cmacro{NULL},並將 \carg{errcode_ret} 置為下列錯誤碼之一:
\startigBase
\item \cenum{CL_INVALID_CONTEXT},
如果 \carg{context} 無效,或者不是由 GL \cnglo{context}創建的。

\item \cenum{CL_INVALID_VALUE},如果 \carg{flags} 的值無效,
或者 \carg{texture_target} 的值不再上面所述範圍內。

\item \cenum{CL_INVALID_MIP_LEVEL},
如果 \carg{miplevel} 小於(OpenGL 實作的) level\low{base} 或者(對於 OpenGL ES 實作而言)是零;
或者大於 q (無論是 OpenGL 還是 OpenGL ES)。
 OpenGL 2.1 規範中的{\ftRef{節 3.8.10}} 和 OpenGL ES 2.0 中的{\ftRef{節 3.7.10}} 中
定義了 level\low{base} 和 q 。

\item \cenum{CL_INVALID_MIP_LEVEL},
如果 \carg{miplevel} 大於零,並且 OpenGL 實作不支持由非零 mipmap 級別創建。

\item \cenum{CL_INVALID_GL_OBJECT},
如果 \carg{texture} 不是 \carg{texture_target} 類型的 GL 材質對象;
或者沒有定義 \carg{miplevel} 的 \carg{texture};
或者指定的 \carg{miplevel} 的寬或高是零。

\item \cenum{CL_INVALID_IMAGE_FORMAT_DESCRIPTOR},
如果 OpenGL 材質內部格式不能映射到所支持的 OpenCL 圖像格式上。

\item \cenum{CL_INVALID_OPERATION},
如果 \carg{texture} 是邊界寬度值大於零的 GL 材質對象。

\item \cenum{CL_OUT_OF_RESOURCES},如果\scdevfailres。

\item \cenum{CL_OUT_OF_HOST_MEMORY},如果\schostfailres。
\stopigBase

如果存在對應的 CL \cnglo{imgobj},
但是通過 GL API (如 \capi{glTexImage2D}、 \capi{glTexImage3D} 或者
修改了材質參數 \cenum{GL_TEXTURE_BASE_LEVEL} 或 \cenum{GL_TEXTURE_MAX_LEVEL})
修改了 GL 材質對象的狀態,
則後續使用 CL \cnglo{imgobj}時會導致\cnglo{undef}的行為。

可使用函式 \clapi{clRetainMemObject} 和 \clapi{clReleaseMemObject} 來
\cnglo{retain}和\cnglo{release}此\cnglo{imgobj}。

\reftab{ImgFmtMap}列出了 GL 材質內部格式與 CL 圖像格式間的對應關係。
只要 GL 材質對象是\reftab{ImgFmtMap}中所列格式,就保證能將其映射到表中對應的 CL 圖像格式上。
如果材質對象是其他 OpenGL 內部格式,
則可能(但不保證)也能映射到 CL 圖像格式上;
如果存在這樣的映射,則保證保留所有顏色組件、數據型別,並且每個組件的位數至少與分配 OpenGL 對象時相同。

\placetable[here][tab:ImgFmtMap]
{GL 內部格式到 CL 圖像格式的映射}
{\startCLOO[GL 內部格式][CL 圖像格式\par(通道次序、通道數據型別)]

\clOO{\cenum{GL_RGBA8}}{%
\cenum{CL_RGBA}、\cenum{CL_UNORM_INT8} 或\par
\cenum{CL_BGRA}、\cenum{CL_UNORM_INT8}}

\clOO{\cenum{GL_RGBA}\par
\cenum{GL_UNSIGNED_INT_8_8_8_8_REV}}{%
\cenum{CL_RGBA}、\cenum{CL_UNORM_INT8}}

\clOO{\cenum{GL_BGRA}\par
\cenum{GL_UNSIGNED_INT_8_8_8_8_REV}}{%
\cenum{CL_BGRA}、\cenum{CL_UNORM_INT8}}

\clOO{\cenum{GL_RGBA16}}{%
\cenum{CL_RGBA}、\cenum{CL_UNORM_INT16}}

\clOO{\cenum{GL_RGBA8I}、\cenum{GL_RGBA8I_EXT}}{%
\cenum{CL_RGBA}、\cenum{CL_SIGNED_INT8}}

\clOO{\cenum{GL_RGBA16I}、\cenum{GL_RGBA16I_EXT}}{%
\cenum{CL_RGBA}、\cenum{CL_SIGNED_INT16}}

\clOO{\cenum{GL_RGBA32I}、\cenum{GL_RGBA32I_EXT}}{%
\cenum{CL_RGBA}、\cenum{CL_SIGNED_INT32}}

\clOO{\cenum{GL_RGBA8UI}、\cenum{GL_RGBA8UI_EXT}}{%
\cenum{CL_RGBA}、\cenum{CL_UNSIGNED_INT8}}

\clOO{\cenum{GL_RGBA16UI}、\cenum{GL_RGBA16UI_EXT}}{%
\cenum{CL_RGBA}、\cenum{CL_UNSIGNED_INT16}}

\clOO{\cenum{GL_RGBA32UI}、\cenum{GL_RGBA32UI_EXT}}{%
\cenum{CL_RGBA}、\cenum{CL_UNSIGNED_INT32}}

\clOO{\cenum{GL_RGBA16F}、\cenum{GL_RGBA16F_ARB}}{%
\cenum{CL_RGBA}、\cenum{CL_HALF_FLOAT}}

\clOO{\cenum{GL_RGBA32F}、\cenum{GL_RGBA32F_ARB}}{%
\cenum{CL_RGBA}、\cenum{CL_FLOAT}}

\stopCLOO

}


\subsection{由 GL 渲染緩衝創建 CL 圖像對象}

\topclfunc{clCreateFromGLRenderBuffer}

\startCLFUNC
cl_mem clCreateFromGLRenderbuffer (
			cl_context context,
			cl_mem_flags flags,
			GLuint renderbuffer,
			cl_int *errcode_ret)
\stopCLFUNC

此函式可以由 OpenGL 渲染緩衝對象創建 OpenCL 2D \cnglo{imgobj}。

\carg{context} 是一個由 OpenGL \cnglo{context}創建的 OpenCL \cnglo{context}。

\carg{flags} 是位欄,用於指定用法資訊。參見\reftab{clmemflags}。
此處只能使用 \cenum{CL_MEM_READ_ONLY}、 \cenum{CL_MEM_WRITE_ONLY} 和 \cenum{CL_MEM_READ_WRITE}。

\carg{renderbuffer} 是 GL 渲染緩衝對象的名字。
創建\cnglo{imgobj}前必須先指定渲染緩衝的存儲。
OpenGL 為 \carg{renderbuffer} 定義的格式和維度
會被用來創建 2D \cnglo{imgobj}。
僅當 GL 渲染緩衝對象的內部格式可以映射到
\reftab{imgChannelOrder}和\reftab{imgChannelDataType}中所列通道次序和數據型別時,
才可用來創建 2D \cnglo{imgobj}。

\carg{errcode_ret} 會返回相應的錯誤碼,見下文。
如果他是 \cmacro{NULL},則不會返回錯誤碼。

如果成功創建了\cnglo{imgobj},則 \clapi{clCreateFromGLRenderbuffer} 會將其返回,
並將 \carg{errcode_ret} 置為 \cenum{CL_SUCCESS}。
否則,返回 \cmacro{NULL},並將 \carg{errcode_ret} 置為下列錯誤碼之一:
\startigBase
\item \cenum{CL_INVALID_CONTEXT},
如果 \carg{context} 無效,或者不是由 GL \cnglo{context}創建的。

\item \cenum{CL_INVALID_VALUE},如果 \carg{flags} 的值無效。

\item \cenum{CL_INVALID_GL_OBJECT},
如果 \carg{renderbuffer} 不是 GL 渲染緩衝對象;
或者 \carg{renderbuffer} 的寬或高是零。

\item \cenum{CL_INVALID_IMAGE_FORMAT_DESCRIPTOR},
如果 OpenGL 渲染緩衝內部格式不能映射到所支持的 OpenCL 圖像格式上。

\item \cenum{CL_INVALID_OPERATION},
如果 \carg{renderbuffer} 是多重採樣的 GL 渲染緩衝對象。

\item \cenum{CL_OUT_OF_RESOURCES},如果\scdevfailres。

\item \cenum{CL_OUT_OF_HOST_MEMORY},如果\schostfailres。
\stopigBase

如果存在對應的 CL \cnglo{imgobj},
但是通過 GL API (如用 GL API \capi{glRenderbufferStorage}
 修改了用以表示 GL 渲染緩衝像素的維度或格式)
修改了 GL 渲染緩衝對象的狀態,
則後續使用 CL \cnglo{imgobj}時會導致\cnglo{undef}的行為。

可使用函式 \clapi{clRetainMemObject} 和 \clapi{clReleaseMemObject} 來
\cnglo{retain}和\cnglo{release}此\cnglo{imgobj}。

\reftab{ImgFmtMap}列出了 GL 渲染緩衝內部格式與 CL 圖像格式間的對應關係。
只要 GL 寫入緩衝對象是\reftab{ImgFmtMap}中所列格式,就保證能將其映射到表中對應的 CL 圖像格式上。
如果渲染緩衝對象是其他 OpenGL 內部格式,
則可能(但不保證)也能映射到 CL 圖像格式上;
如果存在這樣的映射,則保證保留所有顏色組件、數據型別,並且每個組件的位數至少與分配 OpenGL 對象時相同。


\subsection{由 CL 內存對象查詢 GL 對象的資訊}

\topclfunc{clGetGLObjectInfo}

\startCLFUNC
cl_int clGetGLObjectInfo (
		cl_mem memobj,
		cl_gl_object_type *gl_object_type,
		GLuint *gl_object_name)
\stopCLFUNC

此函式可查詢用來創建 OpenCL \cnglo{memobj}的 OpenGL 對象及其對象類型,
即他是材質、渲染緩衝還是緩衝對象。

\carg{gl_object_type} 會返回附到 \carg{memobj} 上的 GL 對象的類型,
可以是:
\startigBase[indentnext=no]
\item \cenum{CL_GL_OBJECT_BUFFER}
\item \cenum{CL_GL_OBJECT_TEXTURE2D}
\item \cenum{CL_GL_OBJECT_TEXTURE3D}
\item \cenum{CL_GL_OBJECT_TEXTURE2D_ARRAY}
\item \cenum{CL_GL_OBJECT_TEXTURE1D}
\item \cenum{CL_GL_OBJECT_TEXTURE1D_ARRAY}
\item \cenum{CL_GL_OBJECT_TEXTURE_BUFFER}
\item \cenum{CL_GL_OBJECT_RENDERBUFFER}
\stopigBase
而如果 \carg{gl_object_type} 是 \cmacro{NULL},則將其忽略。

\carg{gl_object_name} 會返回用來創建 \carg{memobj} 的 GL 對象的名字。
如果 \carg{gl_object_name} 是 \cmacro{NULL},則將其忽略。

如果執行成功,則 \clapi{clGetGLObjectInfo} 會返回 \cenum{CL_SUCCESS}。
否則,返回下列錯誤碼之一:
\startigBase
\item \cenum{CL_INVALID_MEM_OBJECT},
如果 \carg{memobj} 無效。

\item \cenum{CL_INVALID_GL_OBJECT},
如果沒有與 \carg{memobj} 關聯的 GL 對象。

\item \cenum{CL_OUT_OF_RESOURCES},如果\scdevfailres。

\item \cenum{CL_OUT_OF_HOST_MEMORY},如果\schostfailres。
\stopigBase

% clGetGLTextureInfo
\topclfunc{clGetGLTextureInfo}

\startCLFUNC
cl_int clGetGLTextureInfo (
			cl_mem memobj,
			cl_gl_texture_info param_name,
			size_t param_value_size,
			void *param_value,
			size_t *param_value_size_ret)
\stopCLFUNC

此函式會返回與 \carg{memobj} 關聯的 GL 材質對象的附加資訊。

\carg{param_name} 指定要查詢什麼資訊。
\reftab{glTextureInfoType}中列出了 \clapi{clGetGLTextureInfo} 所支持的資訊類型
及 \carg{param_value} 中返回的資訊。

\carg{param_value} 所指內存用於存儲查詢結果。
如果 \carg{param_value} 是 \cmacro{NULL},則將其忽略。

\carg{param_value_size} 即 \carg{param_value} 所指內存的大小。
其值必須大於等於\reftab{glTextureInfoType}中所列返回型別的大小。

\carg{param_value_size_ret} 即拷貝到 \carg{param_value} 中數據的實際大小。
如果 \carg{param_value_size_ret} 是 \cmacro{NULL},則將其忽略。

\placetable[here][tab:glTextureInfoType]
{\clapi{clGetGLTextureInfo} 所支持的 \carg{param_name}}
{\startETD[cl_gl_texture_info][返回型別]

\clETD{CL_GL_TEXTURE_TARGET}{GLenum}{%
\clapi{clCreateFromGLTexture} 的引數 \carg{texture_target}。}

\clETD{CL_GL_MIPMAP_LEVEL}{GLint}{%
\clapi{clCreateFromGLTexture} 的引數 \carg{miplevel}。}

\stopETD

}

如果執行成功, \clapi{clGetGLTextureInfo} 會返回 \cenum{CL_SUCCESS}。
否則,返回下列錯誤碼之一:
\startigBase
\item \cenum{CL_INVALID_MEM_OBJECT},
如果 \carg{memobj} 無效。

\item \cenum{CL_INVALID_GL_OBJECT},
如果沒有與 \carg{memobj} 關聯的 GL 對象。

\item \cenum{CL_INVALID_VALUE},
如果 \carg{param_name} 無效;
或者 \carg{param_value_size} 的值 < \reftab{glTextureInfoType}中所列返回型別的大小,
並且 \carg{param_value} 不是 \cmacro{NULL};
或者 \carg{param_value} 和 \carg{param_value_size_ret} 都是 \cmacro{NULL}。

\item \cenum{CL_OUT_OF_RESOURCES},如果\scdevfailres。

\item \cenum{CL_OUT_OF_HOST_MEMORY},如果\schostfailres。
\stopigBase


\subsection{在 GL 和 CL 上下文間共享已映射到 GL 對象的內存對象}

\topclfunc{clEnqueueAcquireGLObjects}

\startCLFUNC
cl_int clEnqueueAcquireGLObjects (
			cl_command_queue command_queue,
			cl_uint num_objects.
			const cl_mem *mem_objects,
			cl_uint num_events_in_wait_list,
			const cl_event *event_wait_list,
			cl_event *event)
\stopCLFUNC

此函式可用來獲取(acquire)由 OpenGL 對象創建的 OpenCL \cnglo{memobj}。
\cnglo{cmdq}中的任一 OpenCL \cnglo{cmd}要想使用這些對象,都必須先獲取他們。
\carg{command_queue} 所在 OpenCL \cnglo{context}獲取 OpenGL 對象後,
此\cnglo{context}中的所有\cnglo{cmdq}都可以使用這些 OpenGL 對象。

\carg{command_queue} 是一個有效的\cnglo{cmdq}。
用來創建 \carg{command_queue} 所在 OpenCL \cnglo{context}的所有\cnglo{device}
都必須支持獲取共享的 CL/GL 對象。
在創建\cnglo{context}時就要如此。

\carg{num_objects} 即 \carg{mem_objects} 中\cnglo{memobj}的數目。

\carg{mem_objects} 指向 GL 對象所對應的一組 CL \cnglo{memobj}。

\carg{event_wait_list} 和 \carg{num_events_in_wait_list} 中
列出了執行此\cnglo{cmd}前要等待的事件。
如果 \carg{event_wait_list} 是 \cmacro{NULL},
則無須等待任何事件,並且 \carg{num_events_in_wait_list} 必須是0。
如果 \carg{event_wait_list} 不是 \cmacro{NULL},
則其中所有事件都必須是有效的,並且 \carg{num_events_in_wait_list} 必須大於 0。
\carg{event_wait_list} 中的事件充當同步點。

\carg{event} 會返回一個\cnglo{evtobj},
用來識別此\cnglo{cmd},可用來查詢或等待此\cnglo{cmd}完成。
而如果 \carg{event} 是 \cmacro{NULL},就沒辦法查詢此\cnglo{cmd}的狀態或等待其完成了。
如果 \carg{event_wait_list} 和 \carg{event} 都不是 \cmacro{NULL},
則引數 \carg{event} 不能屬於 \carg{event_wait_list}。

如果執行成功, \clapi{clEnqueueAcquireGLObjects} 會返回 \cenum{CL_SUCCESS}。
如果 \carg{num_objects} 是 0,並且 \carg{mem_objects} 是 \cmacro{NULL},
則此函式什麼都不做,並直接返回 \cenum{CL_SUCCESS}。
否則,返回下列錯誤碼之一:
\startigBase
\item \cenum{CL_INVALID_VALUE},
如果 \carg{num_objects} 是零,而 \carg{mem_objects} 不是 \cmacro{NULL};
或者 \carg{num_objects} > 0 而 \carg{mem_objects} 是 \cmacro{NULL}。

\item \cenum{CL_INVALID_MEM_OBJECT},
如果 \carg{mem_objects} 中的\cnglo{memobj}無效。

\item \cenum{CL_INVALID_COMMAND_QUEUE},
如果 \carg{command_queue} 無效。

\item \cenum{CL_INVALID_CONTEXT},
如果 \carg{command_queue} 所在\cnglo{context}不是由 OpenGL \cnglo{context}創建的。

\item \cenum{CL_INVALID_GL_OBJECT},
如果 \carg{mem_objects} 中的\cnglo{memobj}不是由 GL 對象創建的。

\startitem
\cenum{CL_INVALID_EVENT_WAIT_LIST},如果滿足下列條件中的任一項:
\startigBase
\item \carg{event_wait_list} 是 \cmacro{NULL},但 \carg{num_events_in_wait_list} > 0;
\item 或者 \carg{event_wait_list} 不是 \cmacro{NULL},但 \carg{num_events_in_wait_list} 是 0;
\item 或者 \carg{event_wait_list} 中有無效的事件。
\stopigBase
\stopitem

\item \cenum{CL_OUT_OF_RESOURCES},如果\scdevfailres。

\item \cenum{CL_OUT_OF_HOST_MEMORY},如果\schostfailres。
\stopigBase

\topclfunc{clEnqueueReleaseGLObjects}

\startCLFUNC
cl_int clEnqueueReleaseGLObjects (
			cl_command_queue command_queue,
			cl_uint num_objects.
			const cl_mem *mem_objects,
			cl_uint num_events_in_wait_list,
			const cl_event *event_wait_list,
			cl_event *event)
\stopCLFUNC

此函式可用來釋放(release)由 OpenGL 對象創建的 OpenCL \cnglo{memobj}。
要先將其釋放,然後 OpenGL 才能使用他們。
OpenGL 對象由 \carg{command_queue} 所在 OpenCL \cnglo{context} 釋放。

\carg{command_queue} 是一個有效的\cnglo{cmdq}。
用來創建 \carg{command_queue} 所在 OpenCL \cnglo{context}的所有\cnglo{device}
都必須支持獲取共享的 CL/GL 對象。
在創建\cnglo{context}時就要如此。

\carg{num_objects} 即 \carg{mem_objects} 中\cnglo{memobj}的數目。

\carg{mem_objects} 指向 GL 對象所對應的一組 CL \cnglo{memobj}。

\carg{event_wait_list} 和 \carg{num_events_in_wait_list} 中
列出了執行此\cnglo{cmd}前要等待的事件。
如果 \carg{event_wait_list} 是 \cmacro{NULL},
則無須等待任何事件,並且 \carg{num_events_in_wait_list} 必須是0。
如果 \carg{event_wait_list} 不是 \cmacro{NULL},
則其中所有事件都必須是有效的,並且 \carg{num_events_in_wait_list} 必須大於 0。
\carg{event_wait_list} 中的事件充當同步點。

\carg{event} 會返回一個\cnglo{evtobj},
用來識別此\cnglo{cmd},可用來查詢或等待此\cnglo{cmd}完成。
而如果 \carg{event} 是 \cmacro{NULL},就沒辦法查詢此\cnglo{cmd}的狀態或等待其完成了。
如果 \carg{event_wait_list} 和 \carg{event} 都不是 \cmacro{NULL},
則引數 \carg{event} 不能屬於 \carg{event_wait_list}。

如果執行成功, \clapi{clEnqueueReleaseGLObjects} 會返回 \cenum{CL_SUCCESS}。
如果 \carg{num_objects} 是 0,並且 \carg{mem_objects} 是 \cmacro{NULL},
則此函式什麼都不做,並直接返回 \cenum{CL_SUCCESS}。
否則,返回下列錯誤碼之一:
\startigBase
\item \cenum{CL_INVALID_VALUE},
如果 \carg{num_objects} 是零,而 \carg{mem_objects} 不是 \cmacro{NULL};
或者 \carg{num_objects} > 0 而 \carg{mem_objects} 是 \cmacro{NULL}。

\item \cenum{CL_INVALID_MEM_OBJECT},
如果 \carg{mem_objects} 中的\cnglo{memobj}無效。

\item \cenum{CL_INVALID_COMMAND_QUEUE},
如果 \carg{command_queue} 無效。

\item \cenum{CL_INVALID_CONTEXT},
如果 \carg{command_queue} 所在\cnglo{context}不是由 OpenGL \cnglo{context}創建的。

\item \cenum{CL_INVALID_GL_OBJECT},
如果 \carg{mem_objects} 中的\cnglo{memobj}不是由 GL 對象創建的。

\startitem
\cenum{CL_INVALID_EVENT_WAIT_LIST},如果滿足下列條件中的任一項:
\startigBase
\item \carg{event_wait_list} 是 \cmacro{NULL},但 \carg{num_events_in_wait_list} > 0;
\item 或者 \carg{event_wait_list} 不是 \cmacro{NULL},但 \carg{num_events_in_wait_list} 是 0;
\item 或者 \carg{event_wait_list} 中有無效的事件。
\stopigBase
\stopitem

\item \cenum{CL_OUT_OF_RESOURCES},如果\scdevfailres。

\item \cenum{CL_OUT_OF_HOST_MEMORY},如果\schostfailres。
\stopigBase

% Synchronizing OpenCL and OpenGL Access to Shared Objects
\subsubsection[section:syncCLGL]{OpenCL 和 OpenGL 存取共享對象時的同步}

為確保數據完整性,\cnglo{app}要負責通過各自的 API 對共享 CL/GL 對象的存取進行同步。
如果沒有提供同步,則會導致竟態以及其他\cnglo{undef}的行為,包括不可在實作間進行移植。

在調用 \clapi{clEnqueueAcquireGLObjects} 之前,
\cnglo{app}必須確保所有會存取 \carg{mem_objects} 中對象的處於擱置狀態的 GL 操作都完成了。
這可以通過在所有引用這些對象的 GL \cnglo{context}上發起 \clapi{glFinish} 並等待其完成來實現,
同時這種方式也是可移植的。
實作可能會提供其他更加有效的同步方法;
例如,在一些\cnglo{platform}上,調用 \clapi{glFlush} 可能就已經足夠了,
或者在線程中已經有隱式的同步,
或者在特定供應商擴展中可以在 GL 命令流間放置隔柵(fence)以等待 CL \cnglo{cmdq}中的此隔柵完成。
注意:目前只有 \clapi{glFinshi} 才是可移植的,其他方法都不是。

同樣,在調用 \clapi{clEnqueueReleaseGLObjects} 之後,
要想執行後續引用這些對象的 GL 命令,
\cnglo{app}必須確保所有會存取 \carg{mem_objects} 中對象的處於擱置狀態的 CL 操作都完成了。
這可以通過調用 \clapi{clWaitForEvents} 來等待 \clapi{clEnqueueReleaseGLObjects} 所返回的\cnglo{evtobj},或者調用 \clapi{clFinish} 來實現,
這兩種方式都是可移植的。
跟上面一樣,一些實作可能會提供更加有效的方式。

如果 CL 和 GL \cnglo{context} 處於不同的線程中,則\cnglo{app}需要負責維護操作間的正確次序。

如果 GL \cnglo{context}所綁定的線程不是調用 \clapi{clEnqueueReleaseGLObjects} 的線程,
在\cnglo{app}沒有執行其他附加步驟的情況下,
任何 \carg{mem_objects} 中對象的改變對那個\cnglo{context}都是不可見的。
對於 OpenGL 3.1 (或者更高版本)中的\cnglo{context}而言,
OpenGL 3.1 規範中的{\ftRef{附錄 D}}(共享對象以及多個\cnglo{context})描述了相關的要求。
而對於 OpenGL 的早期版本,相關要求\cnglo{impdef}。

在 OpenCL 獲取 OpenGL 對象後並將其釋放之前,
如果 OpenGL 嘗試存取此對象的數據會導致\cnglo{undef}的行為。
類似的,在 OpenCL 獲取共享的 CL/GL 對象前,或者將其釋放後,
如果 OpenCL 嘗試存取此對象也會導致\cnglo{undef}的行為。




\section[section:clEvtObjFromGlSync]{由 GL 同步對象創建 CL 事件對象}

% Overview
\subsection{簡介}

本擴展允許由 OpenGL 隔柵同步對象創建 OpenCL \cnglo{evtobj},
從而潛在地提高在兩種 API 間共享圖像和緩衝的效率。
對應的擴展 \clext{GL_ARB_cl_event} 可以由 OpenCL \cnglo{evtboj}創建 OpenGL 同步對象,
他可以與本擴展互補。

另外,本擴展修改了 \clapi{clEnqueueAcquireGLObjects} 和 \clapi{clEnqueueReleaseGLObjects} 的行為,
如果 OpenGL \cnglo{context} 與 OpenCL \cnglo{context} 綁定到了同一個線程中,
則他可以隱式保證其同步。

如果實作支持本擴展,
則 \cenum{CL_PLATFORM_EXTENSIONS} 或 \cenum{CL_DEVICE_EXTENSIONS} 中
應該有字串 \clext{cl_khr_gl_event},
參見\reftab{cldevquery}。

% New Procedures and Functions
\subsection{新例程和新函式}

\startCLFUNC
cl_event clCreateEventFromGLsyncKHR (
			cl_context context,
			GLsync sync,
			cl_int *errcode_ret);
\stopCLFUNC

% New Tokens
\subsection{新的符記}

調用 \clapi{clGetEventInfo} 時
如果 \carg{param_name} 是 \cenum{CL_EVENT_COMMAND_TYPE},則會返回:
\startclc
CL_COMMAND_GL_FENCE_SYNC_OBJECT_KHR	0x200D
\stopclc

% Additions to Chapter 5 of the OpenCL 1.2 Specification
\subsection{對第五章的補充}

將下面內容添加到\insection[evtObj]的第四段中(\clapi{clCreateUserEvent} 的描述之前):
\startreplacepar
\cnglo{evtobj}也可用來反映 OpenGL 同步對象的狀態。
同步對象指的是 OpenGL 命令流中所執行的隔柵(fence)命令。
這為在 OpenGL 和 OpenCL 間共享緩衝和圖像提供了一種新方法(參見\insection[syncCLGL])。
\stopreplacepar

在\reftab{clGetEventInfo}關於 \clapi{clGetEventInfo} 的描述中,
\carg{param_name} 為 \cenum{CL_EVENT_COMMAND_TYPE} 時,
所返回的 \carg{param_value} 的值增加一項: \cenum{CL_COMMAND_GL_FENCE_SYNC_OBJECT_KHR}。

新添{\ftRef{節 5.9.1}} {\ftEmp{將\cnglo{evtobj}鏈接到 OpenGL 同步對象上}}:
\startreplacepar
可以通過鏈接 OpenGL {\ftEmp{同步對象}}來創建\cnglo{evtobj}。
這種\cnglo{evtobj}的完成就相當於等待與所鏈接 GL 同步對象相關聯隔柵命令的完成。

\topclfunc{clCreateEventFromGLsyncKHR}

\startCLFUNC
cl_event clCreateEventFromGLsyncKHR (
			cl_context context,
			GLsync sync,
			cl_int *errcode_ret)
\stopCLFUNC

此函式會創建一個帶鏈接的\cnglo{evtobj}。

\carg{context} 是一個利用擴展 \clext{cl_khr_gl_sharing},
由 OpenGL \cnglo{context}或共享組創建的的 OpenCL \cnglo{context}。

\carg{sync} 是 \carg{context} 所關聯 GL 共享組中同步對象的名字。

如果成功創建了\cnglo{evtobj},則 \clapi{clCreateEventFromGLsyncKHR} 會將其返回,
並將 \carg{errcode_ret} 置為 \cenum{CL_SUCCESS}。
否則,返回 \cmacro{NULL},並將 \carg{errcode_ret} 置為下列錯誤碼之一:
\startigBase
\item \cenum{CL_INVALID_CONTEXT},
如果 \carg{context} 無效,或者不是由 GL \cnglo{context}創建的。

\item \cenum{CL_INVALID_GL_OBJECT},
如果 \carg{sync} 不是 \carg{context} 所關聯 GL 共享組中同步對象的名字。
\stopigBase

對於這種\cnglo{evtobj}調用 \clapi{clGetEventInfo} 時會返回下列值:
\startigBase
\item 如果查詢的是 \cenum{CL_EVENT_COMMAND_QUEUE},則結果為 \cmacro{NULL},
因為此事件沒有與任何 OpenCL \cnglo{cmdq}關聯。

\item 如果查詢的是 \cenum{CL_EVENT_COMMAND_TYPE},
則結果是 \cenum{CL_COMMAND_GL_FENCE_SYNC_OBJECT_KHR},
表明此事件關聯的是 GL 同步對象,而不是 OpenCL \cnglo{cmd}。

\item 如果查詢的是 \cenum{CL_EVENT_COMMAND_EXECUTION_STATUS},
則結果要麼是 \cenum{CL_SUBMITTED},表明同步對象所關聯的隔柵\cnglo{cmd}還未完成,
要麼是 \cenum{CL_COMPLETE},表明隔柵\cnglo{cmd}完成了。
\stopigBase

\clapi{clCreateEventFromGLsyncKHR} 會在返回的\cnglo{evtobj}上實施隱式的 \clapi{clRetainEvent}。
創建這種帶鏈接的\cnglo{evtobj}時也會在所鏈接的 GL 同步對象上放置一個引用。
當這種\cnglo{evtobj}被刪除時,這個引用也會被移除。

\clapi{clCreateEventFromGLsyncKHR} 所返回的事件可能
只能由 \clapi{clEnqueueAcquireGLObjects} 使用。
將這種事件傳遞給其他 CL API 會生成錯誤 \cenum{CL_INVALID_EVENT}。
\stopreplacepar % stop replace par

% Additions to Chapter 9 of the OpenCL 1.2 Specification
\subsection{對第九章的補充}

對 \clapi{clEnqueueAcquireGLObjects} 的參數 \carg{event} 增加以下描述:
\startreplacepar
如果一個 OpenGL \cnglo{context}綁定到了當前線程上,
那麼同時滿足下列兩個條件的 OpenGL \cnglo{cmd}會在執行
緊跟 \clapi{clEnqueueAcquireGLObjects} 的所有 OpenCL \cnglo{cmd}
(這些 OpenCL \cnglo{cmd}會影響或存取 \carg{mem_objects} 中的\cnglo{memobj})前完成:
\startigNum[indentnext=no]
\item 會影響或存取 \carg{mem_objects} 中\cnglo{memobj}的內容;
\item 是在調用 \clapi{clEnqueueAcquireGLObjects} 之前在此 OpenGL \cnglo{context}上發起的。
\stopigNum
如果所返回的\cnglo{evtobj}不是 \cmacro{NULL},
則只有當這樣的 OpenGL \cnglo{cmd}完成後,他才會報告完成。
\stopreplacepar

對 \clapi{clEnqueueReleaseGLObjects} 的參數 \carg{event} 增加以下描述:
\startreplacepar
如果一個 OpenGL \cnglo{context}綁定到了當前線程上,
那麼只有當 \clapi{clEnqueueReleaseGLObjects} 之前的所有 OpenCL \cnglo{cmd}
(這些 OpenCL \cnglo{cmd}會影響或存取 \carg{mem_objects} 中的\cnglo{memobj})
執行完畢後,
同時滿足下列兩個條件的 OpenGL \cnglo{cmd}才會執行:
\startigNum[indentnext=no]
\item 會影響或存取 \carg{mem_objects} 中\cnglo{memobj}的內容;
\item 是在調用 \clapi{clEnqueueReleaseGLObjects} 之後在此 OpenGL \cnglo{context}上發起的。
\stopigNum
如果所返回的\cnglo{evtobj}不是 \cmacro{NULL},
則只有當他報告完成後,才會執行這些 OpenGL \cnglo{cmd}。
\stopreplacepar

用下列內容取代\insection[syncCLGL]的第二段:
\startreplacepar
調用 \clapi{clEnqueueAcquireGLObjects} 之前,
\cnglo{app}必須確保會存取 \carg{mem_objects} 中對象並且處於擱置狀態的 OpenGL 操作全部完成。

如果支持擴展 \clext{cl_khr_gl_event},
並且 OpenGL \cnglo{context}與 OpenCL \cnglo{context}綁定到了同一線程上,
則 OpenCL 實作會確保(針對此 OpenGL \cnglo{context})這種擱置的 OpenGL 操作全部完成。
這也叫{\ftRef{隱式同步}}。

如果支持擴展 \clext{cl_khr_gl_event},
並且所談及的 OpenGL \cnglo{context}支持隔柵同步對象,
要想確定 OpenGL \cnglo{cmd}完成了,
可以用 \capi{glFenceSync} 在那些\cnglo{cmd}後面放置一個 GL 隔柵\cnglo{cmd},
然後用 \clapi{clCreateEventFromGLsyncKHR} 由所產生的 GL 同步對象創建一個事件,
並通過 \clapi{clEnqueueAcquireGLObjects} 來確定這個\cnglo{evtobj}完成了。
這種方法可能比 \clapi{glFinish} 更加高效,
稱作{\ftRef{顯式同步}}。
當 OpenGL \cnglo{context}綁定的是存取\cnglo{memobj}的另一個線程時,
顯式同步最有用。

如果支持擴展 \clext{cl_khr_gl_event},
要想確定 OpenGL \cnglo{cmd}完成了,
可以在所有帶有對這些對象的擱置引用的 OpenGL \cnglo{context}上
發起並等待\cnglo{cmd} \clapi{glFinish} 的完成。
一些實作可能提供其他有效的同步方法。
如果存在這樣的方法,會在特定\cnglo{platform}的文檔中對其進行描述。

注意,對於所有 OpenGL 實作以及所有 OpenCL 實作,
只有 \clapi{glFinish} 才是可移植的,其他方法都不是。
鑒於 \clapi{glFinish} 是一種代價高昂的操作,
如果在某個\cnglo{platform}上支持擴展 \clext{cl_khr_gl_event},
應盡量避免使用 \clapi{glFinish}。
\stopreplacepar

% Issues
\subsection{問題}

\startQUESTION
如何處理 CL 事件和 GL 同步的相互引用?
\stopQUESTION
\startANSWER
已有提案:帶鏈接的 CL 事件會在 GL 同步對象上放置一個引用。
刪除 CL 事件時會移除這個引用。
還有一些代價更高的方案可以通過 GL 同步來反映 CL 事件\cnglo{refcnt}的變化。
\stopANSWER

\startQUESTION
在其他 API 中如何處理到同步基元的鏈接?
\stopQUESTION
\startANSWER
還未解決。
我們想至少要有一種方式可以將事件鏈接到 EGL 同步對象上。
可能沒有模擬 DX 的概念。
對於所鏈接的每種同步基元應當都有一個入口點,
如 \clapi{clCreateEventFromEGLSyncKHR}。

另一種方案就是通用的 \clapi{clCreateEventFromExternalEvent},可以接受一個特性列。
這個特性列中可以包含外部基元的類型以及其附屬資訊
(GL 同步對象句柄、 EGL display 以及同步對象句柄,等等)。
這樣就可以重用同一個入口點。

這些可能會作為一個獨立的擴展。
\stopANSWER

\startQUESTION
\cenum{CL_EVENT_COMMAND_TYPE} 對應的是\cnglo{cmd}(隔柵)的類型還是
所鏈接同步對象的類型?
\stopQUESTION
\startANSWER
已有提案:所鏈接同步對象的類型。
\stopANSWER

\startQUESTION
是否要同時支持顯式同步和隱式同步?
\stopQUESTION
\startANSWER
已有提案:是的。
隱式同步適用於 GL 和 CL 在同一\cnglo{app}線程中執行的情況。
而顯式同步則適用於在不同線程中執行、但 \capi{glFinish} 代價又太高的情況。
\stopANSWER

\startQUESTION
本擴展是\cnglo{platform}擴展還是\cnglo{device}擴展?
\stopQUESTION
\startANSWER
已有提案:\cnglo{platform}擴展。
這樣的話,要想僅僅使用公開的 GL API 來實現 sync->event 語義需要做大量的工作;
但是,同一運行時中可能會有多個對 GL 支持層級不同的驅動和\cnglo{device}同時存在。
\stopANSWER

\startQUESTION
什麼地方才能使用由 GL 同步對象生成的事件?
\stopQUESTION
\startANSWER
已有提案:僅當調用 \clapi{clEnqueueAcquireGLObjects} 時才能使用,
其他任何地方使用這種事件都會產生錯誤。
其他地方也沒有明確的用例,而且要想支持他還有一個成本問題,
在所有其他將其作為參數的\cnglo{cmd}中檢查事件源都要有相應的成本,
經過權衡,採用了目前的方案。
\stopANSWER


\section{與 Direct3D 10 共享內存對象}

\subsection{簡介}

本擴展是為了使得在 OpenCL 和 Direct3D 10 間可以進行互操作。
其運作方式與\refsection{clShareGl}和\refsection{clEvtObjFromGlSync}中的 OpenGL 類似。
如果實作支持本擴展,
則 \cenum{CL_PLATFORM_EXTENSIONS} 或 \cenum{CL_DEVICE_EXTENSIONS} 中
應包含字串 \clext{cl_khr_d3d10_sharing},參見\reftab{cldevquery}。

\subsection{頭檔}

目前本擴展中的所有接口都由 \ccmm{cl_d3d10.h} 提供。

\subsection{新例程和新函式}

\startCLFUNC
cl_int clGetDeviceIDsFromD3D10KHR (
		cl_platform_id platform,
		cl_d3d10_device_source_khr d3d_device_source,
		void *d3d_object,
		cl_d3d10_device_set_khr d3d_device_set,
		cl_uint num_entries,
		cl_device_id *devices,
		cl_uint *num_devices)
cl_mem clCreateFromD3D10BufferKHR (
		cl_context context,
		cl_mem_flags flags,
		ID3D10Buffer *resource,
		cl_int *errcode_ret)
cl_mem clCreateFromD3D10Texture2DKHR (
		cl_context context,
		cl_mem_flags flags,
		ID3D10Texture2D *resource,
		UINT subresource,
		cl_int *errcode_ret)
cl_mem clCreateFromD3D10Texture3DKHR (
		cl_context context,
		cl_mem_flags flags,
		ID3D10Texture3D *resource,
		UINT subresource,
		cl_int *errcode_ret)
cl_int clEnqueueAcquireD3D10ObjectsKHR (
		cl_command_queue command_queue,
		cl_uint num_objects,
		const cl_mem *mem_objects,
		cl_uint num_events_in_wait_list,
		const cl_event *event_wait_list,
		cl_event *event)
cl_int clEnqueueReleaseD3D10ObjectsKHR (
		cl_command_queue command_queue,
		cl_uint num_objects,
		const cl_mem *mem_objects,
		cl_uint num_events_in_wait_list,
		const cl_event *event_wait_list,
		cl_event *event)
\stopCLFUNC

\subsection{新符記}

\clapi{clGetDeviceIDsFromD3D10KHR} 的參數 \carg{d3d_device_source} 的值可以是:
\startclc
CL_D3D10_DEVICE_KHR		0x4010
CL_D3D10_DXGI_ADAPTER_KHR	0x4011
\stopclc

\clapi{clGetDeviceIDsFromD3D10KHR} 的參數 \carg{d3d_device_set} 的值可以是:
\startclc
CL_PREFERRED_DEVICES_FOR_D3D10_KHR	0x4012
CL_ALL_DEVICES_FOR_D3D10_KHR		0x4013
\stopclc

\clapi{clCreateContext} 和 \clapi{clCreateContextFromType} 的
參數 \carg{properties} 可以是:
\startclc
CL_CONTEXT_D3D10_DEVICE_KHR	0x4014
\stopclc

\clapi{clGetContextInfo} 的參數 \carg{param_name} 可以是:
\startclc
CL_CONTEXT_D3D10_PREFER_SHARED_RESOURCES_KHR	0x402C
\stopclc

\clapi{clGetMemObjectInfo} 的參數 \carg{param_name} 可以是:
\startclc
CL_MEM_D3D10_RESOURCE_KHR	0x4015
\stopclc

\clapi{clGetImageInfo} 的參數 \carg{param_name} 可以是:
\startclc
CL_IMAGE_D3D10_SUBRESOURCE_KHR	0x4016
\stopclc

當 \carg{param_name} 是 \cenum{CL_EVENT_COMMAND_TYPE} 時,
 \clapi{clGetEentInfo} 的參數 \carg{param_alue} 可以是:
\startclc
CL_COMMAND_ACQUIRE_D3D10_OBJECTS_KHR	0x4017
CL_COMMAND_RELEASE_D3D10_OBJECTS_KHR	0x4018
\stopclc

如果指定的 Direct3D 10 \cnglo{device}與\cnglo{context}中的\cnglo{device}不兼容,
則 \clapi{clCreateContext} 和 \clapi{clCreateContextFromType} 會返回:
\startclc
CL_INVALID_D3D10_DEVICE_KHR		-1002
\stopclc

\clapi{clCreateFromD3D10BufferKHR} 的參數 \carg{resource} 必須是 Direct3D 10 \cnglo{bufobj},
而 \clapi{clCreateFromD3D10Texture2DKHR} 和 \clapi{clCreateFromD3D10Texture3DKHR} 的
參數 \carg{resource} 必須是 Direct3D 10 材質對象,
否則這些函式會返回:
\startclc
CL_INVALID_D3D10_RESOURCE_KHR		-1003
\stopclc

如果參數 \carg{mem_objects} 中的任一對象已經由 OpenCL 獲取了,
那麼 \clapi{clEnqueueAcquireD3D10ObjectsKHR} 會返回:
\startclc
CL_D3D10_RESOURCE_ALREADY_ACQUIRED_KHR	-1004
\stopclc

如果參數 \carg{mem_objects} 中的任一對象已經由 OpenCL 獲取了,
那麼 \clapi{clEnqueueReleaseD3D10ObjectsKHR} 會返回:
\startclc
CL_D3D10_RESOURCE_NOT_ACQUIRED_KHR	-1005
\stopclc

% Additions to Chapter 4 of the OpenCL 1.2 Specification
\subsection{對第 4 章的補充}

\refsection{contexts}中,用下列內容取代 \clapi{clCreateContext} 後面對 \carg{properties} 的描述:
\startreplacepar
\carg{properties} 指向一個特性列,即 \ccmm{<特性名, 值>} 的陣列,以零終止。
如果此陣列中沒有某個特性,則使用其缺省值,參見\reftab{prptForclCreateContext}。
如果 \carg{properties} 是 \cmacro{NULL} 或者是空的(第一個值就是零),所有特性都使用缺省值。
\stopreplacepar

\reftab{prptCtx_D3D10}是對\reftab{prptForclCreateContext}的補充。

\placetable[here][tab:prptCtx_D3D10]
{創建上下文時所用特性}
{\startETD[cl_context_properties][屬性值]

\clETD{CL_CONTEXT_D3D10_DEVICE_KHR}{ID3D10Device *}{
使用 \cldt{ID3D10Device *} 來與 Direct3D 10 進行互操作。
缺省值為 \cmacro{NULL}。
}

\stopETD

}

下列內容作為對 \clapi{clCreateContext} 所返回的錯誤列的補充:
\startreplacepar
\startigBase
\item \cenum{CL_INVALID_D3D10_DEVICE_KHR},
如果屬性 \cenum{CL_CONTEXT_D3D10_DEVICE_KHR} 的值不是 \cmacro{NULL},
並且沒有指定一個有效的 Direct3D 10 \cnglo{device}(\carg{cl_device_ids})。

\item \cenum{CL_INVALID_OPERATION},
如果屬性 \cenum{CL_INVALID_D3D10_DEVICE_KHR} 的值不是 \cmacro{NULL},
以表明要與 Direct3D 10 進行互操作,但同時又指定了要與其他圖形 API 進行互操作。
\stopigBase
\stopreplacepar

\clapi{clCreateContextFromType} 所返回的錯誤列中也要添加上面所描述的兩個新錯誤。

\reftab{paramNameclGetContextInfo}要增加\reftab{prpt_clGetContextInfo_d3d10}中所列內容:

\placetable[here][tab:prpt_clGetContextInfo_d3d10]
{\clapi{clGetContextInfo} 所支持的屬性名}
{\startETD[cl_context_info][返回型別]

\clETD{CL_CONTEXT_D3D10_PREFER_SHARED_RESOURCES_KHR}{cl_bool}{
在與 OpenCL 共享 Direct3D 10 資源時,
如果 \carg{MiscFlags} 中設置了 \cenum{D3D10_RESOURCE_MISC_SHARED} (創建時指定)的資源
比沒有設置此標誌的更快,則返回 \cenum{CL_TRUE},
否則返回 \cenum{CL_FALSE}。
}

\stopETD

}

% Additions to Chapter 5 of the OpenCL 1.2 Specification
\subsection{對第五章的補充}

為 \clapi{clGetMemObjectInfo} 所返回的錯誤碼添加以下內容:
\startreplacepar
\startigBase
\startitem
\cenum{CL_INVALID_D3D10_RESOURCE_KHR},
如果 \carg{param_name} 是 \cenum{CL_MEM_D3D10_RESOURCE_KHR},
並且 \carg{memobj} 不是由以下函式創建的:
\startigBase
\item \clapi{clCreateFromD3D10BufferKHR},
\item \clapi{clCreateFromD3D10Texture2DKHR},或
\item \clapi{clCreateFromD3D10Texture3DKHR}。
\stopigBase
\stopitem
\stopigBase
\stopreplacepar

將\reftab{meminfo_d3d10}中的內容添加到\reftab{meminfo}中。

\placetable[here][tab:meminfo_d3d10]
{\capi{clGetMemObjectInfo}所支持的\carg{param_names}}
{\startETD[cl_mem_info][返回类型]

\clETD{CL_MEM_D3D10_RESOURCE_KHR}{ID3D10Resource *}{
如果 \carg{memobj} 是用以下函式創建的,
則返回創建 \carg{memobj} 時所指定的引數 \carg{resource}:
\startigBase
\item \cenum{clCreateFromD3D10BufferKHR},
\item \cenum{clCreateFromD3D10Texture2DKHR},或
\item \cenum{clCreateFromD3D10Texture3DKHR}。
\stopigBase
}

\stopETD

}

為 \clapi{clGetImageInfo} 所返回的錯誤碼添加以下內容:
\startreplacepar
\startigBase
\startitem
\cenum{CL_INVALID_D3D10_RESOURCE_KHR},
如果 \carg{param_name} 是 \cenum{CL_MEM_D3D10_SUBRESOURCE_KHR},
並且 \carg{image} 不是由以下函式創建的:
\startigBase
\item \clapi{clCreateFromD3D10Texture2DKHR},或
\item \clapi{clCreateFromD3D10Texture3DKHR}。
\stopigBase
\stopitem
\stopigBase
\stopreplacepar

將\reftab{clgetimginfo_d3d10}中的內容添加到\reftab{clgetimginfo}中。

\placetable[here][tab:clgetimginfo_d3d10]
{\capi{clGetImageInfo}所支持的\carg{param_names}}
{\startETD[cl_image_info][返回型別]

\clETD{CL_MEM_D3D10_SUBRESOURCE_KHR}{ID3D10Resource *}{
如果 \carg{image} 是用以下函式創建的,
則返回創建 \carg{image} 時所指定的引數 \carg{subresource}:
\startigBase
\item \cenum{clCreateFromD3D10Texture2DKHR},或
\item \cenum{clCreateFromD3D10Texture3DKHR}。
\stopigBase
}

\stopETD
}

在\reftab{clGetEventInfo}中,
當 \carg{cl_event_info} = \cenum{CL_EVENT_COMMAND_TYPE} 時,
對其返回值的描述增加以下內容:
\startreplacepar
\startigBase
\item \cenum{CL_COMMAND_ACQUIRE_D3D10_OBJECTS_KHR}
\item \cenum{CL_COMMAND_RELEASE_D3D10_OBJECTS_KHR}
\stopigBase
\stopreplacepar

% Sharing Memory Objects with Direct3D 10 Resources
\subsection{與 Direct3D 10 資源共享內存對象}

\cnglo{app}可以使用本節所討論 OpenCL 函式將 Direct3D 10 資源當成 OpenCL \cnglo{memobj}使用。
這樣就可以在 OpenCL 和 Direct3D 10 之間有效的共享數據。
由 OpenCL API 所執行的\cnglo{kernel}可以通過讀寫\cnglo{memobj}來操控 Direct3D 10 資源。
可以由 Direct3D 10 材質資源創建 OpenCL \cnglo{imgobj};
也可以由 Direct3D 10 緩衝資源創建 OpenCL \cnglo{bufobj};
但是這樣做有個前提:當且僅當 OpenCL \cnglo{context}是在 Direct3D 10 設備上創建的。

\topclfunc{clGetDeviceIDsFromD3D10KHR}

\startCLFUNC
cl_int clGetDeviceIDsFromD3D10KHR(
		cl_platform_id platform,
		cl_d3d10_device_source_khr d3d_device_source,
		void *d3d_object,
		cl_d3d10_device_set_khr d3d_device_set,
		cl_uint num_entries,
		cl_device_id *devices,
		cl_uint *num_devices)
\stopCLFUNC

\carg{platform} 即 \clapi{clGetPlatformIDs} 所返回的\cnglo{platform} ID。

\carg{d3d_device_source} 指定了 \carg{d3d_object} 的類型,
且必須是\reftab{dxten_device_type} 中的一個。

\carg{d3d_object} 即所查詢 OpenCL \cnglo{device}所對應的對象。
其類型必須是\reftab{dxten_device_type} 中的一個。

\carg{d3d_device_set} 指定所返回的\cnglo{device}集,必須是\reftab{dxten_device_set} 中的值。

\carg{num_entries} 即 \ctype{cl_device_id} 條目(可以添加到 \carg{devices} 中)的數目,
如果 \carg{devices} 不是 \cmacro{NULL},則 \carg{num_entries} 必須大於零。

\carg{devices} 返回所找到的 OpenCL \cnglo{device}清單。
其中的 \ctype{cl_device_id} 可以用來識別一個特定的 OpenCL \cnglo{device}。
如果 \carg{devices} 是 \cmacro{NULL},則忽略此參數。
所返回的 OpenCL \cnglo{device}數量是 \carg{num_entries} 的值
和 \carg{d3d_object} 所對應 OpenCL \cnglo{device}的數量中較小的那個。

\carg{num_devices} 返回的是 \carg{d3d_object} 所對應的所有 OpenCL \cnglo{device}的數目。
如果 \carg{num_devices} 是 \cmacro{NULL},則忽略此參數。

如果執行成功,則 \clapi{clGetDeviceIDsFromD3D10KHR} 會返回 \cenum{CL_SUCCESS};
否則,返回下列錯誤碼之一:
\startigBase
\item \cenum{CL_INVALID_PLATFORM},如果 \carg{platform} 無效。

\item \cenum{CL_INVALID_VALUE},如果 \carg{d3d_device_source} 的值無效、
\carg{d3d_device_set} 的值無效、 \carg{num_entries} 爲零但 \carg{devices} 不是 \cmacro{NULL},
或者 \carg{num_devices} 和 \carg{devices} 都是 \cmacro{NULL}。

\item \cenum{CL_DEVICE_NOT_FOUND},
如果沒有找到任何與 \carg{d3d_object} 所對應的 OpenCL \cnglo{device}。
\stopigBase

\placetable[here][tab:dxten_device_type]
{\clapi{clGetDeviceIDsFromD3D10KHR} 所查詢的 OpenCL \cnglo{device}種類與對應對象的型別}
{\startCLOO[\ctype{cl_d3d_device_source_khr}][\carg{d3d_object} 的型別]

\clOO{\cenum{CL_D3D10_DEVICE_KHR}}{\ctype{ID3D10Device *}}
\clOO{\cenum{CL_D3D10_DXGI_ADAPTER_KHR}}{\ctype{IDXGIAdapter *}}

\stopCLOO
}

\placetable[here][tab:dxten_device_set]
{\clapi{clGetDeviceIDsFromD3D10KHR} 所查詢的 OpenCL \cnglo{device}集}
{\startCLOD[\ctype{cl_d3d_device_set_khr}][\carg{devices} 中返回的\cnglo{device}]

\clOO{\cenum{CL_PREFERRED_DEVICES_FOR_D3D10_KHR}}
{與指定 Direct3D 對象相關聯的 OpenCL \cnglo{device}。}

\clOO{\cenum{CL_ALL_DEVICES_FOR_D3D10_KHR}}
{可以與指定 Direct3D 對象進行互操作的所有 OpenCL \cnglo{device}。
在這些\cnglo{device}上共享數據的性能可能比 preferred \cnglo{device}有相當的下降。}

\stopCLOD
}

\todo{TODO:}

% Issues
\subsection{問題}

\todo{TODO:}


% DX9 Media Surface Sharing
\section{DX9 media surface 共享}

\todo{TODO:}


% Sharing Memory Objects with Direct3D 11
\section{與 Direct3D 11 共享內存對象}

\todo{TODO:}


% OpenCL Installable Client Driver (ICD)
\section{OpenCL 可安裝的客戶端驅動(ICD)}

\subsection{簡介}

這是一個\cnglo{platform}擴展,其中定義了一種比較簡單的機制,
使得 Khronos OpenCL ICD 裝載器可以將多個獨立的 Vendor ICD 暴露給 OpenCL。
針對 ICD 裝載器編寫的\cnglo{app}可以存取所有供應商實作暴露出來的所有 \cldt{cl_platform_ids},
其中 ICD 裝載器扮演譯碼器的角色。
如果實作支持本擴展,
則 \cenum{CL_PLATFORM_EXTENSIONS} 所對應的字串中應當包含 \clext{cl_khr_icd},
參見\reftab{cldevquery}。

% Inferring Vendors from Function Calls from Arguments
\subsection{如何調用供應商提供的函式}

對於任一 OpenCL 函式調用,
 ICD 裝載器都是用其引數調用供應商所實現的函式。
當一個對象是如下結構體時,就說他是與 ICD 相兼容的:
\startclc[indentnext=no]
struct _cl_<object>
{
	struct _cl_icd_dispatch *dispatch;
	// ... remainder of internal data
};
\stopclc
其中 \ccmm{<object>} 是下列字串之一:
\startigBase
\item \ccmm{platform_id}、
\item \ccmm{device_id}、
\item \ccmm{context}、
\item \ccmm{command_queue}、
\item \ccmm{mem}、
\item \ccmm{program}、
\item \ccmm{kernel}、
\item \ccmm{event}、或
\item \ccmm{sampler}。
\stopigBase

結構體 \cldt{_cl_icd_dispatch} 是一個函式指位器分派表,
用來直接調用特定供應商的實作。
所有由 ICD 兼容對象所創建的對象都必須是 ICD 兼容的。

\cldt{_cl_icd_dispatch} 中函式的次序由 ICD 裝載器的源碼確定。
此結構體只能在後面追加新內容。

不帶引數的函式會將引數部分忽略,但 \clapi{clGetExtensionFunctionAddress} 例外,後面有詳細介紹。

% ICD Data
\subsection{ICD 數據}

Vendor ICD 由兩部分數據定義:
\startigBase
\item Vendor ICD 程式庫,包含 OpenCL 實作所實現的所有 OpenCL 入口點。
此程式庫的檔案名應當包含供應商的名字或特定供應商的實作 ID。

\item Vendor ICD 擴展後綴,一個短字串,
此供應商所實現的所有擴展缺省都帶有此後綴。
這個供應商後綴是可選的。
\stopigBase

% ICD Loader Vendor Enumeration on Windows
\subsection{Windows 上 ICD 裝載器所用的供應商列表}

在 Windows 上,為了列舉 Vendor ICD, ICD 裝載器會
掃描註冊表中的鍵值 \ccmm{HKEY_LOCAL_MACHINE\SOFTWARE\Khronos\OpenCL\Vendors}。
對於此鍵值中的任一 DWORD 數據,
只要其值為 0, ICD 裝載器就會用 \clapi{LoadLibraryA} 打開其名字對應的動態鏈接庫。

例如,如果註冊表包含下列值:
\startclc[indentnext=no]
[HKEY_LOCAL_MACHINE\SOFTWARE\Khronos\OpenCL\Vendors]
"c:\\vendor a\\vndra_ocl.dll"=dword:00000000
\stopclc
那麼 ICD 就會打開程式庫“\ccmm{c:\vendor a\vndra_ocl.dll}”。


% ICD Loader Vendor Enumeration on Linux
\subsection{Linux 上 ICD 裝載器所用的供應商列表}

在 Linux 上,為了列舉 Vendor ICD, ICD 裝載器會
掃描路徑 \ccmm{/etc/OpenCL/vendors} 中的所有檔案。
對於此路徑中的任一檔案, ICD 裝載器都會以純文本模式打開此檔案。
此檔案的格式應該是一行文本,指定了 Vendor ICD 的程式庫。
 ICD 裝載器會試圖用 \capi{dlopen()} 打開此程式庫。
指定程式庫時可能是絕對路徑,或者僅僅是檔案名。

例如,存在檔案 \ccmm{/etc/OpenCL/vendors/VendorA.icd},
其中包含文本 \ccmm{libVendorAOpenCL.so},
則 ICD 裝載器會裝載程式庫“\ccmm{libVendorAOpenCL.so}”。

% Adding a Vendor Library
\subsection{添加供應商程式庫}

在成功裝載 Vendor ICD 的程式庫後,
 ICD 裝載器會查詢如下函式:
\startigBase[indentnext=no]
\item \cenum{clIcdGetPlatformIDsKHR},
\item \cenum{clGetPlatformInfo},和
\item \cenum{clGetExtensionFunctionAddress}。
\stopigBase
如果其中任一函式不存在,則 ICD 裝載器會關閉此程式庫並將其忽略。

接下來, ICD 裝載器會用 \clapi{clIcdGetPlatformIDsKHR} 來
查詢程式庫中使能了 ICD 的可用\cnglo{platform}。
對於每個這樣的\cnglo{platform},
 ICD 裝載器都會查詢其擴展字串來驗證他確實支持 \clext{cl_khr_icd},
然後以 \cenum{CL_PLATFORM_ICD_SUFFIX_KHR} 調用 \clapi{clGetPlatformInfo} 查詢
其 Vendor ICD 擴展後綴,

如果上面任一步驟失敗, ICD 裝載器就會忽略這個 Vendor ICD 並繼續裝載下一個。

% New Procedures and Functions
\subsection{新例程和新函式}

\startCLFUNC
cl_int clIcdGetPlatformIDsKHR (
			cl_uint num_entries,
			cl_platform_id *platforms,
			cl_uint *num_platforms);
\stopCLFUNC

% New Tokens
\subsection{新符記}

函式 \clapi{clGetPlatformInfo} 的參數 \carg{param_name} 接受以下值:
\startclc
CL_PLATFORM_ICD_SUFFIX_KHR	0x0920
\stopclc

如果沒有找到可用\cnglo{platform},則 \clapi{clGetPlatformIDs} 會返回:
\startclc
CL_PLATFORM_NOT_FOUND_KHR	-1001
\stopclc

% Additions to Chapter 4 of the OpenCL 1.2 Specification
\subsection{對第四章的補充}

以下列內容取代\refsection{queryPlf}中對 \clapi{clGetPlatformIDs} 的返回值的描述:
\startreplacepar
如果執行成功,並且找到了至少一個可用\cnglo{platform},
則 \clapi{clGetPlatformIDs} 會返回 \cenum{CL_SUCCESS}。
如果沒有找到可用 \cnglo{platform},則返回 \cenum{CL_PLATFORM_NOT_FOUND_KHR}。
如果 \carg{num_entries} 是零,但 \carg{platforms} 不是 \cmacro{NULL};
或者 \carg{num_platforms} 和 \carg{platforms} 都是 \cmacro{NULL},
則返回 \cenum{CL_INVALID_VALUE}。
\stopreplacepar

\refsection{queryPlf}中,在 \clapi{clGetPlatformIDs} 的描述後添加以下內容:
\startreplacepar
\topclfunc{clIcdGetPlatformIDsKHR}

\startCLFUNC
cl_int clIcdGetPlatformIDsKHR (
			cl_uint num_entries,
			cl_platform_id *platforms,
			cl_uint *num_platforms);
\stopCLFUNC

此函式用來獲取可以通過 Khronos ICD 裝載器來訪問的\cnglo{platform}列表。

\carg{num_entries} 即 \carg{platforms} 中可以存放的 \cldt{cl_platform_id} 的數目。
如果 \carg{platforms} 不是 \cmacro{NULL},則 \carg{num_entries} 必須大於零。

\carg{platforms} 用來返回可以通過 Khronos ICD 裝載器來訪問的\cnglo{platform}列表。
其中的 \cldt{cl_platform_id} 是 ICD 兼容的,
可用來識別特定 OpenCL \cnglo{platform}。
如果引數 \carg{platforms} 是 \cmacro{NULL},則將其忽略。
所返回\cnglo{platform}的數目是 \carg{num_entries} 和
實際可用\cnglo{platform}數目中較小的那個。

\carg{num_platforms} 返回實際可用\cnglo{platform}的數目。
如果 \carg{num_platforms} 是 \cmacro{NULL},則將其忽略。

如果執行成功,並且至少有一個可用\cnglo{platform},
則 \clapi{clIcdGetPlatformIDsKHR} 會返回 \cenum{CL_SUCCESS}。
如果沒有找到可用 \cnglo{platform},則返回 \cenum{CL_PLATFORM_NOT_FOUND_KHR}。
如果 \carg{num_entries} 是零,但 \carg{platforms} 不是 \cmacro{NULL};
或者 \carg{num_platforms} 和 \carg{platforms} 都是 \cmacro{NULL},
則返回 \cenum{CL_INVALID_VALUE}。
\stopreplacepar

將\reftab{plf_query_icd}中的內容添加到\reftab{plfquery}中。

\placetable[here][tab:plf_query_icd]
{OpenCL 平台查询}
{\startETD[cl_platform_info][返回型別]

\clETD{CL_PLATFORM_ICD_SUFFIX_KHR}{char[]}{
函式名後綴,由 ICD 裝載器用來識別此\cnglo{platform}中的擴展函式。
}

\stopETD

}

% Additions to Chapter 9 of the OpenCL 1.2 Extension Specification
\subsection{對第九章的補充}

在\refsection{getFuncPtr}的最後添加以下內容:
\startreplacepar
對於 ICD 裝載器所支持的那些函式,
 \clapi{clGetExtensionFunctionAddress} 會返回 ICD 裝載器實作的函式指位器。
而對於 ICD 裝載所不知道的那些函式,
 \clapi{clGetExtensionFunctionAddress} 會根據傳入的字串來確定供應商實作。
 \clapi{clGetExtensionFunctionAddress} 會在 ICD 裝載器所列舉的 Vendor ICD 中進行查詢,
並且查詢時會在所查詢函式名後加上 ICD 後綴。
如果不存在這樣的供應商,或者函式後綴是 \ccmm{KHR} 或 \ccmm{EXT},
則 \clapi{clGetExtensionFunctionAddress} 會返回 \cmacro{NULL}。
\stopreplacepar

% Issues
\subsection{問題}

\startQUESTION
一些 OpenCL 函式沒有對象引數,無法據此識別其供應商程式庫(例如, \clapi{clUnloadCompiler}),
這種情況如何處理?
\stopQUESTION
\startANSWER
已解決:通過 ICD 調用這種函式時,全部為空操作。
\stopANSWER

\startQUESTION
如何處理 OpenCL 擴展?
\stopQUESTION
\startANSWER
已解決:只要有一個供應商實現了某個擴展函式,就應當立刻將其加入 ICD。
對於那些還未加入 ICD 的供應商擴展,使用後綴機制來訪問。
\stopANSWER

\startQUESTION
ICD 如何處理為 \cmacro{NULL} 的 \cldt{cl_platform_id}?
\stopQUESTION
\startANSWER
已解決: ICD 不支持 \cmacro{NULL} \cnglo{platform}。
\stopANSWER

\startQUESTION
無法卸載 ICD,是否應當提供一種機制來做此事?
\stopQUESTION
\startANSWER
已解決:由於沒有一個標準機制來卸載供應商實作,所以也不為 ICD 添加這種機制。
\stopANSWER



\stopcomponent

