\startsection[
  title={Inverting matrices},
]

%e28.2-1
\startEXERCISE
設 \m{M(n)} 是兩個 \m{n\times n} 矩陣相乘所用時間,
 \m{S(n)} 表示求 \m{n\times n} 矩陣平方所需時間。
證明:求矩陣乘機與求矩陣平方實質上難度相同,
即一個 \m{M(n)} 時間的矩陣相乘算法意味着一個 \m{O(M(n))} 時間的矩陣平方算法,
一個 \m{S(n)} 時間的矩陣平方算法意味着一個 \m{O(S(n))} 時間的矩陣相乘算法。
\stopEXERCISE

\startANSWER
\TODO{略。}
\stopANSWER

%e28.2-2
\startEXERCISE
設 \m{M(n)} 是兩個 \m{n\times n} 矩陣相乘所需時間,
 \m{L(n)} 爲計算一個 \m{n\times n} 矩陣的 LUP 分解所需時間。
證明:求矩陣乘機運算與計算矩陣 LUP 分解實質上難度相同,
即一個 \m{M(n)} 時間的矩陣相乘算法意味着一個 \m{O(M(n))} 時間的矩陣 LUP 分解算法,
一個 \m{L(n)} 時間的矩陣 LUP 分解算法意味着一個 \m{O(L(n))} 時間的矩陣相乘算法。
\stopEXERCISE

\startANSWER
\TODO{略。}
\stopANSWER

%e28.2-3
\startEXERCISE
設 \m{M(n)} 是兩個 \m{n\times n} 矩陣相乘所需時間,
 \m{D(n)} 表示求 \m{n\times n} 矩陣行列式值所需時間。
證明:求矩陣乘機運算與求行列式值實質上難度相同,
即一個 \m{M(n)} 時間的矩陣相乘算法意味着一個 \m{O(M(n))} 時間的行列式算法,
一個 \m{D(n)} 時間的行列式算法意味着一個 \m{O(D(n))} 時間的矩陣相乘算法。
\stopEXERCISE

\startANSWER
\TODO{略。}
\stopANSWER

%e28.2-4
\startEXERCISE
設 \m{M(n)} 是兩個 \m{n\times n} 布爾矩陣相乘所需時間,
 \m{T(n)} 爲找出 \m{n\times n} 布爾矩陣的傳遞閉包所需時間(見\insection[floyd_warshall])。
證明:一個 \m{M(n)} 時間的布爾矩陣相乘算法意味着一個 \m{O(M(n)\lg n)} 時間的傳遞閉包算法,
一個 \m{T(n)} 時間的傳遞閉包算法意味着一個 \m{O(T(n))} 時間的布爾矩陣相乘算法。
\stopEXERCISE

\startANSWER
\TODO{略。}
\stopANSWER

%e28.2-5
\startEXERCISE
當矩陣元素屬於整數模 2 所構成的域時,
基於定理 28.2 的矩陣求逆算法是否仍然有效?
請解釋。附定理 28.2:

(矩陣求逆運算不比矩陣乘法運算更難)
如果能在 \m{M(n)} 時間內計算出兩個 \m{n\times n} 實數矩陣的乘機,
其中 \m{M(n)=\Omega(n^2)} 且 \m{M(n)} 滿足兩個正則性條件:
對任意的 \m{k(0\le k\le n)}, \m{M(n+k)=O(M(n))};
對某個常數 \m{c<1/2}, \m{M(n/2)\le cM(n)}。
那麼可以在 \m{O(M(n))} 時間內計算出任何一個 \m{n\times n} 非奇異實數矩陣的逆。
\stopEXERCISE

\startANSWER
\TODO{略。}
\stopANSWER

%e28.2-6
\startEXERCISE
推廣定理 28.2 的矩陣求逆算法,使之哪呢個處理複數矩陣,
並證明你所給出的推廣是正確的。
(\hint 用 \m{A} 的{\EMP 共軛轉置矩陣}
(conjugate transpose) \m{A^*} 來替代 \m{A} 的轉置矩陣,
把 \m{A} 中的每個元素用其共軛複數來替代就得到 \m{A^*}。
考慮用{\EMP 埃爾米特}(Hermitian)矩陣來替代對稱矩陣,
埃爾米特矩陣就是滿足 \m{A=A^*} 的矩陣 \m{A}。)
\stopEXERCISE

\startANSWER
\TODO{略。}
\stopANSWER

\stopsection
