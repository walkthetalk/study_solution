\section[section:conversionRule]{轉換規則}

本節中我們將討論\cnglo{kernel}中讀寫圖像時所用的轉換規則。

% Conversion rules for normalized integer channel data types
\subsection{歸一化整數通道數據類型的轉換規則}

本節我們將討論歸一化整數通道數據類型與浮點值之間的相互轉換。

% Converting normalized integer channel data types to floating-point values
\subsubsection[section:normIntToFloat]{將歸一化整數通道數據類型轉換為浮點值}

如果通道數據類型為 \cenum{CL_UNORM_INT8} 和 \cenum{CL_UNORM_INT16}。
\capi{read_imagef} 將把 8 位或 16 位無符號整數轉換為歸一化浮點值,
其所在區間為 \math{[0.0f \cdots 1.0]}。

如果通道數據類型為 \cenum{CL_SNORM_INT8} 和 \cenum{CL_SNORM_INT16}。
\capi{read_imagef} 將把 8 位或 16 位帶符號整數轉換為歸一化浮點值,
其所在區間為 \math{[-1.0 \cdots 1.0]}。

轉換方式如下:

\startclCmmDesc{\cenum{CL_UNORM_INT8} (8 位無符號整數)-> \ctype{float}}
歸一化浮點值 = (float) c / 255.0f
\stopclCmmDesc

\startclCmmDesc{\cenum{CL_UNORM_INT_101010} (10 位無符號整數)-> \ctype{float}}
歸一化浮點值 = (float) c / 1023.0f
\stopclCmmDesc

\startclCmmDesc{\cenum{CL_UNORM_INT16} (16 位無符號整數)-> \ctype{float}}
歸一化浮點值 = (float) c / 65535.0f
\stopclCmmDesc

\startclCmmDesc{\cenum{CL_SNORM_INT8} (8 位帶符號整數)-> \ctype{float}}
歸一化浮點值 = max(-1.0f, (float)c / 127.0f)
\stopclCmmDesc

\startclCmmDesc{\cenum{CL_SNORM_INT16} (16 位帶符號整數)-> \ctype{float}}
歸一化浮點值 = max(-1.0f, (float)c / 32767.0f)
\stopclCmmDesc

除了下面所列情況,上述轉換的精度均 \math{\leq 1.5 ulp}:
\startclCmmDesc{對於 \cenum{CL_UNORM_INT8}}
0 必須轉換為 0.0f

255 必須轉換為 1.0f
\stopclCmmDesc

\startclCmmDesc{對於 \cenum{CL_UNORM_INT_101010}}
0 必須轉換為 0.0f

1023 必須轉換為 1.0f
\stopclCmmDesc

\startclCmmDesc{對於 \cenum{CL_UNORM_INT16}}
0 必須轉換為 0.0f

65535 必須轉換為 1.0f
\stopclCmmDesc

\startclCmmDesc{對於 \cenum{CL_SNORM_INT8}}
-128 和 -127 必須轉換為 -1.0f

0 必須轉換為 0.0f

127 必須轉換為 1.0f
\stopclCmmDesc

\startclCmmDesc{對於 \cenum{CL_SNORM_INT16}}
-32768 和 -32767 必須轉換為 -1.0f

0 必須轉換為 0.0f

32767 必須轉換為 1.0f
\stopclCmmDesc

% Converting floating-point values to normalized integer channel data types
\subsubsection{將浮點值轉換為歸一化整數通道數據類型}

如果通道數據類型為 \cenum{CL_UNORM_INT8} 和 \cenum{CL_UNORM_INT16}。
\capi{write_imagef} 將把浮點顏色值轉換為 8 位或 16 位無符號整數。

如果通道數據類型為 \cenum{CL_SNORM_INT8} 和 \cenum{CL_SNORM_INT16}。
\capi{write_imagef} 將把浮點顏色值轉換為 8 位或 16 位帶符號整數。

建議按如下方式轉換:

\startclCmmDesc{\ctype{float} -> \cenum{CL_UNORM_INT8} (8 位無符號整數)}
convert_uchar_sat_rte(f * 255.0f)
\stopclCmmDesc

\startclCmmDesc{\ctype{float} -> \cenum{CL_UNORM_INT_101010} (10 位無符號整數)}
min(convert_ushort_sat_rte(f * 1023.0f), 0x3ff)
\stopclCmmDesc

\startclCmmDesc{\ctype{float} -> \cenum{CL_UNORM_INT16} (16 位無符號整數)}
convert_ushort_sat_rte(f * 65535.0f)
\stopclCmmDesc

\startclCmmDesc{\ctype{float} -> \cenum{CL_SNORM_INT8} (8 位帶符號整數)}
convert_char_sat_rte(f * 127.0f)
\stopclCmmDesc

\startclCmmDesc{\ctype{float} -> \cenum{CL_SNORM_INT16} (16 位帶符號整數)}
convert_short_sat_rte(f * 32767.0f)
\stopclCmmDesc

對於溢出時的行為以及飽和轉換規則請參考\insection[oorbasc]。

OpenCL 實作也可能採用其他捨入模式來逼近上述轉換結果。
如果使用的捨入模式不是捨入為最近偶數(\ccmm{_rte}),
則實作所產生的結果與捨入模式 \ccmm{_rte} 所產生結果的絕對誤差必須 \math{\leq 0.6}。

\startclCmmDesc{float -> CL_UNORM_INT8 (8 位無符號整數)}
設 f\low{preferred} = convert_uchar_sat_rte(f * 255.0f)

設 f\low{approx} = convert_uchar_sat_<impl-rounding-mode>(f * 255.0f)

fabs(f\low{preferred} – f\low{fapprox}) 必須 <= 0.6
\stopclCmmDesc

\startclCmmDesc{float -> CL_UNORM_INT_101010 (10 位無符號整數)}
設 f\low{preferred} = convert_ushort_sat_rte(f * 1023.0f)

設 f\low{approx} = convert_ushort_sat_<impl-rounding-mode>(f * 1023.0f)

fabs(f\low{preferred} – f\low{approx}) 必須 <= 0.6
\stopclCmmDesc

\startclCmmDesc{float -> CL_UNORM_INT16 (16 位無符號整數)}
設 f\low{preferred} = convert_ushort_sat_rte(f * 65535.0f)

設 f\low{approx} = convert_ushort_sat_<impl-rounding-mode>(f * 65535.0f)

fabs(f\low{preferred} – f\low{approx}) 必須 <= 0.6
\stopclCmmDesc

\startclCmmDesc{float -> CL_SNORM_INT8 (8 位帶符號整數)}
設 f\low{preferred} = convert_char_sat_rte(f * 127.0f)

設 f\low{approx} = convert_char_sat_<impl_rounding_mode>(f * 127.0f)

fabs(f\low{preferred} – f\low{approx}) 必須 <= 0.6
\stopclCmmDesc

\startclCmmDesc{float -> CL_SNORM_INT16 (16 位帶符號整數)}
設 f\low{preferred} = convert_short_sat_rte(f * 32767.0f)

設 f\low{approx} = convert_short_sat_<impl-rounding-mode>(f * 32767.0f)

fabs(f\low{preferred} – f\low{approx}) 必須 <= 0.6
\stopclCmmDesc

% Conversion rules for half precision floating-point channel data type
\subsection{半精度浮點通道數據類型的轉換規則}

如果圖像通道數據類型為 \cenum{CL_HALF_FLOAT},
則由 \ctype{half} 到 \ctype{float} 的轉換是無損的(參見\insection[dataTypeHalf])。
由 \ctype{float} 到 \ctype{half} 的轉換中,
捨入尾數所用的捨入模式為捨入為最近偶數或向零捨入。
型別為 \ctype{half} 的去規格化數(可能是將 \ctype{float} 轉換為 \ctype{half} 時生成)
可能會被刷成零。
型別為 \ctype{float} 的 NaN 必須轉換為型別為 \ctype{half} 的 NaN。
型別為 \ctype{float} 的 INF 必須轉換為型別為 \ctype{half} 的 INF。

% Conversion rules for floating-point channel data type
\subsection{浮點通道數據類型的轉換規則}

如果圖像通道數據類型為 \cenum{CL_FLOAT},則對其讀寫時要遵守下列規則:
\startigBase
\item NaN 必須轉換為 \cnglo{device} 所支持的 NaN 值。
\item 可以將去規格化數刷成零。
\item 所有其他值都必須保留。
\stopigBase

% Conversion rules for signed and unsigned 8-bit, 16-bit
% and 32-bit integer channel data types
\subsection{帶符號或無符號的 8/16/32 位整數通道數據類型的轉換規則}

如果圖像通道數據類型為 \cenum{CL_SIGNED_INT8}、 \cenum{CL_SIGNED_INT16}
或 \cenum{CL_SIGNED_INT32},
則 \capi{read_imagei} 會返回圖像中指定位置所存儲的原始整數值。

如果圖像通道數據類型為 \cenum{CL_UNSIGNED_INT8}、 \cenum{CL_UNSIGNED_INT16}
或 \cenum{CL_UNSIGNED_INT32},
則 \capi{read_imageui} 會返回圖像中指定位置所存儲的原始整數值。

\capi{write_imagei} 會進行下列轉換:

\startclCmmDesc{32 位帶符號整數 -> 8 位帶符號整數}
convert_char_sat(i)
\stopclCmmDesc
\startclCmmDesc{32 位帶符號整數 -> 16 位帶符號整數}
convert_short_sat(i)
\stopclCmmDesc
\startclCmmDesc{32 位帶符號整數 -> 32 位帶符號整數}
不會實施任何轉換
\stopclCmmDesc

\capi{write_imageui} 會進行下列轉換:
\startclCmmDesc{32 位無符號整數 -> 8 位無符號整數}
convert_uchar_sat(i)
\stopclCmmDesc
\startclCmmDesc{32 位無符號整數 -> 16 位無符號整數}
convert_ushort_sat(i)
\stopclCmmDesc
\startclCmmDesc{32 位無符號整數 -> 32 位無符號整數}
不會實施任何轉換
\stopclCmmDesc

對於本節所描述的轉換,必須使其正確飽和。
