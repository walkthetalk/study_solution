\startsection[
  title={Elements of the greedy strategy},
]

\startEXERCISE
證明:分數揹包問題具有貪心選擇性質。
\stopEXERCISE

\startANSWER
設商品數量爲 \m{n},序號爲 \m{i\in \{1,2,\ldots,n\}} 的商品
價值爲 \m{v_i},重量爲 \m{w_i}。
揹包能承重 \m{W}。目標是讓揹包中的商品價值最大。
不必將整個商品都放到揹包內。
即如果可以將價值最大化,可以只將商品的一部分放入揹包。
先計算每個商品的價值密度,即 \m{v_i / w_i}。
按價值密度對所有商品進行排序。
設商品 \m{j} 的價值密度最大。

\startigBase[n]
\item Case 1:如果 \m{W=w_j},則將商品 \m{j} 放入揹包,問題結束;
\item Case 2:如果 \m{W<w_j},則將儘量多的 \m{j} 放入揹包,問題結束;
\item Case 3:如果 \m{W>w_j},則將商品 \m{j} 放入揹包後。然後揹包還可以放入重量 \m{W-w_j} 的商品。
後面採用貪心策略繼續解決子問題:揹包能承重 \m{W-w_j},有 \m{n-1} 個商品共選擇。
需要注意的是,選擇 \m{j} 時不必考慮子問題的解。
\stopigBase

因此 0-1 揹包問題具有貪婪選擇性質。
\stopANSWER

%e16.2-2
\startEXERCISE
設計動態規劃算法求解 0-1 揹包問題,要求運行時間爲 \m{O(nW)}, \m{n} 爲商品數量,
揹包最大承重爲 \m{W}。
\stopEXERCISE

\startANSWER
\m{w} 和 \m{v} 長度均爲 \m{n};
 \m{W} 是揹包的最大承重;
 \m{B} 是 \m{(w+1)\times(n+1)} 矩陣;
 \m{B[i][j]} 是揹包最大承重爲 \m{j}、只能在前 \m{i} 個商品中選擇時所能得到的最大價值。

\CLRSH{KNAPSACK(w, v, W, n)}
\startCLRS
\\ initialization B[i,j]:
\\ i is index of items, j is weight
for i=1 to n
	B[i,0] = 0
for j=0 to W
	B[0,j] = 0

for i=1 to n
	for j=0 to W
		if w[i] <= j
			B[i,j] = max((v[i] + B[i-1, j-w[i]]), B[i-1,w])
		else
			B[i,j] = B[i-1,w]
\stopCLRS

\CLRSH{KNAPSACK-ITEMS(v, w, B)}
\startCLRS
i=n
k=W

while i>0 and k>0
	if B[i,k] != B[i-1,k]
		add item i in the knapsack
		k = k - w[i]
	--i
\stopCLRS

\stopANSWER

%e16.2-3
\startEXERCISE
假定在 0-1 揹包問題中,商品越重,價值越小,
設計一個高效算法求解此變形問題的最優解,
並證明算法的正確性。
\stopEXERCISE

\startANSWER
按重量由輕到重選擇 \m{k} 個商品,
使得 \m{\sum_{i=1}^k w[i] \le W} 且 \m{\sum_{i=1}^{k+1} w[i] > W},這就是最優解。
如果這 \m{k} 個商品中第 \m{j} 個沒有被選中,而 \m{i>k} 被選中,
則我們可以將 \m{i} 換成 \m{j},總重量不增加但價值會增加。
\stopANSWER

%e16.2-4
\startEXERCISE
Gekko 教授計劃沿 U.S. 2 號高速公路用直排輪滑橫穿 North Dakota,
這條高速公路從 Minnesota 東部邊境的 Grand Forks 一直到達 Montana 西部邊境的 Williston。
教授能夠帶兩公升水,在喝光之前能滑行 \m{m} 英里。
(North Dakota 相當平坦,教授無需擔心上坡路段喝水速度比平地或下坡路段快。)
教授從 Grand Forks 出發,並攜帶兩公升水。
地圖上顯示了高速公訴上所有補給點,以及他們之間的距離。

教授的目標是儘量減少途中補充水的次數。
設計一個高效的方法,幫助教授確定應該在哪些地點補充水。
證明你的策略會生成最優解,並分析其運行時間。
\stopEXERCISE

\startANSWER
我們需要選擇補給站,每次選擇的補給站離上一個的距離應離 \m{m} 最近但不大於 \m{m}。

首先,此問題的解包含子問題的最優解。
令 \m{S} 爲解, \m{G} 爲選中的某個補給站。
如果從起點到 \m{G} 之間,教授停下的次數不是最少,
則我們可以有一個更好的解。
但 \m{S} 是最優解,所以 \m{S} 包含子問題的最優解。

下面來證明貪心選擇會得到最優解。
令貪心選擇法選中的補給站爲 \m{G_1,G_2,\ldots,G_k},且這不是最優解。
則最優解中的第一個站點 \m{O_1} 不是 \m{G_1} 就是 \m{G_1} 之前的站點,
否則從起點到 \m{O_1} 的距離會超過 \m{m}。
由於 \m{O_1} 到 \m{O_2} 的距離小於 \m{m},
因此 \m{O_2} 不是 \m{G_2} 就是 \m{G_2} 之前的站點。
以此類推, \m{O_k} 不是 \m{G_k} 就是 \m{G_k} 之前的站點。
從而最優解選中站點個數不會少於貪心選擇法的解。
即,貪心選擇法的解就是最優解。

貪心選擇法的運行時間爲 \m{O_n},其中 \m{n} 爲站點總數。
\stopANSWER

%e16.2-5
\startEXERCISE
設計一個高效算法,對實數線上給定的一個點集 \m{\{x_1,x_2,\ldots,x_n\}},
求一個單位長度閉區間的集合,包含所有給定的點,並要求此集合最小。
證明你的算法是正確的。
\stopEXERCISE

\startANSWER
貪心選擇法。先對所有點排序,然後以第一個點爲閉區間的起點,
找到單位長度閉區間中最後一個點;以上一個閉區間後面緊挨的點爲新區間的起點,
往後依次搜索。直到遍歷完所有點,得到的區間個數即爲所求。
\stopANSWER

%e16.2-6
\startEXERCISE\DIFFICULT
設計算法,在 \m{O(n)} 時間內求解分數揹包問題。
\stopEXERCISE

\startANSWER
先按 \m{v/w} 從大到小排序,然後依次累加,直到重量超標。
除最後一個外,其他累計物品即爲所求。
\stopANSWER

%e16.2-7
\startEXERCISE
給定兩個集合 \m{A} 和 \m{B},各包含 \m{n} 個正整數。
你可以按需要任意重排每個集合。
重排後,令 \m{a_i} 爲集合 \m{A} 的第 \m{i} 個元素,
 \m{b_i} 爲集合 \m{B} 的第 \m{i} 個元素。
於是你得到回報 \m{\prod_{i=1}^{n}a_i^{b_i}}。
設計算法使回報最大化。
證明算法的正確性,並分析運行時間。
\stopEXERCISE

\startANSWER
將兩個集合均按單調遞減進行排序。

任取兩個索引 \m{i}、 \m{j},其中 \m{i<j}。
則 \m{a_i^{b_i} \times a_j^{b_j} \ge a_i^{b_j}\times a_j^{b_i}}。
\stopANSWER

\stopsection
