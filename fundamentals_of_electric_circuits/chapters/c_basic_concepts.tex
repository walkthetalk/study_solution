\startcomponent c_basic_concepts
\startchapter[
  title={Basic Concepts},
]

% 1.1
\startQ
How much charge is represented by these number of electrons?
\startIG
\item $\sunit{6.482e17}$
\item $\sunit{1.24=e18}$
\item $\sunit{2.46=e19}$
\item $\sunit{1.628e20}$
\stopIG
\stopQ

\startA
\startIG
\item $q = (\sunit{1.602e-19 coulomb}) \times \sunit{6.482e17} = \sunit{_0.10384 coulomb}$
\item $q = (\sunit{1.602e-19 coulomb}) \times \sunit{1.24=e18} = \sunit{_0.19865 coulomb}$
\item $q = (\sunit{1.602e-19 coulomb}) \times \sunit{2.46=e19} = \sunit{_3.941== coulomb}$
\item $q = (\sunit{1.602e-19 coulomb}) \times \sunit{1.628e20} = \sunit{26.08=== coulomb}$
\stopIG
\stopA

% 1.2
\startQ
Determine the current flowing through an element if the charge flow is given by
\startIG
\item $q(t) = (3t+8)               \sunit[l]{milli coulomb}$
\item $q(t) = (8t^2 + 4t − 2)      \sunit[l]{coulomb}$
\item $q(t) = (3e^{−t} − 5e^{−2t}) \sunit[l]{nano coulomb}$
\item $q(t) = 10 \sin 120π t       \sunit[l]{pico coulomb}$
\item $q(t) = 20e^{−4t} \cos 50t   \sunit[l]{micro coulomb}$
\stopIG
\stopQ

\startA
\startIG
\item $i = \frac{d q}{d t} = 3                      \sunit[l]{milli ampere}$
\item $i = \frac{d q}{d t} = (16t + 4)              \sunit[l]{ampere}$
\item $i = \frac{d q}{d t} = (-3e^{−t} + 10e^{−2t}) \sunit[l]{nano ampere}$
\item $i = \frac{d q}{d t} = 1200π \cos 120π t      \sunit[l]{pico ampere}$
\item $i = \frac{d q}{d t} = -(80 \cos 50t + 1000 \sin 50 t) e^{−4t} \sunit[l]{micro ampere}$
\stopIG
\stopA

% 1.3
\startQ
Find the charge $q(t)$ flowing through a device if the current is:
\startIG
\item $i(t) = \sunit{3 ampere}, q(0) = \sunit{1 coulomb}$
\item $i(t) = (2t + 5) \sunit[l]{milli ampere}, q(0) = 0$
\item $i(t) = 20 \cos (10t + π / 6) \sunit[l]{micro ampere}, q(0) = \sunit{2 micro coulomb}$
\item $i(t) = 10e^{−30t} \sin 40t \sunit[l]{ampere}, q(0) = 0$
\stopIG
\stopQ

\startA
\startIG
\item $q(t) = \int_0 i(t) + q(0) = (3t+1) \sunit[l]{coulomb}$
\item $q(t) = \int_0 i(t) + q(0) = (t^2 + 5t) \sunit[l]{milli coulomb}$
\item $q(t) = \int_0 i(t) + q(0) = (2 \sin (10t + π / 6) + 1) \sunit[l]{micro coulomb}$
\item $q(t) = \int_0 i(t) + q(0) = (-e^{-30t}(0.16\cos 40t + 0.12\sin 40t) + 0.16) \sunit[l]{coulomb}$
\stopIG
\stopA

% 1.4
\startQ
A total charge of $\sunit{300 coulomb}$ flows past a given cross section of a conductor in 30 seconds.
What is the value of the current?
\stopQ

\startA
$\sunit[r]{300 coulomb} / \sunit[l]{30 second} = \sunit{10 ampere}$
\stopA

% 1.5
\startQ
Determine the total charge transferred over the time interval
of $0 \le t \le \sunit{10 second}$ when $i(t) = \frac{1}{2}t \sunit[l]{ampere}$.
\stopQ

\startA
\startformula
q(t) = \int_0^{10}\frac{1}{2}t = \left.\frac{t^2}{4}\right|_0^{10} = 25
\stopformula
\stopA

% 1.6
\startQ[Q:1.6]
The charge entering a certain element is shown in \reffig{1.23}. Find the current at:
\startIG
\item $t = \sunit{_1 milli second}$
\item $t = \sunit{_6 milli second}$
\item $t = \sunit{10 milli second}$
\stopIG
\placefigure[here][fig:1.23]{for \refq{1.6}}{
\startMPcode
	numeric ux, uy, ah;
	ux := 10;
	uy := 2;
	ah := .2ux;

	drawarrow (0,0)--(14ux,0);
	for i := 2 step 2 until 12:
		draw (i*ux, 0)--(i*ux, ah);
		mlabel.bot(decimal(i), (i*ux, 0));
	endfor;
	mlabel.bot("0", (0,0));
	mlabel.lrt("t(ms)", (13ux, 0));

	drawarrow (0,0)--(0, 35uy);
	draw (0, 30uy)--(ah, 30uy);
	mlabel.lft("30", (0, 30uy));
	mlabel.lft("q(t)(mC)", (0, 35uy));

	draw (0,0)--(2ux, 30uy)--(8ux, 30uy)--(12ux, 0) withcolor darkred withpen pencircle scaled 1.2;

	draw (2ux, 0)--(2ux, 30uy) dashed evenly;
	draw (8ux, 0)--(8ux, 30uy) dashed evenly;
\stopMPcode
}
\stopQ

\startA
\startIG
\item $i(\sunit{t}) = \frac{dq}{dt} = \frac{d(15t)}{dt}= 15, i(1) = 15 \sunit[l]{coulomb}$
\item $i(\sunit{t}) = \frac{dq}{dt} = \frac{d(30)}{dt} = 0, i(6) = 0 \sunit[l]{coulomb}$
\item $i(\sunit{t}) = \frac{dq}{dt} = \frac{d(90-7.5t)}{dt} = -7.5, i(10) = -7.5 \sunit[l]{coulomb}$
\stopIG
\stopA

% 1.7
\startQ[Q:1.7]
The charge flowing in a wire is plotted in \reffig{1.24}. Sketch the corresponding current.
\placefigure[here][fig:1.24]{for \refq{1.7}}{
\startMPcode
	numeric ux, uy, ah;
	ux := 20;
	uy := 2;
	ah := .1ux;

	drawarrow (0,0)--(5ux,0);
	for i := 1 step 1 until 4:
		draw (i*ux, 0)--(i*ux, ah);
		mlabel.bot(decimal(i), (i*ux, 0));
	endfor;
	mlabel.lrt("t(s)", (5ux, 0));

	mlabel.lft("0", (0,0));

	drawarrow (0,-15uy)--(0, 15uy);
	for i := -10 step 20 until 10:
		draw (0, i*uy)--(ah, i*uy);
		mlabel.lft(decimal(i), (0, i*uy));
	endfor;
	mlabel.lft("q(C)", (0, 15uy));

	draw (0,0)--(1ux, 10uy)--(2ux, -10uy)--(3ux, -10uy)--(4ux, 0)
		withcolor darkred withpen pencircle scaled 1.2;
\stopMPcode
}
\stopQ

\startA
\startMPcode
	numeric ux, uy, ah;
	ux := 20;
	uy := 2;
	ah := .1ux;

	drawarrow (0,0)--(5ux,0);
	for i := 1 step 1 until 4:
		draw (i*ux, 0)--(i*ux, ah);
		mlabel.lrt(decimal(i), (i*ux, 0));
	endfor;
	mlabel.lrt("t(s)", (5ux, 0));

	drawarrow (0,-25uy)--(0, 15uy);
	for i := -20 step 10 until 10:
		draw (0, i*uy)--(ah, i*uy);
		mlabel.lft(decimal(i), (0, i*uy));
	endfor;
	mlabel.lft("q(C)", (0, 15uy));

	draw (0,10uy)--(1ux, 10uy)--(ux, -20uy)--(2ux, -20uy)--(2ux, 0)--(3ux, 0)--(3ux, 10uy)--(4ux, 10uy)
		withcolor darkred withpen pencircle scaled 1.2;
\stopMPcode
\stopA

% 1.8
\startQ[Q:1.8]
 The current flowing past a point in a device is shown in \reffig{1.25}.
 Calculate the total charge through the point.
\placefigure[here][fig:1.25]{for \refq{1.8}}{
\startMPcode
	numeric ux, uy, ah;
	ux := 30;
	uy := 3;
	ah := .1ux;

	drawarrow (0,0)--(2.5ux,0);
	for i := 1 step 1 until 2:
		draw (i*ux, 0)--(i*ux, ah);
		mlabel.bot(decimal(i), (i*ux, 0));
	endfor;
	mlabel.lrt("t(ms)", (2.5ux, 0));

	mlabel.lft("0", (0,0));

	drawarrow (0,0)--(0, 15uy);
	for i := 10 step 10 until 10:
		draw (0, i*uy)--(ah, i*uy);
		mlabel.lft(decimal(i), (0, i*uy));
	endfor;
	mlabel.lft("i(mA)", (0, 15uy));

	draw (0,0)--(1ux, 10uy)--(2ux, 10uy)--(2ux, 0)
		withcolor darkred withpen pencircle scaled 1.2;
\stopMPcode
}
\stopQ

\startA
\startformula
q = \int i dt = \frac{10\times 1}{2} + 10\times 1 = \sunit{15 milli coulomb}
\stopformula
\stopA

% 1.9
\startQ[Q:1.9]
The current through an element is shown in \reffig{1.26}.
Determine the total charge that passed through the element at:
\startcolumns[n=3]
\startIG
\item $t=\sunit{1 second}$
\item $t=\sunit{3 second}$
\item $t=\sunit{5 second}$
\stopIG
\stopcolumns
\placefigure[here][fig:1.26]{for \refq{1.9}}{
\startMPcode
	numeric ux, uy, ah;
	ux := 15;
	uy := 3;
	ah := .1ux;

	drawarrow (0,0)--(6ux,0);
	for i := 0 step 1 until 5:
		draw (i*ux, 0)--(i*ux, ah);
		mlabel.bot(decimal(i), (i*ux, 0));
	endfor;
	mlabel.lrt("t(s)", (6ux, 0));

	drawarrow (0,0)--(0, 15uy);
	for i := 5 step 5 until 10:
		draw (0, i*uy)--(ah, i*uy);
		mlabel.lft(decimal(i), (0, i*uy));
	endfor;
	mlabel.lft("i(A)", (0, 15uy));

	draw (0,10uy)--(1ux, 10uy)--(2ux, 5uy)--(4ux, 5uy)--(5ux, 0)
		withcolor darkred withpen pencircle scaled 1.2;
\stopMPcode
}
\stopQ

\startA
\startIG
\item $q=\int_0^1 i dt = \int_0^1 10 dt = \sunit{10 coulomb}$
\item $q=\int_0^3 i dt
        = \int_0^1 i dt + \int_1^2 i dt         + \int_2^3 i dt
        =\int_0^1 10 dt + \int_1^2 (15 - 5t) dt + \int_2^3 5 dt
        =10             + 7.5                   + 5
        =\sunit{22.5 coulomb}$
\item $q=\int_0^5 i dt
        =\int_0^3 i dt + \int_3^4 i dt + \int_4^5 i dt
        = 22.5         + 5             + 2.5
        =\sunit{30 coulomb}$
\stopIG
\stopA

\stopchapter
\stopcomponent
