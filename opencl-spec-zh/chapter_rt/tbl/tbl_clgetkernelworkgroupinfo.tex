\startETD[cl_kernel_work_group_info][返回型別]

\clETD{CL_KERNEL_GLOBAL_WORK_SIZE}{size_t[3]}{
利用此機制,
\cnglo{app}可以查詢用來在 \carg{device} 上執行\cnglo{kernel}的全局索引空間的大小
(即 \capi{clEnqueueNDRangeKernel} 的引數 \carg{global_work_size})。

這要求 \carg{device} 是\cnglo{customdev}或所執行的\cnglo{kernel}是內建的,
否則 \capi{clGetKernelWorkGroupInfo} 會返回 \cenum{CL_INVALID_VALUE}。
}

\clETD{CL_KERNEL_WORK_GROUP_SIZE}{size_t}{
利用此機制,
\cnglo{app}可以查詢用來在 \carg{device} 上執行\cnglo{kernel}的\cnglo{workgrp}的大小。
OpenCL 實作可以用\cnglo{kernel}的資源需求(寄存器的使用情況等)來確定\cnglo{workgrp}的大小。
}

\clETD{CL_KERNEL_COMPILE_WORK_GROUP_SIZE}{size_t[3]}{
返回限定符 \ccmm{__attribute__((reqd_work_group_size(X, Y, Z)))} 所指定的
\cnglo{workgrp}的大小。參見\insection[optAttrQlf]。

如果沒有用上述特性限定符指定\cnglo{workgrp}的大小,則返回\math{(0, 0, 0)}。
}

\clETD{CL_KERNEL_LOCAL_MEM_SIZE}{cl_ulong}{
返回\cnglo{kernel}所使用的\cnglo{locmem}的大小。
他包括實作執行\cnglo{kernel}所需的內存、
\cnglo{kernel}中所聲明的帶有位址限定符 \cqlf{__local} 的變量、
以及為帶有位址限定符 \cqlf{__local} 的指位器引數所分配的內存
(其大小由 \capi{clSetKernelArg} 指定)。

對於任一帶有位址限定符 \cqlf{__local} 的指位器引數,如果沒有為其指定內存大小,則假定為 0。
}

\clETD{CL_KERNEL_PREFERRED_WORK_GROUP_SIZE_MULTIPLE}{size_t}{
返回所期望的\cnglo{workgrp}大小的粒度。僅為性能建議。
在調用 \capi{clEnqueueNDRangeKernel} 時,
如果給引數 \carg{local_work_size} 所指定的\cnglo{workgrp}大小不是此查詢結果的倍數,
並不會導致函式執行失敗,除非此值超過了\cnglo{device}的數目。
}

\clETD{CL_KERNEL_PRIVATE_MEM_SIZE}{cl_ulong}{
\cnglo{kernel}中每個\cnglo{workitem}至少要使用多少\cnglo{prvmem}。
他包括實作執行\cnglo{kernel}所需的所有\cnglo{prvmem};
包括語言本身所需的\cnglo{prvmem},以及\cnglo{kernel}中聲明的 \cqlf{__private} 變量。
}

\stopETD
