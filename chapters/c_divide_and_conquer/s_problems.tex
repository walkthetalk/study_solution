\startsubject[
  title={Problems},
]

%p4-1
\startPROBLEM(遞迴式例子)
對於下列遞迴式,給出 \m{T(n)} 的漸進上界和下界。
假定 \m{n\le 2} 時 \m{T(n)} 是常數。給出盡量緊確的界,並驗證其正確性。
\startigBase[n]
\item \m{T(n) = 2 T(n/2) + n^4}

\startANSWER
\m{\Theta(n^4)} (主定理)
\stopANSWER

\item \m{T(n) = T(7n/10) + n}

\startANSWER
\m{\Theta(n)} (主定理, \m{\log_{10/7} 1 = 0})
\stopANSWER

\item \m{T(n) = 16 T(n/4) + n^2}

\startANSWER
\m{\Theta(n^2\lg n)} (主定理)
\stopANSWER

\item \m{T(n) = 7 T(n/3) + n^2}

\startANSWER
\m{\Theta(n^2)} (主定理)
\stopANSWER

\item \m{T(n) = 7 T(n/2) + n^2}

\startANSWER
\m{\Theta(n^{\log_2 7})} (主定理)
\stopANSWER

\item \m{T(n) = 2 T(n/4) + n^2}

\startANSWER
\m{\Theta(\sqrt{n}\log_n)} (主定理)
\stopANSWER

\item \m{T(n) = T(n-2) + n^2}

\startANSWER
\m{T(n) = n^2 + T(n-2) = n^2 + (n-2)^2 + T(n-4) = \sum_{i=0}^{n/2}(n-2i)^2 = \Theta(n^3)}
\stopANSWER
\stopigBase
\stopPROBLEM

%p4-2
\startPROBLEM(參數傳遞代價)
本書中自始至終假設:過程調用中的參數傳遞花費常量時間,
即使傳遞一個 \m{N} 個元素的數列亦是如此。
在大多數系統中,這個假設是成立的,因爲傳遞的是數列的指標,而非數列本身。
本題討論三種參數傳遞策略:
\startigBase[n]
\item 通過指標傳遞數列。時間爲 \m{\Theta(1)}。
\item 通過復制元素傳遞數列。時間爲 \m{\Theta(N)},其中 \m{N} 是數列的規模。
\item 傳遞數列時,只復制過程可能存取的子區域。若子數列 \m{A[p..q]} 被傳遞,則時間爲 \m{\Theta(q-p+1)}。
\stopigBase
\startigBase[a]
\item[parameter_pass_cost_item]
考慮在有序數列中查找元素的遞迴二分查找算法(參見\refexercise{bin_search})。
分別給出上述三種策略下,二分查找最壞情況運行時間的遞迴式,並給出遞迴式解的好的上界。
令 \m{N} 爲原問題的規模, \m{n} 爲子問題的規模。
\item 對節 2.3.1 中的 MERGE-SORT 算法重做上一項。
\stopigBase

\startANSWER
二分查找:
\startigBase[n]
\item \m{T(n) = T(n/2) + c = \Theta(\lg n)} (主定理)
\item \m{T(n) = T(n/2) + cN = T(n/4) + 2cN = \sum_{i=0}^{\lg n - 1}(2^i cN/2^i) = cN\lg n = \Theta(n\lg n)}
\item \m{T(n) = T(n/2) + cn = \Theta(n)} (主定理)
\stopigBase
歸並排序:
\startigBase[n]
\item \m{T(n) = 2T(n/2) + cn = \Theta(n\lg{n})} (主定理)
\item
\startformula\startmathalignment
\NC T(n) \NC= 2T(n/2) + cn + 2N = 4N + cn + 2c(n/2) + 4T(n/4) \NR
\NC      \NC= 8N + 2cn + 4c(n/4) + 8T(n/8) \NR
\NC      \NC= \sum_{i=0}^{\lg{n}-1}(cn + 2^iN) \NR
\NC      \NC= \sum_{i=0}^{\lg{n}-1}cn + N\sum_{i=0}^{\lg{n}-1}2^i \NR
\NC      \NC= cn\lg{n} + N\frac{2^{\lg{n}} - 1}{2-1} \NR
\NC      \NC= cn\lg{n} + nN - N \NR
\NC      \NC= \Theta(nN) \NR
\NC      \NC= \Theta(n^2) \NR
\stopmathalignment\stopformula
\item \m{T(n) = 2T(n/2) + cn + 2n/2 = 2T(n/2) + (c+1)n = \Theta(n\lg{n})} (主定理)
\stopigBase
\stopANSWER
\stopPROBLEM

%p4-3
\startPROBLEM(更多遞迴式例子)
對於下列遞迴式,給出 \m{T(n)} 的漸進上界和下界。
假定對足夠小的 \m{n}, \m{T(n)} 是常數。給出盡量緊確的界,並驗證其正確性。
\startigBase[n]
\item \m{T(n) = 4T(n/3) + n\lg{n}}

\startANSWER
\m{\Theta(n^{\log_3 4})} (主定理)
\stopANSWER

\item \m{T(n) = 3T(n/3) + n/\lg{n}}

\startANSWER
\m{\Theta(n \lg \lg n)} (參見\refitem{Tnlgn})
\stopANSWER

\item \m{T(n) = 4T(n/2) + n^2\sqrt{n}}

\startANSWER
\m{\Theta(n^2\sqrt{n}) = \Theta(n^{2.5})} (主定理 \m{\log_2{4} = 2 < 2.5})
\stopANSWER

\item \m{T(n) = 3T(n/3 - 2) + n/2}

\startANSWER
可以忽略 \m{-2},用主定理可得 \m{\Theta(n\lg{n})}。
\stopANSWER

\item[item:Tnlgn] \m{T(n) = 2T(n/2) + n/\lg{n}}

\startANSWER
\startformula\startmathalignment
\NC T(n) \NC= 2T(n/2) + \frac{n}{\lg{n}} \NR
\NC      \NC= 4(n/4) + 2\frac{n/2}{\lg(n/2)} + \frac{n}{\lg{n}} \NR
\NC      \NC= 4T(n/4) + \frac{n}{\lg{n} - 1} + \frac{n}{\lg{n}} \NR
\NC      \NC= nT(1) + \sum_{i=0}^{\lg{n} - 1}\frac{n}{\lg{n}-i} \NR
\NC      \NC= nT(1) + n\sum_{i=1}^{\lg{n}}\frac{1}{\lg{n}} \NR
\NC      \NC= \Theta(n\lg\lg{n}) \NR
\stopmathalignment\stopformula
\stopANSWER

\item \m{T(n) = T(n/2) + T(n/4) + T(n/8) + n}

\startANSWER
猜測 \m{\Theta(n)}:
\startformula\startmathalignment
\NC T(n) \NC = cn/2 + cn/4 + cn/8 + n \le (7/8)cn + n \le cn = O(n) \quad (c \ge 8) \NR
\NC T(n) \NC = cn/2 + cn/4 + cn/8 + n \ge (7/8)cn + n \ge cn = \Omega(n) \quad (c \le 8) \NR
\stopmathalignment\stopformula
\stopANSWER

\item \m{T(n) = T(n - 1) + 1/n}

\startANSWER
\startformula\startmathalignment
\NC T(n) \NC= T(n-1) + 1/n \NR
\NC      \NC= \frac{1}{n} + \frac{1}{n-1} + T(n-2) \NR
\NC      \NC= \frac{1}{n} + \frac{1}{n-1} + \frac{1}{n-2} + T(n-3) \NR
\NC      \NC= \sum_{i=0}^{n-1}\frac{1}{n-i} \NR
\NC      \NC= \sum_{i=1}^n\frac{1}{i} \NR
\NC      \NC= \Theta(\lg{n})
\stopmathalignment\stopformula
\stopANSWER

\item \m{T(n) = T(n - 1) + \lg{n}}

\startANSWER
\startformula\startmathalignment
\NC T(n) \NC= \lg{n} + T(n-1) \NR
\NC      \NC= \lg{n} + \lg{n-1} + T(n-2) \NR
\NC      \NC= \sum_{i=0}^{n-1}\lg(n - i) \NR
\NC      \NC= \sum_{i=1}^{n}\lg{i} \NR
\NC      \NC= \lg(n!) \le \lg{n^n} \NR
\NC      \NC= n\lg{n} \NR
\NC      \NC= \Theta(n\lg{n}) \NR
\stopmathalignment\stopformula
\stopANSWER

\item \m{T(n) = T(n - 2) + 1/\lg{n}}

\startANSWER
\startformula\startmathalignment
\NC T(n) \NC= \frac{1}{\lg{n}} + \frac{1}{\lg{n-2}} + \ldots \NR
\NC      \NC= \sum_{i=1}^{n/2}\frac{1}{\lg(2i)} \NR
\NC      \NC= \sum_{i=1}^{\infty}\frac{1}{\lg{i}} \NR
\NC      \NC= \Theta(\lg\lg{n}) \NR
\stopmathalignment\stopformula
\stopANSWER

\item \m{T(n) = \sqrt{n}T(\sqrt{n}) + n}

\startANSWER
猜測 \m{T(n) \le cn\lg{n}\lg{n}}:
\startformula\startmathalignment
\NC T(n) \NC\le \sqrt{n}c\sqrt{n}\lg\lg\sqrt{n} + n \NR
\NC      \NC= cn\lg\lg\sqrt{n} + n \NR
\NC      \NC= cn\lg\frac{\lg{n}}{2} + n \NR
\NC      \NC= cn\lg\lg{n} - cn\lg{2} + n \NR
\NC      \NC= cn\lg\lg{n} + (1 - c)n \quad (c > 1) \NR
\NC      \NC\le cn\lg\lg{n} \NR
\NC      \NC= \Theta(n\lg\lg{n}) \NR
\stopmathalignment\stopformula
\stopANSWER

\stopigBase
\stopPROBLEM

%p4-4
\startPROBLEM[problem:generating_function] (Fibonacci 數)
本題討論遞迴式(3.22)所定義的 Fibonacci 數的性質。
我們將使用生成函數技術來求解 Fibonacci 遞迴式。
{\EMP 生成函數 generating function}(又稱{\EMP 形式冪級數 formal power series}) \m{{\cal F}} 定義爲:
\startformula\startmathalignment
\NC {\cal F}(z) \NC= \sum_{i=0}^{\infty}F_iz^i \NR
\NC                \NC= 0 + z + z^2 + 2z^3 + 3z^4 + 5z^5 + 8z^6 + 13z^7 + 21z^8 + \ldots \NR
\stopmathalignment\stopformula
其中 \m{F_i} 是第 \m{i} 個 Fibonacci 數。

附遞迴式(3.22):
\startformula\startmathalignment
\NC F_0 \NC = 0 \NR
\NC F_1 \NC = 1 \NR
\NC F_i \NC = F_{i-1} + F_{i-2} \qquad \text{若 \m{i \ge 2}} \NR
\stopmathalignment\stopformula

\startigBase[a]
\item 證明: \m{{\cal F}(z) = z + z{\cal F}(z) + z^2{\cal F}(z)}。

\startANSWER
\startformula\startmathalignment
\NC  \NC z + z{\cal F}(z) + z^2{\cal F}(Z) \NR
\NC =\NC z + z\sum_{i=0}^{\infty}F_iz^i + z^2\sum_{i=0}^{\infty}F_iz^i \NR
\NC =\NC z + \sum_{i=1}^{\infty}F_{i-1}z^i + \sum_{i=2}^{\infty}F_{i-2}z^i \NR
\NC =\NC z + F_1z + \sum_{i=2}^{\infty}(F_{i-1} + F_{i-2})z^i \NR
\NC =\NC z + F_1z + \sum_{i=2}^{\infty}F_iz^i \NR
\NC =\NC {\cal F}(z) \NR
\stopmathalignment\stopformula
\stopANSWER

\startitem 證明:\startformula\startmathalignment
\NC {\cal F}(z) \NC= \frac{z}{1 - z - z^2} \NR
\NC                \NC= \frac{z}{(1 - \phi z)(1 - \hat\phi z)} \NR
\NC                \NC= \frac{1}{\sqrt5}\Big(\frac{1}{1 - \phi z} - \frac{1}{1 - \hat{\phi} z}\Big) \NR
\stopmathalignment\stopformula

其中\startformula\startmathalignment
\NC     \phi \NC= \frac{1 + \sqrt5}{2} = 1.61803\ldots \NR
\NC \hat\phi \NC= \frac{1 - \sqrt5}{2} = -0.61803\ldots \NR
\stopmathalignment\stopformula
\stopitem

\startANSWER
只需注意到 \m{\phi - \hat\phi = \sqrt5}, \m{\phi + \hat\phi = 1}, \m{\phi\hat\phi = -1}:
\startformula\startmathalignment
\NC  \NC {\cal F}(z) \NR
\NC =\NC \frac{{\cal F}(z)(1 - z - z^2)}{1 - z - z^2} \NR
\NC =\NC \frac{{\cal F}(z) - z{\cal F}(z) - z^2{\cal F}(z) - z + z}{1 - z - z^2} \NR
\NC =\NC \frac{{\cal F}(z) - {\cal F}(z) + z}{1 - z - z^2} \NR
\NC =\NC \frac{z}{1 - z - z^2} \NR
\NC =\NC \frac{z}{1 - (\phi + \hat\phi)z + \phi\hat\phi z^2} \NR
\NC =\NC \frac{z}{(1 - \phi z)(1 - \hat\phi z)} \NR
\NC =\NC \frac{\sqrt5 z}{\sqrt5 (1 - \phi z)(1 - \hat\phi z)} \NR
\NC =\NC \frac{(\phi - \hat\phi)z + 1 - 1}{\sqrt5 (1 - \phi z)(1 - \hat\phi z)} \NR
\NC =\NC \frac{(1 - \hat\phi z) - (1 - \phi z)}{\sqrt5 (1 - \phi z)(1 - \hat\phi z)} \NR
\NC =\NC \frac{1}{\sqrt5}\Big(\frac{1}{1 - \phi z} - \frac{1}{1 - \hat\phi z}\Big) \NR
\stopmathalignment\stopformula
\stopANSWER

\item 證明: \m{{\cal F}(z) = \sum_{i=0}^{\infty}\frac{1}{\sqrt5}(\phi^i - \hat{\phi}^i)z^i}。

\startANSWER
如果 \m{|x|<1},則 \m{\frac{1}{1 - x} = \sum_{k=0}^{\infty}x^k}。因此:
\startformula\startmathalignment
\NC {\cal F}(n) \NC= \frac{1}{\sqrt5}\Big(\frac{1}{1 - \phi z} - \frac{1}{1 - \hat\phi z}\Big) \NR
\NC                \NC= \frac{1}{\sqrt5}\Big(\sum_{i=0}^{\infty}\phi^i z^i - \sum_{i=0}^{\infty}\hat{\phi}^i z^i\Big) \NR
\NC                \NC= \sum_{i=0}^{\infty}\frac{1}{\sqrt5}(\phi^i - \hat{\phi}^i) z^i\NR
\stopmathalignment\stopformula
\stopANSWER

\item 利用上一項的結果證明:對 \m{i>0}, \m{F_i = \phi^i / \sqrt5},結果舍入到最近整數。
(\hint \m{|\hat\phi|<1})

\startANSWER
\startformula
{\cal F}(z) = \sum_{i=0}^{\infty}\alpha_iz^i \quad\text{其中} \alpha_i = \frac{\phi^i - \hat{\phi}^i}{\sqrt5}
\stopformula

由於 \m{\alpha_i = F_i},即:
\startformula
F_i = \frac{\phi^i - \hat{\phi}^i}{\sqrt5}  = \frac{\phi^i}{\sqrt5} - \frac{\hat{\phi}^i}{\sqrt5}
\stopformula

對於 \m{i = 0}, \m{\phi^i/\sqrt5 = (\sqrt5 + 5)/10 > 0.5};
對於 \m{i > 2}, \m{|{\hat\phi}^i| < 0.5}。
\stopANSWER

\stopigBase
\stopPROBLEM

%p4-5
\startPROBLEM(芯片檢測)
Diogenes教授有 \m{n} 片可能完全一樣的集成電路芯片,原理上可以用來相互檢測。
教授的測試夾具同時只能容納兩塊芯片。當家具裝載上時,每塊芯片都檢測兩一塊,並報告它是好是壞。
一塊好的芯片總能準確報告另一塊芯片的好壞,但教授不能信任壞芯片報告的結果。
因此, 4 種可能的檢測結果如下:
\bTABLE[align=center]
\bTABLEhead\bTR
	\bTH 芯片 A 報告 \eTH
	\bTH 芯片 B 報告 \eTH
	\bTH 結論 \eTH
\eTR\eTABLEhead
\bTABLEbody\bTR
	\bTD B 是好的 \eTD
	\bTD A 是好的 \eTD
	\bTD 兩片都是好的,或都是壞的 \eTD
\eTR\bTR
	\bTD B 是好的 \eTD
	\bTD A 是壞的 \eTD
	\bTD 至少一塊是壞的 \eTD
\eTR\bTR
	\bTD B 是壞的 \eTD
	\bTD A 是好的 \eTD
	\bTD 至少一塊是壞的 \eTD
\eTR\bTR
	\bTD B 是壞的 \eTD
	\bTD A 是壞的 \eTD
	\bTD 至少一塊是壞的 \eTD
\eTR\eTABLEbody
\eTABLE
\startigBase[a]
\item 證明:如果超過 \m{n/2} 塊芯片是壞的,使用任何基於這種結對檢測的策略都不能確定哪些芯片是好的。
假定壞芯片可以合謀欺騙教授。

\startANSWER
假定有 \m{g < n/2} 塊好芯片。其餘同樣數量的壞芯片可以表現出與好芯片類似的行爲。
也就是說,他們可以相互識別爲好芯片,並將其餘芯片識別爲壞的。
由於這兩組芯片表現出來的行爲一樣,無從區分,也就無法識別哪塊是好芯片。
\stopANSWER

\item 考慮從 \m{n} 塊芯片中尋找一塊好芯片的問題,假定超過 \m{n/2} 塊芯片是好的。
證明:進行 \m{\lfloor n/2 \rfloor} 次結對檢測足以將問題規模減半。

\startANSWER
將芯片分成兩組進行比較。如果結果是第一種(都是好的或都是壞的)我們可以取其一,否則兩塊都棄用。
當兩塊都棄用時,我們至少移除了一塊壞芯片,當然可能同時移除了一塊好芯片。
而當我們選用其中一塊時,好芯片數目多於壞芯片(兩塊都是好芯片的對數更多,因爲好芯片多於壞芯片)。
現在我們最多有 \m{n/2} 塊芯片,其中至少一半是好的。
\stopANSWER

\item 假定超過 \m{n/2} 塊芯片是好的,證明:可以用 \m{\Theta(n)} 次結對檢測找到好的芯片。
給出描述檢測次數的遞迴式,並求解。

\startANSWER
遞迴式爲 \m{T(n) = T(n/2) + n/2}。
由主定理可知其解爲 \m{\Theta(n)}。
找到一個好的後,我們可以用他與其他芯片進行結對檢測,即進行 \m{\Theta(n)} 次運算。
\stopANSWER

\stopigBase
\stopPROBLEM

%p4-6
\startPROBLEM(Monge 陣列)
對一個 \m{m\times n} 的實數陣列 \m{A},
若對所有滿足 \m{1\le i < k \le m} 和 \m{1\le j < l \le n} 的 \m{i}、 \m{j}、 \m{k} 和 \m{l} 有:
\startformula
A[i, j] + A[k, l] \le A[i, l] + A[k, j]
\stopformula
則稱 \m{A} 爲 {\EMP Monge 陣列}。
還句話說,在 Monge 陣列中任選兩行和兩列,對於交叉點上的 4 個元素,
左上角和右下角兩個元素之和總是小於等於左下角和右上角元素之和。
例如,下面就是一個 Monge 陣列:
\startformula\startmatrix
\NC 10 \NC 17 \NC 13 \NC 28 \NC 23 \NR
\NC 17 \NC 22 \NC 16 \NC 29 \NC 23 \NR
\NC 24 \NC 28 \NC 22 \NC 34 \NC 24 \NR
\NC 11 \NC 13 \NC  6 \NC 17 \NC  7 \NR
\NC 45 \NC 44 \NC 32 \NC 37 \NC 23 \NR
\NC 36 \NC 33 \NC 19 \NC 21 \NC  6 \NR
\NC 75 \NC 66 \NC 51 \NC 53 \NC 34 \NR
\stopmatrix\stopformula
\startigBase[a]
\startitem 證明:若一個陣列是 Monge 陣列,當且儘當對所有 \m{i=1,2,\ldots,m-1} 和 {j = 1,2,\ldots,n-1} 都有
\startformula
A[i,j] + A[i+1,j+1] \le A[i,j+1] + A[i+1,j]
\stopformula
(\hint 對於“當”的部分,分別對行和列使用歸納法)
\stopitem

\startANSWER
“儘當”的部分是顯然的,由 Monge 陣列的定義可知。
而對於“當”的部分,先用歸納法證明:
\startformula\startmathalignment[n=1]
\NC A[i,j] + A[i+1, j+1] \le A[i,j+1] + A[i+1, j] \NR
\NC \Downarrow \NR
\NC A[i,j] + A[k, j+1] \le A[i, j+1] + A[k,j] \NR
\stopmathalignment\stopformula
其中 \m{i<k}。

第一步, \m{k=i+1},由已知條件顯然成立。
歸納時,我們假設對於 \m{k=i+n} 成立,我們需要證明對於 \m{k+1=i+n+1} 仍然成立。
\startformula\startmathalignment[n=1]
\NC A[i, j] + A[k, j+1] \le A[i, j+1] + A[k, j] \quad (assumption) \NR
\NC A[k, j] + A[k+1, j+1] \le A[k, j+1] + A[k+1, j] \quad (given) \NR
\NC \Downarrow \NR
\NC A[i, j] + A[k, j+1] + A[k, j] + A[k+1, j+1] \le A[i, j+1] + A[k, j] + A[k, j+1] + A[k+1, j] \NR
\NC \Downarrow \NR
\NC A[i, j] + A[k+1, j+1] \le A[i, j+1] + A[k+1, j] \NR
\stopmathalignment\stopformula
\stopANSWER

\startitem
如下陣列並不是 Monge 陣列,改變其中一個元素,使其成爲 Monge 陣列。(\hint 用上一項的結果)
\startformula\startmatrix
\NC 37 \NC 23 \NC 22 \NC 32 \NR
\NC 21 \NC  6 \NC  7 \NC 10 \NR
\NC 53 \NC 34 \NC 30 \NC 31 \NR
\NC 32 \NC 13 \NC  9 \NC  6 \NR
\NC 43 \NC 21 \NC 15 \NC  8 \NR
\stopmatrix\stopformula
\stopitem

\startANSWER

\startformula\startmatrix
\NC 37 \NC 23 \NC \bf 24 \NC 32 \NR
\NC 21 \NC  6 \NC  7 \NC 10 \NR
\NC 53 \NC 34 \NC 30 \NC 31 \NR
\NC 32 \NC 13 \NC  9 \NC  6 \NR
\NC 43 \NC 21 \NC 15 \NC  8 \NR
\stopmatrix\stopformula
\stopANSWER

\item 令 \m{f(i)} 表示第 \m{i} 行中最左最小元素的列下標。
證明:對任意 \m{m\times n} 的 Monge 陣列, \m{f(1)\le f(2)\le\ldots\le f(m)}。

\startANSWER
令 \m{i} 和 \m{j} 分別爲行 \m{a} 和 \m{b} 中的最左最小元素的列下標,其中 \m{a < b}。
假設 \m{i>j},則:
\startformula
A[j, a] + A[i, b] \le A[i, a] + A[j, b]
\stopformula
但是\startformula\startmathalignment
\NC A[j, a] \ge A[i, a] \NC \quad (A[i,a] \text{ 最小}) \NR
\NC A[i, b] \ge A[j, b] \NC \quad (A[j,b] \text{ 最小}) \NR
\stopmathalignment\stopformula
隱含\startformula\startmathalignment
\NC A[j, a] + A[i, b] \ge A[i, a] + A[j, b] \NR
\NC \Downarrow \NR
\NC A[j, a] + A[i, b] = A[i, a] + A[j, b] \NR
\stopmathalignment\stopformula
又分別隱含\startformula\startmathalignment
\NC A[j, b] < A[i, b] \Rightarrow A[i, a] > A[j, a] \NC \Rightarrow A[i,a] \text{ 不是最小} \NR
\NC A[j, b] = A[i, b] \NC \Rightarrow A[j,b] \text{ 不是最左最小} \NR
\stopmathalignment\stopformula

推出矛盾,因此假設 \m{i>j} 不成立,即 \m{i\le j}。
可以得出結論:
\startformula
a < b \Rightarrow f(a) \le f(b)
\stopformula
即 \m{f(1)\le f(2)\le\ldots\le f(m)}。

\stopANSWER

\startitem
下面所描述的分治算法可用於找出 \m{m\times n} Monge 陣列 \m{A} 中每行的最左最小元素:

提取 \m{A} 的偶數行構造其子矩陣 \m{A'}。
遞迴確定 \m{A'} 每行的最左最小元素。
然後找出 \m{A} 的奇數行最左最小元素。

如何在 \m{O(m+n)} 時間內找出 \m{A} 的奇數行最左最小元素?
(假定已知偶數行的最左最小元素)
\stopitem

\startANSWER
如果 \m{\mu_i} 是第 \m{i} 行最左最小元素的列坐標,由上一項可知 \m{\mu_{i-1} \le \mu_i \le \mu_{i+1}}。

對於 \m{i = 2k+1},其中 \m{k\ge 0},
我們只需要比較此行中列坐標從 \m{\mu_{i-1}} 到 \m{\mu_{i+1}} 中的這些元素,
最多 \m{\mu_{i+1}-\mu_{i-1} + 1} 步就可以找出 \m{\mu_i}。
\startformula\startmathalignment
\NC T(m, n) \NC= \sum_{i=0}^{m/2-1}\Big(\mu_{2i + 2} - \mu_{2i} + 1\Big) \NR
\NC \NC= \sum_{i=0}^{m/2-1}\mu_{2i+2} - \sum_{i=0}^{m/2-1}\mu_{2i} + m/2 \NR
\NC \NC= \sum_{i=1}^{m/2}\mu{2i} - \sum_{i=0}^{m/2-1}\mu{2i} + m/2 \NR
\NC \NC= \mu_m - \mu_0 + m/2 \NR
\NC \NC= n + m/2 \NR
\NC \NC= O(m + n) \NR
\stopmathalignment\stopformula
\stopANSWER

\item 給出上一項中所述算法的運行時間的遞迴式。證明其解爲 \m{O(m+n\log{m})}。

\startANSWER
“分”所用時間爲 \m{O(1)},“治”所用時間爲 \m{T(m/2)},合並所用時間爲 \m{O(m+n)}。
\startformula\startmathalignment
\NC T(m) \NC= T(m/2) + cn + dm \NR
\NC      \NC= cn + dm + cn + dm/2 + cn + dm/4 + \ldots \NR
\NC      \NC= \sum_{i=0}^{\lg{m}-1}cn + \sum_{i=0}^{\lg{m}-1}\frac{dm}{2^i} \NR
\NC      \NC= cn\lg{m} + dm\sum_{i=0}^{\lg{m} - 1}\frac{1}{2^i} \NR
\NC      \NC< cn\lg{m} + 2dm \NR
\NC      \NC= O(n\lg{m} + m) \NR
\stopmathalignment\stopformula
\stopANSWER

\stopigBase
\stopPROBLEM

\stopsubject
