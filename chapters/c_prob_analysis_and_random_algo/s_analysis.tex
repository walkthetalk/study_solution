\startsection[
  title={Probabilistic analysis and further uses of indicator random variables},
]

\startEXERCISE
屋子裏至少要有多少人,才能使得某人與你生日相同的概率不小於 \m{1/2}?
至少要有多少人,才能使得至少兩人生日爲 7 月 4 日的概率大於 \m{1/2}?
\stopEXERCISE
\startANSWER
任一人生日與己不同的概率爲 \m{(n-1)/n},
則 \m{k} 個人生日都與己不同的概率爲 \m{(\frac{n-1}{n})^k}。
利用互補事件求 \m{k}:
\startformula\startmathalignment[n=1]
1 - (\frac{n-1}{k})^k \ge \frac{1}{2} \NR
(\frac{n-1}{k})^k \le \frac{1}{2} \NR
k \lg(\frac{n-1}{n}) \ge \lg\frac{1}{2} \NR
k = \frac{\log(1/2)}{\log(364/365)} \approx 263 \NR
\stopmathalignment\stopformula

另外一個問題:
\startformula\startmathalignment
\NC \Pr\{\text{至少兩人生日爲 7 月 4 日}\} \NC=
        1 - \Pr\{\text{只有 1 人生日爲 7 月 4 日}\} - Pr\{\text{沒有人生日爲 7 月 4 日}\} \NR
\NC \NC= 1 - \frac{k}{n}(\frac{n-1}{n})^{k-1} - (\frac{n-1}{n})^k \NR
\NC \NC= 1 - (\frac{n-1}{n})^{k-1}(\frac{n+k-1}{n}) \NR
\stopmathalignment\stopformula
計算可得 \m{k} 爲 115。
\stopANSWER

\startEXERCISE
假設我們將球投到 \m{b} 個箱子裏,直到某個箱子中有兩個球。
每一次投擲都是獨立的,並且每個球落入任一箱子的機會均等。
請問投擲次數的期望值是多少?
\stopEXERCISE
\startANSWER
本質還是生日問題,更多討論參見 \goto{wikipedia}[url(http://en.wikipedia.org/wiki/Birthday_problem#Average_number_of_people)]。
\stopANSWER

\startEXERCISE \DIFFICULT
生日悖論的分析中,要求各人生日彼此獨立是否很重要?
或者,是否只要兩兩成對獨立就足夠了?
證明你的答案。
\stopEXERCISE
\startANSWER
成對獨立足夠了。對於(5.6)之後的推導,有此即可。
\stopANSWER

\startEXERCISE \DIFFICULT
一次聚會需要邀請多少人,才能讓其中 3 人的生日很可能相同?
\stopEXERCISE
\startANSWER
感覺不大對勁:(

令 \m{X_{ijk}} 爲 \m{i}、 \m{j}、 \m{k} 三人生日相同時間的指示器隨機變量,
其概率爲 \m{1/n^2}。則:
\startformula\startmathalignment
\NC E[X] \NC= \sum_{i=1}^n\sum_{j=i+1}^n\sum_{k=j+1}^nX_{ijk} \NR
\NC      \NC= \sum_{i=1}^n\sum_{j=i+1}^n\sum_{k=j+1}^n\frac{1}{n^2} \NR
\NC      \NC= \binom{n}{3}\frac{1}{n^2} \NR
\NC      \NC= \frac{k(k-1)(k-2)}{6n^2} \NR
\stopmathalignment\stopformula
解得 \m{k = 94}。
\stopANSWER

\startEXERCISE \DIFFICULT
一個長度爲 \m{k} 的字串,其中所有字符均選自一個元素個數爲 \m{n} 的集合,
那麼此字串構成一個 \m{k} 排列的概率是多少?
此問題與生日悖論有何關聯?
\stopEXERCISE
\startANSWER
\startformula
\Pr\{k\text{-perm in }n\} = 1 \cdot
                                 \frac{n-1}{n} \cdot
                                 \frac{n-2}{n} \cdots
                                 \frac{n-k+1}{n}
      = \frac{(n-1)!}{(n-k)!n^k}
\stopformula
這是生日問題的互補事件,即 \m{k} 個人生日各不相同。
\stopANSWER

\startEXERCISE \DIFFICULT
假設將 \m{n} 個球投入 \m{n} 個箱子裏,其中每次投球相互獨立,
並且每個球落入任一箱子的機會均等。
空箱子的數目期望值是多少?
正好有一個球的箱子數目期望值是多少?
\stopEXERCISE
\startANSWER
當 \m{n} 足夠大時,兩個答案都漸進於 \m{n/e}。
首先來看空箱子的數目:

令 \m{X_i} 代表的事件爲:第 \m{i} 個箱子爲空:
\startformula
\Pr\{X_i\} = \left(\frac{n-1}{n}\right)^n
                = \left(1 - \frac{1}{n}\right)^n
                \approx \frac{1}{e}
\stopformula
其期望值爲:
\startformula
E[X] = \sum_{i=1}^n E[X_i] = \frac{n}{e}
\stopformula

箱子裏只有一個球的情況類似,其概率爲:
\startformula
\Pr\{Y_i\} = n\frac{1}{n}\left(\frac{n-1}{n}\right)^{n-1}
                = \left(\frac{n-1}{n}\right)^{n-1} \approx \frac{1}{e}
\stopformula
期望值一樣。

參見 \goto{stackexchange}[url(http://math.stackexchange.com/questions/545920/expectation-of-throwing-n-balls-into-n-bins)]。
\stopANSWER

%e5.4-7
\startEXERCISE \DIFFICULT
为使特征序列长度的下界变得更精确,
请说明在 \m{n} 次硬币的公平抛掷中,
不出现比 \m{\lg{n} - 2\lg\lg{n}} 更长的连续正面特征序列的概率小于 \m{1/n}。
\stopEXERCISE

\startANSWER
\TODO{需要證明}
\stopANSWER

\stopsection
