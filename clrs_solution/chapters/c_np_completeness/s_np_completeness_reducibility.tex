\startsection[
  title={NP-completeness and reducibility},
]

%e34.3-1
\startEXERCISE
驗證圖 34-8(b)中的電路是不可滿足的。
\stopEXERCISE

\startANSWER
\TODO{略。}
\stopANSWER

%e34.3-2
\startEXERCISE
證明: \m{\le_P} 關係是語言上的一種傳遞關係,
即證明如果有 \m{L_1\le_P L_2},且 \m{L_2\le_P L_3} 則有 \m{L_1\le_P L_3}。
\stopEXERCISE

\startANSWER
\TODO{略。}
\stopANSWER

%e34.3-3
\startEXERCISE
證明: \m{L_1\le_P \overbar{P}} 當且僅當 \m{\overbar{L_1}\le_P L}。
\stopEXERCISE

\startANSWER
\TODO{略。}
\stopANSWER

%e34.3-4
\startEXERCISE
證明:在對引理 34.5 的另一種證明中,
可滿足性賦值可以當作證書來使用。
試問哪一個證書可以使證明過程更容易些?
\stopEXERCISE

\startANSWER
\TODO{略。}
\stopANSWER

%e34.3-5
\startEXERCISE
在引理 34.6 的證明中,
假定算法 A 的工作存儲佔用的是一塊具有多項式大小的連續存儲區域。
在該證明的什麼地方用到了這一假設?
論證這一假設的過程要具有普適性。
\stopEXERCISE

\startANSWER
\TODO{略。}
\stopANSWER

%e34.3-6
\startEXERCISE[exercise:34.3-6]
如果對所有 \m{L'\in C},有 \m{L\in C} 且 \m{L'\le_P L},
則相對於多項式時間的歸約來說,
一個語言 \m{L} 對語言類 \m{C} 是{\EMP 完全的}。
證明:相對於多項式時間的歸約來說,
 \m{\phi} 和 \m{\{0,1\}^*} 是 \m{P} 中僅有的對 \m{P} 不完全的語言。
\stopEXERCISE

\startANSWER
\TODO{略。}
\stopANSWER

%e34.3-7
\startEXERCISE[exercise:34.3-7]
證明:關於多項式時間歸約(參見\inexercise[34.3-6]),
 \m{L} 對 NP 是完全的,
當且僅當 \m{\overbar{L}} 對 \ALGO{co-NP} 是完全的。
\stopEXERCISE

\startANSWER
\TODO{略。}
\stopANSWER

%e34.3-8
\startEXERCISE
在引理 34.6 的證明中,歸約算法 \m{F} 基於有關 \m{x}、 \m{A} 和 \m{k} 的信息,
構造了電路 \m{C=f(x)}。
 Sartre 教授觀察到串 \m{x} 是 \m{F} 的輸入,
但只有 \m{A}、 \m{k} 的存在性和運行時間 \m{O(n^k)} 中所隱含的常數因子對 \m{F} 來說
是已知的(因爲語言 \m{L} 屬於 NP),
實際值對 \m{F} 來說卻是未知的。
因此,這位教授就得出了這樣的結論,
即 \m{F} 不可能構造出電路 \m{C},
並且語言 \ALGO{CIRCUIT-SAT} 不一定是 NP 難度的。
試說明在這位教授的推理中存在哪些缺陷?
\stopEXERCISE

\startANSWER
\TODO{略。}
\stopANSWER

\stopsection
