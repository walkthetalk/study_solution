\startsection[
  title={Minimum and maximum},
]

\startEXERCISE
證明:在最壞情況下,
找到 \m{n} 個元素中第二小元素需要 \m{n+\lceil\lg{n}\rceil-2} 次比較。
(\hint 以同時找最小元素。)
\stopEXERCISE

\startANSWER
以錦標賽方式比較元素——將其兩個一組進行比較,然後以同樣的方式對獲勝者進行比較。
需要跟蹤潛在的贏家所參與的每次“賽事”。

通過 \m{n-1} 次比較確定最終贏家。
而第二小元素就在比賽輸於最小元素的 \m{\lceil\lg{n}\rceil} 中——
其中每個元素都是在所參與的最後一次賽事中失利。
因此要找到最小元素還須 \m{\lceil\lg{n}\rceil - 1} 次比較。
\stopANSWER

\startEXERCISE\DIFFICULT
證明:在最壞情況下,同時找到 \m{n} 個元素中最大值和最小值的比較次數的下界
是 \m{\lceil 3n/2 \rceil - 2}。
(\hint 慮有多少個數有成爲最大值或最小值的潛在可能,
然後分析一下每一次比較會如何影響這些計數。)
\stopEXERCISE

\startANSWER
以錦標賽方式,最大元素需要 \m{n-1} 次比較,而最小元素只可能在第一輪比賽中失利的元素中產生,
這樣的元素有 \m{\lceil n/2 \rceil} 個(考慮 \m{n} 爲奇數的情況)。
要產生最小元素還須 \m{\lceil n/2 \rceil - 1} 次比較。
一共需要 \m{\lceil 3n/2 \rceil - 2} 次比較。
\stopANSWER

\stopsection
