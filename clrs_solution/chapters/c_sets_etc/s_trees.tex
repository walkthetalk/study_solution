\startsection[
  title={Trees},
]

%eB.5-1
\startEXERCISE
畫出包含三個頂點 \m{x}、 \m{y} 和 \m{z} 的所有自由樹。
畫出包含節點 \m{x}、 \m{y} 和 \m{z} 的以 \m{x} 爲根的所有有根樹。
畫出包含節點 \m{x}、 \m{y} 和 \m{z} 的以 \m{x} 爲根的所有有序樹。
畫出包含節點 \m{x}、 \m{y} 和 \m{z} 的以 \m{x} 爲根的所有二叉樹。
\stopEXERCISE

\startANSWER
\TODO{略。}
\stopANSWER

%eB.5-2
\startEXERCISE
令 \m{G=(V,E)} 爲一個有向無環圖,
其中存在一個頂點 \m{v_0\in V},
滿足從 \m{v_0} 到其他每個頂點 \m{v\in V} 均有唯一路徑。
證明: \m{G} 的無向版本是一棵樹。
\stopEXERCISE

\startANSWER
\TODO{略。}
\stopANSWER

%eB.5-3
\startEXERCISE
利用歸納法證明:
任何非空二叉樹中 2 度節點數比葉節點數少 1。
證明:
滿二叉樹中的內部節點數目比葉節點數目少 1。
\stopEXERCISE

\startANSWER
\TODO{略。}
\stopANSWER

%eB.5-4
\startEXERCISE
利用歸納法證明:
一個有 \m{n} 個節點的非空二叉樹,其高度至少爲 \m{\lfloor \lg n\rfloor}。
\stopEXERCISE

\startANSWER
\TODO{略。}
\stopANSWER

%eB.5-5
\startEXERCISE\DIFFICULT
一棵滿二叉樹的{\EMP 內部路徑長度}是指所有內部節點深度之和。
類似地,{\EMP 外部路徑長度}是指所有葉節點深度之和。
考慮一個有 \m{n} 個內部節點的滿二叉樹,
其內部路徑長度爲 \m{i},
外路徑長度爲 \m{e}。
證明: \m{e=i+2n}。
\stopEXERCISE

\startANSWER
\TODO{略。}
\stopANSWER

%eB.5-6
\startEXERCISE\DIFFICULT
我們將二叉樹 \m{T} 中的每個深度爲 \m{d} 的葉節點賦予權值 \m{\omega(x)=2^{-d}},
並令 \m{L} 爲 \m{T} 的葉節點集合。
證明 \m{\sum_{x\in L}\omega(x)\le 1}。
(該不等式稱爲 {\EMP Kraft 不等式}。)
\stopEXERCISE

\startANSWER
\TODO{略。}
\stopANSWER

%eB.5-7
\startEXERCISE\DIFFICULT
證明:若 \m{L\ge 2},
則每個葉節點數爲 \m{L} 的二叉樹包含一棵子樹,
該子樹有 \m{L/3} 到 \m{2L/3} 個葉節點。
\stopEXERCISE

\startANSWER
\TODO{略。}
\stopANSWER

\stopsection
