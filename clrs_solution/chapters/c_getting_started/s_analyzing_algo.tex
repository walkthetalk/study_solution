\startsection[
  title={Analyzing algorithms},
]

\startEXERCISE
用 $\Theta$ 標記表示函式 $n^3 / 1000 - 100 n^2 - 100 n + 3$。
\stopEXERCISE
\startANSWER
$\Theta(n^3)$
\stopANSWER

\startEXERCISE[exercise:selection_sort]
對數列 $A[1:n]$ 中的 $n$ 個數進行排序。
先找到 $A[1:n]$ 中的最小元素,並將其與 $A[1]$ 進行交換;
然後找到 $A[2:n]$ 中的最小元素,並將其與 $A[2]$ 進行交換;
再找到 $A[3:n]$ 中的最小元素,並將其與 $A[3]$ 進行交換;
依次类推,對 $A$ 中前 $n-1$ 個元素都用這種方式進行處理。
這種排序方式稱爲{\EMP 選擇排序},試寫出其僞碼。
此算法滿足循環不變性嗎?
爲什麽僅需處理前 $n-1$ 個元素,而不是所有元素?
給出其最壞情況下的運行時間,用 $\Theta$ 表示。
最好情況下的運行時間有改善嗎?
\stopEXERCISE

\startANSWER
\startCLRSCODE
n = A.length
for j = 1 to n - 1
	smallest = j
	for i = j + 1 to n
		if A[i] < A[smallest]
			smallest = i
	// exchange $A[j]$ with $A[smallest]$
	tmp = A[j]
	A[j] = A[smallest]
	A[smallest] = tmp
\stopCLRSCODE

\startformula
\startmathalignment[n=2]
\NC \NC\sum_{i=1}^{n-1}n-i \NR
\NC = \NC n(n-1) - \sum_{i=1}^{n-1}i \NR
\NC = \NC n^2-n-\frac{n^2-n}{2} \NR
\NC = \NC \frac{n^2-n}{2} \NR
\NC = \NC \Theta(n^2) \NR
\stopmathalignment
\stopformula

所用時間爲 $\Theta(n^2)$。
\stopANSWER

\startEXERCISE
回顧\inexercise[linear_earch]中的綫性查找問題。
假設所搜索的值可能是數列中的任一元素,
平均要檢查輸入數列中多少個元素?
最壞情況呢?
用 $\Theta$ 標記表示這兩種情況的運行時間,
並驗證你的答案。
\stopEXERCISE
\startANSWER
一般情況 $\Theta(n)$。
最壞情況 $\Theta(n)$。
\stopANSWER

\startEXERCISE
對於任一排序算法,如何修改使其最好情況下運行時間最短。
\stopEXERCISE
\startANSWER
測試輸入是否滿足某個特例,如果滿足,則直接輸出預先算好的結果。

最好情況下的運行時間一般不能作爲衡量一個算法好壞的標準。
\stopANSWER

\stopsection
