\startsubject[
  title={Problems},
]

%p3-1
\startPROBLEM
(多項式的漸近行爲)
令 $p(n) = \sum_{i = 0}^{d} {a_i n^i}$ 是 $n$ 的 $d$ 次多項式,
其中 $a_d > 0$, $k$是一個常量。
用漸進記號的定義證明下列性質。
\startigBase[a]
\item 如果 $k \ge d$,那麼 $p(n) = O(n^k)$。
\item 如果 $k \le d$,那麼$p(n) = \Omega(n^k)$。
\item 如果 $k = d$,那麼 $p(n) = \Theta(n^k)$。
\item 如果 $k > d$,那麼 $p(n) = o(n^k)$。
\item 如果 $k < d$,那麼 $p(n) = \omega(n^k)$。
\stopigBase
\stopPROBLEM

\startANSWER
取 $c = a_d + b$,滿足下列不等式:
\startformula
p(n) = \sum_{i = 0}^{d}a_i n^i = a_d n^d + a_{d-1}n^{d-1} + \ldots + a_1 n + a_0 \leq cn^d
\stopformula
兩邊同除以 $n^d$,有:
\startformula
c = a_d + b \geq a_d + \frac{a_{d-1}}n + \frac{a_{d-2}}{n^2} + \ldots + \frac{a_0}{n^d}
\stopformula
即
\startformula
b \geq \frac{a_{d-1}}n + \frac{a_{d-2}}{n^2} + \ldots + \frac{a_0}{n^d}
\stopformula
如果使 $b = 1$,則選取 $n_0$,使得:
\startformula
n_0 = \max(da_{d-1}, d\sqrt{a_{d-2}}, \ldots, d\sqrt[d]{a_0})
\stopformula
對於所有 $n\ge n_0$,有 $p(n) \le c n^d$,即 $O(n^d)$;
如果選 $b = -1$,則可得 $\Omega(n^d)$;
綜合可得 $\Theta(n^d)$。
另外兩個的證明類似。
\stopANSWER

%p3-2
\startPROBLEM
(相對漸進增長)
對於下表中每對算式 $(A, B)$,
指明 $A$ 是否是 $B$ 的 $O$、 $o$、 $\Omega$、 $\omega$、 $\Theta$。
假定 $k \ge 1$, $\epsilon > 0$ 且 $c > 1$,均爲常量。

\bTABLE[align=center]
\bTABLEhead
\bTR
	\bTH $A$ \eTH
	\bTH $B$ \eTH
	\bTH $O$ \eTH
	\bTH $o$ \eTH
	\bTH $\Omega$ \eTH
	\bTH $\omega$ \eTH
	\bTH $\Theta$ \eTH
\eTR
\eTABLEhead
\bTABLEbody
\bTR
	\bTD $\lg^k n$ \eTD
	\bTD $n^\epsilon$ \eTD
	\bTD\startANSWER yes \stopANSWER\eTD
	\bTD\startANSWER yes \stopANSWER\eTD
	\bTD\startANSWER no \stopANSWER\eTD
	\bTD\startANSWER no \stopANSWER\eTD
	\bTD\startANSWER no \stopANSWER\eTD
\eTR
\bTR
	\bTD $n^k$ \eTD
	\bTD $c^n$ \eTD
	\bTD\startANSWER yes \stopANSWER\eTD
	\bTD\startANSWER yes \stopANSWER\eTD
	\bTD\startANSWER no \stopANSWER\eTD
	\bTD\startANSWER no \stopANSWER\eTD
	\bTD\startANSWER no \stopANSWER\eTD
\eTR
\bTR
	\bTD $\sqrt{n}$ \eTD
	\bTD $n^{\sin n}$ \eTD
	\bTD\startANSWER no \stopANSWER\eTD
	\bTD\startANSWER no \stopANSWER\eTD
	\bTD\startANSWER no \stopANSWER\eTD
	\bTD\startANSWER no \stopANSWER\eTD
	\bTD\startANSWER no \stopANSWER\eTD
\eTR
\bTR
	\bTD $2^n$ \eTD
	\bTD $2^{n/2}$ \eTD
	\bTD\startANSWER no \stopANSWER\eTD
	\bTD\startANSWER no \stopANSWER\eTD
	\bTD\startANSWER yes \stopANSWER\eTD
	\bTD\startANSWER yes \stopANSWER\eTD
	\bTD\startANSWER no \stopANSWER\eTD
\eTR
\bTR
	\bTD $n^{\lg c}$ \eTD
	\bTD $c^{\lg n}$ \eTD
	\bTD\startANSWER yes \stopANSWER\eTD
	\bTD\startANSWER no \stopANSWER\eTD
	\bTD\startANSWER yes \stopANSWER\eTD
	\bTD\startANSWER no \stopANSWER\eTD
	\bTD\startANSWER yes \stopANSWER\eTD
\eTR
\bTR
	\bTD $\lg(n!)$ \eTD
	\bTD $\lg(n^n)$ \eTD
	\bTD\startANSWER yes \stopANSWER\eTD
	\bTD\startANSWER no \stopANSWER\eTD
	\bTD\startANSWER yes \stopANSWER\eTD
	\bTD\startANSWER no \stopANSWER\eTD
	\bTD\startANSWER yes \stopANSWER\eTD
\eTR
\eTABLEbody
\eTABLE
\stopPROBLEM

\startPROBLEM
(根據漸進增長率排序)
\startigBase[a]
\startitem
根據增長的階爲下列函數排序,即求出滿足
\m{g_1 = \Omega(g_2)}、\m{g_2 = \Omega(g_3)}、\m{\ldots}、\m{g_{29} = \Omega(g_{30})}
的函數的一種排列 \m{g_1, g_2, \ldots, g_{30}}。
並將這些函數劃分成等價類,當且僅當 \m{f(n) = \Theta(g(n))}時,\m{f(n)}和\m{g(n)}才再同一類中。

\bTABLE[align=center]
\bTR \bTD \m{\lg(\lg^{\ast}n)} \eTD \bTD \m{2^{\lg^{\ast}n}} \eTD \bTD \m{(\sqrt{2})^{\lg{n}}} \eTD \bTD \m{n^2} \eTD \bTD \m{n!} \eTD \bTD \m{(\lg{n})!} \eTD \eTR
\bTR \bTD \m{(\frac{3}{2})^n} \eTD \bTD \m{n^3} \eTD \bTD \m{\lg^2{n}} \eTD \bTD \m{\lg(n!)} \eTD \bTD \m{2^{2^n}} \eTD \bTD \m{n^{1/\lg{n}}} \eTD \eTR
\bTR \bTD \m{\ln{\ln{n}}} \eTD \bTD \m{\lg^{\ast}n} \eTD \bTD \m{n \cdot 2^n} \eTD \bTD \m{n^{\lg\lg{n}}} \eTD \bTD \m{\ln{n}} \eTD \bTD \m{1} \eTD \eTR
\bTR \bTD \m{2^{\lg{n}}} \eTD \bTD \m{(\lg{n})^{\lg{n}}} \eTD \bTD \m{e^n} \eTD \bTD \m{4^{\lg{n}}} \eTD \bTD \m{(n + 1)!} \eTD \bTD \m{\sqrt{\lg{n}}} \eTD \eTR
\bTR \bTD \m{\lg^{\ast}(\lg{n})} \eTD \bTD \m{2^{\sqrt{2\lg{n}}}} \eTD \bTD \m{n} \eTD \bTD \m{2^n} \eTD \bTD \m{n\lg{n}} \eTD \bTD \m{2^{2^{n + 1}}} \eTD \eTR
\eTABLE
\stopitem

\startANSWER
\startformula\startalign
\NC 2^{\lg{n}} \NC = \sqrt{n} \NR
\NC \sqrt{2}^{\lg{n}} \NC = 2^{1/2\lg{n}} = 2^{\lg{\sqrt{n}}} = \sqrt{n} \NR
\NC n! < n^n \NC = 2^{\lg{n^n}} = 2^{n\lg{n}} \NR
\NC n^{1/\lg{n}} \NC = n^{\log_n{2}} = 2 \NR
\NC n^{\lg{\lg{n}}} \NC = (2^{\lg{n}})^{\lg\lg{n}} = (2^{\lg\lg{n}})^{\lg{n}} = (\lg{n})^{\lg{n}} \NR
\NC \lg^2{n} \NC = 2^{\lg{\lg^2{n}}} = o(2^{\sqrt{2\lg{n}}}) \NR
\stopalign\stopformula
\startcolumns[n=3,blank=small,distance=2em,balance=yes]
\startigBase[n]
\item \m{2^{2^{n+1}}}
\item \m{2^{2^n}}
\item \m{(n + 1)!}
\item \m{n!}
\item \m{e^n}
\item \m{n \cdot 2^n}
\item \m{2^n}
\item \m{(\frac{3}{2})^n}
\item \m{n^{\lg\lg{n}} = (\lg{n})^{\lg{n}}}
\item \m{(\lg n)!}
\item \m{n^3}
\item \m{n^2 = 4^{\lg{n}}}
\item \m{n\lg{n} \simeq \lg(n!)}
\item \m{n = 2^{\lg{n}}}
\item \m{(\sqrt{2})^{\lg{n}}}
\item \m{2^{\sqrt{2\lg{n}}}}
\item \m{\lg^2{n}}
\item \m{\ln{n}}
\item \m{\sqrt{\lg{n}}}
\item \m{\ln{\ln{n}}}
\item \m{2^{\lg^{\ast}n}}
\item \m{\lg^{\ast}n\simeq \lg^{\ast}{\lg{n}}}
\item \m{\lg(\lg^{\ast}n)}
\item \m{1 = n^{1/\lg n}}
\stopigBase
\stopcolumns
\stopANSWER

\startitem
給出一個非負函數 $f(n)$,
使得對於上一問中所有 $g_i(n)$,
$f(n)$既不是 $O(g_i(n))$,
也不是 $\Omega(g_i(n))$。
\stopitem

\startANSWER
$2^{2^{(n + 1)\sin{n}}}$,或
\startformula
f(n) = \startmathcases
\NC 2^{2^{n+2}} \NC 若 n 是偶數,\NR
\NC 0 \NC 若 n 是奇數。\NR
\stopmathcases
\stopformula
\stopANSWER
\stopigBase
\stopPROBLEM

%p3-4
\startPROBLEM
(漸進記號的性質)
令 $f(n)$ 和 $g(n)$ 是漸進正函數,證明或反駁下面的猜測。
\startigBase[a]
\item $f(n) = O(g(n))$ 蘊含 $g(n) = O(f(n))$。

\startANSWER
錯誤。 $n = O(n^2)$,但是 $n^2 \ne O(n)$。
\stopANSWER

\item $f(n) + g(n) = \Theta(\min(f(n), g(n)))$。

\startANSWER
錯誤。 $n^2 + n \ne \Theta(min(n^2, n)) = \Theta(n)$。
\stopANSWER

\item 如果對於所有足夠大的 $n$,
有 $\lg(g(n))\ge 1$ 且 $f(n)\ge 1$,
那麼 $f(n) = O(g(n))$ 蘊含 $\lg(f(n)) = O(lg(g(n)))$。

\startANSWER
正確。 因爲對於給定 $n \ge n_0$, $f(n) \ge 1$:
\startformula
\exists c, n_0 : \forall n \ge n_0 : 0 \le f(n) \le cg(n)
\stopformula
\startformula
   \Downarrow
\stopformula
\startformula
   0 \le \lg{f(n)} \le \lg(cg(n)) = \lg{c} + \lg{g(n)}
\stopformula
需要證明:
\startformula
\lg{f(n)} \le d\lg{g(n)}
\stopformula
很容易找到 $d$:
\startformula
d = \frac{\lg{c} + \lg{g(n)}}{\lg{g(n)}}
= \frac{\lg{c}}{\lg{g(n)}} + 1
\le \lg{c} + 1
\stopformula
最後一步顯然成立,因爲 \m{\lg{g(n)} \geq 1}。
\stopANSWER

\item \m{f(n) = O(g(n))} 蘊含 \m{2^{f(n)} = O(2^{g(n)})}。

\startANSWER
錯誤。 $2n = O(n)$,但是 $2^{2n} = 4^n \ne O(2^n)$。
\stopANSWER

\item $f(n) = O((f(n))^2)$。

\startANSWER
正確。只要 $f(n) \ge 1$, $0 \le f(n) \le c(f(n))^2$ 是很自然的。
當然如果對於所有 $n$, $f(n) < 1$,則錯誤,但我們通常不考慮這種函數。
\stopANSWER

\item $f(n) = O(g(n))$ 蘊含 $g(n) = \Omega(f(n))$。

\startANSWER
正確。如果 $f(n) = O(g(n))$,
則 $0 \le f(n) \le cg(n)$,我們只需證明:
\startformula
0 \le d f(n) \le g(n)
\stopformula
對於 $d = 1/c$,上式顯然成立。
\stopANSWER

\item $f(n) = \Theta(f(n/2))$。

\startANSWER
錯誤。取 \m{f(n) = 2^n},我們需要證明:
\startformula
\exists c_1, c_2, n: \forall n \geq n_0 : 0 \leq c_1 \cdot 2^{n/2} \leq 2^n
   \leq c_2 \cdot 2^{n/2}
\stopformula
顯然不成立。
\stopANSWER

\item \m{f(n) + o(f(n)) = \Theta(f(n))}。

\startANSWER
正確。令 $g(n) = o(f(n))$,
我們需要證明 $c_1 f(n) \le f(n) + g(n) \le c_2 f(n)$,
我們知道:
\startformula
\forall c \exists n_0 \forall n \geq n_0 : cg(n) < f(n)
\stopformula
因此,只需 $c_1 = 1$, $c_2 = 2$ 即可。
\stopANSWER

\stopigBase
\stopPROBLEM

%p3-5
\startPROBLEM
(掌握漸進記號)
令 $f(n)$ 和 $g(n)$ 爲漸進正函式。
證明:
\startigBase[a]

% a
\startitem
$\Theta(\Theta(f(n))) = \Theta(f(n))$

\startANSWER
\startformula\startmathalignment[n=3]
\NC c_1 f(n) \le \NC \Theta(f(n)) \NC \le c_2 f(n) \NR
\NC c_3 \Theta(f(n)) \le \NC \Theta(\Theta(f(n))) \NC \le c_4 \Theta(f(n)) \NR
\NC c_1 c_3 f(n) \le \NC \Theta(\Theta(f(n))) \NC \le c_2 c_4 f(n) \NR
\stopmathalignment\stopformula
\stopANSWER
\stopitem

% b
\startitem
$\Theta(f(n)) + O(f(n)) = \Theta(f(n))$

\startANSWER
\startformula\startmathalignment[n=3]
\NC c_1 f(n) \le \NC \Theta(f(n)) \NC \le c_2 f(n) \NR
\NC \NC O(f(n)) \NC \le c_3 f(n) \NR
\NC c_1 \Theta(f(n)) \le \NC \Theta(f(n)) + O(f(n)) \NC \le (c_2 + c_3) f(n) \NR
\stopmathalignment\stopformula
\stopANSWER
\stopitem

% c
\startitem
$\Theta(f(n)) + \Theta(g(n)) = \Theta(f(n)+g(n))$

\startANSWER
\startformula\startmathalignment[n=3]
\NC c_1 f(n) \le \NC \Theta(f(n)) \NC \le c_2 f(n) \NR
\NC c_3 g(n) \le \NC \Theta(g(n)) \NC \le c_4 g(n) \NR
\NC \min(c_1,c_3) (f(n)+g(n)) \le
	 \NC \Theta(f(n)) + \Theta(g(n))
	 \NC \le \max(c_2 + c_4) (f(n)+g(n)) \NR
\stopmathalignment\stopformula
\stopANSWER
\stopitem

% d
\startitem
$\Theta(f(n)) \cdot \Theta(g(n)) = \Theta(f(n) \cdot g(n))$

\startANSWER
\startformula\startmathalignment[n=3]
\NC c_1 f(n) \le \NC \Theta(f(n)) \NC \le c_2 f(n) \NR
\NC c_3 g(n) \le \NC \Theta(g(n)) \NC \le c_4 g(n) \NR
\NC c_1 c_3 (f(n)\cdot g(n)) \le
	 \NC \Theta(f(n)) \cdot \Theta(g(n))
	 \NC \le c_2 c_4 (f(n)\cdot g(n)) \NR
\stopmathalignment\stopformula
\stopANSWER
\stopitem

% e
\startitem
分析對於任意常實數 $a_1,a_2 > 0$ 和常整數 $k_1,k_2$,
下列漸進界是否都能成立:
\startformula
(a_1 n)^{k_1} \lg^{k_2}(a_2 n) = \Theta(n^{k_1} \lg^{k_2} n)
\stopformula

\startANSWER
令 $n>n_0=1/a_2$,則有 $\lg(a_2 n)\ge 0$;
令 $0 < c_1 < 1 < c_2$,對於任意 $a_2 > 0$,要使下式成立:
\startformula\startmathalignment[n=3,align={right,middle,left}]
\NC n^{c_1 - 1} \le \NC a_2 \NC \le n^{c_2 - 1} \NR
\stopmathalignment\stopformula
,要求 $n\ge {a_2}^{1/(c_1-1)}$,且 $n\ge {a_2}^{1/(c_2-1)}$,
即 $n \ge \max({a_2}^{1/(c_1-1)}, {a_2}^{1/(c_2-1)})$,

\startformula\startmathalignment[n=3,align={right,middle,left}]
\NC n^{c_1 - 1} \le \NC a_2 \NC \le n^{c_2 - 1} \NR
\NC n^{c_1} \le \NC a_2 n \NC \le n^c_2 \NR
\NC c_1 \lg n \le \NC \lg (a_2 n) \NC \le c_2 \lg n \NR
\NC c_1^{k_2} \lg^{k_2} n \le \NC \lg^{k_2}(a_2 n) \NC \le c_2^{k_2} \lg^{k_2}n \NR
\NC c_1^{k_2} (a_1 n)^{k_1} \lg^{k_2} n \le
    \NC (a_1 n)^{k_1} \lg^{k_2}(a_2 n)
	\NC \le c_2^{k_2} (a_1 n)^{k_1} \lg^{k_2}n \NR
\NC (c_1^{k_2} a_1^{k_1}) n^{k_1} \lg^{k_2} n \le
    \NC (a_1 n)^{k_1} \lg^{k_2}(a_2 n)
	\NC \le (c_2^{k_2} a_1^{k_1}) n^{k_1} \lg^{k_2}n \NR
\stopmathalignment\stopformula
\stopANSWER
\stopitem

% f
\startitem
證明對於所有 $S\subseteq Z$,下式成立:
\startformula
\sum_{k\in S}\Theta(f(k)) = \Theta(\sum_{k\in S}f(k))
\stopformula
假設兩邊求和均收斂。

\startANSWER
\TODO{略。}
\stopANSWER
\stopitem

% g
\startitem
對於 $S\subseteq Z$,下列漸進界不一定成立,
即使兩側求積均收斂,請給出反例。
\startformula
\prod_{k\in S}\Theta(f(k)) = \Theta(\prod_{k\in S}f(k))
\stopformula

\startANSWER
\TODO{略。}
\stopANSWER
\stopitem

\stopigBase
\stopPROBLEM

%p3-6
\startPROBLEM
(\m{O}與\m{\Omega}的一些變形)
某些作者用一種與我們稍微不同的方式來定義 \m{\Omega};
假設我們使用 \m{\mathop{\Omega}\limits^{\infty}} 來表示這種定義。
若存在正常量 \m{c},使得對無窮多個整數 \m{n},有 \m{f(n)\geq cg(n)\geq 0},
則稱 \m{f(n) = \mathop{\Omega}\limits^{\infty}(g(n))}。
\startigBase[a]
\item 證明:對漸進非負的任意兩個函數 \m{f(n)} 和 \m{g(n)},
或者 \m{f(n) = O(g(n))} 或者 \m{f(n) = \mathop{\Omega}\limits^{\infty}(g(n))} 或者二者均成立,
然而,如果用 \m{\Omega} 代替 \m{\mathop{\Omega}\limits^{\infty}},那麼該命題卻不成立。

\startANSWER
我們需要比較 \m{cg(n) \leq f(n)};
如果對於正無窮整數均成立,則有 \m{\mathop{\Omega}\limits^{\infty}};
而如果之對有限的整數成立,則設最大值爲 \m{n_0},有:
\startformula
\forall n > n_0: f(n) < cg(n)
\stopformula
足以說明 \m{f(n) = O(g(n))}。

如果 \m{f(n) = g(n)},顯然兩式均成立。

但是對於 \m{\Omega},卻不一定,
如 \m{n = \mathop{\Omega}\limits^{\infty}(n^{\sin{n}})},
但 \m{n \neq \Omega(n^{\sin{n}})}。
\stopANSWER

\item 描述用 \m{\mathop{\Omega}\limits^{\infty}} 代替 \m{\Omega} 來刻畫程序運行時間的潛在優點和缺點。

\startANSWER
TODO
\stopANSWER

\stopigBase

某些作者也用一種稍微不同的方式定義 \m{O};
假設用 \m{O'} 來表示這種可選的定義。
我們稱 \m{f(n) = O'(g(n))},當且儘當 \m{|f(n)| = O(g(n))}。

\startigBase[a,continue]
\item 如果用 \m{O'} 代替 \m{O},但仍使用 \m{\Omega},定理 3.1 中的“當且儘當”的每個方向將出現什麼情況?

\startANSWER
定理 3.1 中的“當且儘當”要改爲“蘊含”,即 \m{\Theta \Rightarrow O'},反向則不成立。
比如函數 \m{f(n) = n \cdot \sin{n}},即爲 \m{O'(n)},但卻不是 \m{O(n)} 或 \m{\Theta(n)}。
\stopANSWER
\stopigBase

有些作者定義 \m{\tilde{O}} 來意指忽略對數因子的 \m{O}:
\startformula
\tilde{O} = \lbrace f(n) : \exists c > 0, k > 0, n_0 > 0 \forall n \geq n_0: 0 \leq f(n) \leq cg(n)\lg^k(n) \rbrace
\stopformula
\startigBase[a,continue]
\item 用類似方式定義 \m{\tilde{\Omega}} 和 \m{\tilde{\Theta}}。
證明定理 3.1 相對應的類似結論。

\startANSWER
\startformula
\tilde{\Omega} = \lbrace f(n) : \exists c, k, n_0 \forall n > n_0 : 0 \leq cg(n) \lg^{-k}(n) \leq f(n) \rbrace
\stopformula
\startformula
\tilde{\Theta} = \lbrace f(n) : \exists c_1, c_2, k_1, k_2, n_0 \forall n > n_0 : 0 \leq c_1g(n) \lg^{-k_1}(n) \leq f(n) \leq c_2g(n) \lg^{k_2}(n)\rbrace
\stopformula
\stopANSWER
\stopigBase

\stopPROBLEM

\startPROBLEM
(多重函數)我們可以把用於函數 \m{\lg^{\ast}} 中的重復運算符 \m{\ast} 應用於實數集上的任意單調遞增函數 \m{f(n)}。
對給定的常量 \m{c \in R},我們定義多重函數 \m{f_c^{\ast}} 爲:
\startformula
f_c^{\ast}(n) = \min \lbrace i \geq 0 : f^{(i)}(n) \leq c \rbrace
\stopformula
該函數不必在所有情況下都爲良定義的。
換句話說, \m{f_c^{\ast}(n)} 的值就是爲將其參數縮小至 \m{c} 或更小所需要將函數 \m{f} 重復應用的數目。
對如下每個函數 \m{f(n)} 和常量 \m{c},給出 \m{f_c^{\ast}(n)} 的一個盡量緊確的界。

\bTABLE[align=center]
\bTABLEhead
\bTR
	\bTH \m{f(n)} \eTH
	\bTH \m{c} \eTH
	\bTH \m{f_c^{\ast}(n)} \eTH
\eTR
\eTABLEhead
\bTABLEbody
\bTR
	\bTD \m{n - 1} \eTD
	\bTD \m{0} \eTD
	\bTD\startANSWER \m{\Theta(n)} \stopANSWER\eTD
\eTR
\bTR
	\bTD \m{\lg{n}} \eTD
	\bTD \m{1} \eTD
	\bTD\startANSWER \m{\Theta(\lg^{\ast}n)} \stopANSWER\eTD
\eTR
\bTR
	\bTD \m{n/2} \eTD
	\bTD \m{1} \eTD
	\bTD\startANSWER \m{\Theta(\lg{n})} \stopANSWER\eTD
\eTR
\bTR
	\bTD \m{n/2} \eTD
	\bTD \m{2} \eTD
	\bTD\startANSWER \m{\Theta(\lg{n})} \stopANSWER\eTD
\eTR
\bTR
	\bTD \m{\sqrt{n}} \eTD
	\bTD \m{2} \eTD
	\bTD\startANSWER \m{\Theta(\lg\lg{n})} \stopANSWER\eTD
\eTR
\bTR
	\bTD \m{\sqrt{n}} \eTD
	\bTD \m{1} \eTD
	\bTD\startANSWER 無法收斂 \stopANSWER\eTD
\eTR
\bTR
	\bTD \m{n^{1/3}} \eTD
	\bTD \m{2} \eTD
	\bTD\startANSWER \m{\Theta(\log_3\lg{n})} \stopANSWER\eTD
\eTR
\bTR
	\bTD \m{n/\lg{n}} \eTD
	\bTD \m{2} \eTD
	\bTD\startANSWER \m{\omega(\lg\lg{n}), o(\lg{n})} \stopANSWER\eTD
\eTR
\eTABLEbody
\eTABLE

\stopPROBLEM

\stopsubject
