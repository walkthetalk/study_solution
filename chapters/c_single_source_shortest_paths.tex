\startcomponent c_single_source_shortest_paths

\startchapter[
  title={Single-Source Shortest Paths},
]

\startsection[
  title={The Bellman-Ford algorithm},
]

%e24.1-1
\startEXERCISE
在圖 24-4 上運行 Bellman-Ford 算法,
使用節點 \m{z} 作爲源節點。
在每一遍鬆弛過程中,以圖中相同的次序對每條邊進行鬆弛,
給出每遍鬆弛操作後的 \m{d} 值和 \m{\pi} 值。
然後,把邊 \m{(z,x)} 的權重改爲 4,
再次運行該算法,這次使用 \m{s} 作爲源節點。
\stopEXERCISE

\startANSWER
從 \m{z} 點開始:

\startcombination[nx=3,ny=2]
{\externalfigure[output/e24_1_1-1]}
{\externalfigure[output/e24_1_1-2]}
{\externalfigure[output/e24_1_1-3]}
{\externalfigure[output/e24_1_1-4]}
{\externalfigure[output/e24_1_1-5]}
{}
\stopcombination

改了 \m{(z,x)} 的權重後,會出現總權重爲負值的環路。

\externalfigure[output/e24_1_1-10]
\stopANSWER

%e24.1-2
\startEXERCISE
證明推論 24.3。附推論 24.3:

設 \m{G=(V,E)} 是一帶權重的源節點爲 \m{s} 的有向圖,
其權重函數爲 \m{\omega: E\rightarrow R}。
假定圖 \m{G} 不包含從 \m{s} 可以到達的權重爲負值的環路,
則對於所有節點 \m{v\in V},
存在一條從源節點 \m{s} 到節點 \m{v} 的路徑
當且僅當 \ALGO{BELLMAN-FORD} 算法終止時有 \m{v.d < \infty}。
\stopEXERCISE

\startANSWER
首先,假設 \m{G} 中存在 \m{s} 到 \m{v} 的路徑,
證明算法終止時  \m{v.d < \infty}。
根據引理 24.2 可知 \m{v.d=\delta(s,v)},滿足 \m{v.d < \infty}。

現在,假定算法 \ALGO{BELLMAN-FORD} 終止時 \m{v.d<\infty}。
而根據引理 24.2,可知 \m{v.d=\infty} 或者 \m{v.d=\delta(s,v)}。
根據假設可知 \m{v.d=\delta(s,v)}。
因此 \m{G} 中存在從 \m{s} 到 \m{v} 的路徑。

附引理 24.2:

設 \m{G=(V,E)} 爲一個帶權重的源節點爲 \m{s} 的有向圖,
其權重函數爲 \m{\omega: E\rightarrow R}。
假定圖 \m{G} 不包含從 \m{s} 可以到達的權重爲負值的環路。
那麼算法 \ALGO{BELLMAN-FORD} 的第 2~4 行的 {\EMP for} 循環
執行了 \m{|V|-1} 次之後,
對於所有從源節點 \m{s} 可以到達的節點 \m{v},
我們有 \m{v.d=\delta(s,v)}。
\stopANSWER

%e24.1-3
\startEXERCISE
給定有向圖 \m{G=(V,E)} 帶權重且沒有權重爲負值的環路,
對於所有節點 \m{v\in V},
從源節點 \m{s} 到節點 \m{v} 之間的最短路徑中,
包含邊的條數的最大值爲 \m{m}。
(這裏,判斷最短路徑的根據是權重,不是邊的條數。)
請修改算法 \ALGO{BELLMAN-FORD},
讓其可以在 \m{m+1} 遍鬆弛操作後終止,
即使事先不知道 \m{m} 的值。
\stopEXERCISE

\startANSWER
如果一次遍歷過程中沒有任何節點需要鬆弛操作,則終止循環。
\stopANSWER

%e24.1-4
\startEXERCISE[exercise:24.1-4]
修改算法 \ALGO{BELLMAN-FORD},使其對於所有節點 \m{v} 來說,
如果從源節點 \m{s} 到節點 \m{v} 的一條路徑上存在權重爲負值的環路,
則將 \m{v.d} 的值設置爲 \m{-\infty}。
\stopEXERCISE

\startANSWER
\CLRSH{BELLMAN-FORD-B(G,w,s)}
\startCLRS
INITIALIZE-SINGLE-SOURCE(G,s)
for i = 1 to |G.V| - 1
	for each edge (u,v) in G.E
		RELAX(u,v,w)
for each edge (u,v) in G.E
	if v.d > u.d + w(u,v)
		v.d = -infty
		let t = v.pi
		while t != NIL and t.d != -infty
			t.d = -infty
			t = t.pi
for each edge (u,v) in G.E
	if v.d > u.d + w(u,v)
		return FALSE
return TRUE
\stopCLRS
\stopANSWER

%e24.1-5
\startEXERCISE\DIFFICULT
設 \m{G=(V,E)} 爲一個帶權重的有向圖,
其權重函數爲 \m{\omega: E\rightarrow R}。
請給出一個時間複雜度爲 \m{O(VE)} 的算法,
對於每個節點 \m{v\in V},
計算出數值 \m{\delta*(v)=\min_{u\in V}\{\delta(u,v)\}}。
\stopEXERCISE

\startANSWER
\CLRSH{RELAX-MOD(u,v,w)}
\startCLRS
min = w(u,v) + (u.d < 0 ? u.d : 0)
if v.d > min
	v.d = min
\stopCLRS

\ALGO{BELLMAN-FORD} 的時間複雜度不變,仍爲 \m{O(VE)}。
\stopANSWER

%e24.1-6
\startEXERCISE\DIFFICULT
設 \m{G=(V,E)} 爲一個帶權重的有向圖,
且圖中存在權重爲負值的環路。
請給出一個有效的算法來列出所有屬於該環路上的節點。
並證明算法的正確性。
\stopEXERCISE

\startANSWER
利用\refexercise{214.1-4} 的代碼。
\stopANSWER

\stopsection

\startsection[
  title={Single-source shortest paths in directed acyclic graphs},
]

%e24.2-1
\startEXERCISE
請在圖 24-5 上運行 \ALGO{DAG-SHORTEST-PATHS},使用節點 \m{r} 作爲源節點。
\stopEXERCISE

\startANSWER
\externalfigure[output/e24_2_1-6]
\stopANSWER

%e24.2-2
\startEXERCISE
假定將 \ALGO{DAG-SHORTEST-PATHS} 的第 3 行改爲:
\startCLRS
for the first |V| - 1 vertices, taken in topologically sorted order
\stopCLRS
證明:該算法的正確性保持不變。
\stopEXERCISE

\startANSWER
拓撲排序後,沒有從最後一個頂點出發的邊,因此不需要遍歷最後一個頂點。
\stopANSWER

%e24.2-3
\startEXERCISE
上面描述的 PERT 圖的公式有一點不大自然。
在一個更自然的結構下,
圖中的節點代表要執行的工作,
邊代表工作之間的次序限制,
即邊 \m{(u,v)} 表示工作 \m{u} 必須在工作 \m{v} 之前執行。
在這種結構的圖中,我們將權重賦給節點,而不是邊。
請修改 \ALGO{DAG-SHORTEST-PATHS} 過程,
使其可以在線性時間內找出這種有向無環圖一條最長的路徑。
\stopEXERCISE

\startANSWER
替換 \ALGO{DAG-SHORTEST-PATHS} 中的兩個函數:

\CLRSH{NEW-INITIALIZE-SINGLE-SOURCE(G,s,w)}
\startCLRS
for each vertex v in G.V
	v.d = infty
	v.pi = NIL
s.d = w(s)
\stopCLRS

\CLRSH{NEW-RELAX(u,v,w)}
\startCLRS
if v.d > u.d + w(v)
	v.d = u.d + w(v)
	v.pi = u
\stopCLRS
\stopANSWER

%e24.2-4
\startEXERCISE
給出一個有效算法計算有向無環圖中的路徑總數,
並分析其時間複雜度。
\stopEXERCISE

\startANSWER
\m{s.c = 1}, \m{v.c = \sum_{(u,v)\in E}u.c}。

時間複雜度爲 \m{\Theta(V+E)}。

\CLRSH{DAG-PATHS(G,s)}
\startCLRS
topologically sort the vertices of G

for each vertex v in G.V
	v.c = 0
s.c = 1

for each vertex u, taken in topologically sorted order
	for each vertex v in u.Adj
		v.c = v.c + u.c
\stopCLRS
\stopANSWER

\stopsection

\startsection[
  title={Dijkstra’s algorithm},
]

%e24.3-1
\startEXERCISE
在圖 24-2 上運行 Dijkstra 算法,
第一次使用節點 \m{s} 作爲源節點,
第二次使用節點 \m{z} 作爲源節點。
以類似於圖 24-6 的風格,給出每次 {\EMP while} 循環後的 \m{d} 和 \m{\pi},
以及集合 \m{S} 中的所有節點。
\stopEXERCISE

\startANSWER
從 \m{s} 開始:

\startcombination[3*2]
{\externalfigure[output/e24_3_1-1]}{a}
{\externalfigure[output/e24_3_1-2]}{b}
{\externalfigure[output/e24_3_1-3]}{c}
{\externalfigure[output/e24_3_1-4]}{d}
{\externalfigure[output/e24_3_1-5]}{e}
{\externalfigure[output/e24_3_1-6]}{f}
\stopcombination

從 \m{z} 開始:

\startcombination[3*2]
{\externalfigure[output/e24_3_1-7]}{a}
{\externalfigure[output/e24_3_1-8]}{b}
{\externalfigure[output/e24_3_1-9]}{c}
{\externalfigure[output/e24_3_1-10]}{d}
{\externalfigure[output/e24_3_1-11]}{e}
{\externalfigure[output/e24_3_1-12]}{f}
\stopcombination
\stopANSWER

%e24.3-2
\startEXERCISE
請舉出一個包含負權重的有向圖,
使 Dijkstra 算法在騎上運行時將產生不正確的結果。
爲什麼有負權重的情況下,定理 24.6 的證明不能成立?
\stopEXERCISE

\startANSWER
Dijkstra 算法原理:每次新拓展一個最近的點,
就更新與其相鄰的點的距離。
當所有邊權重均爲正值時,不會存在一個距離更短的沒有拓展過的點。
所以這個點的距離永遠不會被改變,因而保證了算法的正確性。
而一旦邊的權重有負值,這個假設就不成立了。
\stopANSWER

%e24.3-3
\startEXERCISE
假定將 Dijkstra 算法第 4 行改爲:
\startCLRS
while |Q| > 1
\stopCLRS
這種改變將讓 {\EMP while} 循環的執行才樹從 \m{|V|} 次降爲 \m{|V|-1} 次。
這樣修改後的算法正確嗎?
\stopEXERCISE

\startANSWER
正確。
\stopANSWER

%e24.3-4
\startEXERCISE
Gaedel 教授寫了一個還曾需,他聲稱該程序實現了 Dijkstra 算法。
對於每個節點 \m{v\in V},
該程序生成 \m{v.d} 和 \m{v.\pi}。
請給出一個時間複雜度爲 \m{O(V+E)} 的算法來檢查教授所編寫程序的輸出。
該算法應該判斷每個節點的 \m{d} 和 \m{\pi} 屬性是否與某棵最短路徑樹中的信息匹配。
這裏可以假設所有邊的權重均非負。
\stopEXERCISE

\startANSWER
沿最短路徑樹上的邊執行鬆弛操作。
\stopANSWER

%e24.3-5
\startEXERCISE
Newman 教授覺得子集發現了 Dijkstra 算法的一個更簡單的證明。
他聲稱 Dijkstra 算法對最短路徑上面的每條邊的鬆弛次序與該條邊在該條最短路徑中的次序相同,
因此,路徑鬆弛性質適用於從源節點可以到達的所有節點。
請構造一個有向圖來說明 Dijkstra 算法並不一定按照最短路徑中邊的出現次序來對邊進行鬆弛,
從而證明教授是錯的。
\stopEXERCISE

\startANSWER
\TODO{略。}
\stopANSWER

%e24.3-6
\startEXERCISE
給定有向圖 \m{G=(V,E)},
每條邊 \m{(u,v)\in E} 有一個關聯值 \m{r(u,v)},
該值是一個實數,範圍爲 \m{0\le r(u,v)\le 1},
表示從節點 \m{u} 到節點 \m{v} 之間通信鏈路的可靠性。
可以認爲, \m{r(u,v)} 代表從 \m{u} 到 \m{v} 的通信鏈路不失效的概率,
並且假設這些概率之間相互獨立。
請給出一個有效的算法找到任意兩個節點之間最可靠的通信鏈路。
\stopEXERCISE

\startANSWER
權重之和最小變成了概率之積最大。
\stopANSWER

%e24.3-7
\startEXERCISE
給定帶權重的有向圖 \m{G=(V,E)},
其權重函數爲 \m{\omega: E\rightarrow \{1,2,\ldots,W\}},
其中 \m{W} 爲某個正整數,
假設圖中從源節點 \m{s} 到任意兩個節點之間的最短路徑權重都不相同。
現在,假設定義一個沒有權重的有向圖 \m{G'=(V\cup V',E')}。
該圖是將每條邊 \m{(u,v)\in E} 予以替換,
替換所用的是 \m{\omega(u,v)} 條具有單位權重的邊。
請問圖 \m{G'} 一共有多少個節點?
現在假設在 \m{G'} 上運行廣度優先搜索算法。
證明: \m{G'} 的廣度優先搜索將 \m{V} 中節點塗上黑色的次序
與 Dijkstra 算法運行在圖 \m{G} 上時從優先隊列中抽取節點的次序相同。
\stopEXERCISE

\startANSWER
對於邊 \m{(u,v)} 會增加 \m{\omega(u,v)-1} 個節點,
因此 \m{G'} 中節點的數目爲:
\startformula
|V| + \sum_{(u,v)\in E}\omega(u,v) - |E|
\stopformula

對於遍歷順序:對於節點 \m{v},假設 Dijkstra 算法算出的值爲 \m{v.d},
則廣度優先搜索時恰好是在第 \m{v.d} 步將 \m{v} 染色。
\stopANSWER

%e24.3-8
\startEXERCISE[exercise:24.3-8]
給定帶權重的有向圖 \m{G=(V,E)},
其權重函數爲 \m{\omega: E\rightarrow \{1,2,\ldots,W\}},
其中 \m{W} 爲某個非負整數。
請修改 Dijkstra 算法來計算從給定源節點 \m{s} 到所有將誒點之間的最短路徑。
該算法時間應爲 \m{O(WV+E)}。
\stopEXERCISE

\startANSWER
用一個數組 \m{A} 實現優先級隊列,下標爲節點的 \m{d} 值,
相應元素存儲的是節點列表,這些節點的 \m{d} 值均與下標相同。

\ALGO{EXTRAC-MIN} 時,所得節點的 \m{d} 值是逐漸增大的,
因此每次從數組中找 \m{d} 值最小元素時只需從上次所得節點的 \m{d} 值開始遍歷即可,
此操作總共需要時間 \m{O(WV)}。

\ALGO{DECREASE-KEY} 總共需要時間 \m{O(E)}。
每次檢查一條邊執行 \ALGO{RELAX} 時,只需根據新的 \m{d} 值移動節點在數組 \m{A} 中的位置即可。
\stopANSWER

%e24.3-9
\startEXERCISE
修改\refexercise{24.3-8} 中的算法,
使其運行時間爲 \m{O((V+E)\lg W)}。
(\hint 在任意時刻,集合 \m{V-S} 裏有多少個不同的最短路徑估計?)
\stopEXERCISE

\startANSWER
改用二叉堆實現優先級隊列。
\stopANSWER

%e24.3-10
\startEXERCISE
假設給定帶權重的有向圖 \m{G=(V,E)},
從源節點 \m{s} 出發的邊的權重可以爲負值,
而其他所有邊的權重全部是非負值,
同時,圖中不包含權重爲負值的環路。
證明: Dijkstra 算法可以正確計算出從源節點 \m{s} 到所有其他節點之間的最短路徑。
\stopEXERCISE

\startANSWER
這種情況下不會破壞 \m{S} 中節點的 \m{d} 值。
\stopANSWER

\stopsection

\startsection[
  title={Difference constraints and shortest paths},
]

%e24.4-1
\startEXERCISE
請給出下面差分約束系統的可行解或證明該系統沒有可行解。
\startformula\startmathalignment
\NC x_1 - x_2 \le \NC 1 \NR
\NC x_1 - x_4 \le \NC -4 \NR
\NC x_2 - x_3 \le \NC 2 \NR
\NC x_2 - x_5 \le \NC 7 \NR
\NC x_2 - x_6 \le \NC 5 \NR
\NC x_3 - x_6 \le \NC 10 \NR
\NC x_4 - x_2 \le \NC 2 \NR
\NC x_5 - x_1 \le \NC -1 \NR
\NC x_5 - x_4 \le \NC 3 \NR
\NC x_6 - x_3 \le \NC -8 \NR
\stopmathalignment\stopformula
\stopEXERCISE

\startANSWER
\externalfigure[output/e24_4_1-8]
\stopANSWER

%e24.4-2
\startEXERCISE
請給出下面差分約束系統的可行解或證明該系統沒有可行解。
\startformula\startmathalignment
\NC x_1 - x_2 \le \NC 4 \NR
\NC x_1 - x_5 \le \NC 5 \NR
\NC x_2 - x_4 \le \NC -6 \NR
\NC x_3 - x_2 \le \NC 1 \NR
\NC x_4 - x_1 \le \NC 3 \NR
\NC x_4 - x_3 \le \NC 5 \NR
\NC x_4 - x_5 \le \NC 10 \NR
\NC x_5 - x_3 \le \NC -4 \NR
\NC x_5 - x_4 \le \NC -8 \NR
\stopmathalignment\stopformula
\stopEXERCISE

\startANSWER
無解,因爲形成了權重爲負值的環路。

\externalfigure[output/e24_4_2-7]
\stopANSWER

%e24.4-3
\startEXERCISE
約束圖中從新節點 \m{v_0} 到其他節點之間的最短路徑權重能夠爲正值嗎?請解釋。
\stopEXERCISE

\startANSWER
不會,因爲從 \m{v_0} 直接到達其他節點的邊權重爲 0,最短路徑的權重不能大於這個值,否則這條邊就是最短路徑了。
\stopANSWER

%e24.4-4
\startEXERCISE
請將單源目的地最短路徑問題表示爲一個線性規劃問題。
\stopEXERCISE

\startANSWER
\TODO{略。}
\stopANSWER

%e24.4-5
\startEXERCISE
請修改 \ALGO{BELLMAN-FORD} 算法,
使其能在 \m{O(nm)} 時間內解決由 \m{n} 個未知變量和 \m{m} 個約束條件所構成的差分約束系統問題。
\stopEXERCISE

\startANSWER
額外添加的 \m{v_0} 及其 \m{n} 條權值爲 0 的邊沒有意義。
我們可以在開始將所有節點 \m{v} 的 \m{d} 初始化爲 0。
\stopANSWER

%e24.4-6
\startEXERCISE
假定在除差分約束系統外,
我們希望處理形式爲 \m{x_i=x_j+b_k} 的{\EMP 相等約束}。
請說明如何修改算法 \ALGO{BELLMAN-FORD} 來解決這種約束系統。
\stopEXERCISE

\startANSWER
每一個等式轉換成兩條邊: \m{x_i-x_j\le b_k} 和 \m{x_j-x_i\le -b_k}。
\stopANSWER

%e24.4-7
\startEXERCISE
說明如何在一個沒有額外節點 \m{v_0} 的約束圖上運行類似 \ALGO{BELLMAN-FORD} 來求解差分約束系統。
\stopEXERCISE

\startANSWER
額外添加的 \m{v_0} 及其 \m{n} 條權值爲 0 的邊沒有意義。
我們可以在開始將所有節點 \m{v} 的 \m{d} 初始化爲 0。
\stopANSWER

%e24.4-8
\startEXERCISE\DIFFICULT
設 \m{Ax\le b} 爲一個有 \m{n} 個變量和 \m{m} 個約束條件的差分約束系統。
證明:在對應的約束圖上運行 \ALGO{BELLMAN-FORD} 將獲得 \m{\sum_{i=1}^{n}x_i} 的最大值,
這裏 \m{Ax\le b} 並且 \m{x_i\le 0}。
\stopEXERCISE

\startANSWER
此算法的解中最大的那個肯定是 0,根據 \m{x_i\le 0} 可知已經最大,其他 \m{x} 根據不等式依次可知均已最大。
因此總和亦爲最大。
\stopANSWER

%e24.4-9
\startEXERCISE\DIFFICULT
設 \m{Ax\le b} 爲一個有 \m{n} 個變量和 \m{m} 個約束條件的差分約束系統。
證明:在對應的約束圖上運行 \ALGO{BELLMAN-FORD} 將獲得 \m{\max\{x_i\} - \min\{x_i\}} 的最小值,
其中 \m{Ax\le b}。
如果該算法被用於安排建設工程的進度,請說明如何應用上述事實。
\stopEXERCISE

\startANSWER
根據上一題可知,如果 \m{x_i\le 0},則得到的所有 \m{x} 都是最大的,
因此 \m{\max\{x_i\} - \min\{x_i\}} 是最小的。
\stopANSWER

%e24.4-10
\startEXERCISE
假定線性規劃問題 \m{Ax\le b} 的矩陣 \m{A} 中每一行對應一個約束條件,
具體來說,對應的是一個形式爲 \m{x_i\le b_k} 的單個變量的約束條件,
或一個形式爲 \m{-x_i\le b_k} 的單變量約束條件。
請說明如何修改算法 \ALGO{BELLMAN-FORD} 來解決這個差分約束系統問題。
\stopEXERCISE

\startANSWER
將新節點 \m{v_0} 加入單變量約束條件。初始化的時候 \m{v_0.d = 0}。

\m{x_i\le b_k}: \m{x_i - x_0 \le b_k}。

\m{-x_i\le b_k}: \m{x_0 - x_i \le b_k}。
\stopANSWER

%e24.4-11
\startEXERCISE
請給出一個有效算法來解決 \m{Ax\le b} 的差分約束系統問題,
這裏 \m{b} 的所有元素爲實數,所有的變量 \m{x_i} 都是整數。
\stopEXERCISE

\startANSWER
將 \m{b} 向下取整。
\stopANSWER

%e24.4-12
\startEXERCISE\DIFFICULT
請給出一個有效算法來解決 \m{Ax\le b} 的差分約束系統問題,
這裏 \m{b} 的所有元素爲實數,所有的變量 \m{x_i} 中某個給定的子集是整數。
\stopEXERCISE

\startANSWER
\TODO{略。}
\stopANSWER

\stopsection

\startsection[
  title={Proofs of shortest-paths properties},
]

%e24.5-1
\startEXERCISE
圖 24-2 中有兩棵最短路徑樹,請給出與其不同的另外兩棵最短路徑樹。
\stopEXERCISE

\startANSWER
\startcombination[nx=2]
{\externalfigure[output/e24_5_1-1]}{a}
{\externalfigure[output/e24_5_1-2]}{b}
\stopcombination
\stopANSWER

%e24.5-2
\startEXERCISE
\m{G=(V,E)} 是一個帶權重的有向圖,
權重函數爲 \m{\omega:E\rightarrow R}。
設 \m{s\in V} 爲某個源節點。
請舉出一個例子,使得圖 \m{G} 滿足下列條件:
對於每條邊 \m{(u,v)\in E},
存在一棵根節點爲 \m{s} 的包含邊 \m{(u,v)} 的最短路徑樹,
也包含一棵根節點爲 \m{s} 的不包含邊 \m{(u,v)} 的最短路徑樹。
\stopEXERCISE

\startANSWER
\stopANSWER

\stopsection

\startsubject[
  title={Problems},
]

\stopsubject%Problems

\stopchapter
\stopcomponent
