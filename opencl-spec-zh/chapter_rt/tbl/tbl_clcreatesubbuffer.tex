

\startbuffer[enumdesc:CL_BUFFER_CREATE_TYPE_REGION]
用 \carg{buffer} 中的特定區域創建\cnglo{bufobj}。

\carg{buffer_create_info} 指向如下數據結構:

\startcintbl
struct _cl_buffer_region {
	size_t origin;
	size_t size;
} cl_buffer_region;
\stopcintbl

\math{(\marg{origin}, \marg{size})} 就是在 \carg{buffer} 中的偏移量和字節數。

如果 \carg{buffer} 是用 \cenum{CL_MEM_USE_HOST_PTR} 創建的,
所返回\cnglo{bufobj}的 \carg{host_ptr} 就是 \math{\marg{host_ptr} + \marg{origin}}。

所返回的\cnglo{bufobj}引用了為 \carg{buffer} 分配的數據存儲空間,
並指向其中的特定區域 \math{(\marg{origin}, \marg{size})}。

如果在 \carg{buffer} 中,區域 \math{(\marg{origin}, \marg{size})} 越界了,
則會在 \carg{errcode_ret} 中返回 \cenum{CL_INVALID_VALUE}。

如果 \carg{size} 是 0,則返回 \cenum{CL_INVALID_BUFFER_SIZE}。

如果與 \carg{buffer} 相關聯的\cnglo{context}中
沒有一個設備的 \cenum{CL_DEVICE_MEM_BASE_ADDR_ALIGN} 與 \carg{origin} 對齊,
則會在 \carg{errcode_ret} 中返回 \cenum{CL_MISALIGNED_SUB_BUFFER_OFFSET}。
\stopbuffer

\startED[\ctype{cl_buffer_create_type}]

\clED{CL_BUFFER_CREATE_TYPE_REGION}{%
\getbuffer[enumdesc:CL_BUFFER_CREATE_TYPE_REGION]}

\stopED

