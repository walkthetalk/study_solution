\startsection[
  title={The heapsort algorithm},
]

\startEXERCISE
參照圖 6-4 的方法,
說明 \ALGO{HEAPSOR} 在數列 \m{A = \langle 5, 13, 2, 25, 7, 17, 20, 8, 4 \rangle} 上的操作過程。
\stopEXERCISE

\startANSWER
\startcombination[4*3]
{\externalfigure[output/e6_4_1-1]}{}
{\externalfigure[output/e6_4_1-2]}{}
{\externalfigure[output/e6_4_1-3]}{}
{\externalfigure[output/e6_4_1-4]}{}
{\externalfigure[output/e6_4_1-5]}{}
{\externalfigure[output/e6_4_1-6]}{}
{\externalfigure[output/e6_4_1-7]}{}
{\externalfigure[output/e6_4_1-8]}{}
{\externalfigure[output/e6_4_1-9]}{}
\stopcombination
\stopANSWER

\startEXERCISE
試分析在使用下列循環不變式時, \ALGO{HEAPSORT} 的正確性:

在算法的第 2~5 行 {\EMP for} 循環每次迭代開始時,
子數列 \m{A[1..i]} 是一個包含了數列 \m{A[1..n]} 中第 \m{i} 小元素的最大堆,
而子數列 \m{A[i+1..n]} 則包含了數列 \m{A[1..n]} 中已排好序的 \m{n-i} 個最大元素。
\stopEXERCISE

\startANSWER
{\EMP 初始化:}子數列 \m{A[i+1..n]} 爲空,不變式成立;

{\EMP 保持:} \m{A[1]} 是 \m{A[1..i]} 中最大的,但是小於 \m{A[i+1..n]} 中所有元素。
將 \m{A[1]} 和 \m{A[i]} 調換後,則 \m{A[i..n]} 中的元素是最大的,且是排好序的。
堆的大小減一,並調用 \ALGO{MAX-HEAPIFY} 會將 \m{A[1..i-1]} 構造成最大堆。
將 \m{i} 減一,繼續下一次迭代;

{\EMP 終止:}待 \m{i=1} 時, \m{A[2..n]} 是排好序的,而 \m{A[1]} 是最小的,
因此整個數列爲排好序的。
\stopANSWER

\startEXERCISE[exercise:heapsort_time]
如果數列 \m{A} 含有 \m{n} 個元素,且已經是升序排列, \ALGO{HEAPSORT} 的時間復雜度是多少?
如果已經按降序排列呢?
\stopEXERCISE

\startANSWER
兩者均爲 \m{\Theta(n\lg{n})}。

如果已經按升序排列,將其轉換成堆需要 \m{O(n)},然後需要調用 \ALGO{MAX-HEAPIFY} 共 \m{n-1} 次,
每次調用需要 \m{\lg{k}} 次運算。因此:
\startformula
\sum_{i=1}^{n-1}\lg{k} = \lg((n-1)!) = \Theta(n\lg{n})
\stopformula

如果已經按降序排列,則 \ALGO{BUILD-MAX-HEAP} 會快一些(常量因子),
但時間主要花在 \ALGO{HEAPSORT} 中的循環上,時間爲 \m{\Theta(n\lg{n})}。
\stopANSWER

\startEXERCISE
證明:最壞情況下, \ALGO{HEAPSORT} 的時間復雜度是 \m{\Omega(n\lg{n})}。
\stopEXERCISE

\startANSWER
與\refexercise{heapsort_time}的第一部分相同。
如果數列已經排好序,我們需要線性時間將其轉換成最大堆,然後需要 \m{n\lg{n}} 的時間排序。
\stopANSWER

\startEXERCISE\DIFFICULT
證明: 在所有元素都不同的情況下, \ALGO{HEAPSORT} 的時間復雜度爲 \m{\Omega(n\lg{n})}。
\stopEXERCISE

\startANSWER
假設堆是完全二叉樹,元素個數爲 \m{n=2^k - 1}。有 \m{2^{k-1}} 個葉子節點以及 \m{2^{k-1} - 1} 個內部節點。

先來看爲堆中前 \m{2^{k-1}} 個元素排序。
給葉子節點着紅色,內部節點着藍色。
已着色的節點是堆的子樹。
由於有 \m{2^{k-1}} 個節點已着色,紅色的最多有 \m{2^{k-2}} 個,藍色的最少 \m{2^{k-2}-1} 個。

紅色節點可以直接跳到根節點位置上,而藍色節點則需要先上移。
讓我們統計以下要將藍色節點移到根節點位置上,需要交換多少次。
交換次數最少的情況必須滿足兩點,
有 \m{2^{k-2} -1} 個藍色節點,
且他們在樹中的相對位置已經滿足最大堆的要求。
如果有 \m{d} 個這樣的藍色節點,則有 \m{i=\lg{d}} 層,
第 \m{i} 層含有 \m{2^i} 個節點。
因此交換次數爲:
\startformula
\sum_{i=0}^{\lg{d}}i2^i = 2 + (\lg{d} - 2)2^{\lg{d}} = \Omega(d\lg{d})
\stopformula

遞迴式:
\startformula
T(n) = T(n/2) + \Omega(n\lg{n})
\stopformula

由主定理可得 \m{T(n)=\Omega(n\lg{n})}。
\stopANSWER

\stopsection
