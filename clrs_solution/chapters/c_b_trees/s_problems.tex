\startsubject[
  title={Problems},
]

%p18-1
\startPROBLEM
(stacks on secondary storage)
對於一臺計算機,如果較快的主存較少,而較慢的磁盤存儲空間較多,
如何在其上實現一個棧。
 \ALGO{PUSH} 和 \ALGO{POP} 的操作對象爲單字。
我們希望計算機支持的棧可以增長的很大,
甚至無法全部裝入主存,即大部分放在磁盤上。

一種簡單但低效的實現方法是將整個棧放在磁盤上。
在主存中保持一個棧的指針,指向棧頂元素的磁盤地址。
如果該指針的值爲 \m{p},
則棧頂元素是磁盤的 \m{\lfloor \frac{p}{m}\rfloor} 頁上的第 \m{p \mod m} 個字,
這裏 \m{m} 爲每頁所含字數。

爲了實現 \ALGO{PUSH} 操作,我們增加棧指針,
從磁盤將適當的頁讀到主存中後,
複製要被壓入棧的元素到該頁上適當字的位置,
最後將該頁寫回到磁盤。
 \ALGO{POP} 操作與之類似。
我們減小棧指針,從磁盤上讀入所需的頁,再返回棧頂元素。
我們不需要寫回該頁,因爲他沒有被修改。

由於磁盤操作代價相對較高,
我們統計任何實現的兩部分代價:
總的磁盤存取次數和總的 CPU 時間。
任何對一個包含 \m{m} 個字的頁面的磁盤存取,
都會引起一次磁盤存取 \m{\Theta(m)} 的 CPU 時間。

\startigBase[a]\startitem
從漸進意義上看,使用這種簡單實現,最壞情況下, \m{n} 個棧操作需要存取多少次磁盤?
 CPU 時間又是多少?
(用 \m{m} 和 \m{n} 來表示這個問題及後面幾個問題的答案)
\stopitem\stopigBase

\startANSWER
\m{2n} 次, \m{\Theta(mn)}。
\stopANSWER

現在考慮棧的另一種實現,
即在主存中始終保持存放棧中的一頁。
(還用少量的主存來記錄當前哪一頁在主存中)
只有相關的磁盤頁面駐留在主存中,才能執行棧操作。
如果需要,可以將當前主存中的頁寫回磁盤,
並且可以從磁盤讀入新的一頁放入主存。
如果相關磁盤頁已經在主存中,那麼就無需任何磁盤存取。

\startigBase[continue]\startitem
最壞情況下, \m{n} 個 \ALGO{PUSH} 操作所需的磁盤存取次數是多少?
所需的 CPU 時間是多少?
\stopitem\stopigBase

\startANSWER
\m{n/m} 次, \m{\Theta(n)}。
\stopANSWER

\startigBase[continue]\startitem
最壞情況下, \m{n} 個棧操作所需的磁盤存取次數是多少?
所需的 CPU 時間是多少?
\stopitem\stopigBase

\startANSWER
\m{n/2} 次, \m{\Theta(mn)}。
\stopANSWER

假設在另一種實現方式中,在主存中保持棧的 2 頁(此外還有少量的字來記錄哪些頁在主存中)。
\startigBase[continue]\startitem
請描述如何管理棧頁,使得任何棧操作的攤還磁盤存取次數爲 \m{O(1/m)},
攤還 CPU 時間爲 \m{O(1)}。
\stopitem\stopigBase

\startANSWER
\TODO{略。}
\stopANSWER

\stopPROBLEM

%p18-2
\startPROBLEM
(joining and splitting 2-3-4 tree)
{\EMP 連接}:輸入爲兩個動態集合 \m{S'} 和 \m{S''},以及一個元素 \m{x},
且對任何 \m{x'\in S'} 和 \m{x''\in S''},
都有 \m{x'.key < x.key < x''.key};
返回一個集合 \m{S=S'\cup \{x\} \cup S''}。

{\EMP 分裂}:{\EMP 連接}的逆操作。
給定一個動態集合 \m{S} 和一個元素 \m{x\in S},
他創建了一個集合 \m{S'},
其包含 \m{S-\{x\}} 中所有關鍵字小於 \m{x.key} 的元素;
同時創建了一個集合 \m{S''},
其包含 \m{S - \{x\}} 中所有關鍵字大於 \m{x.key} 的元素。
在這個問題中,我們討論如何在 2-3-4 樹上實現這些操作。
爲方便起見,假定所有元素都只包含關鍵字,
並且所有的關鍵字都不相同。

\startigBase[a]\startitem
對 2-3-4 樹中的每個節點 \m{x},
說明如何將以 \m{x} 爲根的子樹的高度作爲一個屬性 \m{x.height} 來維護。
要確保所給出的實現不影響查找、插入和刪除的漸進運行時間。
\stopitem\stopigBase

\startANSWER
插入:分裂時給新根節點的 \m{height} 賦值。
\stopANSWER

\startigBase[continue]\startitem
說明如何實現連接操作。
給定兩棵 2-3-4 樹 \m{T'}、 \m{T''} 以及一個關鍵字 \m{k},
連接操作應在 \m{O(1+|h'-h''|)} 運行時間內完成,
其中 \m{h'} 和 \m{h''} 分別是樹 \m{T'} 和 \m{T''} 的高度。
\stopitem\stopigBase

\startANSWER
以較高的樹,從根節點開始沿最左側子樹搜索,
直到 height 屬性比較低樹的根節點 height 大 1,記爲節點 \m{x};
對於路徑上任何一個節點,只要是滿的,就將其分裂。
然後將較低的樹與新關鍵字 \m{k} 加入 \m{x}。
\stopANSWER

\startigBase[continue]\startitem
考慮從一棵 2-3-4 樹 \m{T} 的根到一個給定關鍵字 \m{k} 的簡單路徑 \m{p},
 \m{T} 中小於 \m{k} 的關鍵字集合 \m{S'},
以及 \m{T} 中大於 \m{k} 的關鍵字集合 \m{S''}。
證明: \m{p} 將 \m{S'} 分爲一個樹的集合 \m{\{T_0',T_1',\ldots,T_m'\}} 和
一個關鍵字的集合 \m{\{k_1',k_2',\ldots,k_m'\}},
且對任何關鍵字 \m{y\in T_{i-1}'} 和 \m{z\in T_i'} (\m{i=1,2,\ldots,m}),
都有 \m{y<k_i'<z}。
 \m{T_{i-1}'} 和 \m{T_i'} 的高度之間有什麼關係?
請說明 \m{p} 是如何將 \m{S''} 分爲樹集合和關鍵字集合的。
\stopitem\stopigBase

\startANSWER
從根節點開始沿路徑 \m{p} 搜索 \m{k},
對於路徑上每一個節點 \m{x},假設 \m{x.c_i} 在路徑 \m{p} 上:
如果 \m{i = 1},則分出的子樹爲空,響應關鍵字也爲空;
否則,則分出以 \m{x.c_{i-1}} 爲根的子樹,以及關鍵字 \m{x.key_{i-1}}。
所得即爲樹的集合 \m{\{T_0',T_1',\ldots,T_m'\}} 和
關鍵字的集合 \m{\{k_1',k_2',\ldots,k_m'\}}。

對稱的可得樹的集合 \m{\{T_0'',T_1'',\ldots,T_m''\}} 和
關鍵字的集合 \m{\{k_1'',k_2'',\ldots,k_m''\}}。

\m{T_{i-1}'} 的高度大於等於 \m{T_i'} 的高度。
\stopANSWER

\startigBase[continue]\startitem
請說明如何實現 \m{T} 上的分裂操作。
利用連接操作將 \m{S'} 中的關鍵字拼成一棵簡單的 2-3-4 樹 \m{T'},
將 \m{S''} 中的關鍵字拼成一棵簡單的 2-3-4 樹 \m{T''}。
分裂操作的運行時間要求爲 \m{O(\lg n)},
這裏 \m{n} 是 \m{T} 中關鍵字數目。
(\hint 連接的代價應是套迭的)
\stopitem\stopigBase

\startANSWER
沿到某一個關鍵字的路徑進行分裂。參考上一項的答案。
然後將分裂出的集合進行合併。
所用時間爲:
\startformula
\sum_{i=1}^{m}(1+|h(T_{i-1}') - h(T_i')|)
= m + \sum_{i=1}^{m}(h(T_{i-1}') - h(T_i'))
= m + h(T_0') - h(T_m')
\in O(\lg n)
\stopformula
\stopANSWER

\stopPROBLEM

\stopsubject%Problems
