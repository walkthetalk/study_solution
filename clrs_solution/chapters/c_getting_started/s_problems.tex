\startsubject[
  title={Problems},
]
%p2-1
\startPROBLEM
在歸並排序中對小數列使用插入排序。
\startigBase[a]

\item 證明對 \m{n/k} 個長度爲 \m{k} 的子列進行插入排序的最多需要時間 \m{\Theta(nk)}。

\startANSWER
設對長度爲 \m{k} 的數列進行插入排序所用時間爲 \m{a k^2 + b k + c},則總時間爲:
\startformula
\frac{n}{k}\m{a k^2 + b k + c} = a n k + b n + \frac{c n}{k} = \Theta(nk)
\stopformula
\stopANSWER

\item 證明歸並這些子列最多所需時爲爲 \m{\Theta(n \lg(n/k))}。

\startANSWER
對 \m{a} 個長度爲 \m{k} 的子列進行歸並排序所需時間爲:
\startformula
T(a) = \startcases
\NC 0	\NC \text{若 \m{a = 1},} \NR
\NC 2 T(a/2) + a k \NC \text{若 \m{a = 2^p}, \m{p > 0}。} \NR
\stopcases
\stopformula

設 \m{G(a) = \frac{T(a)}{a}},則:
\startformula\startalign
\NC G(a)	\NC = \frac{T(a)}{a} \NR
\NC 		\NC = \frac{2 T(a/2) + a k}{a} \NR
\NC		\NC = \frac{T(a/2)}{a/2} + k \NR
\NC		\NC = G(a/2) + k \NR
\stopalign\stopformula
即
\startformula
G(a) = \startcases
\NC 0	\NC 若 \m{a = 1},\NR
\NC G(a/2) + k \NC 若 \m{a = 2^p}, \m{p > 0}。\NR
\stopcases
\stopformula

設 \m{H(p) = G(2^p)},則:
\startformula\startalign
\NC H(p)	\NC = G(2^p) \NR
\NC		\NC = G(2^p/2) + k \NR
\NC		\NC = G(2^{p-1}) + k \NR
\NC		\NC = H(p-1) + k \NR
\stopalign\stopformula
即
\startformula
H(p) = \startcases
\NC 0	\NC 若 \m{p = 0}, \NR
\NC H(p-1) + k \NC 若 \m{p > 0}。 \NR
\stopcases
\stopformula

\startformula\startalign
\NC H(p) \NC = k p \NR
\NC G(2^p) \NC = kp \NR
\NC G(a) \NC = k \lg(a) \NR
\NC \frac{T(a)}{a} \NC = k \lg(a) \NR
\NC T(a) \NC = k a \lg(a) \NR
\NC T(\frac{n}{k}) \NC = k \frac{n}{k} \lg(\frac{n}{k}) \NR
\NC		\NC = n \lg(\frac{n}{k}) \NR
\stopalign\stopformula
\stopANSWER

\item 假定修改後的算法運行時間爲 \m{\Theta(nk+n \lg(n/k))},
\m{k} 最大爲多少時(作爲 \m{n} 的函式),新算法與原歸並排序算法所用時間相同?

\startANSWER
\startformula\startalign
\NC k \NC = \lg(n) \NR
\NC \Theta(nk + n \lg(n/k)) \NC = \Theta(n \lg(n) + n \lg(\frac{n}{\lg(n)}) \NR
\NC			\NC = \Theta(n \lg(n)) \NR
\stopalign\stopformula
\stopANSWER

\stopigBase
\stopPROBLEM

%p2-2
\startPROBLEM
冒泡排序的正確性

冒泡排序通俗易懂,但效率不高。
他通過重複交換兩個(次序顛倒的)相鄰元素的方式進行排序。
例程 \ALGO{BUBLLE-SORT} 對數列 $A[1:n]$ 進行排序。

\CLRSH{BUBBLE-SORT(A, n)}
\startCLRSCODE
for i = 1 to n - 1
	for j = n downto i + 1
		if A[j] < A[j - 1]
			exchange A[j] with A[j - 1]
\stopCLRSCODE

\startigBase[a]
\item 在數列 $A$ 上執行 \ALGO{BUBBLE-SORT(A,n)} 後變爲數列 $A'$,
要證明算法的正確性,
除了要證明
\startformula
A'[1]\le A'[2]\le \cdots \le A'[n]
\stopformula
之外,還需要證明什麼?
\stopigBase

\startANSWER
還需要證明排序完成後,新數列包含舊數列中的所有元素。
\stopANSWER

\startigBase[continue]
\item 分析並證明內層 \CLRSCODE{for} 循環的循環不變性。
要求使用本章所展示的證明方式。
\stopigBase

\startANSWER
\TODO{略。}
\stopANSWER

\startigBase[continue]
\item 利用上一問所證明的循環不變性的終止條件,
分析外層循環的循環不變性,並以此證明不等式(2.5)。
要求使用本章所展示的證明方式。
\stopigBase

\startANSWER
\TODO{略。}
\stopANSWER

\startigBase[continue]
\item 冒泡排序在最差情況下運行時間怎樣?
與插入排序相比如何?
\stopigBase

\startANSWER
冒泡排序的比較次數最多爲:$\sum_{i=1}^{n-1}{n-i} = \frac{n(n - 1)}{2}$,
相應的交換次數也一樣,所以最壞情況下所用時間爲 $\Theta(n^2)$,
與插入排序所用時間相同。

通常,兩種算法最好情況下時間均爲 $\Theta(n)$,
但是此處的實現卻爲 $\Theta(n^2)$。
要想在最好情況下達到 $\Theta(n)$,
在外部循環中,如果沒有發生任何交換就直接返回。

另外,冒泡排序會比插入排序慢很多,
因爲交換所引入的賦值操作太多了。
\stopANSWER

\stopPROBLEM

%p2-3
\startPROBLEM
給定如下多項式的係數 $a_0,a_1,a_2,\cdots,a_n$:
\startformula\startalign
\NC P(x) \NC = \sum_{k=0}^{n} a_kx^k \NR
\NC \NC = a_0 + a_1 x + a_2 x^2 + \cdots + a_{n-1}x^{n-1} + a_n x^n \NR
\stopalign\stopformula
Horner 定律按如下方式計算多項式的值:
\startformula
P(x) = a_0 + x(a_1 + x(a_2 + \cdots + x(a_{n-1} + xa_n) \cdots))
\stopformula
例程 \ALGO{HORNER} 是具體實現:

\CLRSH{HORNER(A, n, x)}
\startCLRSCODE
p = 0
for i = n downto 0
	p = A[i] + x \cdot p
return p
\stopCLRSCODE

\startigBase[a]
\item 用 $\Theta$ 表示此例程的運行時間。

\startANSWER
$\Theta(n)$
\stopANSWER

\item 按此多項式的原始定義來計算,需要多少時間?

\startANSWER
\startCLRSCODE
p = 0
for i = 0 to n
	m = 1
	for k = 1 to i
		m = m * x
	p = p + A[i] * m
\stopCLRSCODE
所用時間爲 $\Theta(n^2)$。
\stopANSWER

\item 分析 \ALGO{HORNER} 的循環不變性。
每次迭代前:
\startformula
y = \sum_{k=0}^{n-(i+1)} A_{k+i+1}\dot x^k
\stopformula
將空多項式求和結果定義爲 0。
以本章循環不變性的證明方式,
證明最終結果爲 $y = \sum_{k=0}^n a_kx^k$。

\startANSWER
每次迭代完成後:
\startformula\startalign
\NC p \NC = A_i + x \sum_{k=0}^{n-(i+1)}A_{k+i+1}x^k \NR
\NC	\NC = A_i x^0 + \sum_{k=0}^{n-i-1}A_{k+i+1}x^{k+1} \NR
\NC	\NC = A_i x^0 + \sum_{k=1}^{n-i}A_{k+i}x^{k} \NR
\NC	\NC = \sum_{k=0}^{n-i}A_{k+i}x^{k} \NR
\stopalign\stopformula
最後一次 $i$ 爲 $0$,
代入上式可得 $p = \sum_{k=0}^{n}a_{k}x^{k}$。
\stopANSWER

\stopigBase

\stopPROBLEM

%p2-4
\startPROBLEM[problem:inversion]
逆序對
\startformula
f(i,j) = \startcases
\NC true	\NC \text{若 \m{i<j}, \m{A[i] > A[j]}} \NR
\NC false	\NC \text{否則。} \NR
\stopcases
\stopformula
\startigBase[a]
\item 列出數列 \m{<2,3,8,6,1>} 中的五個逆序對。

\startANSWER
\m{(2,1),(3,1),(8,6),(8,1),(6,1)}
\stopANSWER

\item 如果數列中元素全部取自集合 \m{\{1,2,\cdots,n\}},則最多有多少逆序對。

\startANSWER
\m{<n, n-1, \cdots, 1>},其逆序對數目爲 \m{(n-1)+(n-2)+\cdots+1=\frac{n(n-1)}{2}}。
\stopANSWER

\item 插入排序的運行時間與數列中逆序對的數目有什麼關系?

\startANSWER
\m{\Theta(n+d)},其中 \m{d} 爲逆序對的數目, \m{n} 爲外層循環所用時間。
\stopANSWER

\item 給出所用時間爲 \m{\Theta(n\lg(n))} 的算法,以確定數列中逆序對的數目。

\startANSWER
\CLRSH{MERGE-SORT(A, p, r)}
\startCLRS
if p < r
	inversions = 0
	q = (p + r) / 2
	inversions += merge_sort(A, p, q)
	inversions += merge_sort(A, q + 1, r)
	inversions += merge(A, p, q, r)
	return inversions
else
	return 0

MERGE(A, p, q, r)

n1 = q - p + 1
n2 = r - q
// let L[1..n₁] and R[1..n₂] be new arrays
for i = 1 to n1
	L[i] = A[p + i - 1]
for j = 1 to n2
	R[j] = A[q + j]
i = 1
j = 1
for k = p to r
	if i > n1
		A[k] = R[j]
		j = j + 1
	else if j > n2
		A[k] = L[i]
		i = i + 1
	else if L[i] <= R[j]
		A[k] = L[i]
		i = i + 1
	else
		A[k] = R[j]
		j = j + 1
		inversions += n1 - i
return inversions
\stopCLRS
\stopANSWER

\stopigBase
\stopPROBLEM

\stopsubject
