\startsection[
  title={The substitution method for solving recurrences},
]

%e4.3-1
\startEXERCISE
用替代法證明下列漸進解:
\startigBase[a]
% a
\startitem
$T(n)=T(n-1)+n$ 的解爲 $T(n)=O(n^2)$。

\startANSWER
$T(n) \le c n^2 + d n$,其中 $d\ge c \ge 1/2$:
\startsplitformula\startmathalignment
\NC T(n-1)\NC \le c (n-1)^2 + d (n-1) \NR
\NC T(n) \NC = T(n-1) + n \NR
\NC \NC \le c (n-1)^2 + d(n-1) + n \NR
\NC \NC = c n^2 + (d-2c + 1)n + (c-d) \NR
\NC \NC \le c n^2 + dn \NR
\stopmathalignment\stopsplitformula
\stopANSWER
\stopitem

% b
\startitem
$T(n)=T(n/2)+\Theta(1)$ 的解爲 $T(n)=O(\lg n)$。

\startANSWER
$T(n)\le c \lg n$,其中 $c>\Theta(1)$:
\startsplitformula\startmathalignment
\NC T(n/2)\NC \le c \lg(n/2) \NR
\NC T(n) \NC = T(n/2) + \Theta(1) \NR
\NC \NC \le c \lg(n/2) + \Theta(1) \NR
\NC \NC = c \lg n - (c - \Theta(1)) \NR
\NC \NC \le c \lg n \NR
\stopmathalignment\stopsplitformula
\stopANSWER
\stopitem

% c
\startitem
$T(n)=2T(n/2)+n$ 的解爲 $T(n)=\Theta(n\lg n)$。

\startANSWER
$c_1 n \lg n \le T(n) \le c_2 n\lg n$,其中 $c_1<1, c_2>1$:
\startsplitformula\startmathalignment[n=3,align={right,middle,left}]
\NC c_1 (n/2)\lg(n/2) \le \NC T(n/2) \NC \le c_2 (n/2)\lg(n/2) \NR
\NC 2c_1 (n/2)\lg(n/2) + n \le \NC 2T(n/2) + n \NC \le 2c_2 (n/2)\lg(n/2) + n \NR
\NC c_1n\lg n \le c_1 n \lg n - (c_1 - 1)n \le
 \NC T(n)
 \NC \le c_2 n \lg n - (c_2-1)n \le c_2n\lg n\NR
\stopmathalignment\stopsplitformula
\stopANSWER
\stopitem

% d
\startitem
$T(n)=2T(n/2+17)+n$ 的解爲 $T(n)=O(n\lg n)$。

\startANSWER
$T(n)\le cn \lg n + dn$,其中 $d < c$:
\startsplitformula\startmathalignment
\NC T(n/2+17)\NC \le c (n/2+17)\lg(n/2+17) \NR
\NC T(n) \NC = 2T(n/2+17) + n \NR
\NC \NC \le 2c(n/2+17) \lg(n/2+17) + n \NR
\NC \NC = cn\lg(n/2+17) + 34c\lg(n/2+17) + n \NR
\NC \NC = cn\lg(n\frac{n+34}{2n}) + 34c\lg\frac{n+34}{2} + n \NR
\NC \NC = cn\lg n + cn\lg\frac{n+34}{2n} + 34c\lg\frac{n+34}{2} + n \NR
\stopmathalignment\stopsplitformula
現在我們需要滿足: $cn\lg\frac{n+34}{2n} + 34c\lg\frac{n+34}{2} + n\le 0$:
\startsplitformula\startmathalignment
\NC \NC cn\lg\frac{n+34}{2n} + 34c\lg\frac{n+34}{2} + n \NR
\NC < \NC cn\lg\frac{n+34}{2n} + 34c\lg(n+34) + n \NR
\NC = \NC c(-n + \lg\frac{n+34}{n} + 34\lg(n+34)) + n\NR
\NC = \NC c(\frac{1-c}{c}n + \lg\frac{n+34}{n} + 34\lg(n+34))\NR
\NC = \NC c(\frac{1-c}{c}n + 35\lg(n+34) - \lg n)\NR
\NC \le \NC c(-c_1 n + c_2 \lg n) \qquad
 \text{令} c_1 = \frac{c-1}{c}, 35\lg(n+34)-\lg n\le c_2 \lg n\NR
\stopmathalignment\stopsplitformula
即要滿足 $c_2\lg n -c_1 n \le 0$。

由於 $35\lg(n+34)-\lg n = O(\lg n)$,
所以 $35\lg(n+34)-\lg n\le c_2 \lg n$ 中的 $c_2>0$ 必定有解。

又由於 $c_2\lg n = O(n)$,因此 $-c_1n + c_2\lg n \le 0$ 中的 $c_1>0$ 必定有解。

因此 $(c-1)/c = c_1$ 中的 $c > 1$ 必定有解。
\stopANSWER
\stopitem

% e
\startitem
$T(n)=2T(n/3)+\Theta(n)$ 的解爲 $T(n)=\Theta(n)$。

\startANSWER
令 $c_3n\le\Theta(N)\le c_4n$, $c_1n\le T(n)\le c_2n$:
\startsplitformula\startmathalignment[n=3,align={right,middle,left}]
\NC c_1 (n/3) \le \NC T(n/3) \NC \le c_2 (n/3) \NR
\NC 2 c_1 (n/3) + \Theta(n) \le
  \NC 2T(n/3) + \Theta(n)
  \NC \le 2c_2(n/3) + \Theta(n) \NR
\NC \frac{2}{3}c_1n + c_3 n \le \NC T(n)
  \NC \le \frac{2}{3}c_2 n + c_4 n \NR
\NC (\frac{2}{3}c_1 + c_3)n \le \NC T(n)
  \NC \le (\frac{2}{3}c_2 + c_4)n \NR
\stopmathalignment\stopsplitformula
需要滿足 $c_1 \le \frac{2}{3}c_1 + c_3$ 且 $\frac{2}{3}c_2 + c_4 \le c_2$,
即 $c_1 \le 3c_3$, $c_2\ge 3c_4$。
\stopANSWER
\stopitem

% f
\startitem
$T(n)=4T(n/2)+\Theta(n)$ 的解爲 $T(n)=\Theta(n^2)$。

\startANSWER
令 $c_5n\le\Theta(N)\le c_6n$, $c_1n^2 + c_2n\le T(n)\le c_3n^2 + c_4n$:
\startsplitformula\startmathalignment[n=3,align={right,middle,left}]
\NC c_1 (n/2)^2 + c_2(n/2) \le \NC T(n/2) \NC \le c_3 (n/2)^2 + c_4(n/2) \NR
\NC 4 c_1 (n/2)^2 + 4c_2(n/2) + c_5 n \le
  \NC 4T(n/2) + \Theta(n)
  \NC \le 4c_3(n/2)^2 + 4c_4(n/2) + c_6n \NR
\NC c_1n^2 + (2c_2+c_5) n \le \NC T(n)
  \NC \le c_3 n^2 + (2c_4+c_6) n \NR
\stopmathalignment\stopsplitformula
需要滿足 $c_2\le 2c_2+c_5$ 及 $2c_4+c6\le c_4$,
即 $c_2\ge 0$ 且 $c_4\le -c_6$。
\stopANSWER
\stopitem

\stopigBase
\stopEXERCISE

%e4.3-2
\startEXERCISE
遞迴式 $T(n)=4T(n/2)+n$ 的解爲 $T(n)=\Theta(n^2)$。
請分析說明:
如果只是假設 $T(n)\le cn^2$,用替代法無法證明此結論。
但如果在此基礎上再減去一個低階項,替代法就能生效了。
\stopEXERCISE
\startANSWER
令 $c_1 n^2 \le T(n) \le c_2 n^2$,其中 $c_1,c_2>0$,則:
\startsplitformula\startmathalignment[n=3,align={right,middle,left}]
\NC c_1 (n/2)^2 \le \NC T(n/2) \NC \le c_2 (n/2)^2 \NR
\NC 4 c_1 (n/2)^2 + n \le \NC 4T(n/2) + n \NC \le 4 c_2 (n/2)^2 + n \NR
\NC c_1 n^2 + n \le \NC T(n) \NC \le c_2 n^2 + n \NR
\stopmathalignment\stopsplitformula
顯然不一定成立,而減去一個低階項后:
\startsplitformula\startmathalignment[n=3,align={right,middle,left}]
\NC c_1 n^2 - n \le \NC T(n) \NC \le c_2 n^2 - n) \NR
\NC c_1 (n/2)^2 - n/2 \le \NC T(n/2) \NC \le c_2 (n/2)^2 - n/2 \NR
\NC 4 c_1 (n/2)^2 -2n + n \le \NC 4T(n/2) + n \NC \le 4 c_2 (n/2)^2 -2n + n \NR
\NC c_1 n^2 - n \le \NC T(n) \NC \le c_2 n^2 - n \NR
\stopmathalignment\stopsplitformula
命題得證。
\stopANSWER

%e4.3-3
\startEXERCISE
遞迴式 $T(n)=2T(n-1)+1$ 的解爲 $T(n)=O(2^n)$。
請分析說明:
如果只是假設 $T(n)=c 2^n$ (其中 $c\ge 0$),用替代法無法證明此結論。
但如果在此基礎上再減去一個低階項,替代法就能生效了。
\stopEXERCISE
\startANSWER
令 $T(n)\le c 2^n$,其中 $c>0$,則:
\startsplitformula\startmathalignment
\NC T(n) \NC \le c 2^n \NR
\NC T(n-1) \NC \le c 2^{n-1} \NR
\NC 2T(n-1)+1 \NC \le 2 c2^{n-1} + 1 \NR
\NC T(n) \NC \le c 2^n + 1 \NR
\stopmathalignment\stopsplitformula
與猜測不一致,而如果減去一個低階項($d\ge 0$)后:
\startsplitformula\startmathalignment
\NC T(n) \NC \le c 2^n - n \NR
\NC T(n-1) \NC \le c 2^{n-1} - (n-1) \NR
\NC 2T(n-1)+1 \NC \le c 2^n - 2n + 2 + 1 \NR
\NC T(n) \NC \le c 2^n - n - (n-3) \le c 2^n - n \NR
\stopmathalignment\stopsplitformula
只要 $n-3\ge 0$,即 $n\ge 3$ 即可。
\stopANSWER

\stopsection
