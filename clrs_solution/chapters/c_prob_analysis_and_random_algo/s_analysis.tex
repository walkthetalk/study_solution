\startsection[
  title={Probabilistic analysis and further uses of indicator random variables},
]

%e5.4-1
\startEXERCISE
屋子裏至少要有多少人,才能保證有人與你生日相同的概率不小於 \m{1/2}?
至少要有多少人,才能保證至少兩人生日爲 7 月 4 日的概率大於 \m{1/2}?
\stopEXERCISE

\startANSWER
令縂人數為 $n$,
任一人生日與己不同的概率爲 \m{364/365},記爲 $p$,
則 \m{k} 個人生日都與己不同的概率爲 \m{p^k}。
利用互補事件求 \m{n}:
\startformula\startmathalignment
\NC 1 - p^n \NC \ge \frac{1}{2} \NR
\NC p^n     \NC \le \frac{1}{2} \NR
\NC n \lg p  \NC \ge \lg\frac{1}{2} \NR
\NC n = \lceil \log_p \frac{1}{2}\rceil \NC = 254 \NR
\stopmathalignment\stopformula

另外一個問題,令生日為 7 月 4 日的人數為 $k$:
\startformula\startmathalignment
\NC \Pr\{k\ge 2\} \NC=
        1 - \Pr\{k=1\} - Pr\{k=0\} \NR
\NC \NC= 1 - \binom{n}{k}p^{n-k}|_{k=1} - p^{n-k}|_{k=0} \NR
\NC \NC= 1 - (n-p)p^{n-1} \NR
\stopmathalignment\stopformula
計算可得 \m{k} 爲 115。
\stopANSWER

%e5.4-2
\startEXERCISE
屋子裏至少需要多少人,才能保證有兩人生日相同的概率至少為 0.99?
在此人數下,記共有 $k$ 對人生日相同, $k$ 的期望值是多少?
\stopEXERCISE

\startANSWER
令人數為 $m$,則兩人生日相同的概率為:
\startformula\startmathalignment
\NC 1 - \binom{n}{m} \NC \ge 0.99 \NR
\NC \frac{m}{n} \NC \le 0.01 \NR
\NC m \NC \le 
\stopmathalignment\stopformula
\stopANSWER

\startEXERCISE
假設我們將球投到 \m{b} 個箱子裏,直到某個箱子中有兩個球。
每一次投擲都是獨立的,並且每個球落入任一箱子的機會均等。
請問投擲次數的期望值是多少?
\stopEXERCISE

\startANSWER
本質還是生日問題,更多討論參見 \simpleurl{http://en.wikipedia.org/wiki/Birthday_problem#Average_number_of_people}。
\stopANSWER

%e5.4-3
\startEXERCISE \DIFFICULT
生日悖論的分析中,要求各人生日彼此獨立是否很重要?
或者,是否只要兩兩成對獨立就足夠了?
證明你的答案。
\stopEXERCISE
\startANSWER
成對獨立足夠了。對於(5.6)之後的推導,有此即可。
\stopANSWER

%e5.4-4
\startEXERCISE \DIFFICULT
一次聚會需要邀請多少人,才能讓其中 3 人的生日很可能相同?
\stopEXERCISE

\startANSWER
令生日有 \m{m=365} 種可能,人數爲 \m{n},只有 \m{i} 對人生日相同的概率:
\startformula\startmathalignment
\NC \NC
\frac{\left(\binom{n}{2}\frac{m}{m} \frac{1}{m}\right)
      \left(\binom{n-2}{2}\frac{m-1}{m} \frac{1}{m}\right)
      \left(\binom{n-4}{2}\frac{m-2}{m} \frac{1}{m}\right)
      \cdots
      \left(\binom{n-2(i-1)}{2}\frac{m-(i-1)}{m} \frac{1}{m}\right)}
     {i!} \NR
\NC \NC \qquad \frac{m-i}{m} \frac{m-(i+1)}{m} \cdots \frac{m-(n-i-1)}{m} \NR
\NC = \NC \frac{1}{i!}
          \frac{n!}{2^i (n-2i)!}
          \frac{m!}{(m-n+i)!}
          \frac{1}{m^n} \NR
\NC = \NC \frac{m!n!}{i!(n-2i)!(m-n+i)! 2^i m^n} \NR
\stopmathalignment\stopformula
至少有三人生日相同的概率爲:
\startformula\startmathalignment
\NC \NC 1 - \sum_{i=0}^{\lfloor n/2\rfloor}\frac{m!n!}{i!(n-2i)!(m-n+i)! 2^i m^n} \NR
\NC = \NC 1 - \frac{m!n!}{m^n}\sum_{i=0}^{\lfloor n/2\rfloor}
              \frac{1}{i!(n-2i)!(m-n+i)! 2^i} \NR
\stopmathalignment\stopformula
請參考:\simpleurl{http://math.stackexchange.com/questions/25876/probability-of-3-people-in-a-room-of-30-having-the-same-birthday}
\stopANSWER

\startEXERCISE \DIFFICULT
一個長度爲 \m{k} 的字串,其中所有字符均選自一個元素個數爲 \m{n} 的集合,
那麼此字串構成一個 \m{k} 排列的概率是多少?
此問題與生日悖論有何關聯?
\stopEXERCISE
\startANSWER
\startformula
\Pr\{k\text{-perm in }n\} = 1 \cdot
                                 \frac{n-1}{n} \cdot
                                 \frac{n-2}{n} \cdots
                                 \frac{n-k+1}{n}
      = \frac{(n-1)!}{(n-k)!n^k}
\stopformula
這是生日問題的互補事件,即 \m{k} 個人生日各不相同。
\stopANSWER

\startEXERCISE \DIFFICULT
假設將 \m{n} 個球投入 \m{n} 個箱子裏,其中每次投球相互獨立,
並且每個球落入任一箱子的機會均等。
空箱子的數目期望值是多少?
正好有一個球的箱子數目期望值是多少?
\stopEXERCISE
\startANSWER
當 \m{n} 足夠大時,兩個答案都漸進於 \m{n/e}。
首先來看空箱子的數目:

令 \m{X_i} 代表的事件爲:第 \m{i} 個箱子爲空:
\startformula
\Pr\{X_i\} = \left(\frac{n-1}{n}\right)^n
                = \left(1 - \frac{1}{n}\right)^n
                \approx \frac{1}{e}
\stopformula
其期望值爲:
\startformula
E[X] = \sum_{i=1}^n E[X_i] = \frac{n}{e}
\stopformula

箱子裏只有一個球的情況類似,其概率爲:
\startformula
\Pr\{Y_i\} = n\frac{1}{n}\left(\frac{n-1}{n}\right)^{n-1}
                = \left(\frac{n-1}{n}\right)^{n-1} \approx \frac{1}{e}
\stopformula
期望值一樣。

參見 \simpleurl{http://math.stackexchange.com/questions/545920/expectation-of-throwing-n-balls-into-n-bins}。
\stopANSWER

%e5.4-7
\startEXERCISE \DIFFICULT
爲使特徵序列長度的下界更精確,
請說明公平拋擲 \m{m} 次硬幣,
如果所有連續正面特徵序列的長度均不大於 \m{\lg{n} - 2\lg\lg{n}},
其概率小於 \m{1/n}。
\stopEXERCISE

\startANSWER
\TODO{略。}
\stopANSWER

\stopsection
