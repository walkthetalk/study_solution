\startsection[
  title={What is a binary search tree?},
]

%e12.1-1
\startEXERCISE
對於關鍵字集合 \m{\{1,4,5,10,16,17,21\}},分別畫出高度爲 2、 3、 4、 5 和 6 的二叉搜索樹。
\stopEXERCISE

\startANSWER
\externalfigure[output/e12_1_1-1]
\externalfigure[output/e12_1_1-2]
\externalfigure[output/e12_1_1-3]
\externalfigure[output/e12_1_1-4]
\externalfigure[output/e12_1_1-5]
\stopANSWER

%e12.1-2
\startEXERCISE
二叉搜索樹的性質與最小堆(參見\insection[heaps])性質有什麼不同?
能使用最小堆性質在 \m{O(n)} 時間內按序輸出一棵有 n 個節點的二叉搜索樹的所有關鍵字嗎?
可以的話,請說明如何做,否則解釋理由。
\stopEXERCISE

\startANSWER
二叉搜索樹:左子樹所有節點小於自己,右子樹的所有節點大於自己。

最小堆:左子樹和右子樹的所有節點均大於自己。

不能。原因是最小堆中左右子樹的大小關係不固定。
結果就是不能按照樹的前、中、後序遍歷在 \m{O(n)} 時間內按序輸出。
而二叉搜索樹,左右子樹大小關係固定,可以用中序遍歷按序輸出。
\stopANSWER

%e12.1-3
\startEXERCISE
設計一個執行中序遍歷的非遞迴算法。
(\hint 種較爲簡單的方法是使用棧作爲輔助結構;
另一種較複雜但比較簡潔的做法是不用棧,但要假設能測試兩個指針是否相等。)
\stopEXERCISE

\startANSWER
\CLRSH{INORDER-TREE-WALK(root)}
\startCLRS
prev = NIL
node = root
while node != NIL
	if prev == node.p
		if node.left != NIL
			prev = node
			node = node.left
		else
			output node.key
			if node.right != NIL
				prev = node
				node = node.right
			else
				prev = node
				node = node.p
	else if prev == node.left
		output node.key
		if node.right != NIL
			prev = node
			node = node.right
		else
			prev = node
			node = node.p
	else if prev == node.right
		prev = node
		node = node.p
\stopCLRS
\stopANSWER

%e12.1-4
\startEXERCISE
對於一棵有 n 個節點的二叉搜索樹,
設計可以在 \m{\Theta(n)} 時間內完成的先序遍歷算法和後序遍歷算法。
\stopEXERCISE

\startANSWER
\CLRSH{PREORDER-TREE-WALK(root)}
\startCLRS
prev = NIL
node = root
while node != NIL
	if prev == node.p
		output node.key
		if node.left != NIL
			prev = node
			node = node.left
		else
			if node.right != NIL
				prev = node
				node = node.right
			else
				prev = node
				node = node.p
	else if prev == node.left
		if node.right != NIL
			prev = node
			node = node.right
		else
			prev = node
			node = node.p
	else if prev == node.right
		prev = node
		node = node.p
\stopCLRS

\CLRSH{POSTORDER-TREE-WALK(root)}
\startCLRS
prev = NIL
node = root
while node != NIL
	if prev == node.p
		if node.left != NIL
			prev = node
			node = node.left
		else
			if node.right != NIL
				prev = node
				node = node.right
			else
				output node.key
				prev = node
				node = node.p
	else if prev == node.left
		if node.right != NIL
			prev = node
			node = node.right
		else
			output node.key
			prev = node
			node = node.p
	else if prev == node.right
		output node.key
		prev = node
		node = node.p
\stopCLRS
\stopANSWER

%e12.1-5
\startEXERCISE
因爲在基於比較的排序模型中,完成 n 個元素的排序,
其最壞情況下需要 \m{\Omega(n\lg{n})} 的時間。
試證明:任何基於比較的算法從 n 個元素的任意序列中構造一棵二叉搜索樹,
其最壞情況下需要 \m{\Omega(n\lg{n})} 的時間。
\stopEXERCISE

\startANSWER
構造二叉搜索樹的過程本質就是一個排序的過程,
所以最壞情況需要的時間與比較排序相同,都是 \m{\Omega(n\lg{n})}。
\stopANSWER

\stopsection
