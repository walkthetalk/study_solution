\startsection[
  title={Algorithms},
]

%e1.1-1
\startEXERCISE
描述兩個真實範例,一個需要排序,另一個需要求兩點之間最短距離。
\stopEXERCISE

\startANSWER
需要排序的例子:字典索引。需要求兩點之間最短距離的例子:打的。
\stopANSWER

%e1.1-2
\startEXERCISE
真實環境下,要衡量效率,除了速度,我們還需要考慮哪些因素?
\stopEXERCISE

\startANSWER
内存占用情況、編碼效率、人力成本等。
\stopANSWER

%e1.1-3
\startEXERCISE
選一個你所見過的數據結構,討論一下他的優勢和使用限制。
\stopEXERCISE

\startANSWER
鏈表:
\startigBase
\item 優勢:插入、刪除;
\item 限制:隨機存取。
\stopigBase
\stopANSWER

%e1.1-4
\startEXERCISE
最短路徑問題和郵遞員問題有哪些異同點?
\stopEXERCISE

\startANSWER
\startigBase
\item 同:都是查找最短路徑,可以用帶權圖描述;
\item 異:前者只考慮兩點,而後者需要考慮更多頂點,且最終必須回到出發點。
\stopigBase
\stopANSWER

%e1.1-5
\startEXERCISE
現實世界中有什麽問題需要得到最優解,什麽問題只需要近似最優解?
\stopEXERCISE

\startANSWER
\startigBase
\item 需要最優解:求兩個正整數的最大公約數;
\item 近似最優解:微分方程的近似解。
\stopigBase
\stopANSWER

%e1.1-6
\startEXERCISE
描述一個現實問題,在解決問題前,有時可以拿到所有輸入信息,
而有時輸入信息並不完整,會不斷更新。
\stopEXERCISE

\startANSWER
\TODO{略。}
\stopANSWER

\stopsection
