\startsection[
  title={Duality},
]

%e29.4-1
\startEXERCISE
給出練習 29.3-5 中線性規劃的對偶問題。
\stopEXERCISE

\startANSWER
\TODO{略。}
\stopANSWER

%e29.4-2
\startEXERCISE
假設我們有一個線性規劃不是標準型。
我們需要先將其轉換成標準型,
然後才能轉換爲對偶。
然而,如果能直接產生對偶,將更爲方便。
說明我們如何能夠直接構造一個任意線性規劃的對偶。
\stopEXERCISE

\startANSWER
\TODO{略。}
\stopANSWER

%e29.4-3
\startEXERCISE
對式(29.47)~(29.50)給出的最大流線性規劃,構造其對偶。
說明如何將此形式解釋爲一個最小割問題。
\stopEXERCISE

\startANSWER
\TODO{略。}
\stopANSWER

%e29.4-4
\startEXERCISE
對式(29.51)~(29.52)給出的最小費用流線性規劃,構造其對偶。
說明如何用圖和流拉來解釋這個問題。
\stopEXERCISE

\startANSWER
\TODO{略。}
\stopANSWER

%e29.4-5
\startEXERCISE
證明:一個線性規劃對偶的對偶是原始線性規劃。
\stopEXERCISE

\startANSWER
\TODO{略。}
\stopANSWER

%e29.4-6
\startEXERCISE
第 26 章哪一個結果可以被解釋成最大流問題的弱對偶?
\stopEXERCISE

\startANSWER
\TODO{略。}
\stopANSWER

\stopsection
