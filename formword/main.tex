% ==================== Main text ====================>

\define\prdname{生字组词}

\setupinteraction[
  state=start,
  color=blue,
  contrastcolor=blue,
  style=bold,
  title={生字组词},
  author={倪慶亮},
  subtitle={},
  keyword={生字 组词},
]

\setuplayout[
  % vertical
  top=0mm,
    topdistance=0mm,
      header=0mm,
        headerdistance=0mm,
        footerdistance=0mm,
      footer=0mm,
    bottomdistance=0mm,
  bottom=0mm,
  topspace=0mm,%\dimexpr(\topheight + \topdistance),
  bottomspace=0mm,%\dimexpr(\bottomheight + \bottomdistance),
  height=fit,
  % horizontal
  leftedge=0mm,
    leftedgedistance=0mm,
      leftmargin=0mm,
        leftmargindistance=0mm,
        rightmargindistance=0mm,
      rightmargin=0mm,
    rightedgedistance=0mm,
  rightedge=0mm,
  backspace=0mm,%\dimexpr(\leftedgewidth + \leftedgedistance + \leftmarginwidth + \leftmargindistance + 5mm),
  cutspace=0mm,%\dimexpr(\rightedgewidth + \rightedgedistance + \rightmarginwidth + \rightmargindistance),
  width=fit,
  % misc
  location=middle,
  marking=on
]

\startluacode
function commands.splitwords(str)
	words = string.split(str, ' ')
	groupsize = 4
	for i=1,#words,groupsize do
		tex.sprint('\\startxrow[frame=on]')
		tex.sprint('\\startxcell')
		for k = 0,3 do
			idx = i + k
			if (idx > #words) then
				break
			end
			chars = utf.split(words[idx])
			for j=1,#chars do
				tex.sprint(chars[j])
			end
			if (i ~= #words) and (k ~= 3) then
				tex.sprint('\\kern.5em')
			end
		end
		tex.sprint('\\stopxcell')
		tex.sprint('\\stopxrow')
	end
end
\stopluacode

\define[1]\fword{
\ctxlua{commands.splitwords('#1')}
}

% main text
\starttext

\startxtable[
  align={lohi},
  bodyfont={tt,22pt},
  %width=48pt,
  height=26pt,
  split=yes,
  %strut=none,
  loffset=0pt,
  roffset=0pt,
  toffset=1pt,
  %boffset=1pt,
  frame=off,
]
  \fword{可以 可是 认可 许可}
  \fword{东西 房东 东方 做东}
  \fword{东西 西风 西服 西域}
\stopxtable


\stoptext
