%Integer Functions
\subsection[section:integerFunc]{整數函式}

\reftab{svIntegerFunc}中列出了內建的整數函式,其引數即可為標量,亦可為矢量。
矢量版本的整數函式按組件逐一運算。其中的描述針對單個組件的。

泛型 \ctype{gentype} 表示函式引數的型別可以是:
\startigBase[indentnext=no]
\item \ctype{char}、 \ctype{char{2|3|4|8|16}}、
\item \ctype{uchar}、 \ctype{uchar{2|3|4|8|16}}、
\item \ctype{short}、 \ctype{short{2|3|4|8|16}}、
\item \ctype{ushort}、 \ctype{ushort{2|3|4|8|16}}、
\item \ctype{int}、 \ctype{int{2|3|4|8|16}}、
\item \ctype{uint}、 \ctype{uint{2|3|4|8|16}}、
\item \ctype{long}、 \ctype{long{2|3|4|8|16}}、
\item \ctype{ulong} 或 \ctype{ulong{2|3|4|8|16}}。
\stopigBase
而泛型 \ctype{ugentype} 則指代無符號版本的 \ctype{gentype}。
例如,如果 \ctype{gentype} 為 \ctype{char4},則 \ctype{ugentype} 為 \ctype{uchar4}。
同時,泛型 \ctype{sgentype} 指明函式的引數可以是標量(即
 \ctype{char}、 \ctype{uchar}、 \ctype{short}、 \ctype{ushort}、
 \ctype{int}、 \ctype{uint}、 \ctype{long} 或 \ctype{ulong})。
對於既有 \ctype{gentype} 引數,又有 \ctype{sgentype} 引數的內建整數函式,
 \ctype{gentype} 必須是標量或矢量版本的 \ctype{sgentype}。
例如,如果 \ctype{sgentype} 是 \ctype{uchar},
則 \ctype{gentype} 必須是 \ctype{uchar} 或 \ctype{uchar{2|3|4|8|16}}。
對於矢量版本, \ctype{sgentype} 只是簡單的拓寬成 \ctype{gentype},
參見\insection[operator]中的\initem[arithoperator]。

對於任一函式的任一具體應用,所有引數以及返回值的型別均相同,除非明確指定了其型別。

\placetable[here][tab:svIntegerFunc]
{引數既可為標量整數,也可為矢量整數的內建函式}
{% abs
\startbuffer[funcproto:abs]
ugentype abs(gentype x)
\stopbuffer
\startbuffer[funcdesc:abs]
返回 \math{|\marg{x}|}。
\stopbuffer

% abs_diff
\startbuffer[funcproto:abs_diff]
ugentype abs_diff(gentype x,
		gentype y)
\stopbuffer
\startbuffer[funcdesc:abs_diff]
返回 \math{|\marg{x} - \marg{y}|},不會有模上溢。
\stopbuffer

% add_sat
\startbuffer[funcproto:add_sat]
gentype add_sat(gentype x,
		gentype y)
\stopbuffer
\startbuffer[funcdesc:add_sat]
返回 \math{\marg{x} + \marg{y}},並使結果飽和。
\stopbuffer

% hadd
\startbuffer[funcproto:hadd]
gentype hadd (gentype x,
		gentype y)
\stopbuffer
\startbuffer[funcdesc:hadd]
返回 \math{(\marg{x} + \marg{y})>>1},中間的加法不會有模上溢。
\stopbuffer

% rhadd
\startbuffer[funcproto:rhadd]
gentype rhadd (gentype x,
		gentype y)
\stopbuffer
\startbuffer[funcdesc:rhadd]
返回 \math{(\marg{x} + \marg{y} + 1)>>1},中間的加法不會有模上溢\footnote{
矢量運算經常需要 \math{n+1} 個臨時位才能算出結果。
而 \capi{rhadd} 指令具有這個額外的位,從而無需上升取樣(upsample)或下降取樣(downsample)。
其性能優勢非常大。}。
\stopbuffer

% clamp
\startbuffer[funcproto:clamp]
gentype clamp (gentype x,
		gentype minval,
		gentype maxval)
gentype clamp (gentype x,
		sgentype minval,
		sgentype maxval)
\stopbuffer
\startbuffer[funcdesc:clamp]
返回 \math{\mapiemp{min}(\mapiemp{max}(\marg{x}, \marg{minval}), \marg{maxval})}。
如果 \math{\marg{minval} > \marg{maxval}},則結果未定義。
\stopbuffer

% clz
\startbuffer[funcproto:clz]
gentype clz (gentype x)
\stopbuffer
\startbuffer[funcdesc:clz]
返回 \carg{x} 中前導 0 的位數,從最高有效位開始。
\stopbuffer

% mad_hi
\startbuffer[funcproto:mad_hi]
gentype mad_hi (gentype a,
		gentype b,
		gentype c)
\stopbuffer
\startbuffer[funcdesc:mad_hi]
返回 \math{\mapiemp{mul_hi}(\marg{a}, \marg{b}) + \marg{c}}。
\stopbuffer

% mad_sat
\startbuffer[funcproto:mad_sat]
gentype mad_sat (gentype a,
		gentype b,
		gentype c)
\stopbuffer
\startbuffer[funcdesc:mad_sat]
返回 \math{\marg{a} \times \marg{b} + \marg{c}},並使結果飽和。
\stopbuffer

% max
\startbuffer[funcproto:max]
gentype max (gentype x, gentype y)
gentype max (gentype x, sgentype y)
\stopbuffer
\startbuffer[funcdesc:max]
如果 \math{\marg{x} < \marg{y}},則返回 \carg{y},否則返回 \carg{x}。
\stopbuffer

% min
\startbuffer[funcproto:min]
gentype min (gentype x, gentype y)
gentype min (gentype x, sgentype y)
\stopbuffer
\startbuffer[funcdesc:min]
如果 \math{\marg{y} < \marg{x}},則返回 \carg{y},否則返回 \carg{x}。
\stopbuffer

% mul_hi
\startbuffer[funcproto:mul_hi]
gentype mul_hi (gentype x,
		gentype y)
\stopbuffer
\startbuffer[funcdesc:mul_hi]
計算 \math{\marg{x} \times \marg{y}},並返回乘積的高位半部(high half)。
\stopbuffer

% rotate
\startbuffer[funcproto:rotate]
gentype rotate (gentype v, gentype i)
\stopbuffer
\startbuffer[funcdesc:rotate]
將 \carg{v} 中的每個元素都左移,其位數就是 \carg{i} 中對應元素的值
(遵守\insection[operator]中的移位取模規則)。
由左側移出的位再從右側移入。
\stopbuffer

% sub_sat
\startbuffer[funcproto:sub_sat]
gentype sub_sat (gentype x, gentype y)
\stopbuffer
\startbuffer[funcdesc:sub_sat]
返回 \math{\marg{x} - \marg{y}},並使結果飽和。
\stopbuffer

% upsample
\startbuffer[funcproto:upsampleshort]
short upsample (char hi, uchar lo)
ushort upsample (uchar hi, uchar lo)
shortn upsample (charn hi, ucharn lo)
ushortn upsample (ucharn hi, ucharn lo)
\stopbuffer
\startbuffer[funcdesc:upsampleshort]
\math{result[i] = ((short)hi[i] << 8) | lo[i]}\par
\math{result[i] = ((ushort)hi[i] << 8) | lo[i]}
\stopbuffer
\startbuffer[funcproto:upsampleint]
int upsample (short hi, ushort lo)
uint upsample (ushort hi, ushort lo)
intn upsample (shortn hi, ushortn lo)
uintn upsample (ushortn hi, ushortn lo)
\stopbuffer
\startbuffer[funcdesc:upsampleint]
\math{result[i] = ((int)hi[i] << 16) | lo[i]}\par
\math{result[i] = ((uint)hi[i] << 16) | lo[i]}
\stopbuffer
\startbuffer[funcproto:upsamplelong]
long upsample (int hi, uint lo)
ulong upsample (uint hi, uint lo)
longn upsample (intn hi, uintn lo)
ulongn upsample (uintn hi, uintn lo)
\stopbuffer
\startbuffer[funcdesc:upsamplelong]
\math{result[i] = ((long)hi[i] << 32) | lo[i]}\par
\math{result[i] = ((ulong)hi[i] << 32) | lo[i]}
\stopbuffer

% popcount
\startbuffer[funcproto:popcount]
gentype popcount (gentype x)
\stopbuffer
\startbuffer[funcdesc:popcount]
返回 \carg{x} 中非零位的數目。
\stopbuffer

% begin TABLE
\startCLFD

\clFD{abs}
\clFD{abs_diff}
\clFD{add_sat}
\clFD{hadd}
\clFD{rhadd}
\clFD{clamp}
\clFD{clz}
\clFD{mad_hi}
\clFD{mad_sat}
\clFD{max}
\clFD{min}
\clFD{mul_hi}
\clFD{rotate}
\clFD{sub_sat}
\clFD{upsampleshort}
\clFD{upsampleint}
\clFD{upsamplelong}
\clFD{popcount}

\stopCLFD

}

\reftab{fastIntegerFunc}中列出了優化\cnglo{kernel}性能時可用的快速整數函式。
泛型 \ctype{gentype} 表明函式引數的型別可以是
 \ctype{int}、 \ctype{int2}、 \ctype{int3}、
 \ctype{int4}、 \ctype{int8}、 \ctype{int16}、
 \ctype{uint}、 \ctype{uint2}、 \ctype{uint3}、
 \ctype{uint4}、 \ctype{uint8} 或 \ctype{uint16}。

\placetable[here][tab:fastIntegerFunc]
{內建的快速整數函式}
{\input{chapter_lgg/tbl/tbl_fastintegerfunc.tex}}

OpenCL C 編程語言定義了如下巨集,他們必須使用指定的值。
可以在預處理指示 \ccmm{#if} 中使用這些常量算式。

\startclc
#define CHAR_BIT	8
#define CHAR_MAX	SCHAR_MAX
#define CHAR_MIN	SCHAR_MIN
#define INT_MAX		2147483647
#define INT_MIN		(-2147483647 - 1)
#define LONG_MAX	0x7fffffffffffffffL
#define LONG_MIN	(-0x7fffffffffffffffL - 1)
#define SCHAR_MAX	127
#define SCHAR_MIN	(-127 - 1)
#define SHRT_MAX	32767
#define SHRT_MIN	(-32767 - 1)
#define UCHAR_MAX	255
#define USHRT_MAX	65535
#define UINT_MAX	0xffffffff
#define ULONG_MAX	0xffffffffffffffffUL
\stopclc

\reftab{integerMacroAndApp}中給出了上面所列巨集與\cnglo{app}所用的巨集名字之間的對應關係。

\placetable[here][tab:integerMacroAndApp]
{整形巨集與應用程式所用巨集之間的對應關係}
{\input{chapter_lgg/tbl/tbl_mminteger.tex}}

