\startsection[
  title={Properties of red-black trees},
]

%e13.1-1
\startEXERCISE
有關鍵字 \m{\left{1,2,\ldots,15\right}},以圖 13.1(a)的形式,畫出高度爲 3 的完全二叉樹。
以三種不同的方式給此樹塗色並添加 NIL 葉子,使其變成紅黑樹,其黑高分別爲 2,3 和 4。
\stopEXERCISE

\startANSWER
\externalfigure[output/e13_1_1-1]
\stopANSWER

%e13.1-2
\startEXERCISE
有紅黑樹如圖 13-1,畫出插入關鍵字 36 後的結果。
如果插入的節點標爲紅色,所得樹是不是紅黑樹?如果該節點標爲黑色呢?
\stopEXERCISE

\startANSWER
無論標記爲紅色,還是黑色,都不再是紅黑樹。
\stopANSWER

%e13.1-3
\startEXERCISE
定義{\EMP 鬆弛紅黑樹(relaxed red-black tree)}爲滿足紅黑性質 1、 3、 4 和 5 的二叉搜索樹。
換句話說,根節點的顏色沒有限制,即可以是紅色,也可以是黑色。
考慮根節點爲紅色的鬆弛紅黑樹 T,如果將 T 的根節點標爲黑色,而其他都不變,
那麼所得是否是紅黑樹?
\stopEXERCISE

\startANSWER
是。
\stopANSWER

%e13.1-4
\startEXERCISE
如果將紅黑樹的每一個紅色節點都“吸收”進其黑色父節點中,
使得紅色節點的子節點變成黑色父節點的子節點(忽略關鍵字的變化)。
當一個黑色節點的所有紅色子節點都被吸收後,他的度會是多少?
所得樹的葉節點深度如何?
\stopEXERCISE

\startANSWER
黑度不變,深度減半。
\stopANSWER

%e13.1-5
\startEXERCISE
證明:在紅黑樹中,從某節點 \m{x} 到其後代葉節點的所有簡單路徑中,
最長的一條至多是最短一條的 2 倍。
\stopEXERCISE

\startANSWER
最短路徑:全黑,包括葉子節點,長度爲 \m{bh+1}。

最長路徑:黑紅黑紅……黑,紅比黑少 1,長度爲 \m{2bh+1}。
\stopANSWER

%e13.1-6
\startEXERCISE
有一紅黑樹,其黑高爲 \m{k},則最多有多少內部節點?最少又是多少?
\stopEXERCISE

\startANSWER
內部節點最多:樹高爲 \m{2k+1},偶數層紅色,奇數層黑色,內部節點個數爲 \m{2^{2k}-1}。

內部節點最少:高爲 \m{k} 的滿二叉樹,全部爲黑色節點,內部節點數爲 \m{2^k-1},可用歸納法證明,參見定理 13.1。

附定理13.1:若紅黑樹內部節點個數爲 \m{n},則樹高最多爲 \m{2\lg(n+1)}。
\stopANSWER

%e13.1-7
\startEXERCISE
有一紅黑樹,含 \m{n} 個關鍵字,其內部節點中紅色的個數和黑色的個數比值最大是多少?最小又是多少?
\stopEXERCISE

\startANSWER
紅黑樹內部節點個數爲 \m{n},則樹高爲:
\startformula
h=\lfloor \lg{n} \rfloor + 1
\stopformula

若紅色節點最多:
\startigBase[a]
\startitem
若 \m{h} 爲偶數,從根節點到數據節點的最長路徑爲 BRBR...BR:
\startformula\startmathalignment
\NC \#(B) \NC =2^0+2^2+\ldots + 2^{h-2} \NR
\NC ratio \NC = \frac{n-\#(B)}{\#(B)} \NR
\stopmathalignment\stopformula
\stopitem

\startitem
若 \m{h} 爲奇數,從根結點到數據結點的最長路徑爲 BRBR...BBR:
\startformula\startmathalignment
\NC \#(B) \NC = \startmathcases
\NC 1		\MC h=1 \NR
\NC 2^0+2^2+\ldots +2^{h-3}+2^{h-2}	\MC h\ge 3 \NR
\stopmathcases \NR
\NC ratio \NC = \frac{n-\#(B)}{\#(B)} \NR
\stopmathalignment\stopformula
\stopitem
\stopigBase

若紅色節點最少:
\startformula\startmathalignment
\NC \#(R) \NC = \startmathcases
\NC 0		\MC h=2^h-1 \NR
\NC n-(2^{n-1}-1)	\MC n\ne 2^h-1 \NR
\stopmathcases \NR
\NC ratio \NC = \frac{n-\#(R)}{\#(R)} \NR
\stopmathalignment\stopformula

當 \m{n=3} 時,最多有 2 個紅色節點,紅黑比值爲 2。
當 \m{n} 生成滿二叉樹時,紅黑比值爲 0。
\stopANSWER
\stopsection
