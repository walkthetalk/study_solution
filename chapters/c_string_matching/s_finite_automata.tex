\startsection[
  title={String matching with finite automata},
]

%e32.3-1
\startEXERCISE
對模式 \m{P=aabab} 構造出相應的字串匹配自動機,
並說明他在文本字串 \m{T=aaababaabaababaab} 上的操作過程。
\stopEXERCISE

\startANSWER
\TODO{略。}
\stopANSWER

%e32.3-2
\startEXERCISE
對字母表 \m{\sum = \{a,b\}},
畫出與模式 \m{ababbabbababbababbabb} 對應的字串匹配自動機的狀態轉換圖。
\stopEXERCISE

\startANSWER
\TODO{略。}
\stopANSWER

%e32.3-3
\startEXERCISE
如果由 \m{P_k \sqsupset P_q} 導出 \m{k=0} 或 \m{k=q},
則稱模式 \m{P} 是{\EMP 不可重疊的}。
試描述與不可重疊模式相應的字串匹配自動機的狀態轉換圖。
\stopEXERCISE

\startANSWER
\TODO{略。}
\stopANSWER

%e32.3-4
\startEXERCISE
已知兩個模式 \m{P} 和 \m{P'},
試描述如何構造一個有限自動機,
使之能確定其中任意一個模式的所有出現位置。
儘量使自動機的狀態數最小。
\stopEXERCISE

\startANSWER
\TODO{略。}
\stopANSWER

%e32.3-5
\startEXERCISE
給定一個包括間隔字符(\refexercise{32.1-4})的模式 \m{P},
說明如何構造一個有限自動機,
使其在 \m{O(n)} 時間內找出 \m{P} 在文本 \m{T} 中的一次出現位置,
其中 \m{n=|T|}。
\stopEXERCISE

\startANSWER
\TODO{略。}
\stopANSWER


\stopsection
