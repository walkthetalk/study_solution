\subsection[section:relationFunc]{關係函式}

可以使用關係算子和相等算子(<、 <=、 >、 >=、 !=、 ==)對內建標量和矢量型別進行關係運算,
所產生的結果分別為標量或矢量帶符號整形,參見\insection[operator]。

\reftab{svRelationalFunc}中所列函式\footnote{
如果實作對規範進行了擴充,從而支持 IEEE-754 标志和異常,
則當有一個或多個算數是 NaN 時,
\reftab{svRelationalFunc}中所定義的內建函式不會引發{\ftRef{無效(invalid)}}浮點異常。}
可以內建標量或矢量型別為引數,返回的結果為標量或矢量整形。
泛型 \ctype{gentype} 指代下列內建型別:
\startigBase[indentnext=no]
\item \cldt{char}、 \cldt[n]{char}、 \cldt{uchar}、 \cldt[n]{uchar}、
\item \cldt{short}、 \cldt[n]{short}、 \cldt{ushort}、 \cldt[n]{ushort}、
\item \cldt{int}、 \cldt[n]{int}、 \cldt{uint}、 \cldt[n]{uint}、
\item \cldt{long}、 \cldt[n]{long}、 \cldt{ulong}、 \cldt[n]{ulong}、
\item \cldt{float}、 \cldt[n]{float}、 \cldt{double} 和 \cldt[n]{double}。
\stopigBase
泛型 \ctype{igentype} 指代內建帶符號整形,即:
\startigBase[indentnext=no]
\item \cldt{char}、 \cldt[n]{char}、
\item \cldt{short}、 \cldt[n]{short}、
\item \cldt{int}、 \cldt[n]{int}、
\item \cldt{long} 和 \cldt[n]{long}。
\stopigBase
泛型 \ctype{ugentype} 指代內建無符號整形,即:
\startigBase[indentnext=no]
\item \cldt{uchar}、 \cldt[n]{uchar}、
\item \cldt{ushort}、 \cldt[n]{ushort}、
\item \cldt{uint}、 \cldt[n]{uint}、
\item \cldt{ulong} 和 \cldt[n]{ulong}。
\stopigBase
其中 \ccmmsuffix{n} 為 2、 3、 4、 8 或 16。

對於標量型別的引數,如果所指定的關係為 {\ftRef{false}},則下列函式(參見\reftab{svRelationalFunc})會返回 0,否則返回 1:
\startigBase[indentnext=no]
\item \capi{isequal}、 \capi{isnotequal}、
\item \capi{isgreater}、 \capi{isgreaterequal}、
\item \capi{isless}、 \capi{islessequal}、
\item \capi{islessgreater}、
\item \capi{isfinite}、 \capi{isinf}、
\item \capi{isnan}、 \capi{isnormal}、
\item \capi{isordered}、 \capi{isunordered} 和
\item \capi{signbit}。
\stopigBase
而對於矢量型別的引數,如果所指定的關係為 {\ftRef{false}},則返回 0,
否則返回 -1 (即所有位都是 1)。

如果任一引數為 NaN,則下列關係函式返回 0:
\startigBase[indentnext=no]
\item \capi{isequal}、
\item \capi{isgreater}、 \capi{isgreaterequal}、
\item \capi{isless}、 \capi{islessequal} 和
\item \capi{islessgreater}。
\stopigBase
如果引數為標量,則當任一引數為 NaN 時, \capi{isnotequal} 返回 1;
而如果引數為矢量,則當任一引數為 NaN 時, \capi{isnotequal} 返回 -1。

\placetable[here,split][tab:svRelationalFunc]
{標量和矢量關係函式}
{\startCLFD

\clFD{isequal}
\clFD{isnotequal}
\clFD{isgreater}
\clFD{isgreaterequal}
\clFD{isless}
\clFD{islessequal}
\clFD{islessgreater}
\clFD{isfinite}
\clFD{isinf}
\clFD{isnan}
\clFD{isnormal}
\clFD{isordered}
\clFD{isunordered}
\clFD{signbit}
\clFD{any}
\clFD{all}
\clFD{bitselect}
\clFD{select}

\stopCLFD
}

