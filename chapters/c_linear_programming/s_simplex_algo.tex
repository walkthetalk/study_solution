\startsection[
  title={The simplex algorithm},
]

%e29.3-1
\startEXERCISE
請完成引理 29.4 的證明,說明必有 \m{c=c'} 和 \m{v=v'}。
\stopEXERCISE

\startANSWER
\TODO{略。}
\stopANSWER

%e29.3-2
\startEXERCISE
請說明在 \ALGO{SIMPLEX} 的第 12 行對 \ALGO{PIVOT} 的調用永遠不會減小 \m{v} 的值。
\stopEXERCISE

\startANSWER
\TODO{略。}
\stopANSWER

%e29.3-3
\startEXERCISE
證明:對 \ALGO{PIVOT} 過程給定的鬆弛型和該過程返回的鬆弛型是等價的。
\stopEXERCISE

\startANSWER
\TODO{略。}
\stopANSWER

%e29.3-4
\startEXERCISE
假設把一個標準型的線性規劃 \m{(A,b,c)} 轉換成鬆弛型。
證明:基本解是可行的當且僅當對 \m{i=1,2,\ldots,m},有 \m{b_i\ge 0}。
\stopEXERCISE

\startANSWER
\TODO{略。}
\stopANSWER

%e29.3-5
\startEXERCISE
採用 \ALGO{SIMPLEX} 求解下面的線性規劃:

最大化 \m{18x_1 + 12.5x_2}

滿足約束:
\startformula\startmathalignment[n=5]
\NC x_1 \NC + \NC x_2 \NC \le \NC 20 \NR
\NC x_1 \NC   \NC     \NC \le \NC 12 \NR
\NC     \NC   \NC x_2 \NC \le \NC 16 \NR
\NC     \NC x_1, \NC x_2 \NC \ge \NC 0 \NR
\stopmathalignment\stopformula
\stopEXERCISE

\startANSWER
\TODO{略。}
\stopANSWER

%e29.3-6
\startEXERCISE
採用 \ALGO{SIMPLEX} 求解下面的線性規劃:

最大化 \m{5x_1 - 3x_2}

滿足約束:
\startformula\startmathalignment[n=5]
\NC  x_1 \NC - \NC x_2 \NC \le \NC 1 \NR
\NC 2x_1 \NC + \NC x_2 \NC \le \NC 2 \NR
\NC     \NC x_1, \NC x_2 \NC \ge \NC 0 \NR
\stopmathalignment\stopformula
\stopEXERCISE

\startANSWER
\TODO{略。}
\stopANSWER

%e29.3-7
\startEXERCISE
採用 \ALGO{SIMPLEX} 求解下面的線性規劃:

最大化 \m{x_1 + x_2 + x_3}

滿足約束:
\startformula\startmathalignment[n=7,
align={right,right,right,right,right,right,right}]
\NC  2x_1 \NC + \NC 7.5x_2 \NC + \NC  3x_3 \NC \ge \NC 10000 \NR
\NC 20x_1 \NC + \NC   5x_2 \NC + \NC 10x_3 \NC \ge \NC 30000 \NR
\NC     \NC x_1, \NC x_2, \NC x_3 \NC      \NC \ge \NC 0 \NR
\stopmathalignment\stopformula
\stopEXERCISE

\startANSWER
\TODO{略。}
\stopANSWER

%e29.3-8
\startEXERCISE
在引理 29.5 的證明中,我們聲明至多存在 \m{\binom{m+n}{n}} 種方法來選取一個基本變量集合 \m{B}。
給出一個線性規劃的例子,其中有嚴格少於 \m{\binom{m+n}{n}} 種方法來選取此集合 \m{B}。
\stopEXERCISE

\startANSWER
\TODO{略。}
\stopANSWER

\stopsection
