% Math Functions
\subsection[section:mathFunc]{數學函式}

\reftab{svMathFunc}中列出了內建的數學函式。
內建的數學函式分為兩種:
\startigBase
\item 第一種函式有兩個版本,一個版本的引數是標量,另一個版本的引數是矢量;

\item 第二種函式只有一個版本,引數為標量浮點數。
\stopigBase

矢量版本的數學函式按組件逐一進行運算。
描述也是針對單個組件的。

無論調用環境中使用哪種捨入模式,
內建數學函式始終捨入為最近偶數,返回的結果也始終如一。

\reftab{svMathFunc}中所列函式即可接受標量引數,也可接受矢量引數。
泛型 \ctype{gentype} 表示函式引數的型別可以是
 \ctype{float}、 \ctype{float2}、 \ctype{float3}、 \ctype{float4}、
 \ctype{float8}、 \ctype{float16}、 \ctype{double}、 \ctype{double2}、
 \ctype{double3}、 \ctype{double4}、 \ctype{double8} 或 \ctype{double16}。
泛型 \ctype{gentypef} 表示函式引數的型別可以是
 \ctype{float}、 \ctype{float2}、 \ctype{float3}、 \ctype{float4}、
 \ctype{float8} 或 \ctype{float16}。
泛型 \ctype{gentyped} 表示函式引數的型別可以是
 \ctype{double}、 \ctype{double2}、 \ctype{double3}、 \ctype{double4}、
 \ctype{double8} 或 \ctype{double16}。
如果沒有特殊說明,函式的返回值與引數的型別都相同。

\placetable[here,split][tab:svMathFunc]
{引數既可為標量,也可為矢量的內建數學函式表}
{\input{chapter_lgg/tbl/tbl_svmathfunc.tex}}

% half_ & native_ math function
\reftab{hnMathFunc}中列出了下列函式:
\startigBase
\item \reftab{svMathFunc}中的部分函式,但定義時帶有前綴 \ccmm{half_}。
實現這些函式時,精度至少要有 10 位,即所有 ULP 值都要小於等於 8192 ulp
(ULP:units in the last place,最後一位的進退位)。

\item \reftab{svMathFunc}中的部分函式,但定義時帶有前綴 \ccmm{native_}。
這些函式可能會映射到一條或多條原生的\cnglo{device}指令上,
性能通常比對應的不帶前綴 \ccmm{native_} 的函式更好。
這些函式的精度(以及某些情況下的輸入範圍)\cnglo{impdef}。

\item 用於除法和倒數運算的 \ccmm{half_} 和 \ccmm{native_} 函式。
\stopigBase

\startnotepar
對於帶前綴 \ccmm{native_} 的函式,
參數的取值範圍以及取最大值時會產生什麼錯誤都\cnglo{impdef}。
\stopnotepar

在\reftab{hnMathFunc}中,泛型 \ctype{gentype} 表示函式引數的型別可以是
 \ctype{float}、 \ctype{float2}、 \ctype{float3}、 \ctype{float4}、
 \ctype{float8} 或 \ctype{float16}。

\placetable[here][tab:hnMathFunc]
{內建的 \ccmm{half_} 和 \ccmm{native_} 數學函式}
{\input{chapter_lgg/tbl/tbl_hnmathfunc.tex}}

實作可以自行決定 \capi{half_} 函式是否支持去規格化值。
如果引數是去規格化數, \capi{half_} 函式可以返回任何值,只要\insection[ftzmECB]允許就行,
無論 \ccmm{-cl-denorms-are-zero} (參見\insection[MathIntrinsicsOption])是否有效。

有下列符號常量可用。這些值的型別都是 \ctype{float},在單精度浮點數的精度內是準確的。
\input{chapter_lgg/tbl/tbl_floatconstant.tex}

如果\cnglo{device}支持雙精度浮點數,還有下列符號常量可用:
\input{chapter_lgg/tbl/tbl_doubleconstant.tex}

% Floating-point macros and pragmas
\subsubsection{浮點巨集和雜注}

雜注(pragma) \cpragmaemp{FP_CONTRACT} 可用來允許(如果狀態是 \ccmm{on})
或禁止(如果狀態是 \ccmm{off})實作化簡算式。
他可位於外部聲明的外面,也可位於複合語句中的顯式聲明或語句的前面。
當在外部聲明外面時,在遇到下一個 \cpragmaemp{FP_CONTRACT} 或者翻譯單元結束時就無效了。
當在複合語句中時,在遇到下一個 \cpragmaemp{FP_CONTRACT}
(包括嵌套的複合語句中的 \cpragmaemp{FP_CONTRACT})或者複合語句結束時就無效了;
在複合語句末尾處,會恢復成此語句之前的狀態。
在其他任何上下文中使用此雜注,其行為都是未定義的。

這樣設置 \cpragmaemp{FP_CONTRACT}:
\startclc
#pragma OPENCL FP_CONTRACT on-off-switch

on-off-switch is one of:
	/BTEX\ftEmp{ON}/ETEX, /BTEX\ftEmp{OFF}/ETEX or /BTEX\ftEmp{DEFAULT}/ETEX.
	The /BTEX\ftEmp{DEFAULT}/ETEX value is /BTEX\ftEmp{ON}/ETEX.
\stopclc

% float - single precision
巨集 \cmacroemp{FP_FAST_FMAF} 用來指明對於單精度浮點數,
函式 \capi{fma} 是否比直接編碼更快。
如果定義了此巨集,則表明對算元為 \ctype{float} 的乘、加運算,
函式 \capi{fma} 一般跟直接編碼一樣快,或者更快。

OpenCL C 編程語言定義了如下巨集,他們必須使用指定的值。
可以在預處理指示 \ccmm{#if} 中使用這些常量算式。
\startclc
#define FLT_DIG		6
#define FLT_MANT_DIG	24
#define FLT_MAX_10_EXP	+38
#define FLT_MAX_EXP	+128
#define FLT_MIN_10_EXP	-37
#define FLT_MIN_EXP	-125
#define FLT_RADIX	2
#define FLT_MAX		0x1.fffffep127f
#define FLT_MIN		0x1.0p-126f
#define FLT_EPSILON	0x1.0p-23f
\stopclc

\reftab{tblFltMacroAndApp}中給出了上面所列巨集與\cnglo{app}所用的巨集名字之間的對應關係。

\placetable[here][tab:tblFltMacroAndApp]
{單精度浮點巨集與應用程式所用巨集的對應關係}
{\input{chapter_lgg/tbl/tbl_mmfloat.tex}}

下列兩個巨集將會展開成整數常量算式。
如果 \carg{x} 是 0 或 NaN,則 \math{\mapiemp{ilogb}(x)} 會分別返回這兩個值。
\startigBase
\item \cmacroemp{FP_ILOGB0} 為 \ccmm{{INT_MIN}} 或 \ccmm{-{INT_MAX}}。
\item \cmacroemp{FP_ILOGBNAN} 為 \ccmm{{INT_MAX}} 或 \ccmm{{INT_MIN}}。
\stopigBase

除此之外,還有一些常量可用,如\reftab{tblFltMacroConst}所示。
他們的型別都是 \ctype{float},在 \ctype{float} 型別的精度內是準確的。

\placetable[here][tab:tblFltMacroConst]
{單精度浮點常量}
{\input{chapter_lgg/tbl/tbl_constfloat.tex}}

% double - double precision
如果\cnglo{device}支持雙精度浮點數,還有下列巨集和常量可用:
\startigBase
\item 巨集 \cmacroemp{FP_FAST_FMA} 指明處理雙精度浮點數時,
 \capi{fma} 系列函式是否比直接編碼更快。
如果定義了此巨集,則表明對 \ctype{double} 算元的乘、加運算,
函式 \capi{fma} 一般跟直接編碼一樣快,或者更快。
\stopigBase

OpenCL C 編程語言定義了如下巨集,他們必須使用指定的值。
可以在預處理指示 \ccmm{#if} 中使用這些常量算式。
\startclc
#define DBL_DIG		15
#define DBL_MANT_DIG	53
#define DBL_MAX_10_EXP	+308
#define DBL_MAX_EXP	+1024
#define DBL_MIN_10_EXP	-307
#define DBL_MIN_EXP	-1021
#define DBL_MAX		0x1.fffffffffffffp1023
#define DBL_MIN		0x1.0p-1022
#define DBL_EPSILON	0x1.0p-52
\stopclc

\reftab{tblDblMacroAndApp}中給出了上面所列巨集與\cnglo{app}所用的巨集名字之間的對應關係。

\placetable[here][tab:tblDblMacroAndApp]
{雙精度浮點巨集與應用程式所用巨集的對應關係}
{\input{chapter_lgg/tbl/tbl_mmdouble.tex}}

除此之外,還有一些常量可用,如\reftab{tblDblMacroConst}所示。
他們的型別都是 \ctype{double},在 \ctype{double} 型別的精度內是準確的。

\placetable[here][tab:tblDblMacroConst]
{雙精度浮點常量}
{\input{chapter_lgg/tbl/tbl_constdouble.tex}}

