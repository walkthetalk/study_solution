\startsubject[
  title={Problems},
]

%p29-1
\startPROBLEM
(Linear-inequality feasibility)
給定 \m{m} 個線性不等式,其中有 \m{n} 個變量 \m{x_1,x_2,\ldots,x_n} ,
{\EMP 線性不等式可行性問題}關注是否有變量的一個設置,能夠同時滿足每個不等式。

\startigBase[a]\startitem
證明:如果有一個線性規劃的算法,
那麼可以利用他來解一個線性不等式可行性問題,
在線性規劃問題中,你用到的變量和約束的個數應該是 \m{n} 和 \m{m} 的多項式。
\stopitem\stopigBase

\startANSWER
\TODO{略。}
\stopANSWER

\startigBase[continue]\startitem
證明:如果有一個線性不等式可行性問題的算法,
那麼可以利用他來求解線性規劃問題。
在線性不等式可行性問題中,
你用到的變量和線性不等式的個數應該是 \m{n} 和 \m{m} 的多項式,
即線性規劃中變量和約束的數目。
\stopitem\stopigBase

\startANSWER
\TODO{略。}
\stopANSWER
\stopPROBLEM

%p29-2
\startPROBLEM
(Complementary slackness)
{\EMP 互補鬆弛性}描述原始變量值和對偶約束,
以及對偶變量值與原始約束之間的關係。
設 \m{\bar{x}} 表示式(29.16)~(29.18)中給出的原始線性規劃的一個可行解,
 \m{\bar{y}} 表示式(29.83)~(29.85)中給出的對偶線性規劃的可行解。
互補鬆弛闡述下面的條件是 \m{\bar{x}} 和 \m{\bar{y}} 爲最優的充分必要條件:
\startformula
\sum_{i=1}^{m}a_{ij}\bar{y}_i = c_j \text{或者} \bar{x}_j=0, j=1,2,\ldots,n
\stopformula
以及
\startformula
\sum_{i=1}^{n}a_{ij}\bar{x}_j = b_i \text{或者} \bar{y}_i=0, i=1,2,\ldots,m
\stopformula

\startigBase[a]\startitem
對式(29.53)~(29.57)中的線性規劃驗證互補鬆弛性成立。
\stopitem\stopigBase

\startANSWER
\TODO{略。}
\stopANSWER

\startigBase[continue]\startitem
證明:對任意原始線性規劃和他相應的對偶,互補鬆弛性成立。
\stopitem\stopigBase

\startANSWER
\TODO{略。}
\stopANSWER

\startigBase[continue]\startitem
證明:式(29.16)~(29.18)中給出的原始線性規劃的一個可行解 \m{\bar{x}} 是最優的,
當且僅當存在值 \m{\bar{y}=(\bar{y}_1,\bar{y}_2,\ldots,\bar{y}_m)} 使得:
\startigBase[n]\startitem
\m{\bar{y}} 是式(29.83)~(29.85)中給出的對偶線性規劃的一個可行解。
\stopitem\stopigBase

\startigBase[continue]\startitem
對於所有的 \m{j} 有 \m{\sum_{i=1}^{m}a_{ij}\bar{y}_i=c_j},於是 \m{\bar{x}_j > 0},以及
\stopitem\stopigBase

\startigBase[continue]\startitem
對於所有的 \m{i} 有 \m{\bar{y}_i=0},於是 \m{\sum_{j=1}^{n}a_{ij}\bar{x}_j < b_i}。
\stopitem\stopigBase
\stopitem\stopigBase

\startANSWER
\TODO{略。}
\stopANSWER

\stopPROBLEM

\stopsubject%Problems
