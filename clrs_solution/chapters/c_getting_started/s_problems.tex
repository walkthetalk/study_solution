\startsubject[
  title={Problems},
]
%p2-1
\startPROBLEM
在歸並排序中對小數列使用插入排序。

雖然最壞情況下,歸并排序運行時間為 $\Theta(n\lg n)$,
而插入排序為 $\Theta(n^2)$,
但在許多機器上對於小規模問題,
由於插入排序的常係數很小,
反而比歸并排序還要快一些。
因此在歸并排序中,儅子問題規模足夠小時,
可以改用插入排序來處理。
嘗試改進歸并排序,
對於 $n/k$ 個子數列分別進行插入排序,
然後再用標準的歸并算法進行合并。
其中 $k$ 待定。
\startigBase[a]

\item 證明對 $n/k$ 個長度爲 $k$ 的子列進行插入排序,所需時間最長為 $\Theta(nk)$。

\startANSWER
設對長度爲 $k$ 的數列進行插入排序所用時間爲 \m{a k^2 + b k + c},則總時間爲:
\startformula
\frac{n}{k}(a k^2 + b k + c) = a n k + b n + \frac{c n}{k} = \Theta(nk)
\stopformula
\stopANSWER

\item 證明歸並這些子列所需時間最多為 $\Theta(n \lg(n/k))$。

\startANSWER
對 $a$ 個長度爲 $k$ 的子列進行歸並排序所需時間爲:
\startformula
T(a) = \startcases
\NC 0	\NC \text{若 $a = 1$,} \NR
\NC 2 T(a/2) + a k \NC \text{若 $a = 2^p$, $p > 0$。} \NR
\stopcases
\stopformula

設 $G(a) = \frac{T(a)}{a}$,則:
\startformula\startalign
\NC G(a)	\NC = \frac{T(a)}{a} \NR
\NC 		\NC = \frac{2 T(a/2) + a k}{a} \NR
\NC		\NC = \frac{T(a/2)}{a/2} + k \NR
\NC		\NC = G(a/2) + k \NR
\stopalign\stopformula
即
\startformula
G(a) = \startcases
\NC 0	\NC 若 $a = 1$,\NR
\NC G(a/2) + k \NC 若 $a = 2^p$, $p > 0$。\NR
\stopcases
\stopformula

設 $H(p) = G(2^p)$,則:
\startsplitformula\startalign
\NC H(p)	\NC = G(2^p) \NR
\NC		\NC = G(2^p/2) + k \NR
\NC		\NC = G(2^{p-1}) + k \NR
\NC		\NC = H(p-1) + k \NR
\stopalign\stopsplitformula
即
\startformula
H(p) = \startcases
\NC 0	\NC 若 $p = 0$, \NR
\NC H(p-1) + k \NC 若 $p > 0$。 \NR
\stopcases
\stopformula

\startsplitformula\startalign
\NC H(p) \NC = k p \NR
\NC G(2^p) \NC = kp \NR
\NC G(a) \NC = k \lg(a) \NR
\NC \frac{T(a)}{a} \NC = k \lg(a) \NR
\NC T(a) \NC = k a \lg(a) \NR
\NC T(\frac{n}{k}) \NC = k \frac{n}{k} \lg(\frac{n}{k}) \NR
\NC		\NC = n \lg(\frac{n}{k}) \NR
\NC     \NC \Theta(n \lg(n/k)) \NR
\stopalign\stopsplitformula
\stopANSWER

\item 假定修改後的算法運行時間爲 $\Theta(nk+n \lg(n/k))$,
$k$ 最大爲多少時(作爲 $n$ 的函式),新算法與原歸並排序算法所用時間相同?

\startANSWER
\startformula\startalign
\NC k \NC = \lg(n) \NR
\NC \Theta(nk + n \lg(n/k)) \NC = \Theta(n \lg(n) + n \lg(\frac{n}{\lg(n)}) \NR
\NC			\NC = \Theta(n \lg(n)) \NR
\stopalign\stopformula
\stopANSWER

\stopigBase
\stopPROBLEM

%p2-2
\startPROBLEM
冒泡排序的正確性

冒泡排序通俗易懂,但效率不高。
他通過重複交換兩個(次序顛倒的)相鄰元素的方式進行排序。
例程 \ALGO{BUBLLE-SORT} 對數列 $A[1:n]$ 進行排序。

\CLRSH{BUBBLE-SORT(A, n)}
\startCLRSCODE
for i = 1 to n - 1
	for j = n downto i + 1
		if A[j] < A[j - 1]
			// exchange A[j] with A[j - 1]
\stopCLRSCODE

\startigBase[a]
\item 在數列 $A$ 上執行 \ALGO{BUBBLE-SORT(A,n)} 後變爲數列 $A'$,
要證明算法的正確性,
除了要證明
\startformula
A'[1]\le A'[2]\le \cdots \le A'[n]
\stopformula
之外,還需要證明什麼?
\stopigBase

\startANSWER
還需要證明排序完成後,新數列包含舊數列中的所有元素。
\stopANSWER

\startigBase[continue]
\item 分析並證明內層 \CLRSCODE{for} 循環的循環不變性。
要求使用本章所展示的證明方式。
\stopigBase

\startANSWER
\TODO{略。}
\stopANSWER

\startigBase[continue]
\item 利用上一問所證明的循環不變性的終止條件,
分析外層循環的循環不變性,並以此證明不等式(2.5)。
要求使用本章所展示的證明方式。
\stopigBase

\startANSWER
\TODO{略。}
\stopANSWER

\startigBase[continue]
\item 冒泡排序在最差情況下運行時間怎樣?
與插入排序相比如何?
\stopigBase

\startANSWER
冒泡排序的比較次數最多爲:$\sum_{i=1}^{n-1}{n-i} = \frac{n(n - 1)}{2}$,
相應的交換次數也一樣,所以最壞情況下所用時間爲 $\Theta(n^2)$,
與插入排序所用時間相同。

通常,兩種算法最好情況下時間均爲 $\Theta(n)$,
但是此處的實現卻爲 $\Theta(n^2)$。
要想在最好情況下達到 $\Theta(n)$,
在外部循環中,如果沒有發生任何交換就直接返回。

另外,冒泡排序會比插入排序慢很多,
因爲交換所引入的賦值操作太多了。
\stopANSWER

\stopPROBLEM

%p2-3
\startPROBLEM
现有多项式:
\startformula\startalign
\NC P(x) \NC = \sum_{k=0}^{n} a_kx^k \NR
\NC \NC = a_0 + a_1 x + a_2 x^2 + \cdots + a_{n-1}x^{n-1} + a_n x^n \NR
\stopalign\stopformula
给定系数 $a_0,a_1,a_2,\cdots,a_n$ 后,
可以按 Horner 定律計算多項式的值:
\startformula
P(x) = a_0 + x(a_1 + x(a_2 + \cdots + x(a_{n-1} + xa_n) \cdots))
\stopformula
具體實現如下:

\CLRSH{HORNER(A, n, x)}
\startCLRSCODE
p = 0
for i = n downto 0
	p = A[i] + x \cdot p
return p
\stopCLRSCODE

\startigBase[a]
\item 用 $\Theta$ 表示此例程的運行時間。

\startANSWER
$\Theta(n)$
\stopANSWER

\item 按此多項式的原始定義來計算,需要多少時間?

\startANSWER
\startCLRSCODE
p = 0
for i = 0 to n
	m = 1
	for k = 1 to i
		m = m \cdot x
	p = p + A[i] \cdot m
\stopCLRSCODE
所用時間爲 $\Theta(n^2)$。
\stopANSWER

\item 分析 \ALGO{HORNER} 的循環不變性。
每次迭代前:
\startformula
y = \sum_{k=0}^{n-(i+1)} A_{k+i+1}\cdot x^k
\stopformula
將空多項式求和結果定義爲 0。
以本章循環不變性的證明方式,
證明最終結果爲:
\startformula
y = \sum_{k=0}^n a_kx^k
\stopformula

\startANSWER
每次迭代完成後:
\startformula\startalign
\NC p \NC = A_i + x \sum_{k=0}^{n-(i+1)}A_{k+i+1}x^k \NR
\NC	\NC = A_i x^0 + \sum_{k=0}^{n-i-1}A_{k+i+1}x^{k+1} \NR
\NC	\NC = A_i x^0 + \sum_{k=1}^{n-i}A_{k+i}x^{k} \NR
\NC	\NC = \sum_{k=0}^{n-i}A_{k+i}x^{k} \NR
\stopalign\stopformula
最後一次 $i$ 爲 $0$,
代入上式可得 $p = \sum_{k=0}^{n}a_{k}x^{k}$。
\stopANSWER

\stopigBase

\stopPROBLEM

%p2-4
\startPROBLEM[problem:inversion]
逆序對(Inversion)

數列 $A[1:n]$ 中的 $n$ 個數各不相同。
如果 $i<j$ 且 $A[i]>A[j]$,
則稱 $(i,j)$ 為 $A$ 的逆序對。
\startigBase[a]
\item 列出數列 $\langle 2,3,8,6,1\rangle$ 中的五個逆序對。

\startANSWER
$(2,1),(3,1),(8,6),(8,1),(6,1)$
\stopANSWER

\item 如果數列 $A$ 中所有元素全部取自集合 $\{1,2,\cdots,n\}$,
如何使 $A$ 中的逆序對最多,最多是多少?

\startANSWER
$\langle n, n-1, \cdots, 1\rangle$,
其逆序對數目爲 $(n-1)+(n-2)+\cdots+1=\frac{n(n-1)}{2}$。
\stopANSWER

\item 插入排序的運行時間與數列中逆序對的數目有什麼關系?
試驗證你的結論。

\startANSWER
$\Theta(n+d)$,其中 $d$ 爲逆序對的數目,
$n$ 爲外層循環所用時間。
\stopANSWER

\item 給出算法以確定數列中逆序對的數目。
要求最壞情況下所用時間爲 $\Theta(n\lg n)$,
$n$為數列中元素數目。

\startANSWER
用歸并排序解決此問題。
空間占用為 $\Theta(n)$。
最差運行時間爲 $\Theta(n\lg n)$。

\CLRSH{INVERSIONS_COUNT(A, p, r)}
\startCLRSCODE
if p < r
	inversions = 0
	q = (p + r) / 2
	inversions += \ALGO{INVERSIONS-COUNT(A, p, q)}
	inversions += \ALGO{INVERSIONS-COUNT(A, q + 1, r)}
	inversions += \ALGO{INVERSIONS-MERGE(A, p, q, r)}
	return inversions
else
	return 0
\stopCLRSCODE

\CLRSH{INVERSIONS_MERGE(A, p, q, r)}
\startCLRSCODE
n_1 = q - p + 1
n_2 = r - q
// let $L[1..n_1]$ and $R[1..n_2]$ be new arrays
for i = 1 to n_1
	L[i] = A[p + i - 1]
for j = 1 to n_2
	R[j] = A[q + j]
i = 1
j = 1
for k = p to r
	if i > n_1
		A[k] = R[j]
		j = j + 1
	else if j > n_2
		A[k] = L[i]
		i = i + 1
	else if L[i] \le R[j]
		A[k] = L[i]
		i = i + 1
	else
		A[k] = R[j]
		j = j + 1
		inversions += n_1 - i
return inversions
\stopCLRSCODE
\stopANSWER

\stopigBase
\stopPROBLEM

\stopsubject
