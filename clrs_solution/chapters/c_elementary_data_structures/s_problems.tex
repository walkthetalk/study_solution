\startsubject[
  title={Problems},
]

%p10-1
\startPROBLEM
(Comparisons among lists)
對於下表中的 4 種鏈表,所列的每種動態集合操作在最壞情況下的漸進運行時間是多少?
\stopPROBLEM

\startANSWER
\input{tbl/tbl10-1.tex}

對於已排序的單向鏈表,如果跟蹤鏈表爲節點,則 \ALGO{MAXIMUM} 的運行時間爲常數。
\stopANSWER

%p10-2
\startPROBLEM
(Mergeable heaps using linked lists)
{\EMP 可合並堆}({\EMP mergeable heap})支持以下操作: \ALGO{MAKE-HEAP}(創建一個空的可合並堆)、
 \ALGO{INSERT}、 \ALGO{MINIMUM}、 \ALGO{EXTRACT-MIN} 和 \ALGO{UNION}\footnote{%
由於我們定義的是 \ALGO{MINIMUM} 和 \ALGO{EXTRACT-MIN},因此我們也可以將其稱爲{\EMP 可合並最小堆};
如果我們定義的是 \ALGO{MAXIMUM} 和 \ALGO{EXTRACT-MAX},則可以將其稱爲{\EMP 可合並最大堆}。
}。
試說明下列各種情況下如何用鏈表實現可合並堆。
試者使各種操作盡可能高效。
根據動態集合的規模分析每種操作的運行時間。
\startigBase[a]
\item 鏈表已排序;
\item 鏈表未排序;
\item 鏈表未排序,且待合並的動態集合是不相交的。
\stopigBase
\stopPROBLEM

\startANSWER
\input{tbl/tbl10-2.tex}

其中,如果可以跟蹤鏈表的尾節點,對於未排序的鏈表的 \ALGO{UNION} 操作可以是常數時間。
\stopANSWER

%p10-3
\startPROBLEM
(Searching a sorted compact list)
\inexercise[compact_list] 提出了如何使鏈表保持緊湊的問題。
我們假設所有 key 各不相同,且緊湊鏈表已排序,即,
 \m{key[i] < key[next[i]]},其中 \m{i=1,2,\ldots,n},且 \m{next[i] \ne NIL}。
同時假設有一個變量 \m{L} 保存鏈表種第一個元素的下標值。
在這些假設下,我們可以用下列隨機化算法以期望時間 \m{O(\sqrt{n})} 搜索鏈表。

\CLRSH{COMPACT-LIST-SEARCH(L, n, k)}
\startCLRS
i = L
while i ≠ NIL and key[i] < k
	j = RANDOM(1, n)
	if key[i] < key[j] and key[j] ≤ k
		i = j
		if key[i] == k
			return i
	i = next[i]
if i == NIL or key[i] > k
	return NIL
else
	return i
\stopCLRS

如果忽略 3~7 行,那麼他就退化成一個普通的搜索已排序鏈表的算法,
其中下標 \m{i} 依次指向鏈表的各個位置。
一旦下標 \m{i} 落到鏈表以外,或者 \m{key[i]\ge k},算法終止。
在 \m{key[i]\ge k} 這種情況下,如果 \m{key[i] = k},顯然我們找到了值爲 \m{k} 的 key;
而如果 \m{key[i] > k},那麼我們就無法找到值爲 \m{k} 的 key 了;
因此算法終止。

3~7 行試圖跳到前面隨機選擇一個位置 \m{j}。
如果 \m{key[j]} 大於 \m{key[i]},且不大於 \m{k},那麼這樣處理是有好處的;
這種情況下, \m{j} 所指位置是普通搜索時 \m{i} 的必經之地。
由於鏈表是緊湊的,我們知道 1 到 n 之間的任意值所指位置都是鏈表中的元素,而不會是空閒的。

我們不直接分析 \ALGO{COMPACT-LIST-SEARCH} 的性能,而是先分析一個相關的算法, \ALGO{COMPACT-LIST-SEARCH'},
他會執行兩個獨立的循環。
這個算法會接受一個額外的參數 \m{t},此參數決定了第一個循環迭代次數的上界。

\CLRSH{COMPACT-LIST-SEARCH‘(L, n, k, t)}
\startCLRS
i = L
for q = 1 to t
	j = RANDOM(1, n)
	if key[i] < key[j] and key[j] ≤ k
		i = j
		if key[i] == k
			return i
while i ≠ NIL and key[i] < k
	i = next[i]
if i == NIL or key[i] > k
	return NIL
else
	return i
\stopCLRS

爲了比較這兩個算法,我們假設兩個算法中調用 \ALGO{RANDOM(1, n)} 所返回的值是相同的。
\startigBase[a]
\startitem
假設 \ALGO{COMPACT-LIST-SEARCH(L, n, k)} 中 2~8 行的 while 循環進行了 t 次迭代。
證明 \ALGO{COMPACT-LIST-SEARCH'(L, n, k, t)} 會返回同樣的結果,
且 for 和 while 循環迭代的次數之和至少爲 t。
\stopitem

\startANSWER
\TODO{}
\stopANSWER
\stopigBase

在調用 \ALGO{COMPACT-LIST-SEARCH'(L, n, k, t)} 的時候,
令隨機變數 \m{X_t} 代表 2~7 行的 for 迴圈經 t 次迭代後鏈表中
(即通過 \m{next} 指針鏈)從位置 i 到目標關鍵字 k 之間的距離。
\startigBase[a,continue]
\startitem
論證 \ALGO{COMPACT-LIST-SEARCH'(L, n, k, t)} 的期望運行時間為 \m{O(t+E[X_t])}。
\stopitem

\startANSWER
要不就是在 t 次迭代後找到了 k,否則 while 迴圈還要進行 \m{X_t} 次迭代。
\stopANSWER

\startitem
證明: \m{E[X_t]\le \sum_{r=1}^{n}(1-r/n)^t}。(\hint 用等式 C.25)
\stopitem

\startANSWER
當 \m{t=1} 時,距離大於等於 r 的概率为 \m{(n-r)/n}。
\startformula
\Pr\{X_t\ge r\} = (\frac{n-r}{n})^t = (1-\frac{r}{n})^t
\stopformula

即,有一次調用 \ALGO{RANDOM} 會得到期望的距離,而其他的則會不斷接近期望距離。

利用 C.25:
\startformula
E[X_t] = \sum_{r=1}^{\infty} \Pr\{X_t \ge r\}
        = \sum_{r=1}^n \Pr\{X_t \ge r\}
        = \sum_{r=1}^n \left(1 - \frac{r}{n}\right)^t
\stopformula
\stopANSWER

\startitem
證明: \m{E[X_t]\le n/(t+1)}。
\stopitem

\startANSWER
由 A.11 可知:
\startformula
\sum_{r=0}^{n-1} r^t \le \int_0^n x^t dx = \frac{n^{t+1}}{t+1}
\stopformula
因此:
\startformula\startmathalignment
\NC E[X_t]
    \NC= \sum_{r=1}^n \bigg(1 - \frac{r}{n}\bigg)^t \NR
\NC \NC= \sum_{r=0}^{n-1} \bigg(\frac{r}{n}\bigg)^t \NR
\NC \NC= \frac{1}{n^t} \sum_{r=0}^{n-1} r^t \NR
\NC \NC\le \frac{1}{n^t} \cdot \frac{n^{t+1}}{t + 1} \NR
\NC \NC= \frac{n}{t+1} \NR
\stopmathalignment\stopformula
\stopANSWER

\startitem
證明: \ALGO{COMPACT-LIST-SEARCH'(L, n, k, t)} 的期望運行時間爲 \m{O(t+n/t)}。
\stopitem

\startANSWER
\startformula
O(t + E[X_t]) = O(t + n/(t+1)) = O(t + n/t)
\stopformula
\stopANSWER

\startitem
論證 \ALGO{COMPACT-LIST-SEARCH'(L, n, k, t)} 的期望運行時間爲 \m{O(\sqrt{n})}。
\stopitem

\startANSWER
我們只需找到 \m{t+n/t} 的最小值。針對 \m{t} 求導爲 \m{1-n/t^2}, \m{t=\sqrt{n}} 時導數爲零。
因此期望運行時間爲 \m{O(\sqrt{n}}。
\stopANSWER

\startitem
爲什麼我們假設 \ALGO{COMPACT-LIST-SEARCH} 中的所有關鍵字互不相同?
論證如果包含相同關鍵字,則隨機跳躍不一定能改進漸進時間。
\stopitem

\startANSWER
如果有相同關鍵字,則無法得到 \m{E[X_t]\le n/(t+1)}。
只有當 \ALGO{RANDOM} 找到的值大於當前值時才會發生躍變。
例如,如果鏈表中全都是 0,而我們要找 1,算法仍然需要一直迭代,直到鏈表尾部。
\stopANSWER

\stopigBase
\stopPROBLEM

\stopsubject%Problems
