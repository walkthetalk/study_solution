\subsection{圖像對象查詢}

對於所有\cnglo{memobj}都適用的資訊可以使用 \capi{clGetMemObjectInfo} 來獲取,
參見\insection[memObjQuery]。

而對於由 \capi{clCreateImage} 創建的\cnglo{imgobj}而言,一些特定資訊使用如下函式獲取:

\topclfunc{clGetImageInfo}

\startCLFUNC
cl_int clGetImageInfo (cl_mem image,
			cl_image_info param_name,
			size_t param_value_size,
			void *param_value,
			size_t *param_value_size_ret)
\stopCLFUNC

\carg{image} 即被查詢的\cnglo{imgobj}。

\carg{param_name} 指明要查詢什麼資訊。
他所支持的類型以及在 \carg{param_value} 中返回的資訊如\reftab{clgetimginfo}所示。

\carg{param_value} 用來存儲查詢結果。
如果 \carg{param_value} 是 \cmacro{NULL},則忽略。

\carg{param_value_size},即 \carg{param_value} 內存塊的大小。
其值必須 >= \reftab{clgetimginfo}中返回型別的大小。

\carg{param_value_size_ret},即查詢結果的實際大小。
如果 \carg{param_value_size_ret} 是 \cmacro{NULL},則忽略。

如果執行成功, \capi{clGetImageInfo} 會返回 \cenum{CL_SUCCESS}。
否則,返回下列錯誤碼之一:
\startigBase
\item \cenum{CL_INVALID_VALUE},如果 \cenum{param_name} 的值無效,
或者 \cenum{param_value_size} 的值 < \reftab{clgetimginfo}中返回型別的大小
並且 \cenum{param_value} 不是 \cmacro{NULL},
或者 \carg{param_name} 指的是某個擴展,但是此\cnglo{device}不支持相應擴展。

\item \cenum{CL_INVALID_MEM_OBJECT},如果 \carg{image} 無效。

\item \cenum{CL_OUT_OF_RESOURCES},如果\scdevfailres。
\item \cenum{CL_OUT_OF_HOST_MEMORY},如果\schostfailres。
\stopigBase

\placetable[here][tab:clgetimginfo]
{\capi{clGetImageInfo} 所支持的 \carg{param_names}}
{\input{chapter_rt/tbl/tbl_clgetimginfo.tex}}

