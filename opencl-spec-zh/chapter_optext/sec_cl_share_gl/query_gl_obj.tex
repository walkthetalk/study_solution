\subsection{由 CL 內存對象查詢 GL 對象的資訊}

\topclfunc{clGetGLObjectInfo}

\startCLFUNC
cl_int clGetGLObjectInfo (
		cl_mem memobj,
		cl_gl_object_type *gl_object_type,
		GLuint *gl_object_name)
\stopCLFUNC

此函式可查詢用來創建 OpenCL \cnglo{memobj}的 OpenGL 對象及其對象類型,
即他是材質、渲染緩衝還是緩衝對象。

\carg{gl_object_type} 會返回附到 \carg{memobj} 上的 GL 對象的類型,
可以是:
\startigBase[indentnext=no]
\item \cenum{CL_GL_OBJECT_BUFFER}
\item \cenum{CL_GL_OBJECT_TEXTURE2D}
\item \cenum{CL_GL_OBJECT_TEXTURE3D}
\item \cenum{CL_GL_OBJECT_TEXTURE2D_ARRAY}
\item \cenum{CL_GL_OBJECT_TEXTURE1D}
\item \cenum{CL_GL_OBJECT_TEXTURE1D_ARRAY}
\item \cenum{CL_GL_OBJECT_TEXTURE_BUFFER}
\item \cenum{CL_GL_OBJECT_RENDERBUFFER}
\stopigBase
而如果 \carg{gl_object_type} 是 \cmacro{NULL},則將其忽略。

\carg{gl_object_name} 會返回用來創建 \carg{memobj} 的 GL 對象的名字。
如果 \carg{gl_object_name} 是 \cmacro{NULL},則將其忽略。

如果執行成功,則 \clapi{clGetGLObjectInfo} 會返回 \cenum{CL_SUCCESS}。
否則,返回下列錯誤碼之一:
\startigBase
\item \cenum{CL_INVALID_MEM_OBJECT},
如果 \carg{memobj} 無效。

\item \cenum{CL_INVALID_GL_OBJECT},
如果沒有與 \carg{memobj} 關聯的 GL 對象。

\item \cenum{CL_OUT_OF_RESOURCES},如果\scdevfailres。

\item \cenum{CL_OUT_OF_HOST_MEMORY},如果\schostfailres。
\stopigBase

% clGetGLTextureInfo
\topclfunc{clGetGLTextureInfo}

\startCLFUNC
cl_int clGetGLTextureInfo (
			cl_mem memobj,
			cl_gl_texture_info param_name,
			size_t param_value_size,
			void *param_value,
			size_t *param_value_size_ret)
\stopCLFUNC

此函式會返回與 \carg{memobj} 關聯的 GL 材質對象的附加資訊。

\carg{param_name} 指定要查詢什麼資訊。
\reftab{glTextureInfoType}中列出了 \clapi{clGetGLTextureInfo} 所支持的資訊類型
及 \carg{param_value} 中返回的資訊。

\carg{param_value} 所指內存用於存儲查詢結果。
如果 \carg{param_value} 是 \cmacro{NULL},則將其忽略。

\carg{param_value_size} 即 \carg{param_value} 所指內存的大小。
其值必須大於等於\reftab{glTextureInfoType}中所列返回型別的大小。

\carg{param_value_size_ret} 即拷貝到 \carg{param_value} 中數據的實際大小。
如果 \carg{param_value_size_ret} 是 \cmacro{NULL},則將其忽略。

\placetable[here][tab:glTextureInfoType]
{\clapi{clGetGLTextureInfo} 所支持的 \carg{param_name}}
{\startETD[cl_gl_texture_info][返回型別]

\clETD{CL_GL_TEXTURE_TARGET}{GLenum}{%
\clapi{clCreateFromGLTexture} 的引數 \carg{texture_target}。}

\clETD{CL_GL_MIPMAP_LEVEL}{GLint}{%
\clapi{clCreateFromGLTexture} 的引數 \carg{miplevel}。}

\stopETD

}

如果執行成功, \clapi{clGetGLTextureInfo} 會返回 \cenum{CL_SUCCESS}。
否則,返回下列錯誤碼之一:
\startigBase
\item \cenum{CL_INVALID_MEM_OBJECT},
如果 \carg{memobj} 無效。

\item \cenum{CL_INVALID_GL_OBJECT},
如果沒有與 \carg{memobj} 關聯的 GL 對象。

\item \cenum{CL_INVALID_VALUE},
如果 \carg{param_name} 無效;
或者 \carg{param_value_size} 的值 < \reftab{glTextureInfoType}中所列返回型別的大小,
並且 \carg{param_value} 不是 \cmacro{NULL};
或者 \carg{param_value} 和 \carg{param_value_size_ret} 都是 \cmacro{NULL}。

\item \cenum{CL_OUT_OF_RESOURCES},如果\scdevfailres。

\item \cenum{CL_OUT_OF_HOST_MEMORY},如果\schostfailres。
\stopigBase

