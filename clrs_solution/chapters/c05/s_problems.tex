\startsubject[
  title={Problems},
]

%p5-1
\startPROBLEM(Probabilstic counting)
利用一個 $b$ 位的計數器,
我們一般只能計數到 $2^b -1$。
而用 R.Morris 的\emph{概率記數法},
我們可以計數到一個大得多的值,
代價是精度有所損失。

對 $i=0,1,\ldots,2^b-1$,令計數器的值 $i$ 表示 $n_i$ 的計數,
其中 $n_i$ 構成了一個非負的遞增序列。
假設計數器初值爲 $0$,表示計數 $n_0 = 0$。
現有這樣一個計數器,其值爲 $i$,
INCREMENT 運算單元以概率的方式對其進行運算。
如果 $i = 2^b-1$,則該運算單元報告溢出錯誤;
否則,該運算單元以概率 $1/(n_{i+1}-n_i)$ 將計數器增加 $1$,
而以概率 $1-1/(n_{i+1}-n_i)$ 保持計數器不變。

對所有的 $i\ge 0$,若選擇 $n_i = i$,此計數器就退化爲一個普通的計數器。
若選擇 $n_i=2^{i-1}(i>0)$,或者 $n_i=F_i$(第 $i$ 個 Fibonacci 數,參見\insection[notationfunction]),
則會出現更多有趣的情形。

對於這個問題,假設 $n_{2^b-1}$ 已足夠大,發生溢出錯誤的概率可以忽略。

\startigBase[a]
\startitem
請說明在執行 $n$ 次 INCREMENT 運算後,計數器所表示的數期望值正好是 $n$;
\stopitem
\stopigBase

\startANSWER
假設第 $j$ 次 INCREMENT 前,計數器的值爲 $i$,表示 $n_i$。
如果計數器增加,那麼其值會增大 $n_{i+1} - n_i$,
其概率爲 $1/(n_{i+1} - n_i)$,增加值的期望爲:
\startsplitformula\startmathalignment
\NC E[X_j] \NC= 0 \cdot \Pr\{\text{stays same}\} + 1 \cdot \Pr\{\text{increases}\} \NR
\NC        \NC= 0 \cdot \left(1 - \frac{1}{n_{i+1} - n_i}\right) +
                1 \cdot \left((n_{i+1} - n_i) \cdot \frac{1}{n_{i+1} - n_i}\right) \NR
\NC        \NC= 1 \NR
\stopmathalignment\stopsplitformula
這是每次 INCREMENT 的期望值,由於有 $n$ 次 INCREMNT,其期望值爲 $n$。
\stopANSWER

\startigBase[continue]
\startitem
此計數器所表示的計數的方差依賴於 $n_i$ 序列。
我們來看一個簡單情形:對所有 $i\ge 0$, $n_i = 100i$。
在執行 $n$ 次 INCREMENT 運算後,請估計計數器所表示數的方差。
\stopitem
\stopigBase

\startANSWER
單次 INCREMENT 運算的協方差:
\startsplitformula\startmathalignment
\NC Var[X_j] \NC= E[X_j^2] - E^2[X_j] \NR
\NC           \NC= \left(0^2 \cdot \frac{99}{100} + 100^2 \frac{1}{100}\right) - 1 \NR
\NC           \NC= 99 \NR
\stopmathalignment\stopsplitformula

$n$ 次運算的協方差:
\startformula
Var[X] = Var[X_1 + X_2 + \ldots + X_n] = \sum_{i=1}^n Var[X_i] = 99n
\stopformula
\stopANSWER

\stopPROBLEM

%p5-2
\startPROBLEM(Searching an unsorted array)
本題將分析三個算法,他們在無序數列 $A$ 的 $n$ 個元素中搜索 $x$。

考慮如下隨機策略:隨機挑選 $A$ 中的一個下標 $i$。
如果 $A[i] = x$,則終止;否則,繼續挑選 $A$ 中的一個新的隨機下標。
重復隨機挑選下標,直到找到一個下標 $j$,使得 $A[j] = x$,
或者直到我們已檢查過 $A$ 中的所有元素。
注意,我們每次都是從全部下標集合中挑選,
因此可能會對某個元素檢查多次。
\startigBase[a]
\item 請寫出過程 \ALGO{RANDOM-SEARCH} 的僞碼以實現上述策略。
要確保當 $A$ 中所有下標都被挑選過後,算法應停止。
\stopigBase

\startANSWER
\CLRSH{RANDOM-SEARCH(x, A, n)}
\startCLRSCODE
v = \emptyset
while |v|\ne n
	i = \ALGO{RANDOM(1, n)}
	if A[i] = x
		return i
	else:
		v = v\cup i
return -1
\stopCLRSCODE
其中可以多種方式實現 $v$——如散列表、樹或位圖。位圖的時間、空間性能應當都是最好的。
\stopANSWER

\startigBase[continue]
\item 假定恰好有一個下標 $i$ 滿足 $A[i] = x$。
在我們找到 $x$,結束 \ALGO{RANDOM-SEARCH} 之前,
所檢查過的下標的數目期望值是多少?
\stopigBase

\startANSWER
參考 \simpleurl{http://en.wikipedia.org/wiki/Bernoulli_trial},期望值爲 $n$。
另 $N$ 爲代表搜索次數的隨即變量。則:
\startsplitformula\startmathalignment
\NC E[N] \NC = \sum_{i\ge 1}i \Pr\{i\} \NR\allowbreak
\NC \NC = \sum_{i\ge i} i \frac{n-1}{n}^{i-1} \frac{1}{n} \NR
\NC \NC = \frac{1}{n} \frac{1}{\left(1-\frac{n-1}{n}\right)^2} \NR
\NC \NC = n \NR
\stopmathalignment\stopsplitformula
\stopANSWER

\startigBase[continue]
\item 假設有 $k\ge 1$ 個下標 $i$ 使得 $A[i] = x$,
推廣你對上一項的解答。
在找到 $x$ 或 \ALGO{RANDOM-SEARCH} 結束之前,
必須挑選 $A$ 的下標的數目期望值是多少?
你的答案應該是 $n$ 和 $k$ 的函數。
\stopigBase

\startANSWER
答案爲 $n/k$。
\stopANSWER

\startigBase[continue]
\item 假設沒有下標 $i$ 使得 $A[i] = x$,
在檢查完 $A$ 的所有元素或 \ALGO{RANDOM-SEARCH} 結束之前,
我們必須挑選 $A$ 的下標的數目期望值是多少?
\stopigBase

\startANSWER
答案爲 $n(\ln{n} + O(1))$,參見節 5.4.2。
\stopANSWER

現在考慮一個確定性的線性查找算法,我們稱之爲 \ALGO{DETERMINISTIC-SEARCH}。
具體地說,這個算法在 $A$ 中順序查找 $x$,
考慮 $A[1]$、 $A[2]$、 $A[3]$、 $\ldots$、 $A[n]$,
直到找到 $A[i] = x$,或者到達數列的末尾。
假設輸入數列的所有排列都是等可能的。
\startigBase[continue]
\item 假設恰好有一個下標 $i$ 使得 $A[i] = x$。
\ALGO{DETERMINISTIC-SEARCH} 的平均運行時間是多少?
最壞運行時間又是多少?
\stopigBase

\startANSWER
最壞運行時間爲 $n$。平均運行時間爲 $(n+1)/2$。
\stopANSWER

\startigBase[continue]
\item 假設有 $k\ge 1$ 個下標 $i$ 使得 $A[i] = x$,推廣你對上一項的解答。
\ALGO{DETERMINISTIC-SEARCH} 的平均運行時間是多少?最壞運行時間又是多少?
你的答案應該是 $n$ 和 $k$ 的函數。
\stopigBase

\startANSWER
最壞運行時間爲 $n-k+1$。
平均運行時間爲 $(n+1)/(k+1)$。
令 $X_i$ 爲指示器隨機變量,代表第 $i$ 個元素匹配,則 $\Pr\{X_i\} = 1/(k+1)$。
令 $Y$ 爲指示器隨機變量,代表在前 $n-k+1$ 個元素中找到了匹配項($\Pr{\{Y\}}=1$)。
因此:
\startsplitformula\startmathalignment
\NC E[X] \NC= E[X_1 + X_2 + \ldots + X_{n-k} + Y] \NR
\NC      \NC= 1 + \sum_{i=1}^{n-k}E[X_i] \NR
\NC      \NC= 1 + \frac{n-k}{k+1} \NR
\NC      \NC= \frac{n+1}{k+1} \NR
\stopmathalignment\stopsplitformula
\stopANSWER

\startigBase[continue]
\item 假設沒有下標 $i$ 使得 $A[i] = x$,
\ALGO{DETERMINISTIC-SEARCH} 的平均運行時間是多少?最壞運行時間又是多少?
\stopigBase

\startANSWER
兩種情況都是 $n$。
\stopANSWER

最後,考慮一個隨機算法 \ALGO{SCRAMBLE-SEARCH},
他縣將輸入數列隨機變換排列,然後在變換後的數列上,運行上面的確定性線性查找算法。
\startigBase[continue]
\item 設 $k$ 是滿足 $A[i] = x$ 的下標的數目,
請給出在 $k=0$ 和 $k=1$ 兩種情況下,
算法 \ALGO{SCRAMBLE-SEARCH} 最壞情形的運行時間和運行時間期望值。
推廣你的解答以處理 $k\ge 1$ 的情況。
\stopigBase

\startANSWER
與 \ALGO{DETERMINISTIC-SEARCH} 一樣,只是將“平均情況”換成了“期望”。
\stopANSWER

\startigBase[continue]
\item 你將會使用 3 種查找算法中的哪一個?解釋你的答案。
\stopigBase

\startANSWER
當然是 \ALGO{DETERMINISTIC-SEARCH}。
雖然 \ALGO{SCRAMBLE-SEARCH} 可以有更好的期望值,
但是排列變換需要額外的時間,而這個時間是線性的操作。
在進行排列變換的時間李,我們可能已經掃描完了整個數列並得到了結果。
\stopANSWER

\stopPROBLEM

\stopsubject
