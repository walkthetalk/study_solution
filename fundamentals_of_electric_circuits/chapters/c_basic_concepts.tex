\startcomponent c_basic_concepts
\startchapter[
  title={Basic Concepts},
]

% 1.1
\startQ
How much charge is represented by these number of electrons?
\startIG
\item $\munit{6.482e17}$
\item $\munit{1.24=e18}$
\item $\munit{2.46=e19}$
\item $\munit{1.628e20}$
\stopIG
\stopQ

\startA
\startIG
\item $q = (\munit{1.602e-19 coulomb}) \times \munit{6.482e17} = \munit{_0.10384 coulomb}$
\item $q = (\munit{1.602e-19 coulomb}) \times \munit{1.24=e18} = \munit{_0.19865 coulomb}$
\item $q = (\munit{1.602e-19 coulomb}) \times \munit{2.46=e19} = \munit{_3.941== coulomb}$
\item $q = (\munit{1.602e-19 coulomb}) \times \munit{1.628e20} = \munit{26.08=== coulomb}$
\stopIG
\stopA

% 1.2
\startQ
Determine the current flowing through an element if the charge flow is given by
\startIG
\item $q(t) = (3t+8)               \munit[l]{milli coulomb}$
\item $q(t) = (8t^2 + 4t − 2)      \munit[l]{coulomb}$
\item $q(t) = (3e^{−t} − 5e^{−2t}) \munit[l]{nano coulomb}$
\item $q(t) = 10 \sin 120π t       \munit[l]{pico coulomb}$
\item $q(t) = 20e^{−4t} \cos 50t   \munit[l]{micro coulomb}$
\stopIG
\stopQ

\startA
\startIG
\item $i = \frac{d q}{d t} = 3                      \munit[l]{milli ampere}$
\item $i = \frac{d q}{d t} = (16t + 4)              \munit[l]{ampere}$
\item $i = \frac{d q}{d t} = (-3e^{−t} + 10e^{−2t}) \munit[l]{nano ampere}$
\item $i = \frac{d q}{d t} = 1200π \cos 120π t      \munit[l]{pico ampere}$
\item $i = \frac{d q}{d t} = -(80 \cos 50t + 1000 \sin 50 t) e^{−4t} \munit[l]{micro ampere}$
\stopIG
\stopA

% 1.3
\startQ
Find the charge $q(t)$ flowing through a device if the current is:
\startIG
\item $i(t) = \munit{3 ampere}, q(0) = \munit{1 coulomb}$
\item $i(t) = (2t + 5) \munit[l]{milli ampere}, q(0) = 0$
\item $i(t) = 20 \cos (10t + π / 6) \munit[l]{micro ampere}, q(0) = \munit{2 micro coulomb}$
\item $i(t) = 10e^{−30t} \sin 40t \munit[l]{ampere}, q(0) = 0$
\stopIG
\stopQ

\startA
\startIG
\item $q(t) = \int_0 i(t) + q(0) = (3t+1) \munit[l]{coulomb}$
\item $q(t) = \int_0 i(t) + q(0) = (t^2 + 5t) \munit[l]{milli coulomb}$
\item $q(t) = \int_0 i(t) + q(0) = (2 \sin (10t + π / 6) + 1) \munit[l]{micro coulomb}$
\item $q(t) = \int_0 i(t) + q(0) = (-e^{-30t}(0.16\cos 40t + 0.12\sin 40t) + 0.16) \munit[l]{coulomb}$
\stopIG
\stopA

% 1.4
\startQ
A total charge of $\munit{300 coulomb}$ flows past a given cross section of a conductor in 30 seconds.
What is the value of the current?
\stopQ

\startA
$\munit[r]{300 coulomb} / \munit[l]{30 second} = \munit{10 ampere}$
\stopA

% 1.5
\startQ
Determine the total charge transferred over the time interval
of $0 \le t \le \munit{10 second}$ when $i(t) = \frac{1}{2}t \munit[l]{ampere}$.
\stopQ

\startA
\startformula
q(t) = \int_0^{10}\frac{1}{2}t = \left.\frac{t^2}{4}\right|_0^{10} = 25
\stopformula
\stopA

% 1.6
\startQ[Q:1.6]
The charge entering a certain element is shown in \reffig{1.23}. Find the current at:
\startIG
\item $t = \munit{_1 milli second}$
\item $t = \munit{_6 milli second}$
\item $t = \munit{10 milli second}$
\stopIG
\placefigure[here][fig:1.23]{for \refq{1.6}}{
\startMPcode
	defcoord(10,2,15,15,1.5);
	defdim.x("t(ms)")(0,2,4,6,8,10,12);
	defdim.y("q(t)(mC)")(30);
	draw coordsys;
	drawemp((0,0)--(2,30)--(8,30)--(12,0));
	drawaux((2,0)--(2,30));
	drawaux((8,0)--(8,30));
\stopMPcode
}
\stopQ

\startA
\startIG
\item $i(\munit{t}) = \frac{dq}{dt} = \frac{d(15t)}{dt}= 15, i(1) = 15 \munit[l]{coulomb}$
\item $i(\munit{t}) = \frac{dq}{dt} = \frac{d(30)}{dt} = 0, i(6) = 0 \munit[l]{coulomb}$
\item $i(\munit{t}) = \frac{dq}{dt} = \frac{d(90-7.5t)}{dt} = -7.5, i(10) = -7.5 \munit[l]{coulomb}$
\stopIG
\stopA

% 1.7
\startQ[Q:1.7]
The charge flowing in a wire is plotted in \reffig{1.24}. Sketch the corresponding current.
\placefigure[here][fig:1.24]{for \refq{1.7}}{
\startMPcode
	defcoord(20,2,15,15,1.5);
	defdim.x("t(s)")(1,2,3,4);
	defdim.y("q(C)")(-10,0,10);
	draw coordsys;
	drawemp((0,0)--(1,10)--(2,-10)--(3,-10)--(4,0));
\stopMPcode
}
\stopQ

\startA
\startMPcode
	defcoord(20,2,15,15,1.5);
	defdim.x("t(s)")(1,2,3,4);
	setloc.x("lrt")(1,2);
	defdim.y("q(C)")(-20,-10,0,10);
	draw coordsys;
	drawemp((0,10)--(1,10)--(1,-20)--(2,-20)--(2,0)--(3,0)--(3,10)--(4,10));
\stopMPcode
\stopA

% 1.8
\startQ[Q:1.8]
 The current flowing past a point in a device is shown in \reffig{1.25}.
 Calculate the total charge through the point.
\placefigure[here][fig:1.25]{for \refq{1.8}}{
\startMPcode
	defcoord(30,3,15,15,1.5);
	defdim.x("t(ms)")(0,1,2);
	defdim.y("i(mA)")(10);
	draw coordsys;
	drawemp((0,0)--(1,10)--(2,10)--(2,0));
\stopMPcode
}
\stopQ

\startA
\startformula
q = \int i dt = \frac{10\times 1}{2} + 10\times 1 = \munit{15 milli coulomb}
\stopformula
\stopA

% 1.9
\startQ[Q:1.9]
The current through an element is shown in \reffig{1.26}.
Determine the total charge that passed through the element at:
\startcolumns[n=3]
\startIG
\item $t=\munit{1 second}$
\item $t=\munit{3 second}$
\item $t=\munit{5 second}$
\stopIG
\stopcolumns
\placefigure[here][fig:1.26]{for \refq{1.9}}{
\startMPcode
	defcoord(15,3,15,15,1.5);
	defdim.x("t(s)")(0,1,2,3,4,5);
	defdim.y("i(A)")(5,10);
	draw coordsys;
	drawemp((0,10)--(1,10)--(2,5)--(4,5)--(5,0));
\stopMPcode
}
\stopQ

\startA
\startIG
\item $q=\int_0^1 i dt = \int_0^1 10 dt = \munit{10 coulomb}$
\item $q=\int_0^3 i dt
        = \int_0^1 i dt + \int_1^2 i dt         + \int_2^3 i dt
        =\int_0^1 10 dt + \int_1^2 (15 - 5t) dt + \int_2^3 5 dt
        =10             + 7.5                   + 5
        =\munit{22.5 coulomb}$
\item $q=\int_0^5 i dt
        =\int_0^3 i dt + \int_3^4 i dt + \int_4^5 i dt
        = 22.5         + 5             + 2.5
        =\munit{30 coulomb}$
\stopIG
\stopA

% 1.10
\startQ
A lightning bolt with \munit{10 kilo ampere} strikes an object for \munit{15 micro second}.
How much charge is deposited on the object?
\stopQ

\startA
$q = (\munit{10e3 ampere}) \times (\munit{15e-6 second})
   = \munit{150e-3 coulomb} = \munit{0.15 coulomb}$
\stopA

% 1.11
\startQ
A rechargeable flashlight battery is capable of delivering
\munit{90 milli ampere} for about \munit{12 hour}.
How much charge can it release at that rate?
If its terminal voltage is \munit{1.5 volt},
how much energy can the battery deliver?
\stopQ
\startA
$q = (\munit{90e-3 ampere}) \times (12 \times 3600 \munit{hour})
   = \munit{3888 coulomb}$

$E = pt = ivt = qv = \munit{3888 coulomb} \times \munit{1.5 volt} = \munit{5832 joule}$
\stopA

% 1.12
\startQ
If the current flowing through an element is given by
\startformula
i(t)=\startmathcases
\NC 3t \munit[l]{ampere}, \MC 0 \le t < \munit{6 second} \NR
\NC \munit{18 ampere},    \MC 6 \le t < \munit{10 second} \NR
\NC \munit{-12 ampere},   \MC 10\le t < \munit{15 second} \NR
\NC 0, \MC t \ge \munit{15 second} \NR
\stopmathcases
\stopformula
Plot the charge stored in the element over $0 < t < \munit{20 second}$.

\startMPcode
	defcoord(5,2,15,15,1.5);
	defdim.x("(t) s")(6,10,15);
	setloc.x("lrt")(10,15);
	defdim.y("i(t) A")(-12,0,18);
	draw coordsys;
	drawemp((0,0)--(6,18)--(10,18)--(10,-12)--(15,-12)--(15,0)--(17,0));
\stopMPcode
\stopQ

\startA
\startformula
q(t)=\startmathcases
\NC 1.5t^2 \munit[l]{coulomb},  \MC 0 \le t < \munit{6 second} \NR
\NC 18t-54 \munit[l]{ampere},   \MC 6 \le t < \munit{10 second} \NR
\NC -12t+246 \munit[l]{ampere}, \MC 10\le t < \munit{15 second} \NR
\NC 66 \munit[l]{ampere},       \MC t \ge \munit{15 second} \NR
\stopmathcases
\stopformula

\startMPcode
	defcoord(7,0.7,15,15,1.5);
	defdim.x("(t) s")(0,6,10,15);
	setloc.x("lrt")(10,15);
	defdim.y("q(t) C")(18,54,66,126);
	draw coordsys;

	polydef.main(0,6,1,20)(0,0,1.5);
	drawauxp.p((6,54),(10,126),(15,66));
	drawemp(polycurve.main--(10,126)--(15,66)--(16,66));
\stopMPcode
\stopA

% 1.13
\startQ
The charge entering the positive terminal of an element is
$q = 5 \sin 4 π t \sunit[l]{milli coulomb}$,
while the voltage across the element (plus to minus) is
$v = 3 \cos 4 π t \sunit[l]{volt}$,
\startIG
\item Find the power delivered to the element at $t = \sunit{0.3 second}$.
\item Calculate the energy delivered to the element between 0 and \sunit{0.6 second}.
\stopIG
\stopQ

\startA
\startIG
\item $i = \frac{dq}{dt} = 20π\cos 4π t \sunit[l]{milli ampere}$,
$p = vi = 60π\cos^2 4π t \sunit[l]{milli watt}$,
at $t = \sunit{0.3 second}$:
\startformula
p(0.3) = 60π\cos^2 (4π \times 0.3) = 123.37 \sunit[l]{milli watt}
\stopformula

\item $E = \int p d t = \int 60π\cos^2 4π t dt = 60π\int \cos^2 4π t dt
	 = 60π \int (1 + \cos 8π t) dt \sunit[l]{milli joule}$
\startformula
E = 60π (\left.t + \frac{\sin 8π t}{8π}\right|_0^{0.6}) = 117.5 \sunit[l]{milli joule}
\stopformula
\stopIG
\stopA

% 1.14
\startQ
The voltage v(t) across a device and the current i(t) through it are
$v(t) = 20 \sin (4t) \sunit[l]{volt}$ and $i(t) = 10(1 + e^{−2t}) \sunit[l]{milli ampere}$,
Calculate:
\startIG
\item the total charge in the device at $t = \sunit{1 second}$, $q(0) = 0$.
\item the power consumed by the device at $t = \sunit{1 second}$.
\stopIG
\stopQ

\startA
\startIG
\item $q = \int i(t) dt
         = \int 10(1+e^{-2t}) dt
         = \left.(10t - 5 e^{-2t})\right|_0^1
         = 5-5e^{-2}
         = 4.323 \sunit{milli ampere}$
\item $p = v(t)i(t)
         = 20\sin(4t) \times 10 (1 + e^{-2t})
         = 200\sin(4t) \times (1 + e^{-2t})$, $p(1) = -171.845 \sunit[l]{milli watt}$
\stopIG
\stopA

% 1.15
\startQ
The current entering the positive terminal of a device is
$i(t) = 6e^{−2t} \sunit[l]{milli ampere}$
and the voltage across the device is
$v(t) = 10\frac{di}{dt}\sunit[l]{volt}$.
\startIG
\item Find the charge delivered to the device between
$t = 0$ and $t = \sunit{2 second}$.
\item Calculate the power absorbed.
\item Determine the energy absorbed in \sunit{3 second}.
\stopIG
\stopQ

\startA
\startIG
\item $q = \int i(t) dt
         = \left.-3e^{-2t}\right|_0^2
         = 3 - 3 e^{-4}
         = 2.945 \sunit[l]{milli coulomb}$
\item $v(t) = 10\frac{di}{dt} = -120e^{-2t}\sunit[l]{volt}$,
$p = v(t) i(t) = -720e^{-4t} \sunit[l]{milli watt}$
\item $E = \int p dt = -0.99889 \sunit[l]{milli joule}$
\stopIG
\stopA

% 1.16
\startQ[Q:1.16]
\reffig{1.27} shows the current through and the voltage across an element.
\startIG
\item Sketch the power delivered to the element for $t>0$.
\item Fnd the total energy absorbed by the element for the period of $0 < t < \sunit{4 second}$.
\stopIG
\placefigure[here][fig:1.27]{for \refq{1.16}}{
\startcombination[2*1]
{\startMPcode
	defcoord(20,1,15,15,1.5);
	defdim.x("t(s)")(0,2,4);
	defdim.y("i(mA)")(60);
	draw coordsys;

	drawemp((0,0)--(2,60)--(4,0));
\stopMPcode}{a}{\startMPcode
	defcoord(20,5,15,15,1.5);
	defdim.x("t(s)")(2,4);
	setloc.x("lrt")(2,4);
	defdim.y("v(V)")(-5,0,5);
	draw coordsys;

	drawemp((0,5)--(2,5)--(2,-5)--(4,-5)--(4,0));
\stopMPcode}{b}
\stopcombination
}
\stopQ

\startA
\startformula\startmathalignment[align={left}]
\NC
  i(t) = \startmathcases
         \NC 30t \sunit[l]{milli ampere} \MC 0 < t < 2 \NR
         \NC 120-30t \sunit[l]{milli ampere} \MC 2 < t < 4 \NR
         \stopmathcases
\NC
  v(t) = \startmathcases
         \NC 5 \sunit[l]{volt} \MC 0 < t < 2 \NR
         \NC -5\sunit[l]{volt} \MC 2 < t < 4 \NR
         \stopmathcases
\NR

\NC
  p(t) = \startmathcases
         \NC 150t    \sunit[l]{milli watt} \MC 0 < t < 2 \NR
         \NC 150t-600\sunit[l]{milli watt} \MC 2 < t < 4 \NR
         \stopmathcases
\NC
  E = \int p(t) dt = \startmathmatrix[left={\left(},right={\right)}]
                     \NC \left.75t^2\right|_0^2 \NR
                     \NC + \NR
                     \NC \left.75t^2-600t\right|_2^4 \NR
                     \stopmathmatrix
                   = \sunit{0 joule}
\NR
\stopmathalignment\stopformula

\startMPcode
	defcoord(20,0.1,15,15,1.5);
	defdim.x("t(s)")(2,4);
	setloc.x("lrt")(2);
	defdim.y("p(mW)")(-300,0,300);
	draw coordsys;

	drawemp((0,0)--(2,300)--(2,-300)--(4,0));
\stopMPcode
\stopA

% 1.17
\startQ[Q:1.17]
\reffig{1.28} shows a circuit with four elements,
$p_1 = \sunit{60 watt}$ absorbed, $p_3 = -145\sunit[l]{watt}$ absorbed,
and $p_4 = \sunit{75 watt}$ absorbed.
How many watts does element $2$ absorb?
\placefigure[here][fig:1.28]{for \refq{1.17}}{
\startMPcode
	clearObjsForTwiceRun;
	setX("fig128");

	newR.X(r1) "angle(90)";
	ObjLabel.X(r1)(textext("1")) "labcolor(FireBrick)";

	newR.X(r2) "angle(90)";
	ObjLabel.X(r2)(textext("2")) "labcolor(FireBrick)";

	newR.X(r4) "angle(90)";
	ObjLabel.X(r4)(textext("4")) "labcolor(FireBrick)";

	newR.X(r3) "angle( 0)";
	ObjLabel.X(r3)(textext("3")) "labcolor(FireBrick)";

	numeric uc;
	uc := 10;
	X(r2).c = origin;
	X(r1).c = X(r2).c+(-6uc,0);
	X(r4).c = X(r2).c+( 6uc,0);
	X(r3).c = (X(r2).c+X(r4).c)/2 + (0,3uc);

	drawObj(X(r1),X(r2),X(r3),X(r4));

	string armL; armL:= "armA(ypart(X(r3).c-X(r2).ie))";
	pcsimple(X(r3))(2)(X(r4))(2);
	pcsimple(X(r2))(2)(X(r3))(1);
	pcsimple(X(r2))(2)(X(r1))(2) armL;
	pcsimple(X(r2))(1)(X(r4))(1) armL;
	pcsimple(X(r2))(1)(X(r1))(1) armL;
\stopMPcode
}
\stopQ

\startA
because:
\startformula
\sum p = p_1 + p_2 + p_3 + p_4 = 0
\stopformula
so:
\startformula
p_2 = 0 - p_1 - p_3 - p_4 = 0 - 60 - (-145) - 75 = 10 \sunit[l]{watt}
\stopformula
\stopA

% 1.18
\startQ[Q:1.18]
Find the power absorbed by each of the elements in \reffig{1.29}.

\placefigure[here][fig:1.29]{for \refq{1.18}}{
\startMPcode
	clearObjsForTwiceRun;
	setX("fig129");

	newIVS.X(u1) "angle(90)";
	ObjMLabel.X(u1)("p_1")
		"labpoint(is)", "labdir(rt)";
	ObjMLabel.X(u1)("30V")
		"labpoint(in)", "labdir(lft)";

	newR.X(r2);
	ObjMLabel.X(r2)("p_2")
		"labpoint(is)", "labdir(bot)";
	ObjMLabel.X(r2)("+")
		"labpoint(inw)", "labdir(top)";
	ObjMLabel.X(r2)("-")
		"labpoint(ine)", "labdir(top)";
	ObjMLabel.X(r2)("10V")
		"labpoint(in)", "labdir(top)", "labshift((0,10))";

	newR.X(r3) "angle(90)";
	ObjMLabel.X(r3)("p_3")
		"labpoint(is)", "labdir(rt)";
	ObjMLabel.X(r3)("+")
		"labpoint(ine)", "labdir(lft)";
	ObjMLabel.X(r3)("-")
		"labpoint(inw)", "labdir(lft)";
	ObjMLabel.X(r3)("20V")
		"labpoint(in)", "labdir(lft)";

	newR.X(r4);
	ObjMLabel.X(r4)("p_4")
		"labpoint(is)", "labdir(bot)";
	ObjMLabel.X(r4)("+")
		"labpoint(inw)", "labdir(top)";
	ObjMLabel.X(r4)("-")
		"labpoint(ine)", "labdir(top)";
	ObjMLabel.X(r4)("8V")
		"labpoint(in)", "labdir(top)", "labshift((0,10))";

	newDCS.X(u2) "angle(90)";
	ObjMLabel.X(u2)("p_5")
		"labpoint(is)", "labdir(urt)";
	ObjMLabel.X(u2)("0.4I")
		"labpoint(is)", "labdir(lrt)";
	ObjMLabel.X(u2)("12V")
		"labpoint(in)", "labdir(lft)";
	ObjMLabel.X(u2)("+")
		"labpoint(ie)", "labdir(ulft)";
	ObjMLabel.X(u2)("-")
		"labpoint(iw)", "labdir(llft)";

	newCDM.X(cdm0) "angle(180)";
	ObjMLabel.X(cdm0)("4A")
		"labpoint(is)", "labdir(top)";
	newCDM.X(cdm1);
	ObjMLabel.X(cdm1)("I=10A")
		"labpoint(in)", "labdir(top)";
	newCDM.X(cdm2) "angle(-90)";
	ObjMLabel.X(cdm2)("14A")
		"labpoint(in)", "labdir(rt)";

	numeric uc;
	uc := 10;
	X(r3).c = origin;
	X(u1).c = X(r3).c+(-9uc,0);
	X(u2).c = X(r3).c+( 9uc,0);
	X(r2).c = X(r3).c+(-3uc,5uc);
	X(r4).c = X(r3).c+( 3uc,5uc);

	X(cdm0).nnw = (xpart(X(u2).c), ypart(X(r4).c));
	X(cdm1).ssw = (xpart(X(u1).c), ypart(X(r2).c));
	X(cdm2).se = X(r3).ie;

	pcsimple(X(u2))(2)(X(r4))(2);
	pcsimple(X(r3))(1)(X(u2))(1) "armA(ypart(X(r4).iw-X(r3).ie))";
	pcsimple(X(r3))(1)(X(u1))(1) "armA(ypart(X(r4).iw-X(r3).ie))";
	pcsimple(X(r3))(2)(X(r4))(1);
	pcsimple(X(r3))(2)(X(r2))(2);
	pcsimple(X(u1))(2)(X(r2))(1);

	drawObj(X(u1),X(r2),X(r3),X(r4),X(u2));
	drawObj(X(cdm0),X(cdm1),X(cdm2));
\stopMPcode
}
\stopQ

\startA
\startformula\startmathalignment
\NC p_1 \NC = \sunit{30 volt} \times (-10)\sunit[l]{ampere} = -300 \sunit[l]{watt} \NR
\NC p_2 \NC = \sunit{10 volt} \times 10\sunit[l]{ampere} = 100 \sunit[l]{watt} \NR
\NC p_3 \NC = \sunit{20 volt} \times 14\sunit[l]{ampere} = 280 \sunit[l]{watt} \NR
\NC p_4 \NC = \sunit{8 volt} \times (-4)\sunit[l]{ampere} = -32 \sunit[l]{watt} \NR
\NC p_5 \NC = \sunit{12 volt} \times (-4)\sunit[l]{ampere} = -48 \sunit[l]{watt} \NR
\stopmathalignment\stopformula
\stopA

% 1.19
\startQ[Q:1.19]
Find $I$ and the power absorbed by each element in the network of \reffig{1.30}.
\placefigure[here][fig:1.30]{for \refq{1.19}}{
\startMPcode
	clearObjsForTwiceRun;
	setX("fig130");

	newR.X(r0) "angle(90)";
	ObjMLabel.X(r0)("-")
		"labpoint(ise)", "labdir(rt)";
	ObjMLabel.X(r0)("+")
		"labpoint(isw)", "labdir(rt)";
	ObjMLabel.X(r0)("15V")
		"labpoint(is)", "labdir(rt)";
	ObjMLabel.X(r0)("p_2")
		"labpoint(ine)", "labdir(ulft)";

	newR.X(r1) "angle(90)";
	ObjMLabel.X(r1)("-")
		"labpoint(ise)", "labdir(rt)";
	ObjMLabel.X(r1)("+")
		"labpoint(isw)", "labdir(rt)";
	ObjMLabel.X(r1)("9V")
		"labpoint(is)", "labdir(rt)";
	ObjMLabel.X(r1)("p_3")
		"labpoint(ine)", "labdir(lft)";

	newIVS.X(u2) "angle(-90)";
	ObjMLabel.X(u2)("6V")
		"labpoint(in)", "labdir(rt)";
	ObjMLabel.X(u2)("p_4")
		"labpoint(ise)", "labdir(llft)";

	newICS.X(u1) "angle(-90)";
	ObjMLabel.X(u1)("15V")
		"labpoint(in)", "labdir(rt)";
	ObjMLabel.X(u1)("10A")
		"labpoint(is)", "labdir(lft)";
	ObjMLabel.X(u1)("-")
		"labpoint(inw)", "labdir(rt)";
	ObjMLabel.X(u1)("+")
		"labpoint(ine)", "labdir(rt)";
	ObjMLabel.X(u1)("p_1")
		"labpoint(isw)", "labdir(ulft)";

	newCDM.X(cdm0) "angle(-90)";
	ObjMLabel.X(cdm0)("I")
		"labpoint(in)", "labdir(rt)";
	newCDM.X(cdm1) "angle(90)";
	ObjMLabel.X(cdm1)("4A")
		"labpoint(is)", "labdir(rt)";

	numeric uc;
	uc := 10;
	X(r0).c = origin;
	X(r1).c = X(r0).c + (5uc,2uc);
	X(u2).c = X(r0).c + (5uc,-2uc);
	X(u1).c = X(r0).c + (-5uc,0);
	X(cdm0).ssw = (xpart X(r1).c, ypart X(r0).ie + 5uc);
	X(cdm1).ne  = X(r0).ie + (0,5uc);

	drawObj(X(r0),X(r1),X(u1),X(u2));
	pcsimple(X(r0))(2)(X(r1))(2) "armA(5uc)";
	pcsimple(X(r0))(2)(X(u1))(1) "armA(5uc)";
	pcsimple(X(r0))(1)(X(u1))(2) "armA(3uc)";
	pcsimple(X(r0))(1)(X(u2))(2) "armA(3uc)";
	pcsimple(X(r1))(1)(X(u2))(1);
	drawObj(X(cdm0),X(cdm1));
\stopMPcode
}
\stopQ

\startA
because $\sunit{10 ampere} + I = \sunit{4 ampere}$, so $I = -6\sunit[l]{ampere}$.

\startformula\startmathalignment
\NC p_1 \NC = \sunit{15 volt} \times (-10)\sunit[l]{ampere} = -150 \sunit[l]{watt} \NR
\NC p_2 \NC = \sunit{15 volt} \times 4\sunit[l]{ampere} = 60 \sunit[l]{watt} \NR
\NC p_3 \NC = \sunit{9 volt} \times (-6)\sunit[l]{ampere} = 54 \sunit[l]{watt} \NR
\NC p_4 \NC = \sunit{6 volt} \times (6)\sunit[l]{ampere} = 36 \sunit[l]{watt} \NR
\stopmathalignment\stopformula
\stopA

% 1.20
\startQ[Q:1.20]
Find $V_0$ and the power absorbed by each element in the circuit of \reffig{1.31}.
\placefigure[here][fig:1.31]{for \refq{1.20}}{
\startMPcode
	clearObjsForTwiceRun;
	setX("fig131");

	newR.X(r0);
	ObjMLabel.X(r0)("-")
		"labpoint(ine)", "labdir(ulft)";
	ObjMLabel.X(r0)("+")
		"labpoint(inw)", "labdir(urt)";
	ObjMLabel.X(r0)("12V")
		"labpoint(n)", "labdir(top)", "labshift((0,10))";
	ObjMLabel.X(r0)("p_2")
		"labpoint(is)", "labdir(bot)";

	newR.X(r1) "angle(90)";
	ObjMLabel.X(r1)("+")
		"labpoint(ine)", "labdir(lft)";
	ObjMLabel.X(r1)("-")
		"labpoint(inw)", "labdir(lft)";
	ObjMLabel.X(r1)("V_0")
		"labpoint(n)", "labdir(lft)";
	ObjMLabel.X(r1)("p_3")
		"labpoint(is)", "labdir(rt)";

	newR.X(r2);
	ObjMLabel.X(r2)("-")
		"labpoint(ise)", "labdir(llft)";
	ObjMLabel.X(r2)("+")
		"labpoint(isw)", "labdir(lrt)";
	ObjMLabel.X(r2)("28V")
		"labpoint(is)", "labdir(bot)", "labshift((0,-10))";
	ObjMLabel.X(r2)("p_4")
		"labpoint(ie)", "labdir(urt)";

	newR.X(r3);
	ObjMLabel.X(r3)("-")
		"labpoint(ise)", "labdir(llft)";
	ObjMLabel.X(r3)("+")
		"labpoint(isw)", "labdir(lrt)";
	ObjMLabel.X(r3)("28V")
		"labpoint(is)", "labdir(bot)", "labshift((0,-10))";
	ObjMLabel.X(r3)("p_5")
		"labpoint(ie)", "labdir(urt)";

	newDVS.X(u2) "angle(-90)";
	ObjMLabel.X(u2)("5I_0")
		"labpoint(in)", "labdir(rt)";
	ObjMLabel.X(u2)("p_6")
		"labpoint(is)", "labdir(lft)";

	newIVS.X(u1) "angle(90)";
	ObjMLabel.X(u1)("30V")
		"labpoint(in)", "labdir(lft)";
	ObjMLabel.X(u1)("p_1")
		"labpoint(is)", "labdir(rt)";

	newCDM.X(cdm0) "angle(90)";
	ObjMLabel.X(cdm0)("6A")
		"labpoint(in)", "labdir(lft)";
	newCDM.X(cdm1);
	ObjMLabel.X(cdm1)("6A")
		"labpoint(in)", "labdir(top)";
	newCDM.X(cdm2) "angle(-90)";
	ObjMLabel.X(cdm2)("3A")
		"labpoint(is)", "labdir(lft)";
	newCDM.X(cdm3);
	ObjMLabel.X(cdm3)("1A")
		"labpoint(in)", "labdir(top)";
	newCDM.X(cdm4);
	ObjMLabel.X(cdm4)("I_0=2A")
		"labpoint(in)", "labdir(top)";
	newCDM.X(cdm5) "angle(-90)";
	ObjMLabel.X(cdm5)("3A")
		"labpoint(in)", "labdir(rt)";

	numeric uc;
	uc := 10;
	X(r1).c = origin;
	X(r0).c = X(r1).c + (-5uc, 4uc);
	X(r2).c = X(r1).c + ( 5uc, 4uc);
	X(r3).c = X(r2).c + (   0, 5uc);
	X(u1).c = X(r1).c + (-10uc,0);
	X(u2).c = X(r1).c + ( 10uc,0);
	X(cdm0).ssw = (xpart X(u1).c, ypart X(r1).c - 4uc);
	X(cdm1).ssw = (xpart X(u1).c, ypart X(r1).c + 4uc);
	X(cdm2).ne  = X(r1).e;
	X(cdm3).sw  = (xpart X(r1).c, ypart X(r2).c);
	X(cdm4).sw  = (xpart X(r1).c, ypart X(r3).c);
	X(cdm5).sse = (xpart X(u2).c, ypart X(r1).c - 4uc);

	drawObj(X(r0),X(r1),X(r2),X(r3),X(u1),X(u2));

	save _sl; string _sl; _sl := "armA(4uc - ypart(X(r1).c - X(r1).iw))";
	pcsimple(X(u1))(2)(X(r0))(1);
	pcsimple(X(r0))(2)(X(r1))(2);
	pcsimple(X(r1))(2)(X(r2))(1);
	pcsimple(X(r1))(2)(X(r3))(1);
	pcsimple(X(r2))(2)(X(u2))(1);
	pcsimple(X(r3))(2)(X(u2))(1);
	pcsimple(X(r1))(1)(X(u2))(2) _sl;
	pcsimple(X(r1))(1)(X(u1))(1) _sl;
	drawObj(X(cdm0), X(cdm1), X(cdm2), X(cdm3), X(cdm4), X(cdm5));
\stopMPcode
}
\stopQ

\startA
because $-30\sunit[l]{volt} + 12\sunit[l]{volt} + V_0 = 0\sunit[l]{volt}$, so $V_0 = 18\sunit[l]{volt}$.

\startformula\startmathalignment
\NC p_1 \NC = \sunit{30 volt} \times (-6)\sunit[l]{ampere} = -180 \sunit[l]{watt} \NR
\NC p_2 \NC = \sunit{12 volt} \times 6\sunit[l]{ampere} = 72 \sunit[l]{watt} \NR
\NC p_3 \NC = \sunit{18 volt} \times 3\sunit[l]{ampere} = 54 \sunit[l]{watt} \NR
\NC p_4 \NC = \sunit{28 volt} \times 1\sunit[l]{ampere} = 28 \sunit[l]{watt} \NR
\NC p_5 \NC = \sunit{28 volt} \times 2\sunit[l]{ampere} = 56 \sunit[l]{watt} \NR
\NC p_6 \NC = \sunit{10 volt} \times (-3)\sunit[l]{ampere} = -30 \sunit[l]{watt} \NR
\stopmathalignment\stopformula
\stopA

% 1.21
\startQ
A \sunit{60 watt} incandescent bulb operates at \sunit{120 volt}.
How many electrons and coulombs flow through the bulb in one day?
\stopQ

\startA
\startformula\startmathalignment
\NC p \NC = v i \rightarrow i = \frac{p}{v} = \frac{60}{120} = 0.5 \sunit[l]{ampere} \NR
\NC q \NC = i t = 0.5 \times 60 \times 60 \times 24 = 43200 \sunit[l]{coulomb} \NR
\NC N_e \NC = \sunit{6.24e18} \times 43200 = \sunit{2.696e23} \NR
\stopmathalignment\stopformula
\stopA

% 1.22
\startQ
A lightning bolt strikes an airplane with \sunit{40 kilo ampere} for \sunit{1.7 milli second}.
How many coulombs of charge are deposited on the plane?
\stopQ

\startA
$q=it=\sunit{40e3} \times \sunit{1.7e-3} = 68 \sunit[l]{coulomb}$
\stopA

% 1.23
\startQ
A \sunit{1.8 kilo watt} electric heater takes \sunit{15 minute} to boil a quantity of water.
If this is done once a day and power costs \sunit{10 cent per kilo watt hour},
what is the cost of its operation for 30 days?
\stopQ

\startA
\startformula
1.8 \times \frac{15}{60} \times 30 \times \frac{10}{100} = \$1.35
\stopformula
\stopA

% 1.24
\startQ
A utility company charges 8.2 cents/kWh.
If a consumer operates a 60-W light bulb continuously for one day,
how much is the consumer charged?
\stopQ

\startA
\startformula
\sunit{60e-3} \times 24 \times 8.2 = 11.808 \sunit[l]{cent}
\stopformula
\stopA

% 1.25
\startQ
A \sunit{1.2 kilo watt} toaster takes roughly \sunit{4 minute} to heat four slices of bread.
Find the cost of operating the toaster twice per day for 2 weeks (14 days).
Assume energy costs 9 cents/kWh.
\stopQ

\startA
\startformula
1.2 \times \frac{4}{60} \times 2 \times 14 \times 9 = 20.16 \sunit[l]{cent}
\stopformula
\stopA

% 1.26
\startQ
A cell phone battery is rated at \sunit{3.85 volt} and can store 10.78 watt-hours of energy.
\startIG
\item How much average current can it deliver over a period of 3 hours
if it is fully discharged at the end of that time?
\item How much average power is delivered in part (a)?
\item What is the ampere-hour rating of the battery?
\stopIG
\stopQ

\startA
\startIG
\item $10.78 / 3 / 3.85 = \frac{14}{15} = 0.93 \sunit[l]{ampere}$
\item $10.78 / 3 = 3.59 \sunit[l]{watt}$
\item $10.78 / 3.85 = 2.8 \sunit[l]{ampere hour}$
\stopIG
\stopA

% 1.27
\startQ
A constant current of \sunit{3 ampere} for 4 hours is required to charge an automotive battery.
If the terminal voltage is $10 + \frac{t}{2} \sunit[l]{volt}$, where $t$ is in hours,
\startIG
\item how much charge is transported as a result of the charging?
\item how much energy is expended?
\item how much does the charging cost? Assume electricity costs 9 cents/kWh.
\stopIG
\stopQ

\startA
\startIG
\startitem
Let $T = 4 \sunit[l]{hour} = 4 \times 3600 \sunit[l]{second}$,
\startformula
q = \int i dt = \int_0^T 3 dt = 3 \times 4 \times 3600 = 43.2 \sunit[l]{kilo coulomb}
\stopformula
\stopitem

\startitem
\startformula
W = \int p dt = \int v i dt
  = \int_0^T 3 \times (10 + \frac{t/3600}{2}) dt
  = \left.(30t+\frac{t^2}{4800})\right|_0^{4\times 3600}
  = 475.2 \sunit[l]{kilo joule}
\stopformula
\stopitem

\startitem
$\sunit{joule} = \sunit[l]{watt second}$, so
\startformula
\frac{475.2}{3600} \sunit[l]{kilo watt hour} \times 9 \sunit[l]{cent} = 1.188 \sunit[l]{cent}
\stopformula
\stopitem
\stopIG
\stopA

% 1.28
\startQ
A \sunit{150 watt} incandescent outdoor lamp is connected to
a \sunit{120 volt} source and is left burning continuously for
an average of 12 hours per day. Determine:
\startIG
\item the current through the lamp when it is lit.
\item the cost of operating the light for one non-leap year
if electricity costs 9.5 cents per \sunit{kilo watt hour}.
\stopIG
\stopQ

\startA
\startIG
\item $150 / 120 = 1.25 \sunit[l]{ampere}$
\item $150 / 1000 \times 12 \time 365 \times 9.5 / 100 = \$62.415$
\stopIG
\stopA

% 1.29
\startQ
An electric stove with four burners and an oven is
used in preparing a meal as follows.
\startIG
\item Burner 1: 20 minutes
\item Burner 2: 40 minutes
\item Burner 3: 15 minutes
\item Burner 4: 45 minutes
\item Oven: 30 minutes
\stopIG
If each burner is rated at \sunit{1.2 kilo watt} and the oven at \sunit{1.8 kilo watt},
and electricity costs 12 cents per \sunit{kilo watt hour},
calculate the cost of electricity used in preparing the meal.
\stopQ

\startA
\startformula\startmathalignment
\NC w \NC = pt = 1.2 \sunit[l]{kilo watt} \times \frac{20 + 40 + 15 + 45}{60} \sunit[l]{hour}
     + 1.8\sunit[l]{kilo watt} \times \frac{30}{60}\sunit[l]{hour}
     = 3.3 \sunit[l]{kilo watt hour} \NR
\NC \text{Cost} \NC = 12 \times 3.3 = 39.6 \text{cents} \NR
\stopmathalignment\stopformula
\stopA

% 1.30
\startQ
Reliant Energy (the electric company in Houston, Texas) charges customers as follows:
\startIG
\item Monthly charge \$6
\item First 250 kWh @ \$0.02/kWh
\item All additional kWh @ \$0.07/kWh
\stopIG
If a customer uses 2,436 kWh in one month,
how much will Reliant Energy charge?
\stopQ

\startA
\startformula
6 + 0.02 \times 250 + 0.07 \times (2436-250) = \$164.02
\stopformula
\stopA

% 1.31
\startQ
In a household, a business is run for an average of 6 h/day.
The total power consumed by the computer and its printer is \sunit{230 watt}.
In addition, a 75-W light runs during the same 6 h.
If their utility charges 11.75 cents per kWh,
how much do the owners pay every 30 days?
\stopQ

\startA
\startformula\startmathalignment
\NC p \NC = \frac{230 + 75}{1000}\sunit[l]{kilo watt} = 0.305\sunit[l]{kilo watt} \NR
\NC \text{Cost} \NC = 11.75 \times (0.305 \times 6 \times 30) = 645.075 \text{cents}
\stopmathalignment\stopformula
\stopA

% 1.32
\startQ
A telephone wire has a current of \sunit{20 micro ampere} flowing through it.
How long does it take for a charge of \sunit{15 coulomb} to pass through the wire?
\stopQ

\startA
\startformula
\frac{15}{\sunit{20e-6}} = \sunit{7.5e5 second}
\stopformula
\stopA

% 1.33
\startQ
A lightning bolt carried a current of \sunit{2 kilo ampere} and lasted for \sunit{3 milli second}.
How many coulombs of charge were contained in the lightning bolt?
\stopQ

\startA
\startformula
i = \frac{d q}{d t} \rightarrow q = \int i dt = 2000 \times 3 \times 10^{-3} = \sunit{6 coulomb}
\stopformula
\stopA

% 1.34
\startQ[Q:1.34]
\reffig{1.32} shows the power consumption of a certain household in 1 day.
Calculate:
\startIG
\item the total energy consumed in \sunit{kilo watt hour},
\item the average power over the total 24 hour period.
\stopIG
\placefigure[here][fig:1.32]{for \refq{1.34}}{
\startMPcode
	defcoord(8,0.05,15,15,1.5);
	defdim.x("t(h)")(0,2,4,6,8,10,12,14,16,18,20,22,24);
	defdim.y("p")(200,800,1200);
	draw coordsys;
	drawemp((0,200)--(6,200)--(6,800)--(8,800)--(8,200)
		--(18,200)--(18,1200)--(22,1200)--(22,200)--(24,200));
	coordmlabel.top("800 W", (7, 800));
	coordmlabel.top("200 W", (13, 200));
	coordmlabel.top("1200 W", (20, 1200));
\stopMPcode
}
\stopQ

\startA
\startIG
\item $\frac{200}{1000} \times (6 + 10 + 2)
  + \frac{800}{1000} \times 2
  + \frac{1200}{1000} \times 4
  = 10 \sunit[l]{kilo watt hour}$
\item $10 \times 1000 / 24 = 416.7 \sunit[l]{watt}$
\stopIG
\stopA

% 1.35
\startQ[Q:1.35]
The graph in \reffig{1.33} represents the power drawn
by an industrial plant between 8:00 and 8:30 a.m.
Calculate the total energy in MWh consumed by the plant.
\placefigure[here][fig:1.33]{for \refq{1.35}}{
\startMPcode
	defcoord(4.5,8,15,15,1.5);
	defdim.x("t")(0,5,10,15,20,25,30);
	defdim.y("p(MW)")(3,4,5,8);
	setlabel.x("8.00")( 0);
	setlabel.x("8.05")( 5);
	setlabel.x("8.10")(10);
	setlabel.x("8.15")(15);
	setlabel.x("8.20")(20);
	setlabel.x("8.25")(25);
	setlabel.x("8.30")(30);
	draw coordsys;
	drawemp((0,5)--(5,5)--(5,4)--(10,4)--(10,3)--(15,3)--(15,8)--(20,8)--(20,4)--(30,4));
\stopMPcode
}
\stopQ

\startA
\startformula
8 \times \frac{5}{60} + 5 \times \frac{5}{60}
+ 4 \times\frac{15}{60} + 3 \times \frac{5}{60}
= 2.333 \sunit[l]{mega watt hour}
\stopformula
\stopA

% 1.36
\startQ
A battery can be rated in ampere-hours (Ah) or watt-hours (Wh).
The ampere hours can be obtained from the watt hours
by dividing watt hours by a nominal voltage of 12 V.
If an automobile battery is rated at 20 Ah:
\startIG
\item What is the maximum current that can be supplied for 15 minutes?
\item How many days will it last if it is discharged at a rate of \sunit{2 milli ampere}?
\stopIG
\stopQ

\startA
\startIG
\item $20 / \frac{15}{60} = 80\sunit[l]{ampere}$
\item $\frac{20}{2\times 10^{-3}} / 24 = 416.7\sunit[l]{day}$
\stopIG
\stopA

% 1.37
\startQ
A total of \sunit{2 mega joule} are delivered to an automobile
battery (assume \sunit{12 volt}) giving it an additional charge.
How much is that additional charge?
Express your answer in ampere-hours.
\stopQ

\startA
$\sunit{joule}=\sunit{watt second}$:
\startformula
2 \times 10^6 / 12 / 3600 = 46.3\sunit[l]{ampere hour}
\stopformula
\stopA

% 1.38
\startQ
How much energy does a 10-hp motor deliver in 30 minutes?
Assume that 1 horsepower = 746 W.
\stopQ

\startA
$746\times 10 \times 30 \times 60 = \sunit{1.3428e7 joule}$
\stopA

% 1.39
\startQ
A 600-W TV receiver is turned on for 4 h with nobody watching it.
If electricity costs 10 cents/kWh,
how much money is wasted?
\stopQ

\startA
$E = \frac{600}{1000}\times 4 = 2.4\sunit[l]{kilo watt hour}$

$\text{Cost} = 10 \times 2.4 = 24 \text{cents}$
\stopA

\stopchapter
\stopcomponent
