\startsection[
  title={Indicator random variables},
]

%e5.2-1
\startEXERCISE
在 \ALGO{HIRE-ASSISTANT} 中,假設應聘者以隨機順序出現,
你只僱傭一次的概率是多少?正好僱傭 $n$ 次的概率又是多少?
\stopEXERCISE
\startANSWER
當最好的應聘者出現在第一個時,我們只需僱傭一次,
其概率爲 $1/n$ (一共有 $n!$ 種全排列,
而最好的出現在第一個的全排列共 $(n-1)!$ 種)。
而正好僱傭 $n$ 次,則需應聘者必須以增序出現,概率爲 $1/n!$。
\stopANSWER

%e5.2-2
\startEXERCISE
在 \ALGO{HIRE-ASSISTANT} 中,假設應聘者以隨機順序出現,
你正好僱傭兩次的概率是多少?
\stopEXERCISE
\startANSWER
如果正好僱傭了兩次,那麼假設第一個僱傭的人級別爲 $i$,
第二次僱傭的人級別肯定爲 $n$,且 $i < n$。

比第一個僱傭的人級別高的應聘者有 $n-i$ 個,
這些人中級別爲 $n$ 的那個排在最前面,其概率爲 $1/(n-i)$
(我們可以忽略其他應聘者,他們不影響概率)。
因此在第一個應聘者級別爲 $i$,則正好僱傭兩次的概率爲:
\startformula
\Pr\{T_i\} = \frac{1}{n}\frac{1}{n-i}
\stopformula
其中第一部分反映的是級別爲 $i$ 的應聘者排在第一個的概率。

正好僱傭兩次的概率爲:
\startformula\startmathalignment
\NC \Pr\{T\} \NC= \sum_{i=1}^{n-1}\Pr\{T_i\} \NR
\NC        \NC= \sum_{i=1}^{n-1}\frac{1}{n}\frac{1}{n-i} \NR
\NC        \NC= \frac{1}{n} \sum_{i=1}^{n-1}\frac{1}{i} \NR
\NC        \NC= \frac{1}{n} (\lg(n-1) + O(1)) \NR
\NC        \NC= \Omega(\frac{\lg n}{n})
\stopmathalignment\stopformula
\stopANSWER

%e5.2-3
\startEXERCISE
利用指示器隨機變量計算擲 $n$ 個骰子之和的期望值。
\stopEXERCISE
\startANSWER
令第 $k$ 個骰子的點數爲 $X_k$,
則 $X_k = 1,2,3,4,5,6$。
每個點數的概率相等 $\Pr\{X_k=i\}=1/6$。

如果是單個骰子,期望值 $X_k$ 爲:
\startformula\startmathalignment
\NC \E[X_k]
   \NC= \sum_{i=0}^6 i \Pr\{X_k = i\} \NR
\NC\NC= \frac{1 + 2 + 3 + 4 + 5 + 6}{6} \NR
\NC\NC= \frac{21}{6} \NR
\NC\NC= 3.5 \NR
\stopmathalignment\stopformula

對於多個骰子,則:
\startformula\startmathalignment
\NC \E[X] \NC= \E[\sum_{i=1}^nX_i] \NR
\NC      \NC= \sum_{i=1}^n \E[X_i] \NR
\NC      \NC= \sum_{i=1}^n 3.5 \NR
\NC      \NC= 3.5 \cdot n \NR
\stopmathalignment\stopformula
\stopANSWER

%e5.2-4
\startEXERCISE
本練習要求你驗證:
即使隨機變量並不相互獨立,期望值的線性性質依然成立。
假設有兩個六面骰子,獨立投擲,投擲結果之和的期望值是多少?
如果第一個骰子正常投擲,而將第二個設置成和第一個同樣的值。總和期望值又是多少?
如果第一個骰子正常投擲,而令第二個的值爲 $7$ 減去第一個的值,總和期望值又是多少?
\stopEXERCISE
\startANSWER
如果相互獨立,結果爲 $\E[X] = 3.5 \times 2 = 7$。

如果兩者相同,結果依然是 $\E[X] = 3.5 \times 2 = 7$。

如果要保證和爲 $7$,結果依然是 $\E[X] = 3.5 \times 2 = 7$。
\stopANSWER

%e5.2-5
\startEXERCISE
利用指示器隨機變量來求解如下{\EMP 帽子覈對問題}(hat-check problem):
$n$ 位顧客,每個人都將自己的帽子交給服務生。
服務生以隨機次序將帽子歸還給顧客。
請問拿到自己帽子的顧客數期望值是多少?
\stopEXERCISE
\startANSWER
設 $X$ 爲最後拿到自己帽子的顧客總數,
 $X_i$ 表示第 $i$ 位顧客拿回了自己的帽子:
\startformula
X_i = \startmathcases
\NC 1 \NC 如果顧客 $i$ 拿回了自己的帽子;\NR
\NC 0 \NC 如果顧客 $i$ 沒有拿回自己的帽子。\NR
\stopmathcases
\stopformula

$\Pr\{X_i=1\}=1/n$, $\Pr\{X_i=0\}=1-1/n$

$\E[X_i]=1\cdot \Pr\{X_i=1\} + 0\cdot \Pr\{X_i=0\}=1/n$

每位顧客拿回自己帽子的概率都是 $1/n$。
\startformula\startmathalignment
\NC \E[X] \NC= \E[X_1 + X_2 + \ldots + X_n] \NR
\NC \NC = \sum_{i=1}^n E[X_i] \NR
\NC \NC = \sum_{i=1}^n \frac{1}{n} \NR
\NC \NC = 1 \NR
\stopmathalignment\stopformula
\stopANSWER

%e5.2-6
\startEXERCISE
設 $A[1..n]$ 是由 $n$ 個不同的數構成的數列。
如果 $i < j$ 且 $A[i] > A[j]$,則稱 $(i,j)$ 爲 $A$ 的一個{\EMP 逆序對}(inversion)。
(參見\refproblem{inversion} 中更多關於逆序對的例子)
假設 $A$ 的元素構成 $\langle1,2,\ldots,n\rangle$ 上的一個均勻隨機排列,
請用指示器隨機變量來計算其中逆序對數目的期望值。
\stopEXERCISE
\startANSWER
令 $X_{i,j}$ 表示 $A[i]$ 和 $A[j]$ 爲逆序對。
其中 $\Pr\{A[i]>A[j]\} = \Pr\{A[i]<A[j]\} = 1/2$。
\startformula\startmathalignment
\NC \E[\sum_{i<j}X_{i,j}]
   \NC = \sum{i<j}\E[X_{i,j}] \NR
\NC\NC= \sum_{i=1}^{n-1} \sum_{j=i+1}^n \Pr\{A[i]>A[j]\} \NR
\NC      \NC= \frac{1}{2} \sum_{i=1}^{n-1} \sum_{j=i+1}^n 1 \NR
\NC      \NC= \frac{1}{2} \sum_{i=1}^{n-1} (n-i) \NR
\NC      \NC= \frac{1}{2} \sum_{i=1}^{n-1} (n-i) \NR
\NC      \NC= \frac{n(n-1)}{4} \NR
\NC      \NC= \binom{n}{2}/2 \NR
\stopmathalignment\stopformula
\stopANSWER

\stopsection
