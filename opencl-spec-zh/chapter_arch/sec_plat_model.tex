\section{平台模型}

OpenCL 平台模型的定義可以查看\reffig{plfmodel}。
此模型中有一個\cngloemp{host},
並且有一個或多個 {\ftEmp{OpenCL}} \cnglo{device}與其相連。
每個 OpenCL \cnglo{device}可劃分成一個或多個\cnglo{computeunit}(CU),
每個\cnglo{computeunit}又可劃分成一個或多個\cnglo{prcele}(PE)。
\cnglo{device}上的計算是在\cnglo{prcele}中進行的。

\input{fig/PlfMode.tex}
\placefigure[here][fig:plfmodel]
{平台模型}
{\reuseMPgraphic{plfMode}}

OpenCL \cnglo{app}會按照\cnglo{host}\cnglo{platform}的原生模型
在這個\cnglo{host}上運行。
\cnglo{host}上的 OpenCL \cnglo{app}提交\cngloemp{cmd}
給\cnglo{device}中的\cnglo{prcele}以執行計算任務。
\cnglo{computeunit}中的\cnglo{prcele}會作為 SIMD 單元(執行指令流的步伐一致)
或 SPMD 單元(每個 PE 維護自己的程式計數器)執行指令流。

\subsection{平台對多版本的支持}
OpenCL 的設計目標就是要支持同一\cnglo{platform}中多種具有不同能力的設備。
這些設備可以符合不同版本的 OpenCL 規範。
對於一個 OpenCL 系統,有三個重要的版本 ID 要考慮:
\cnglo{platform}版本、\cnglo{device}版本、
\cnglo{device}所支持的 OpenCL C 語言的版本。

\cnglo{platform}版本表明了所支持的 OpenCL 運行時的版本。
這包括可由\cnglo{host}用來與 OpenCL 運行時交互的所有 API,
像\cnglo{context}、\cnglo{memobj}、\cnglo{device}、\cnglo{cmdq}。

\cnglo{device}版本表明了\cnglo{device}的能力,
獨立於運行時和編譯器的版本,由 \capi{clGetDeviceInfo} 所返回的\cnglo{device}資訊來描述。
有很多特性都與\cnglo{device}版本有關,如資源限制和擴展功能。
所返回的版本號即為此\cnglo{device}所符合的 OpenCL 規範的最高版本號,
但不會高於\cnglo{platform}版本。

語言版本,可以讓開發人員據此知道此\cnglo{device}所支持的 OpenCL 編程語言具備哪些特性。
此版本會是所支持語言的最高版本。

OpenCL C 被設計為向後兼容的,
因此對於一個\cnglo{device}而言,只要支持語言的某一個版本,就可以說他符合標準。
如果某個\cnglo{device}支持多個語言版本,編譯器缺省使用最高的那個版本。
語言的版本不會高於\cnglo{platform}的版本,
但可能高於\cnglo{device}的版本(參見\insection[ctrlcveroption])。

