\startcomponent raise
\product prd-srd

\chapter{RAISE PROBLEM}
\placecontent
When developing software, especially for hardware driver, the register operation
got you into mess usually. Let's see following example:
\startccode
#define RX_STATUS (ioaddr + 0x06)
#define RX_CMD    RX_STATUS
#define TX_STATUS (ioaddr + 0x07)
#define TX_CMD    TX_STATUS
#define GP_LOW    (ioaddr + 0x08)
#define GP_HIGH   (ioaddr + 0x09)

#define TX_COLLISION 0x02
#define TX_16COLLISIONS 0x04
#define TX_READY 0x08
\stopccode
There is one MACRO for every register, whose value is the offset of its' address.
and there is one MACRO for every field for every register, whose value is the
mask of the field, used for getting/setting.

In C (the hardware driver is usually developed using 'C' language, of course,
asm sometime), we can use \code{structure} to represent the register, which
looks very nice, making more sense than \code{#define}.
\startccode
struct TEST_CHIP {
	TEST_REG1_FIELD1	: 2;
	/* unused */		: 1;
	TEST_REG1_FIELD2	: 2;
	TEST_REG1_FIELD3	: 3;

	
	TEST_REG2_FIELD1	: 2;
	TEST_REG2_FIELD2	: 1;
	/* unused */		: 2;
	TEST_REG2_FIELD3	: 3;
};
\stopccode

\stopcomponent

