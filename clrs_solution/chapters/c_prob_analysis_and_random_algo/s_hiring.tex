\startsection[
  title={The hiring problem},
]

%e5.1-1
\startEXERCISE
證明:假設在 \ALGO{HIRE-ASSISTANT} 算法的第 4 行中,
我們總能確定哪一個應聘者更佳,
則意味着我們知道應聘者排名的全部次序。

\CLRSH{HIRE-ASSISTANT(n)}
\startCLRSCODE
best = 0	// candidate 0 is a least-qualified dummy candidate
for i = 1 to n
	// interview candidate $i$
	if C_i > C_{best}
		best = i
		// hire candidate $i$
\stopCLRSCODE
\stopEXERCISE
\startANSWER
全序是一種包含了所有關係的偏序,
一個關係若是偏序關系,他必須是自反的、反對稱的以及可傳遞的。
令 $R$ 是 $A$ 上的偏序關係,即 $\forall a,b \in A$,有 $a R b$ 或 $b R a$。
這個關系爲“同樣好或更好”:
\startigBase[1]
\item 自反:顯然,任何人與自身相比都滿足此關系,即 $\forall x\in A, x R x$;
\item 反對稱:若候選人 A 比候選人 B 優秀,則 B 沒有 A 優秀, $\forall x,y\in A$,若 $xRy$ 且 $yRx$,則 $x=y$;
\item 可傳遞:若 A 比 B 優秀, B 比 C 優秀,則 A 比 C 優秀, $\forall x,y,z\in A$,若 $xRy$ 且 $yRz$,則 $xRz$。
\stopigBase

又由於我們假設可以比較任意兩個應聘者,也就說包含了所有關系,即全序。
\stopANSWER

%e5.1-2
\startEXERCISE \DIFFICULT
利用 \ALGO{RANDOM(0,1)} 實現過程 \ALGO{RANDOM(a,b)}。
作爲 $a$ 和 $b$ 的函數,其運行時間是多少?
\stopEXERCISE
\startANSWER
令 $n=b-a$,算法爲:
\startigBase[n]
\item 找到最小整數 $c$,使得 $2^c\ge n$ ($c=\ceil{\lg{n}}$);
\item 調用 \ALGO{RANDOM(0,1)} $c$ 次,得到一個 $c$ 比特的二進制數 $r$;
\item 如果 $r>n$,跳轉到上一步;
\item 否則返回 $a+r$。
\stopigBase
這會在範圍 $[a, b]$ 內均勻的生成一個隨機數。
但是,第二步可能會重復多次。
第二步所生成的值有效的概率是 $p=n/2^c$。

令生成合法隨機數的總輪數爲 $X$,其數學期望 $\E(X)$ 爲:
\startformula\startmathalignment
\NC \E(X)
    \NC = p + 2(1-p)p + 3(1-p)^2p + \cdots + n(1-p)^{n-1}p + \cdots \NR
\NC \NC = \sum_{i=1}^{\infty}i(1-p)^{i-1}p \NR
\NC \NC = p \sum_{i=1}^{\infty}i(1-p)^{i-1} \NR
\NC \NC = p \frac{1}{(1-(1-p))^2} \NR
\NC \NC = \frac{1}{p} \NR
\NC \NC = \frac{2^\ceil{\lg n}}{n} \NR
\stopmathalignment\stopformula
由於 $1 \le \frac{2^{\ceil{\lg n}}}{n} \le 2$,所以 $E(X)=\Theta(1)$。

而每輪所用時間爲 $\ceil{\lg n}$,因此運行時間爲:
\startformula
\ceil{\lg n}\Theta(1) = \Theta(\ceil{\lg n}) = \Theta(\lg n) = \Theta(\lg(b-a))
\stopformula
\stopANSWER

%e5.1-3
\startEXERCISE \DIFFICULT
假設你想寫一個程序,輸出 $0$ 和 $1$,概率均爲 $1/2$。
已有一個可以輸出 0 或 1 的過程 \ALGO{BIASED-RANDOM} 可用。
此過程輸出 1 的概率爲 $p$,輸出 0 的概率爲 $1-p$,
其中 $0<p<1$,但 $p$ 的值未知。
請給出一個算法,返回一個無偏的結果,
即各以 1/2 的概率返回 0 和 1,
可以利用 \ALGO{BIASED-RANDOM} 作爲子程式。
用 $p$ 的函數表示此算法的運行時間。
\stopEXERCISE
\startANSWER
簡單!
\startigBase[n]
\item 生成兩個隨機數 $x$ 和 $y$;
\item 如果這兩個數不同,則返回 $x$;
\item 否則,重復第一步。
\stopigBase
$x=0, y=1$ 和 $x=1, y=0$ 的概率相同。
第二步直接返回的概率是 $2pq$。
重試次數的期望值是 $1/(2pq)$,
期望運行時間爲 $\Theta(\frac{1}{p(1-p)})$。
\stopANSWER

\stopsection
