\startsection[
  title={Modular arithmetic},
]

%e31.3-1
\startEXERCISE
畫出羣\m{(\integers_4, +_4)} 和羣 \m{(\integers_5^\ast, \cdot_5)} 的運算表。
通過找這兩個羣的元素間的一一對應關係 \alpha,
滿足 \m{a+b\equiv c (\mod 4)} 當且僅當 \m{\alpha(a)\cdot \alpha(b) \equiv \alpha(c) (\mod 5)},
來證明這兩個羣是同構的。
\stopEXERCISE

\startANSWER
先找到各自的單位元(Identify) \m{a_0},再找到 \m{a_1},使得 \m{a_1^{(2)} = a_0}。
以此類推,找到 \m{a_2} 和 \m{a_3}。

\startcolumns[n=3, rule=on]
\startformula\startmathalignment[n=3]
\NC +_4 \NC \rightarrow \NC \cdot_5 \NR
\NC 0 \NC \rightarrow \NC 1 \NR
\NC 1 \NC \rightarrow \NC 3 \NR
\NC 2 \NC \rightarrow \NC 4 \NR
\NC 3 \NC \rightarrow \NC 1 \NR
\stopmathalignment\stopformula
\column

\startxtable[
    option=max,
    align={middle,lohi},
    split=yes,
    header=repeat,
    footer=repeat,
    offset=.25em,
]

% head
\startxtablehead[frame=off,bottomframe=on]
\startxrow[foregroundstyle=bold,]
  \xcell[rightframe=on]{\m{+_4}}\processcommalist[0, 1, 2, 3]{\xcell[align={middle}]}
\stopxrow
\stopxtablehead

% body
\startxtablebody[frame=off]
\startxrow \xcell[rightframe=on]{0}\processcommalist[0,1,2,3]\xcell \stopxrow
\startxrow \xcell[rightframe=on]{1}\processcommalist[1,2,3,0]\xcell \stopxrow
\startxrow \xcell[rightframe=on]{2}\processcommalist[2,3,0,1]\xcell \stopxrow
\startxrow \xcell[rightframe=on]{3}\processcommalist[3,0,1,2]\xcell \stopxrow
\stopxtablebody

\stopxtable

\column
\startxtable[
    option=max,
    align={middle,lohi},
    split=yes,
    header=repeat,
    footer=repeat,
    offset=.25em,
]

% head
\startxtablehead[frame=off,bottomframe=on]
\startxrow[foregroundstyle=bold,]
  \xcell[rightframe=on]{\m{\cdot_5}}\processcommalist[1, 2, 3, 4]{\xcell[align={middle}]}
\stopxrow
\stopxtablehead

% body
\startxtablebody[frame=off]
\startxrow \xcell[rightframe=on]{1}\processcommalist[1,2,3,4]\xcell \stopxrow
\startxrow \xcell[rightframe=on]{2}\processcommalist[2,4,1,3]\xcell \stopxrow
\startxrow \xcell[rightframe=on]{3}\processcommalist[3,1,4,2]\xcell \stopxrow
\startxrow \xcell[rightframe=on]{4}\processcommalist[4,3,2,1]\xcell \stopxrow
\stopxtablebody

\stopxtable


\stopcolumns
\stopANSWER

%e31.3-2
\startEXERCISE
列出 \m{\integers_9} 和 \m{\integers_{13}^\ast} 的所有子羣。
\stopEXERCISE

\startANSWER
\startcolumns[n=2, rule=on]
\startformula\startmathalignment
\NC \langle 0 \rangle \NC= \{ 0 \} \NR
\NC \langle 1 \rangle \NC= \{ 0, 1, 2, 3, 4, 5, 6, 7, 8 \} \NR
\NC \langle 2 \rangle \NC= \{ 0, 1, 2, 3, 4, 5, 6, 7, 8 \} \NR
\NC \langle 3 \rangle \NC= \{ 0, 3, 6 \} \NR
\NC \langle 4 \rangle \NC= \{ 0, 1, 2, 3, 4, 5, 6, 7, 8 \} \NR
\NC \langle 5 \rangle \NC= \{ 0, 1, 2, 3, 4, 5, 6, 7, 8 \} \NR
\NC \langle 6 \rangle \NC= \{ 0, 3, 6 \} \NR
\NC \langle 7 \rangle \NC= \{ 0, 1, 2, 3, 4, 5, 6, 7, 8 \} \NR
\NC \langle 8 \rangle \NC= \{ 0, 1, 2, 3, 4, 5, 6, 7, 8 \} \NR
\stopmathalignment\stopformula
\column
\startformula\startmathalignment
\NC \langle 1 \rangle \NC= \{ 1 \} \NR
\NC \langle 2 \rangle \NC= \{ 1, 2, 3, 4, 5, 6, 7, 8, 9, 10, 11, 12 \} \NR
\NC \langle 3 \rangle \NC= \{ 1, 3, 9 \} \NR
\NC \langle 4 \rangle \NC= \{ 1, 3, 4, 9, 10, 12 \} \NR
\NC \langle 5 \rangle \NC= \{ 1, 5, 8, 12 \} \NR
\NC \langle 6 \rangle \NC= \{ 1, 2, 3, 4, 5, 6, 7, 8, 9, 10, 11, 12 \} \NR
\NC \langle 7 \rangle \NC= \{ 1, 2, 3, 4, 5, 6, 7, 8, 9, 10, 11, 12 \} \NR
\NC \langle 8 \rangle \NC= \{ 1, 5, 8, 12 \} \NR
\NC \langle 9 \rangle \NC= \{ 1, 3, 9 \} \NR
\NC \langle 10 \rangle \NC= \{ 1, 3, 4, 9, 10, 12 \} \NR
\NC \langle 11 \rangle \NC= \{ 1, 2, 3, 4, 5, 6, 7, 8, 9, 10, 11, 12 \} \NR
\NC \langle 12 \rangle \NC= \{ 1, 12 \} \NR
\stopmathalignment\stopformula
\stopcolumns
\stopANSWER

%e31.3-3
\startEXERCISE
證明定理 31.14。附:

定理 31.14 (一個有限羣的非空封閉子集是一個子羣):如果 \m{(S,\oplus)} 是一個有限羣,
 \m{S'} 是 \m{S} 的任意一個非空子集並滿足對所有 \m{a,b\in S'},都有 \m{a\oplus b\in S'},
則 \m{(S',\oplus)} 是 \m{(S,\oplus)} 的一個子羣。
\stopEXERCISE

\startANSWER
只需證明 \m{S'} 是一個羣即可。
\startigBase[n]
\item {\EMP 封閉性}:假設已經滿足;
\item {\EMP 單位元}:\TODO{};
\item {\EMP 結合律}:繼承自 \m{S},也滿足;
\item {\EMP 逆元}:\TODO{}。
\stopigBase
\stopANSWER

%e31.3-4
\startEXERCISE
證明:如果 p 是素數且 e 是正整數,則:
\startformula
\phi(p^e) = p^{e-1}(p-1)
\stopformula
\stopEXERCISE

\startANSWER
\startformula
\phi(p^e) = p^e (1-\frac{1}{p}) = p^{e-1}(p-1)
\stopformula
\stopANSWER

%e31.3-5
\startEXERCISE
證明:對任意 \m{n>1} 和任意 \m{a\in\integers_n^\ast},
由 \m{f_a(x)=ax \mod n} 所定義的函數 \m{f_a: \integers_n^\ast \rightarrow \integers_n^\ast} 是 \m{\integers_n^\ast} 的一個置換。
\stopEXERCISE

\startANSWER
對於任一 a 和 n,每一個 x 都有一個唯一的 y 與其對應,也就是 \m{f(x)} 是一個雙射函數。
\stopANSWER

\stopsection
