% Image Read and Write Functions
\subsection[section:imgRwFunc]{圖像讀寫函式}

本節中所定義的內建函式僅能與\cnglo{imgobj}一起使用。

聲明可被\cnglo{kernel}讀取的\cnglo{imgobj}時應帶有限定符 \cqlf{__read_only}。
對帶有 \cqlf{__read_only} 的\cnglo{imgobj}調用 \capi{write_image} 會造成編譯錯誤。
聲明可被\cnglo{kernel}寫入的\cnglo{imgobj}時應帶有限定符 \cqlf{__write_only}。
對帶有 \cqlf{__write_only} 的\cnglo{imgobj}調用 \capi{read_image} 會造成編譯錯誤。
不支持在同一\cnglo{kernel}中對同一\cnglo{imgobj}
同時調用 \capi{read_image} 和 \capi{write_image}。

\capi{read_image} 返回的是一個四組件矢量浮點數、整數或無符號整數顏色值。
此值用 \ccmm{x}、 \ccmm{y}、 \ccmm{z}、 \ccmm{w} 來標識,
其中 \ccmm{x} 指代紅色分量, \ccmm{y} 指代綠色分量, \ccmm{z} 指代藍色分量,
 \ccmm{w} 指代 alpha 分量,每個分量都是一個矢量組件。

% Samplers
\subsubsection[section:samplerInImgRw]{採樣器}

圖像讀取函式的引數中有一個就是\cnglo{sampler}。
\cnglo{sampler}可作為引數由 \capi{clSetKernelArg} 傳給\cnglo{kernel},
也可以在 \cqlf{kernel} 函式的最外層聲明\cnglo{sampler},
或者是\cnglo{program}源碼中聲明的型別為 \ctype{sampler_t} 的常量。

\cnglo{program}中所聲明\cnglo{sampler}變量的型別為 \ctype{sampler_t}。
這種變量必須用 32 位無符號整形常數進行初始化,
按位欄解釋此常量,其位欄指定了下列屬性:
\startigBase[indentnext=no]
\item 尋址模式
\item 濾波模式
\item 歸一化坐標
\stopigBase
這些屬性控制着 \capi{read_image{f|i|ui}} 如何讀取圖像中的元素。

也可在\cnglo{program}源碼中將\cnglo{sampler}聲明為全局常量,下列三種語法均有效:
\startclc
const sampler_t		<sampler name> = <value>
constant sampler_t	<sampler name> = <value>
__constant sampler_t	<sampler_name> = <value>
\stopclc

\startnotepar
在計算每個\cnglo{device}中指向常數位址空間的引數數目或常數位址空間的大小時,
不考慮帶有限定符 \cqlf{constant} 的\cnglo{sampler}(參見\reftab{cldevquery}中的
 \cenum{CL_DEVICE_MAX_CONSTANT_ARGS} 和
 \cenum{CL_DEVICE_MAX_CONSTANT_BUFFER_SIZE})。
\stopnotepar

\placetable[here][tab:samplerDesc]
{採樣器描述符}
{\startCLOD[\cnglo{sampler}屬性][描述]

\clOD{\ccmm{<normalized coords>}}{
指定所傳入的坐標 \math{x}、 \math{y} 和 \math{z} 是否已歸一化。
他必須是常值,可以是下列預定義枚舉中的一個:
\startigBase
\item \cenum{CLK_NORMALIZED_COORDS_TRUE}
\item \cenum{CLK_NORMALIZED_COORDS_FALSE}
\stopigBase

在單個\cnglo{kernel}中針對同一圖像多次調用 \capi{read_image{f|i|ui}} 時,
所用\cnglo{sampler}中 \ccmm{<normalized coords>} 的值必須相同。
}

\clOD{\ccmm{<addressing mode>}}{
指定圖像的尋址模式,即圖像坐標溢出時如何處置。
他必須是常值,可以是下列預定義枚舉中的一個:
\startigBase
\item \cenum{CLK_ADDRESS_MIRRORED_REPEAT}——在整數接點處翻轉圖像坐標。
這種尋址模式只能用於歸一化坐標。
如果使用的不是歸一化坐標,則此模式生成的圖像坐標未定義。

\item \cenum{CLK_ADDRESS_REPEAT}——溢出的坐標會繞回到有效區間內。
這種尋址模式只能用於歸一化坐標。
如果使用的不是歸一化坐標,則此模式生成的圖像坐標未定義。

\item \cenum{CLK_ADDRESS_CLAMP_TO_EDGE}——溢出的坐標會被壓入有效範圍內。

\item \cenum{CLK_ADDRESS_CLAMP}\footnote{
與尋址模式\cenum{CLK_ADDRESS_CLAMP_TO_EDGE}類似。}——溢出的坐標會返回顏色極值。

\item \cenum{CLK_ADDRESS_NONE}——此模式下,由程式員保證坐標不會溢出,否則結果未定義。
\stopigBase

對於 1D 和 2D 圖像陣列,尋址模式僅對坐標 \math{x} 和 \math{(x,y)} 有效。
坐標中的陣列索引所用尋址模式始終是 \cenum{CLK_ADDRESS_CLAMP_TO_EDGE}。
}

\clOD{\ccmm{filter mode}}{
指定所用的濾波模式。
他必須是常值,可以是下列預定義枚舉中的一個:
\startigBase
\item \cenum{CLK_FILTER_NEAREST}
\item \cenum{CLK_FILTER_LINEAR}
\stopigBase

對於這些濾波模式的描述,請參見\insection[addressFilterMode]。
}

\stopCLOD

}

例:
\startclc[indentnext=no]
const sampler_t /BTEX\ftEmp{samplerA}/ETEX = CLK_NORMALIZED_COORDS_TRUE
			| CLK_ADDRESS_REPEAT
			| CLK_FILTER_NEAREST;
\stopclc
\cnglo{sampler} {\ftEmp{samplerA}} 用的是規格化坐標、重複尋址模式和最近濾波。

對於一個\cnglo{kernel}中所能聲明\cnglo{sampler}的最大數目,
可以用 \capi{clGetDeviceInfo} 以 \cenum{CL_DEVICE_MAX_SAMPLERS} 進行查詢。

% Determining the border color
\subsubsubsection{確定顏色極值}

如果\cnglo{sampler}中的 \ccmm{<addressing mode>} 是 \cenum{CLK_ADDRESS_CLAMP},
則溢出的圖像坐標會返回顏色極值。
所選顏色極值取決於圖像通道順序,可以是下列中的一個:
\startigBase
\item 如果圖像通道順序為 \cenum{CL_A}、 \cenum{CL_INTENSITY}、 \cenum{CL_Rx}、
\cenum{CL_RA}、 \cenum{CL_RGx}、 \cenum{CL_RGBx, CL_ARGB}、 \cenum{CL_BGRA}
 或 \cenum{CL_RGBA},則所選顏色極值為 \ccmm{(0.0f, 0.0f, 0.0f, 0.0f)}。

\item 如果圖像通道順序為 \cenum{CL_R}、 \cenum{CL_RG}、 \cenum{CL_RGB}
 或 \cenum{CL_LUMINANCE},則所選顏色極值為 \ccmm{(0.0f, 0.0f, 0.0f, 1.0f)}。
\stopigBase

% Built-in Image Read Functions
\subsubsection[section:builtInImgReadFunc]{內建圖像讀取函式}

下列帶有\cnglo{sampler}的內建函式可用於讀取圖像\footnote{%
这些函式不支持型别为 \ctype{mage1d_buffer_t} 的图像。%
}。

\placetable[here,split][tab:imgReadFunc]
{內建圖像讀取函式}
{\input{chapter_lgg/tbl/tbl_imgreadfunc.tex}}

% Built-in Image Sampler-less Read Functions
\subsubsection{內建無採樣器圖像讀取函式}

下列無\cnglo{sampler}的內建函式也可用於讀取圖像。
其行為與\insection[builtInImgReadFunc]中坐標為整數、並帶有\cnglo{sampler}的對應函式一樣,
相當於\cnglo{sampler}的濾波模式為 \cenum{CLK_FILTER_NEAREST}、
歸一化坐標為 \cenum{CLK_NORMALIZED_COORDS_FALSE}、
尋址模式為 \cenum{CLK_ADDRESS_NONE}。

\placetable[here,split][tab:imgReadWithoutSamplerFunc]
{內建無採樣器圖像讀取函式}
{\input{chapter_lgg/tbl/tbl_imgreadwithoutsamplerfunc.tex}}

% Built-in Image Write Functions
\subsubsection{內建圖像寫入函式}

下列內建函式可用於寫入圖像。

\placetable[here,split][tab:imgWriteFunc]
{內建圖像寫入函式}
{\input{chapter_lgg/tbl/tbl_imgwritefunc.tex}}

% Built-in Image Query Functions
\subsubsection{內建圖像查詢函式}

下列內建函式可用於查詢圖像資訊。

\placetable[here,split][tab:imgQueryFunc]
{內建圖像查詢函式}
{\input{chapter_lgg/tbl/tbl_imgqueryfunc.tex}}

\reftab{imgQueryFunc}中,
\capi{get_image_channel_data_type} 和 \capi{get_image_channel_order} 所返回的帶有前綴 \cenum{CLK_} 的值
分別對應於\reftab{imgChannelDataType}和\reftab{imgChannelOrder}中帶有前綴 \cenum{CL_} 的值。
例如, \cenum{CL_UNORM_INT8} 和 \cenum{CLK_UNORM_INT8} 都是指通道數據類型為非歸一化的 8 位整數。

\subsubsection{将图像通道映射到 read_image 所返回的颜色值上、%
将传递给 write_image 的颜色值映射到图像通道上}

\reftab{ecComponentAndImgChannel}中列出了圖像元素的各通道的顏色值
與 \ctype{float4}、 \ctype{int4} 或 \ctype{uint4} 中組件的映射關係,
這些矢量由 \capi{read_image{f|i|ui}} 返回或作為 \capi{write_image{f|i|ui}} 的參數 \carg{color}。
對於未映射的組件,如果是紅、綠、藍幾個通道,則將其值置為 \ccmm{0.0},
而如果是 alpha 通道,則將其值置為 \math{1.0}。

\placetable[here][tab:ecComponentAndImgChannel]
{矢量組件與圖像通道的對應關係}{
\startCLOO[通道順序][矢量組件中的通道數據]
\clOO{\cenum{CL_R}、 \cenum{CL_Rx}}{\ccmm{(r, 0.0, 0.0, 1.0)}}
\clOO{\cenum{CL_A}}{\ccmm{(0.0, 0.0, 0.0, a)}}

\clOO{\cenum{CL_RGB}、 \cenum{CL_RGBx}}{\ccmm{(r, g, 0.0, 1.0)}}
\clOO{\cenum{CL_RA}}{\ccmm{(r, 0.0, 0.0, a)}}

\clOO{\cenum{CL_RG}、 \cenum{CL_RGx}}{\ccmm{(r, g, b, 1.0)}}
\clOO{\cenum{CL_RGBA}、 \cenum{CL_BGRA}、 \cenum{CL_ARGB}}{\ccmm{(r, g, b, a)}}

\clOO{\cenum{CL_INTENSITY}}{\ccmm{(I, I, I, I)}}
\clOO{\cenum{CL_LUMINANCE}}{\ccmm{(L, L, L, 1.0)}}
\stopCLOO
}

如果\cnglo{kernel}對多個圖像使用同一個尋址模式為 \cenum{CL_ADDRESS_CLAMP} 的\cnglo{sampler},
則可能導致實作內部使用額外的\cnglo{sampler}。
如果通過 \capi{read_image{f | i | ui}} 對多個圖像使用同一\cnglo{sampler},
則實作可能需要分配額外的\cnglo{sampler}來處理不同的顏色極值(這取決於所用的圖像格式)。
在計算\cnglo{device}所支持\cnglo{sampler}的最大數目時(\cenum{CL_DEVICE_MAX_SAMPLERS}),
會將這些實作自行分配的\cnglo{sampler}考慮在內。
如果所入隊的\cnglo{kernel}需要的\cnglo{sampler}超過了這個最大值,
則會導致返回 \cenum{CL_OUT_OF_RESOURCES}。
