\section{命令隊列}

OpenCL 對象,
如\cnglo{memobj}、\cnglo{programobj}和\cnglo{kernelobj}都是用\cnglo{context}創建的。
對這些對象的操控都是通過\cnglo{cmdq}實施的。
用\cnglo{cmdq}可以將一系列操作(叫做\cnglo{cmd})按序排隊。
如果有多個\cnglo{cmdq},\cnglo{app}可以將多個相互獨立的\cnglo{cmd}分別排隊而無須同步。
然而這僅在沒有共享任何對象時才成立。
如果要在多個\cnglo{cmdq}間共享對象,就要求\cnglo{app}實施相應的同步。
這在\inappendix[sharedClObj]中會有所描述。

\topclfunc{clCreateCommandQueue}

\startCLFUNC
cl_command_queue clCreateCommandQueue(
			cl_context context,
			cl_device_id device,
			cl_command_queue_properties properties,
			cl_int *errcode_ret)
\stopCLFUNC

函式 \capi{clCreateCommandQueue} 可用來在某個\cnglo{device}上創建\cnglo{cmdq}。

\carg{context} 必須是一個有效的 OpenCL \cnglo{context}。

\placetable[here][tab:clcmdprop]
{\ctype{cl_command_queue_property} 的有效值及其描述}
{\startED[命令隊列屬性]

\clED{CL_QUEUE_OUT_OF_ORDER_EXEC_MODE_ENABLE}{
用來確定\cnglo{cmdq}中的\cnglo{cmd}是\cnglo{inordexec}還是\cnglo{outordexec}。
如果設置了此屬性,就\cnglo{outordexec},否則\cnglo{inordexec}。

詳情請參考\insection[oooe]。
}

\clED{CL_QUEUE_PROFILING_ENABLE}{
使能或去能對\cnglo{cmd}的評測(profiling)。
如果設置了此屬性,則使能,否則去能。

詳情請參考\insection[profileMoKernel]。
}

\stopED

}

\carg{device} 必須是與 \carg{context} 關聯的\cnglo{device}。
他要麼是用 \capi{clCreateContext} 創建 \carg{context} 時所指定的\cnglo{device}清單中的一個,
要麼其類型與用 \capi{clCreateContextFromType} 創建 \carg{context} 時所指定的\cnglo{device}類型相同。

\carg{properties} 指定了\cnglo{cmdq}的一系列屬性。
他是位欄,參見\reftab{clcmdprop}。
其值只能從\reftab{clcmdprop}所列屬性中選取,否則無效。

\carg{errcode_ret} 用來返回錯誤碼。
當然如果 \carg{errcode_ret} 是 \cmacro{NULL},就不會返回錯誤碼了。

如果成功創建了\cnglo{cmdq},則 \capi{clCreateCommandQueue} 會將其返回,同時將 \carg{errcode_ret} 置為 \cenum{CL_SUCCESS}。
否則返回 \cmacro{NULL},同時將 \carg{errcode_ret} 置為下列錯誤碼之一:
\startigBase
\item \cenum{CL_INVALID_CONTEXT},如果 \carg{context} 無效。
\item \cenum{CL_INVALID_DEVICE},如果 \carg{device} 無效或未與 \carg{context} 關聯。
\item \cenum{CL_INVALID_VALUE},如果 \carg{properties} 的值有效,但是此\cnglo{device}不支持。
\item \cenum{CL_INVALID_QUEUE_PROPERTIES},如果 \carg{properties} 本身沒問題,但 \carg{device} 不支持。
\item \cenum{CL_OUT_OF_RESOURCES},如果\scdevfailres。
\item \cenum{CL_OUT_OF_HOST_MEMORY},如果\schostfailres。
\stopigBase

\topclfunc{clRetainCommandQueue}

\startCLFUNC
cl_int clRetainCommandQueue(
		cl_command_queue command_queue)
\stopCLFUNC

此函式會使 \carg{command_queue} 的\cnglo{refcnt}增一。

如果執行成功,\capi{clRetainCommandQueue} 會返回 \cenum{CL_SUCCESS}。
否則,返回下列錯誤碼之一:
\startigBase
\item \cenum{CL_INVALID_COMMAND_QUEUE},如果 \carg{command_queue} 無效。
\item \cenum{CL_OUT_OF_RESOURCES},如果\scdevfailres。
\item \cenum{CL_OUT_OF_HOST_MEMORY},如果\schostfailres。
\stopigBase

\capi{clCreateCommandQueue} 會實施隱式的 \englo{retain}。
這對第三方庫非常有用,這樣\cnglo{app}可以將\cnglo{cmdq}傳給他們使用。
然而,\cnglo{app}可能在沒有通知庫的情況下刪除\cnglo{cmdq}。
通過\cnglo{retain}或\cnglo{release}\cnglo{cmdq},
在庫所使用的\cnglo{cmdq}不再有效時就不會出現問題。

\topclfunc{clReleaseCommandQueue}

\startCLFUNC
cl_int clReleaseCommandQueue(
		cl_command_queue command_queue)
\stopCLFUNC

此函式會使 \carg{command_queue} 的\cnglo{refcnt}減一。

如果執行成功,\capi{clReleaseCommandQueue} 會返回 \cenum{CL_SUCCESS}。
否則,返回下列錯誤碼之一:
\startigBase
\item \cenum{CL_INVALID_COMMAND_QUEUE},如果 \carg{command_queue} 無效。
\item \cenum{CL_OUT_OF_RESOURCES},如果\scdevfailres。
\item \cenum{CL_OUT_OF_HOST_MEMORY},如果\schostfailres。
\stopigBase

当 \carg{command_queue} 的\cnglo{refcnt}降為 0,並且其中的所有\cnglo{cmd}全部執行完畢
(如執行\cnglo{kernel}、更新\cnglo{memobj}等)時,此\cnglo{cmdq}就會被刪除。

\capi{clReleaseCommandQueue}會實施隱式的刷新(flush),
這會觸發所有之前入隊的 OpenCL \cnglo{cmd}。

\topclfunc{clGetCommandQueueInfo}

\startCLFUNC
cl_int clGetCommandQueueInfo(
		cl_command_queue command_queue,
		cl_command_queue_info param_name,
		size_t param_value_size,
		void *param_value,
		size_t *param_value_size_ret)
\stopCLFUNC

此函式可用來查詢\cnglo{cmdq}的資訊。

\carg{command_queue} 指定要查詢哪個\cnglo{cmdq}。

\carg{param_name} 指定要查詢什麼資訊。

\carg{param_value} 所指內存用來存儲查詢結果。如果是 \cmacro{NULL},則忽略。

\carg{param_value_size} 即 \carg{param_value} 所指內存塊的大小(單位:字節)。
其值必須 >= \reftab{cmdqinfo}中返回型別的大小。
如果 \carg{param_value} 是 \cmacro{NULL},則將其忽略。

\carg{param_value_size_ret} 會返回查詢結果的實際大小。如果是 \cmacro{NULL},則忽略。

\capi{clGetCommandQueueInfo} 所支持的 \carg{param_name}
以及 \carg{param_value} 中所返回的資訊如\reftab{cmdqinfo}所示。

\placetable[here][tab:cmdqinfo]
{\capi{clGetCommandQueueInfo}所支持的\carg{param_names}}
{\input{chapter_rt/tbl/tbl_cmdq_getinfo.tex}}

如果執行成功,\capi{clGetCommandQueueInfo} 會返回 \cenum{CL_SUCCESS}。
否則,返回下列錯誤碼之一:
\startigBase
\item \cenum{CL_INVALID_COMMAND_QUEUE},如果 \carg{command_queue} 無效。

\item \cenum{CL_INVALID_VALUE},如果 \carg{param_name} 不在支持之列,
或者 \carg{param_value_size} 的值 < \reftab{cmdqinfo}中返回型別的大小
且 \carg{param_value} 不是 \cmacro{NULL}。

\item \cenum{CL_OUT_OF_RESOURCES},如果\scdevfailres。

\item \cenum{CL_OUT_OF_HOST_MEMORY},如果\schostfailres。
\stopigBase

\startnotepar
如果創建了\cnglo{context}和\cnglo{cmdq}並且已經有\cnglo{cmd}入隊了,
而這時他們所使用的\cnglo{device}可能變的不可用了。
這種情況下,對於使用這個\cnglo{context}(和\cnglo{cmdq})的 OpenCL API 而言,
其行為\cnglo{impdef}。
在\cnglo{device}變的不可用時,如果創建\cnglo{context}時用戶指定了回調函式,
則可以在傳遞給他的引數 \carg{errinfo}、\carg{private_info} 中記錄相應的資訊。
\stopnotepar

