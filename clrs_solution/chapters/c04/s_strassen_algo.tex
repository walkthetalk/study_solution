\startsection[
  title={Strassen’s algorithm for matrix multiplication},
]

\startEXERCISE
用 Strassen 算法計算矩陣積:
\startformula
\startpmatrix%[location=low]
\NC1\NC3\NR
\NC7\NC5\NR
\stoppmatrix
\startpmatrix%[location=low]
\NC6\NC8\NR
\NC4\NC2\NR
\stoppmatrix
\stopformula
\stopEXERCISE

\startANSWER
十個矩陣分別爲:
\startformula\startmathalignment[n=10]
\NC S_1 \NC= 6 \qquad
\NC S_2 \NC=  4 \qquad
\NC S_3 \NC= 12 \qquad
\NC S_4 \NC= -2 \qquad
\NC S_5 \NC= 5 \NR
\NC S_6 \NC= 8 \qquad
\NC S_7 \NC= -2 \qquad
\NC S_8 \NC=  6 \qquad
\NC S_9 \NC= -6 \qquad
\NC S_{10} \NC= 14 \NR
\stopmathalignment\stopformula
七個矩陣積分別爲:
\startformula\startmathalignment[n=8]
\NC P_1 \NC= 1 \cdot 6 = 6 \qquad
\NC P_2 \NC= 4 \cdot 2 = 8 \qquad
\NC P_3 \NC= 6 \cdot 12 = 72 \qquad
\NC P_4 \NC= (-2) \cdot 5 = -10 \NR
\NC P_5 \NC= 6 \cdot 8 = 48 \qquad
\NC P_6 \NC= (-2) \cdot 6 = -12 \qquad
\NC P_7 \NC= (-6) \cdot 14 = -84 \qquad
\NC \NC \NR
\stopmathalignment\stopformula
結果爲:
\startformula\startmathalignment[n=4]
\NC C_{11} \NC = 48 + (-10) - 8 + (-12) = 18 \qquad
\NC C_{12} \NC = 6 + 8 = 14 \NR
\NC C_{21} \NC = 72 + (-10) = 62 \qquad
\NC C_{22} \NC = 48 + 6 - 72 - (-84) = 66 \NR
\stopmathalignment\stopformula
即:
\startformula
\startpmatrix
\NC18\NC44\NR
\NC62\NC66\NR
\stoppmatrix
\stopformula
\stopANSWER

\startEXERCISE
爲 Strassen 算法編寫僞碼。
\stopEXERCISE

\startANSWER
\CLRSH{STRASSEN(A,B)}
\startCLRSCODE
if A.row == 1
	C = A * B
	return C
// 劃分子陣
S_1 = B_{12} - B_{22}
S_2 = A_{11} + A_{12}
S_3 = A_{21} + A_{22}
S_4 = B_{21} - B_{11}
S_5 = A_{11} + A_{22}
S_6 = B_{11} + B_{22}
S_7 = A_{12} - A_{22}
S_8 = B_{21} + B_{22}
S_9 = A_{11} - A_{21}
S_{10} = B_{11} + B_{12}

P_1 = \ALGO{STRASSEN(A_{11},S_1)}
P_2 = \ALGO{STRASSEN(S_2,B_{22})}
P_3 = \ALGO{STRASSEN(S_3,B_{11})}
P_4 = \ALGO{STRASSEN(A_{22},S_4)}
P_5 = \ALGO{STRASSEN(S_5,S_6)}
P_6 = \ALGO{STRASSEN(S_7,S_8)}
P_7 = \ALGO{STRASSEN(S_9,S_{10})}

C_{11} = P_5 + P_4 - P_2 + P_6
C_{12} = P_1 + P_2
C_{21} = P_3 + P_4
C_{22} = P_5 + P_1 - P_3 - P_7

return C
\stopCLRSCODE
\stopANSWER

%e4.2-3
\startEXERCISE
要想在 $o(n^{\lg 7})$ 時間內完成 $n \times n$ 矩陣相乘,
且完成兩個 $3 \times 3$ 矩陣相乘需要 $k$ 次乘法運算(不滿足乘法交換律),
滿足這一條件的 $k$ 最大是多少?
此算法運行時間如何?
\stopEXERCISE

\startANSWER
由分治法可得遞迴算法運行時間爲:
\startformula
T(n) = k T(n/3) + O(1)
\stopformula
如果 \m{k} 足夠大,則求解得 \m{T(n) = \Theta(n^{\log_3^k})}。則:
\startformula\startmathalignment[n=1,align=center]
\NC n^{\log_3 k} < n^{\lg7} \NR
\NC \Downarrow \NR
\NC \log_3k < \lg7 \NR
\NC \Downarrow \NR
\NC k < 3^{\lg7} \approx 21.84986\NR
\stopmathalignment\stopformula
所以 $k$ 最大爲 $21$。
\stopANSWER

%e4.2-4
\startEXERCISE
V.Pan 發現一種方法可以用 $132464$ 次乘法操作完成大小爲 $68\times 68$ 的矩陣乘法;
還發現一種方法可以用 $143640$ 次乘法操作完成大小爲 $70\times 70$ 的矩陣乘法;
還發現一種方法可以用 $155424$ 次乘法操作完成大小爲 $72\times 72$ 的矩陣乘法。
若用於矩陣相乘的分治算法,上述哪種方法會得到最佳的漸進運行時間?
與 Strassen 算法相比,性能如何?
\stopEXERCISE
\startANSWER
\startformula\startalign
\NC \log_{68} 132464 \NC \approx 2.795128 \NR
\NC \log_{70} 143640 \NC \approx 2.795122 \NR
\NC \log_{72} 155424 \NC \approx 2.795147 \NR
\stopalign\stopformula
顯然 $70 \times 70$ 的方法漸進運行時間最佳。比 Strassen 算法要好。
\stopANSWER

%e4.2-5
\startEXERCISE
如何只用三次實數相乘得到兩個複數 $a+bi$ 和 $c+di$ 的乘積。
以 $a,b,c,d$ 作爲算法的輸入,
結果分別爲實部 $ac-bd$ 和虛部 $ad+bc$。
\stopEXERCISE

\startANSWER
\startCLRSCODE
S_1 = a \cdot (c+d)
S_2 = (a+b) \cdot d
S_3 = (a-b) \cdot c
C_r = S_1 - S_2	// $ac-bd$
C_i = S_1 - S_3	// $ad+bc$
\stopCLRSCODE
也可以是:
\startCLRSCODE
S_1 = (a+b)\cdot (c+d)
S_2 = a\cdot c
S_3 = b\cdot d
C_r = S_2 - S_3
C_i = S_1 - S_2 - S_3
\stopCLRSCODE
均用了三次乘法運算,五次加減運算
\stopANSWER

%e4.2-6
\startEXERCISE
現有一算法用於求 $n\times n$ 矩陣的平方,
時間複雜度爲 $\Theta(n^{\alpha})$,
其中 $\alpha \ge 2$。
根據此算法計算兩個不同的 $n\times n$ 矩陣的積,
要求時間複雜度依然是 $\Theta(n^{\alpha})$。
\stopEXERCISE

\startANSWER
若此算法爲分治算法,
僅需改造 $n=1$ 的情況下代碼的計算方式即可。
\stopANSWER

\stopsection
