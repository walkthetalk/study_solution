\startsection[
  title={Maintaining the heap property},
]

\startEXERCISE
參照圖 6-2 的方法,說明 \ALGO{MAX-HEAPIFY(A, 3)} 在
數列 \m{A = \langle 27, 17, 3, 16, 13, 10, 1, 5, 7, 12, 4, 8, 9, 0\rangle} 上的操作過程。
\stopEXERCISE

\startANSWER
\startcombination[2*2]
{\externalfigure[output/e6_2_1-1]}{}
{\externalfigure[output/e6_2_1-2]}{}
{\externalfigure[output/e6_2_1-3]}{}
{}{}
\stopcombination
\stopANSWER

\startEXERCISE
參考 \ALGO{MAX-HEAPIFY},寫出能維護相應最小堆的 \ALGO{MIN-HEAPIFY(A, i)} 的僞碼,
並比較 \ALGO{MIN-HEAPIFY} 和 \ALGO{MAX-HEAPIFY} 的運行時間。
\stopEXERCISE

\startANSWER
\CLRSH{MIN-HEAPIFY(A, i)}
\startCLRS
l = LEFT(i)
r = RIGHT(i)
if l ≤ A.heap-size and A[l] < A[i]
	smallest = l
else
	smallest = i
if r ≤ A.heap-size and A[r] < A[i]
	smallest = r
if smallest ≠ i
	exchange A[i] with A[smallest]
	MIN-HEAPIFY(A, smallest)
\stopCLRS

\ALGO{MIN-HEAPIFY} 和 \ALGO{MAX-HEAPIFY} 的運行時間的運行時間相同。
\stopANSWER

\startEXERCISE
當元素 \m{A[i]} 比其兩個子節點的值都大時,調用 \ALGO{MAX-HEAPIFY(A, i)} 會有什麼結果?
\stopEXERCISE

\startANSWER
沒有變化。執行比較後,發現 \m{A[i]} 本身就是最大的,直接返回了。
\stopANSWER

\startEXERCISE
當 \m{i > A.heap-size / 2} 時,調用 \ALGO{MAX-HEAPIFY(A, i)} 會有什麼結果?
\stopEXERCISE

\startANSWER
沒有變化。因爲 \m{A[i]} 是葉子節點。 \ALGO{LEFT} 和 \ALGO{RIGHT} 所返回的值都大於 \m{A} 的元素個數;
 \m{i} 即 \m{largest},然後就返回了。
\stopANSWER

\startEXERCISE
就常量因子而言, \ALGO{MAX-HEAPIFY} 的執行效率很不錯,
但第 \m{10} 行可能時隔例外,有的編譯器可能會產生低效的代碼。
請用循環控制結構取代遞迴,重寫 \ALGO{MAX-HEAPIFY}。
\stopEXERCISE

\startANSWER
\CLRSH{MAX-HEAPIFY-LOOP(A, i)}
\startCLRS
while true
	l = LEFT(i)
	r = RIGHT(i)
	if l ≤ A.heap-size and A[l] < A[i]
		largest = l
	else
		largest = i
	if r ≤ A.heap-size and A[r] < A[i]
		largest = r
	if largest == i
		return
	exchange A[i] with A[largest]
	i = largest
\stopCLRS
\stopANSWER

\blank

\startEXERCISE
證明:對於一個大小爲 \m{n} 的堆, \ALGO{MAX-HEAPIFY} 的最壞情況運行時間爲 \m{\Omega(\lg{n})}。
(\hint 於 \m{n} 個節點的堆,可以通過對每個節點設定恰當的值,
使得從根節點到葉節點路徑上的每個節點都會遞迴調用 \ALGO{MAX-HEAPIFY}。)
\stopEXERCISE

\startANSWER
結論是顯然的。以最左邊的路徑爲例,如果堆中最大元素在此路徑上,而根節點是最小元素,
則爲將最小元素放到葉子節點上,此路徑上的每一個節點都會調用一次 \ALGO{MAX-HEAPIFY}。
由於堆的高度爲 \m{\lfloor \lg{n} \rfloor} \inexercise[heap_height],最壞情況運行時間爲 \m{\Omega(\lg{n})}。
\stopANSWER

\stopsection
