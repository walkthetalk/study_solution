\startsection[
  title={Hash functions},
]

%e11.3-1
\startEXERCISE
假設我們希望查找一個長度爲 n 的鏈表,
其中每一個元素都包含一個關鍵字 k 並具有散列值 \m{h(k)}。
每一個關鍵字都是長字串。
那麼在表中查找具有給定關鍵字的元素時,如何利用各元素的散列值?
\stopEXERCISE

\startANSWER
比較關鍵字散列值而不是比較關鍵字。
\stopANSWER

%e11.3-2
\startEXERCISE
假設將一個長度爲 r 的字串散列到 m 個槽位中,
並將其視爲一個以 128 爲基數的數,要求應用除法散列法。
我們可以很容易地把數 m 表示爲一個 32 位的機器字,
但是對於長度爲 r 的字串,由於他被當作以 128 爲基數的數來處理,
就要佔用若幹個機器字。
假設應用除法散列法來計算一個字串的散列值,
那麼如何才能在除了該字串本身所佔空間外,只利用常數個機器字?
\stopEXERCISE

\startANSWER
程序如下,其中 p、 b、 c 均小於 m, c 爲最終結果:
\startCLRS
p = 1
b = 128 mod m
c = 0
for i = 0 upto r - 1
	c = (c + a[i] * p mod m) mod m
	p = p * b mod m
return c
\stopCLRS
\stopANSWER

%e11.3-3
\startEXERCISE
考慮除法散列法的另一個版本,
其中 \m{h(k) = k \mod m}, \m{m=2^p-1}, \m{k} 爲按基數 \m{2^p} 表示的字串。
試證明:如果串 x 可由串 y 通過其自身的字符置換排列導出,
則 x 和 y 具有相同的散列值。
給出一個應用的例子,其中這一特性在散列函數中是不希望出現的。
\stopEXERCISE

\startANSWER
若 \m{m=2^p-1},則 \m{k = \sum_{i=0}^{n}a_i(m+1)^i}。
因此 \m{k \mod m = \left(\sum_{i=0}^{n}a_i\right)\mod m}。

因此只要字串 x 和 y 可相互置換,則散列值必定相同。
\stopANSWER

%e11.3-4
\startEXERCISE
考慮一個大小爲 \m{m=1000} 的散列表和一個對應的散列函數 \m{h(k)=\lfloor m(kA\mod 1)\rfloor},
其中 \m{A=(\sqrt{5}-1)/2},
試計算關鍵字 61、 62、 63、 64 和 65 被映射到的位置。
\stopEXERCISE

\startANSWER
用 maxima 執行下列命令:
\startcodebox
for a:61 step 1 thru 65 do display(floor(mod((sqrt(5)-1)/2 * a, 1) * 1000));
\stopcodebox
結果爲:
\startcodebox
                               61 (sqrt(5) - 1)
                   floor(1000 (---------------- - 37)) = 700
                                      2

                   floor(1000 (31 (sqrt(5) - 1) - 38)) = 318

                               63 (sqrt(5) - 1)
                   floor(1000 (---------------- - 38)) = 936
                                      2

                   floor(1000 (32 (sqrt(5) - 1) - 39)) = 554

                               65 (sqrt(5) - 1)
                   floor(1000 (---------------- - 40)) = 172
                                      2
\stopcodebox
\stopANSWER

%e11.3-5
\startEXERCISE
定義一個從有限集合 U 到有限集合 B 上的散列函數族 \m{{\cal H}} 爲 {\EMP \m{\epsilon} 全域}的,
則必須滿足:
如果對 U 中任意兩個不同元素 k 和 l,都有:
\startformula
\Pr\{h(k)=h(l)\}\le\epsilon
\stopformula
其中概率是相對從函數族 \m{{\cal H}} 中隨機抽取的散列函數 \m{h} 而言的。
證明:一個 \m{\epsilon} 全域的散列函數族必定滿足:
\startformula
\epsilon \ge \frac{1}{|B|} - \frac{1}{|U|}
\stopformula
\stopEXERCISE

\startANSWER
令 \m{n=|U|}、 \m{m=|B|}。
對於 B 中的第 i 個槽位,其元素個數爲 \m{x_i},
則 \m{C(x_i, 2) = x_i * (x_i - 1) / 2} 種衝突。
則衝突總數爲:
\startformula\startmathalignment
\NC \sum_{i=1}^{m}x_i (x_i - 1)/2
     =\NC \frac{1}{2}\sum_{i=1}^{m}x_i^2 - \frac{1}{2}\sum_{i=1}^{m}x_i \NR
\NC\ge\NC \frac{n^2}{2m} - \frac{n}{2}
\stopmathalignment\stopformula

因此:
\startformula\startmathalignment
\NC \Pr\{h(k)=h(l)\}
    \ge \NC \frac{\frac{n(n-m)}{2m}}{C(n,2)} \NR
\NC =   \NC \frac{n-m}{m(n-1)} \NR
\NC \ge \NC \frac{n-m}{mn} \NR
\NC =   \NC \frac{1}{m} - \frac{1}{n} \NR
\NC =   \NC \frac{1}{|B|} - \frac{1}{|U|} \NR
\stopmathalignment\stopformula
\stopANSWER

%e11.3-6
\startEXERCISE\DIFFICULT
設 U 爲 n 元組集合,其中所有元素均取自 \m{\integers_p},且 \m{B=\integers_p},
其中 p 爲素數。
對於一個取自 U 的輸入 n 元組 \m{\langle a_0,a_1,\ldots,a_{n-1}\rangle},
定義其上的散列函數 \m{h_b: U\rightarrow B(b\in \integers_p)} 爲:
\startformula
h_b(\langle a_0, a_1, \ldots, a_{n-1} \rangle) =
   \left(\sum_{j=0}^{n-1} a_j b^j \right) \mod p
\stopformula
令 \m{{\cal H} = \{h_b : b \in \integers_p\}}。
根據上一個練習種 \m{\epsilon} 全域的定義,
證明 \m{{\cal H}} 是 \m{((n-1)/p)} 全域的。
(\hint 參考\inexercise[model_zero])
\stopEXERCISE

\startANSWER
令 \m{k=\langle k_0,k_1,\ldots,k_{n-1}\rangle \in U},
 \m{l=\langle l_0,l_1,\ldots,l_{n-1}\rangle \in U},
且 \m{k_i \ne l_i}, \m{b\in \mathbb{Z}_p}。
則:
\startformula\startmathalignment
\NC \Pr\{h_b(k)=h_b(l)\}
   = \NC \Pr\{h_b(k)-h_b(l)=0\} \NR
\NC= \NC \Pr\{\sum_{j=0}^{n-1}(k_j-l_j)b^j = 0\} \NR
\stopmathalignment\stopformula
其中 \m{\sum_{j=0}^{n-1}(k_j-l_j)b^j = 0} 爲 \m{n-1} 次多項式,最多有 \m{n-1} 個解。
因此:
\startformula
\Pr\{h_b(k)=h_b(l)\} \le \frac{n-1}{p} \qquad b \in \integers_p
\stopformula
所以 \m{{\cal H}} 是 \m{((n-1)/p)} 全域的。
\stopANSWER

\stopsection
