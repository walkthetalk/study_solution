\startsection[
  title={Standard and slack forms},
]

%e29.1-1
\startEXERCISE
如果將式(29.24)~(29.28)中的線性規劃表示成式(29.19)~(29.21)中的緊湊記號形式,
則 \m{n}、 \m{m}、 \m{A}、 \m{b} 和 \m{c} 分別是什麼?
附:

\startformula\startmathalignment[n=6,
  align={right,right,right,middle,left,right}]
\NC 2x_1 - \NC 3x_2 + \NC 3x_3 \NC     \NC    \NC \qquad (29.24) \NR
\NC  x_1 + \NC  x_2 - \NC  x_3 \NC \le \NC  7 \NC \qquad (29.25) \NR
\NC -x_1 - \NC  x_2 + \NC  x_3 \NC \le \NC -7 \NC \qquad (29.26) \NR
\NC  x_1 - \NC 2x_2 + \NC 2x_3 \NC \le \NC  4 \NC \qquad (29.27) \NR
\NC  x_1,  \NC  x_2,  \NC  x_3 \NC \ge \NC  0 \NC \qquad (29.28) \NR
\NC        \NC        \NC c^Tx \NC     \NC    \NC \qquad (29.19) \NR
\NC        \NC        \NC  Ax  \NC \le \NC  b \NC \qquad (29.20) \NR
\NC        \NC        \NC   x  \NC \ge \NC  0 \NC \qquad (29.21) \NR
\stopmathalignment\stopformula
\stopEXERCISE

\startANSWER
\startformula\startmathalignment
\NC n \NC = 3 \NR

\NC m \NC = 3 \NR

\NC A \NC = \left[\startmatrix
\NC 1 \NC 1 \NC -1 \NR
\NC -1 \NC -1 \NC 1 \NR
\NC 1 \NC -2 \NC 2 \NR
\stopmatrix\right] \NR

\NC b \NC = \left[\startmatrix
\NC 7 \NR
\NC -7 \NR
\NC 4 \NR
\stopmatrix\right] \NR

\NC c \NC = \left[\startmatrix
\NC 2 \NR
\NC -3 \NR
\NC 3 \NR
\stopmatrix\right] \NR
\stopmathalignment\stopformula
\stopANSWER

%e29.1-2
\startEXERCISE
請給出式(29.24)~(29.28)中線性規劃的三個可行解。
每個解的目標值是多少?
\stopEXERCISE

\startANSWER
\TODO{略。}
\stopANSWER

%e29.1-3
\startEXERCISE
在式(29.38)~(29.41)的鬆弛型中, \m{N}、 \m{B}、 \m{A}、 \m{b}、 \m{c} 和 \m{v} 是什麼?
附:

\startformula\startmathalignment[n=6,
  align={right,right,right,right,right,right}]
\NC   z = \NC      \NC 2x_1 - \NC 3x_2 + \NC 3x_3 \NC \qquad (29.38) \NR
\NC x_4 = \NC  7 - \NC  x_1 - \NC  x_2 + \NC  x_3 \NC \qquad (29.39) \NR
\NC x_5 = \NC -7 + \NC  x_1 + \NC  x_2 - \NC  x_3 \NC \qquad (29.40) \NR
\NC x_6 = \NC  4 - \NC  x_1 + \NC 2x_2 - \NC 2x_3 \NC \qquad (29.41) \NR
\stopmathalignment\stopformula
\stopEXERCISE

\startANSWER
\TODO{略。}
\stopANSWER

%e29.1-4
\startEXERCISE
將下面線性規劃轉換成標準型:

最小化 \m{2x_1 + 7x_2 + x_3}

滿足約束:
\startformula\startmathalignment[n=7,
  align={right,right,right,right,right,right, middle}]
\NC  x_1 \NC   \NC     \NC - \NC x_3 \NC = \NC 7 \NR
\NC 3x_1 \NC + \NC x_2 \NC   \NC     \NC \ge \NC 24 \NR
\NC      \NC   \NC x_2 \NC   \NC     \NC \ge \NC 0 \NR
\NC      \NC   \NC     \NC   \NC x_3 \NC \le \NC 0 \NR
\stopmathalignment\stopformula
\stopEXERCISE

\startANSWER
\TODO{略。}
\stopANSWER

%e29.1-5
\startEXERCISE
將下面線性規劃轉換成鬆弛型:

最小化 \m{2x_1 - 6x_3}

滿足約束:
\startformula\startmathalignment[n=7,
  align={right,right,right,right,right,right, middle}]
\NC  x_1 \NC + \NC  x_2 \NC - \NC  x_3 \NC \le \NC 7 \NR
\NC 3x_1 \NC - \NC  x_2 \NC   \NC      \NC \ge \NC 8 \NR
\NC -x_1 \NC + \NC 2x_2 \NC + \NC 3x_3 \NC \ge \NC 0 \NR
\NC      \NC x_1, \NC x_2, \NC x_3 \NC \NC \ge \NC 0 \NR
\stopmathalignment\stopformula
其中基本變量和非基本變量是什麼?
\stopEXERCISE

\startANSWER
\TODO{略。}
\stopANSWER

%e29.1-6
\startEXERCISE
說明下面線性規劃是不可解的:

最大化 \m{3x_1 - 2x_2}

滿足約束:
\startformula\startmathalignment[n=5,
  align={right,right,right,right,right}]
\NC   x_1 \NC + \NC  x_2 \NC \le \NC 2 \NR
\NC -2x_1 \NC - \NC 2x_2 \NC \le \NC -10 \NR
\NC      \NC x_1, \NC x_2, \NC \ge \NC 0 \NR
\stopmathalignment\stopformula
\stopEXERCISE

\startANSWER
\TODO{略。}
\stopANSWER

%e29.1-7
\startEXERCISE
說明下面線性規劃是無界的:

最大化 \m{x_1 - x_2}

滿足約束:
\startformula\startmathalignment[n=5,
  align={right,right,right,right,right}]
\NC -2x_1 \NC + \NC  x_2 \NC \le \NC -1 \NR
\NC  -x_1 \NC - \NC 2x_2 \NC \le \NC -2 \NR
\NC      \NC x_1, \NC x_2, \NC \ge \NC 0 \NR
\stopmathalignment\stopformula
\stopEXERCISE

\startANSWER
\TODO{略。}
\stopANSWER

%e29.1-8
\startEXERCISE
假設有一個 \m{n} 個變量和 \m{m} 個約束的一般線性規劃,
並且假設將其轉換稱標準型。
請給出所得線性規劃中變量和約束數目的一個上界。
\stopEXERCISE

\startANSWER
\TODO{略。}
\stopANSWER

%e29.1-9
\startEXERCISE
請給出一個線性規劃的例子,
其中可行區域是無界的,
但最優目標值是有界的。
\stopEXERCISE

\startANSWER
\TODO{略。}
\stopANSWER

\stopsection
