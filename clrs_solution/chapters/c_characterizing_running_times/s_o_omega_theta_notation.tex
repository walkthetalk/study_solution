\startsection[
  title={\m{O}, \m{\Omega}, \m{\Theta} notation},
]

\startEXERCISE
重新分析插入排序的下界,使得輸入的大小不必爲 $3$ 的倍數。
\stopEXERCISE
\startANSWER
對於完全逆序的數列,
其迭代次數和爲 $1+2+3+\dots+(n-1)=n (n-1)/2$,
對於數列大小沒有要求。
\stopANSWER

\startEXERCISE
回顧我們是如何分析插入排序的運行時間的,
用類似的方法分析\refexercise{selection_sort}中選擇排序的運行時間。
\stopEXERCISE
\startANSWER
構造數列 $\langle 2,3,4,5,\cdots,n,1\rangle$,
其運行時間(下界)爲 $\sum_{i=n-1}^{1}i = n(n-1)/2$。
上界爲 $O(n^2)$,
下界依舊是 $\Omega(n^2)$,
緊界爲 $\Theta(n^2)$。
\stopANSWER

\startEXERCISE
有小數 $\alpha$ 滿足 $0<\alpha<1$。
數列 $A=\langle A_1,A_2,\cdots,A_n\rangle$,
其中最大的 $\alpha n$ 個數位於前 $\alpha n$ 個位置上。
用此數列重新分析插入排序的下界。
對於 $\alpha$ 還需要什麼限制?
最大的 $\alpha n$ 個值必須
穿越中間所有 $(1-2\alpha)n$ 個位置,
要想使穿越次數最大, $\alpha$ 應取何值?
\stopEXERCISE
\startANSWER
下界爲 $\alpha n \times (1-2\alpha)n$。

$\alpha$ 須使得 $\alpha n$ 爲整數。

$\alpha\times(1-2\alpha)$ 爲二次多項式,
$\alpha\times(1-2\alpha)=0$ 的兩個根爲 $0$ 和 $1/2$,
則 $\alpha$ 爲 $1/4$ 即爲所求。
\stopANSWER

\stopsection
