\startsection[
  title={Linked-list representation of disjoint sets},
]

%e21.2-1
\startEXERCISE
使用鏈表表示和加權合併啓發式策略,
寫出 \ALGO{MAKE-SET}、 \ALGO{FIND-SET} 和 \ALGO{UNION} 操作的僞碼。
並指定你在集合對象和表對象中所使用的對象。
\stopEXERCISE

\startANSWER
\TODO{略。}
\stopANSWER

%e21.2-2
\startEXERCISE[exercise:21.2-2]
給出下面程序的結果數據結構,
並回答該程序中 \ALGO{FIND-SET} 操作返回的答案。
這裏使用加權合併啓發式策略的鏈表表示。

\startCLRS
for i=1 to 16
	MAKE-SET(x[i])
for i=1 to 15 by 2
	UNION(x[i], x[i+1])
for i=1 to 13 by 4
	UNION(x[i], x[i+2])
UNION(x[1], x[5])
UNION(x[11], x[13])
UNION(x[1], x[10])
FIND-SET(x[2])
FIND-SET(x[9])
\stopCLRS

假定如果包含 \m{x_i} 和 \m{x_j} 的集合有相同的大小,
則 \ALGO{UNION(x_i, x_j)} 表示將 \m{x_j} 所在的表鏈接到 \m{x_i} 所在的表後。
\stopEXERCISE

\startANSWER
兩個 \ALGO{FIND-SET} 返回的都是第一個集合。
\stopANSWER

%e21.2-3
\startEXERCISE
對定理 21.1 的整體證明進行改造,
得到使用鏈表表示和加權合併啓發式策略下的 \ALGO{MAKE-SET} 和 \ALGO{FIND-SET} 的攤還時間上界爲 \m{O(1)},
以及 \ALGO{UNION} 的攤還時間上界爲 \m{O(\lg n)}。
\stopEXERCISE

\startANSWER
\ALGO{MAKE-SET} 只是初始化一個空集合, \ALGO{FIND-SET} 只是返回節點的父指針,
兩者攤還時間上界爲 \m{O(1)}。

\ALGO{UNION} 的總時間爲 \m{O(n\lg n)},因此其攤還時間上界爲 \m{O(\lg n)}。
\stopANSWER

%e21.2-4
\startEXERCISE
請給出圖 21-3 所示操作序列的一個運行時間的漸進緊確界,
假定使用鏈表表示和加權合併啓發式策略。
附圖 21-3:

\startxtable[
    option=max,
    align={middle,lohi},
    split=yes,
    header=repeat,
    footer=repeat,
    offset=.25em,
]

% head
\startxtablehead[frame=off,bottomframe=on]
\startxrow[foregroundstyle=bold,]
  \xcell{Operation}\xcell{Number of objects updated}
\stopxrow
\stopxtablehead

% body
\startxtablebody[frame=off]
\startxrow \xcell{\ALGO{MAKE-SET(x_1)}}\xcell{1} \stopxrow
\startxrow \xcell{\ALGO{MAKE-SET(x_2)}}\xcell{1} \stopxrow
\startxrow \xcell{\m{\vdots}}\xcell{\m{\vdots}} \stopxrow
\startxrow \xcell{\ALGO{MAKE-SET(x_n)}}\xcell{1} \stopxrow
\startxrow \xcell{\ALGO{UNION(x_2,x_1)}}\xcell{1} \stopxrow
\startxrow \xcell{\ALGO{UNION(x_3,x_2)}}\xcell{2} \stopxrow
\startxrow \xcell{\ALGO{UNION(x_4,x_3)}}\xcell{3} \stopxrow
\startxrow \xcell{\m{\vdots}}\xcell{\m{\vdots}} \stopxrow
\startxrow \xcell{\ALGO{UNION(x_n,x_{n-1})}}\xcell{\m{n-1}} \stopxrow
\stopxtablebody

\stopxtable

\stopEXERCISE

\startANSWER
\m{O(n)}
\stopANSWER

%e21.2-5
\startEXERCISE
Gompers 教授猜想也許有可能在每個集合對象中僅使用一個指針,
而不是兩個指針(\m{head} 和 \m{tail}),
同時仍然保留每個鏈表元素的 2 個指針。
請說明教授的猜想是有道理的,
並通過描述如何使用一個鏈表來表示每個集合,
使得每個操作與本章中描述的操作有相同的運行時間,來加以解釋。
同時描述這些操作是如何工作的。
你的方法應該允許使用加權合併啓發式策略,
並與本節所描述的有相同效果。
(\hint 使用一個鏈表的尾作爲集合的代表。)
\stopEXERCISE

\startANSWER
\TODO{略。}
\stopANSWER

%e21.2-6
\startEXERCISE
如何對 \ALGO{UNION} 過程做一個簡單的改動,
在採用鏈表表示中拿掉讓集合對象的 \m{tail} 指針總是指向每個表的最後一個對象的要求。
無論是否使用加權合併啓發式策略,
這個修改不應該改變 \ALGO{UNION} 過程的漸進運行時間。
(\hint 而不是把一個表連接到另一個表後面,將他們拼接在一起。)
\stopEXERCISE

\startANSWER
\TODO{略。}
\stopANSWER

\stopsection
