\subsection[section:atomicFunc]{原子函式}

OpenCL C 編程語言實現了\reftab{atomicFunc}中所列函式,
可用來對位於 \cqlf{__global} 或 \cqlf{__local} 內存中的 32 位帶符號、
無符號整數以及單精度浮點數\footnote{
只有 \capi{atomic_xchg} 才支持單精度浮點數據型別。}進行原子操作。

\startnotepar
OpenCL 1.0 規範的節 9.5 和節 9.6 中列有如下擴展:
\startigBase
\item \clext{cl_khr_global_int32_base_atomics}
\item \clext{cl_khr_global_int32_extended_atomics}
\item \clext{cl_khr_local_int32_base_atomics}
\item \clext{cl_khr_local_int32_extended_atomics}
\stopigBase
其中所定義的帶有前綴 \capi{atom_} 的內建原子函式也在支持之列。
\stopnotepar

\placetable[here,split][tab:atomicFunc]
{內建原子函式}
{% atomic_add
\startbuffer[funcproto:atomic_add]
int atomic_add (
	volatile __global int *p,
	int val)
unsigned int atomic_add (
	volatile __global unsigned int *p,
	unsigned int val)

int atomic_add (
	volatile __local int *p,
	int val)
unsigned int atomic_add (
	volatile __local unsigned int *p,
	unsigned int val)
\stopbuffer
\startbuffer[funcdesc:atomic_add]
讀取 \carg{p} 所指向的 32 位值(記為 \math{old})。
計算 \math{(old + \marg{val})} 並將結果存儲到 \carg{p} 所指位置中。
此函式返回 \math{old}。
\stopbuffer

% atomic_sub
\startbuffer[funcproto:atomic_sub]
int atomic_sub (
	volatile __global int *p,
	int val)
unsigned int atomic_sub (
	volatile __global unsigned int *p,
	unsigned int val)

int atomic_sub (
	volatile __local int *p,
	int val)
unsigned int atomic_sub (
	volatile __local unsigned int *p,
	unsigned int val)
\stopbuffer
\startbuffer[funcdesc:atomic_sub]
讀取 \carg{p} 所指向的 32 位值(記為 \math{old})。
計算 \math{(old - \marg{val})} 並將結果存儲到 \carg{p} 所指位置中。
此函式返回 \math{old}。
\stopbuffer

% atomic_xchg
\startbuffer[funcproto:atomic_xchg]
int atomic_xchg (
	volatile __global int *p,
	int val)
unsigned int atomic_xchg (
	volatile __global unsigned int *p,
	unsigned int val)
float atomic_xchg (
	volatile __global float *p,
	float val)

int atomic_xchg (
	volatile __local int *p,
	int val)
unsigned int atomic_xchg (
	volatile __local unsigned int *p,
	unsigned int val)
float atomic_xchg (
	volatile __local float *p,
	float val)
\stopbuffer
\startbuffer[funcdesc:atomic_xchg]
將位置 \carg{p} 中所存儲的值 \math{old} 和 \carg{val} 中的新值相互交換。
返回 \math{old}。
\stopbuffer

% atomic_inc
\startbuffer[funcproto:atomic_inc]
int atomic_inc (volatile __global int *p)
unsigned int atomic_inc (
	volatile __global unsigned int *p)

int atomic_inc (volatile __local int *p)
unsigned int atomic_inc (
	volatile __local unsigned int *p)
\stopbuffer
\startbuffer[funcdesc:atomic_inc]
讀取 \carg{p} 所指向的 32 位值(記為 \math{old})。
計算 \math{(old+1)} 並將結果存儲到 \carg{p} 所指位置中。
此函式返回 \math{old}。
\stopbuffer

% atomic_dec
\startbuffer[funcproto:atomic_dec]
int atomic_dec (volatile __global int *p)
unsigned int atomic_dec (
	volatile __global unsigned int *p)

int atomic_dec (volatile __local int *p)
unsigned int atomic_dec (
	volatile __local unsigned int *p)
\stopbuffer
\startbuffer[funcdesc:atomic_dec]
讀取 \carg{p} 所指向的 32 位值(記為 \math{old})。
計算 \math{(old-1)} 並將結果存儲到 \carg{p} 所指位置中。
此函式返回 \math{old}。
\stopbuffer

% atomic_cmpchg
\startbuffer[funcproto:atomic_cmpxchg]
int atomic_cmpxchg (
	volatile __global int *p,
	int cmp, int val)
unsigned int atomic_cmpxchg (
	volatile __global unsigned int *p,
	unsigned int cmp,
	unsigned int val)

int atomic_cmpxchg (
	volatile __local int *p,
	int cmp,
	int val)
unsigned int atomic_cmpxchg (
	volatile __local unsigned int *p,
	unsigned int cmp,
	unsigned int val)
\stopbuffer
\startbuffer[funcdesc:atomic_cmpxchg]
讀取 \carg{p} 所指向的 32 位值(記為 \math{old})。
計算 \math{(old == cmp) ? val : old} 並將結果存儲到 \carg{p} 所指位置中。
此函式返回 \math{old}。
\stopbuffer

% atomic_min
\startbuffer[funcproto:atomic_min]
int atomic_min (
	volatile __global int *p,
	int val)
unsigned int atomic_min (
	volatile __global unsigned int *p,
	unsigned int val)

int atomic_min (
	volatile __local int *p,
	int val)
unsigned int atomic_min (
	volatile __local unsigned int *p,
	unsigned int val)
\stopbuffer
\startbuffer[funcdesc:atomic_min]
讀取 \carg{p} 所指向的 32 位值(記為 \math{old})。
計算 \math{\mapiemp{min}(old, \marg{val})} 並將結果存儲到 \carg{p} 所指位置中。
此函式返回 \math{old}。
\stopbuffer

% atomic_max
\startbuffer[funcproto:atomic_max]
int atomic_max (
	volatile __global int *p,
	int val)
unsigned int atomic_max (
	volatile __global unsigned int *p,
	unsigned int val)

int atomic_max (
	volatile __local int *p,
	int val)
unsigned int atomic_max (
	volatile __local unsigned int *p,
	unsigned int val)
\stopbuffer
\startbuffer[funcdesc:atomic_max]
讀取 \carg{p} 所指向的 32 位值(記為 \math{old})。
計算 \math{\mapiemp{max}(old, \marg{val})} 並將結果存儲到 \carg{p} 所指位置中。
此函式返回 \math{old}。
\stopbuffer

% atomic_and
\startbuffer[funcproto:atomic_and]
int atomic_and (
	volatile __global int *p,
	int val)
unsigned int atomic_and (
	volatile __global unsigned int *p,
	unsigned int val)

int atomic_and (
	volatile __local int *p,
	int val)
unsigned int atomic_and (
	volatile __local unsigned int *p,
	unsigned int val)
\stopbuffer
\startbuffer[funcdesc:atomic_and]
讀取 \carg{p} 所指向的 32 位值(記為 \math{old})。
計算 \math{(old \mcmm{&} \marg{val})} 並將結果存儲到 \carg{p} 所指位置中。
此函式返回 \math{old}。
\stopbuffer

% atomic_or
\startbuffer[funcproto:atomic_or]
int atomic_or (
	volatile __global int *p,
	int val)
unsigned int atomic_or (
	volatile __global unsigned int *p,
	unsigned int val)

int atomic_or (
	volatile __local int *p,
	int val)
unsigned int atomic_or (
	volatile __local unsigned int *p,
	unsigned int val)
\stopbuffer
\startbuffer[funcdesc:atomic_or]
讀取 \carg{p} 所指向的 32 位值(記為 \math{old})。
計算 \math{(old \mcmm{|} \marg{val})} 並將結果存儲到 \carg{p} 所指位置中。
此函式返回 \math{old}。
\stopbuffer

% atomic_xor
\startbuffer[funcproto:atomic_xor]
int atomic_xor (
	volatile __global int *p,
	int val)
unsigned int atomic_xor (
	volatile __global unsigned int *p,
	unsigned int val)

int atomic_xor (
	volatile __local int *p,
	int val)
unsigned int atomic_xor (
	volatile __local unsigned int *p,
	unsigned int val)
\stopbuffer
\startbuffer[funcdesc:atomic_xor]
讀取 \carg{p} 所指向的 32 位值(記為 \math{old})。
計算 \math{(old \mcmm{^} \marg{val})} 並將結果存儲到 \carg{p} 所指位置中。
此函式返回 \math{old}。
\stopbuffer


% begin table
\startCLFD
\clFD{atomic_add}
\clFD{atomic_sub}
\clFD{atomic_xchg}
\clFD{atomic_inc}
\clFD{atomic_dec}
\clFD{atomic_cmpxchg}
\clFD{atomic_min}
\clFD{atomic_max}
\clFD{atomic_and}
\clFD{atomic_or}
\clFD{atomic_xor}
\stopCLFD

}
