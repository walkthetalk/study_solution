\startcomponent c_fibonacci_heaps

\startchapter[
  title={Fibonacci Heaps},
]

\startsection[
  title={Structure of Fibonacci heaps},
]
\stopsection

\startsection[
  title={Mergeable-heap operations},
]

%e19.2-1
\startEXERCISE
給出圖 19-4(m)中的 Fibonacci 堆調用 \ALGO{FIB-HEAP-EXTRACT-MIN} 後得到的 Fibonacci 堆。
附圖 19-4(m):

\externalfigure[output/e19_2_1-2]
\stopEXERCISE

\startANSWER
\externalfigure[output/e19_2_1-2]
\stopANSWER

\stopsection

\startsection[
  title={Decreasing a key and deleting a node},
  reference=section:fib_heap_delete,
]

%e19.3-1
\startEXERCISE
假定 Fibonacci 堆中一個根 \m{x} 被標記了。
解釋 \m{x} 是如何成爲一個被標記的根的。
試說明 \m{x} 是否被標記對分析並沒有影響,
即使他沒有丟失過孩子。
\stopEXERCISE

\startANSWER
\TODO{略。}
\stopANSWER

%e19.3-2
\startEXERCISE
使用聚合分析來證明 \ALGO{FIB-HEAP-DECREASE-KEY} 的 \m{O(1)} 攤還時間是每一個操作的平均代價。
\stopEXERCISE

\startANSWER
\TODO{略。}
\stopANSWER

\startANSWER
\stopANSWER

\stopsection

\startsection[
  title={Bounding the maximum degree},
]

%e19.4-1
\startEXERCISE
Pinocchio 教授聲稱一個 \m{n} 節點的 Fibonacci 堆的高度是 \m{O(\lg n)} 的。
對於任意的正整數 \m{n},
試給出經過一系列 Fibonacci 堆操作後,
可以創建出一個 Fibonacci 堆,
該堆僅僅包含一棵具有 \m{n} 個節點的線性鏈的樹,
以此來說明該教授是錯誤的。
\stopEXERCISE

\startANSWER
\stopANSWER

%e19.4-2
\startEXERCISE
假定對級聯切斷操作進行推廣,
對於某個整數常數 \m{k},
只要一個節點失去了他的第 \m{k} 個孩子,
就將其從他的父節點上剪切掉(\refsection{fib_heap_delete} 中爲 \m{k=2} 的情形)。
 \m{k} 取什麼值時,有 \m{D(n)=O(\lg n)}。
\stopEXERCISE

\startANSWER
\TODO{略。}
\stopANSWER

\stopsection

\startsubject[
  title={Problems},
]

%p19-1
\startPROBLEM
(Alternative implementation of deletion)
Pisano 教授提出了下面的 \ALGO{FIB-HEAP-DELETE} 過程的一個變種,
聲稱如果刪除的節點不是由 \m{H.min} 指向的節點,那麼程序運行得更快。

\CLRSH{(PISANO-DELETE(H, x)}
\startCLRS
if x == H.min
	FIB-HEAP-EXTRACT-MIN(H)
else
	y = x.p
	if y != NIL
		CUT(H, x, y)
		CASCADING-CUT(H, y)
	add x's child list to the root list of H
	remove x from the root list of H
\stopCLRS

\startigBase[a]\startitem
該教授的聲稱是基於第 7 行可以在 \m{O(1)} 實際時間內完成的這一假設,
他的程序可以運行得更快。該假設有什麼問題嗎?
\stopitem\stopigBase

\startANSWER
運行時間應爲 \m{O(x.degree)},即 \m{O(\lg n)}。
因爲雖然整個鏈表可以一次性完成插入,但是各節點的屬性 \m{p} 卻需要獨立修改。
\stopANSWER

\startigBase[continue]\startitem
當 \m{H.min} 沒有指向 \m{x} 時,給出 \ALGO{PISANO-DELETE} 實際時間的一個好的上界。
你給出的上界應該以 \m{x.degree} 和調用 \ALGO{CASCADING-CUT} 的次數 \m{c} 這兩個參數表示。
\stopitem\stopigBase

\startANSWER
\startformula
O(x.degree + c)
\stopformula
\stopANSWER

\startigBase[continue]\startitem
假定調用 \ALGO{PISANO-DELETE(H, x)},並設 \m{H'} 是執行後得到的 Fibonacci 堆。
假定節點 \m{x} 不是一個根,
用 \m{x.degree}、 \m{c}、 \m{t(H)} 和 \m{m(H)} 來表示 \m{H'} 勢的界。
\stopitem\stopigBase

\startANSWER
\startformula
\Phi(H') = [t(H) + x.degree + c] + 2[m(H) - c + 2]
\stopformula
\stopANSWER

\startigBase[continue]\startitem
證明: \ALGO{PISANO-DELETE} 的攤還時間漸進值不會比 \ALGO{FIB-HEAP-DELETE} 的攤還時間更好,
即使 \m{x\ne H.min} 也是如此。
\stopitem\stopigBase

\startANSWER
\startformula
O(x.degree + c) + x.degree + 4 - c = O(x.degree + c) = O(\lg n) \gt O(1)
\stopformula
\stopANSWER

\stopPROBLEM

%p19-2
\startPROBLEM
(Binomial trees and binomial heaps)
二項樹 \m{B_k} 是一棵遞歸定義的有序樹(參看 B.5.2 節)。
如圖 19-6(a)所示,二項樹 \m{B_0} 僅包含一個節點。
二項樹 \m{B_k} 是由兩個二項樹 \m{B_{k-1}} 組成的,
這兩棵樹按照一棵樹的根是另一棵樹根的最左孩子的方式鏈接。
圖 19-6(b)所示爲二項樹 \m{B_0} 到 \m{B_4}。

附圖 19-6(a):

\midaligned{\externalfigure[output/p19_2-1][scale=600]}

附圖 19-6(b):

\midaligned{\externalfigure[output/p19_2-2][scale=600]}

附圖 19-6(c):

\midaligned{\externalfigure[output/p19_2-3][scale=600]}

\startigBase[a]\startitem
對於二項樹 \m{B_k},證明:
\stopitem
  \startigBase[n]
  \startitem
  一共有 \m{2^k} 個節點。
  \stopitem

\startANSWER
\startformula\startmathalignment
\NC n_0 \NC = 1 \NR
\NC n_k \NC = n_{k-1} + n_{k-1} = 2 n_{k-1} \NR
\NC n_k \NC = 2^k \NR
\stopmathalignment\stopformula
\stopANSWER

  \startitem
  樹的高度是 \m{k}。
  \stopitem

\startANSWER
\startformula\startmathalignment
\NC h_0 \NC = 0 \NR
\NC h_k \NC = h_{k-1} + 1 \NR
\NC h_k \NC = k \NR
\stopmathalignment\stopformula
\stopANSWER

  \startitem
  對於 \m{i=0,1,\ldots,k},恰有 \m{\binom{k}{i}} 個深度爲 \m{i} 的節點。
  \stopitem

\startANSWER
\startformula\startmathalignment
\NC n(k, i) \NC = n(k-1, i) + n(k-1, i-1) \NR
\NC n(k, 0) \NC = 1 \NR
\NC n(k, i) \NC = \binom{k}{i} \NR
\stopmathalignment\stopformula
\stopANSWER

  \startitem
  根的度數爲 \m{k},他比其他任意節點的度數都大。
  此外,如圖 19-6(c)所示,
  如果把根的孩子從左至右編號爲 \m{k-1}, \m{k-2}, \m{\ldots}, 0,
  那麼孩子 \m{i} 是子樹 \m{B_i} 的根。
  \stopitem

\startANSWER
\startformula\startmathalignment
\NC d_k \NC = d_{k-1} + 1 \NR
\NC d_0 \NC = 0 \NR
\NC d_k \NC = k \NR
\stopmathalignment\stopformula
\stopANSWER
  \stopigBase
\stopigBase

{\EMP 二項堆}(binomial heap) \m{H} 是具備如下性質的二項樹的集合:
\startigNum[n]
\item 每個節點具有一個關鍵字(與 Fibonacci 堆相同)。
\item  \m{H} 中的每一個二項樹遵循最小堆性質。
\item 對於任意的非負整數 \m{k}, \m{H} 中最多有一個二項樹的根的度數爲 \m{k}。
\stopigNum

\startigBase[continue]\startitem
假定一個二項堆 \m{H} 一共有 \m{n} 個節點。
討論 \m{H} 中包含的二項樹與 \m{n} 的二進制表示之間的關係。
並證明 \m{H} 最多由 \m{\lfloor\lg n\rfloor +1} 棵二項樹組成。
\stopitem\stopigBase

\startANSWER
令 \m{n} 的二進制表示爲 \m{a_k,a_{k-1},\ldots,a_1,a_0},
如果 \m{a_i} (\m{0\le i\le k})爲 0,則此堆不包含二項樹 \m{B_i},否則包含二項樹 \m{B_i}。

假定 \m{k} 是最高位,則 \m{2^k \le n < 2^{k+1}}。
令二項樹的數目爲 \m{n},則 \m{n \le k+1}。
即:
\startformula
n \le (k+1) \le \lfloor (\lg n\rfloor + 1)
\stopformula
\stopANSWER

假定按如下方式表述二項堆。
用節 10.4 節中的左孩子、右兄弟方案表示二項堆中的每一棵二項樹。
每個節點包含一個關鍵字,指向他父節點的指針、指向他最左孩子的指針
和指向他右兄弟的指針(這些指針某些情況下是 NIL),
以及他的度數(同 Fibonacci 堆,表示爲有多少個孩子)。
這些根組成了一個單向連接的根鏈表,
並以根的度數從小到大排列。
可以通過一個指向根鏈表第一個節點的指針來訪問二項堆。

\startigBase[continue]\startitem
完整描述如何表示一個二項堆(例如,對屬性進行命名,描述屬性值什麼時候爲 NIL,定義根鏈表是怎麼組織的),
並說明如何用與本章中實現 Fibonacci 堆一樣的方式實現二項堆上同樣的 7 個操作。
每一個操作的最壞時間應該爲 \m{O(\lg n)},
其中 \m{n} 爲二項堆中的節點數目(或對於 \ALGO{UNION} 操作,爲要被合併的兩個二項堆中的節點數)。
 \ALGO{MAKE-HEAP} 操作應爲常數時間。
\stopitem\stopigBase

\startANSWER
\TODO{略。}
\stopANSWER

\startigBase[continue]\startitem
假定僅僅要實現在一個 Fibonacci 堆上的可合併操作
(即並不實現 \ALGO{DECREASE-KEY} 和 \ALGO{DELETE} 操作)。
 Fibonacci 堆中的樹與二項堆中的樹有何相似之處?
有什麼區別?
證明在一個 \m{n} 個節點的 Fibonacci 堆中最大度數最多爲 \m{\lfloor \lg n\rfloor}。
\stopitem\stopigBase

\startANSWER
\TODO{略。}
\stopANSWER

\startigBase[continue]\startitem
McGee 教授提出了一個基於 Fibonacci 堆的新的數據結構。
一個 McGee 堆具有與 Fibonacci 堆相同的結構,
並且只支持可合併操作。
除了插入和合併在最後一步中合併根鏈表外,
其他操作的實現方式均與 Fibonacci 堆中的實現方式相同。
 McGee 堆上個操作的最壞運行時間是多少?
\stopitem\stopigBase

\startANSWER
\TODO{略。}
\stopANSWER

\stopPROBLEM

%p19-3
\startPROBLEM
(More Fibonacci-heap operations)
要擴展 Fibonacci 堆 \m{H} 支持兩個新操作,
要求不改變 Fibonacci 堆其他操作的攤還時間。
\startigBase[a]\startitem
操作 \ALGO{FIB-HEAP-CHANGE-KEY(H, x, k)} 將節點 \m{x} 中關鍵字的值改爲 \m{k}。
給出 \ALGO{FIB-HEAP-CHANGE-KEY} 的一個有效實現,
並分析當 \m{k} 大於、小於或等於 \m{x.key} 時,各情形下的攤還運行時間。
\stopitem\stopigBase

\startANSWER
\TODO{略。}
\stopANSWER

\startigBase[continue]\startitem
給出 \ALGO{FIB-HEAP-PRUNE(H, r)} 的一個有效實現,
該操作從 \m{H} 中刪除 \m{q=\min(r, H.n)} 個節點。
可以選擇任意 \m{q} 個節點來刪除。
試分析你的實現的攤還運行時間。
(\hint 可能需要修改數據結構以及勢函數。)
\stopitem\stopigBase

\startANSWER
\TODO{略。}
\stopANSWER

\stopPROBLEM

%p19-4
\startPROBLEM
(2-3-4 heaps)
第 18 章介紹了 2-3-4 樹,
樹中每個內部節點(而不是根)有 2 個、 3 個或 4 個孩子,
且所有的葉節點有相同的深度。
本問題中,實現支持可合併堆操作的 2-3-4 堆。

2-3-4 堆以下幾點與 2-3-4 樹不同。
在 2-3-4 堆中,僅僅葉節點存儲關鍵字,
並且每個葉節點 \m{x} 僅僅在屬性 \m{x.key} 中存儲一個關鍵字。
葉節點中的關鍵字可能以任意順序存在。
每個內部節點 \m{x} 包含一個值 \m{x.small},
他等於以 \m{x} 爲根的子樹中葉節點存儲的最小的關鍵字。
根 \m{r} 包含一個屬性 \m{r.height},存儲樹的高度。
最後, 2-3-4 堆設計爲存放在主存中,
這樣就不需要讀寫磁盤了。

實現下面的 2-3-4 堆操作。
在一個具有 \m{n} 個元素的 2-3-4 堆上,
(a)~(e)中每一個操作應該在 \m{O(\lg n)} 時間內完成。
(f) 中的 \ALGO{UNION} 操作應該在 \m{O(\lg n)} 時間內完成,
其中 \m{n} 爲輸入的兩個堆元素個數之和。

\startigBase[a]
\item \ALGO{MINIMUM},該操作返回一個指向具有最小關鍵字的葉節點的指針;
\item \ALGO{DECREASE-KEY},該操作將一個給定的葉節點 \m{x} 的關鍵字減小爲給定的值 \m{k\le x.key};
\item \ALGO{INSERT},該操作插入一個關鍵字爲 \m{k} 的葉節點 \m{x};
\item \ALGO{DELETE},該操作刪除一個給定的葉節點 \m{x};
\item \ALGO{EXTRACT-MIN},該操作抽取具有最小關鍵字的葉節點;
\item \ALGO{UNION},該操作合併兩個 2-3-4 堆,並返回一個單獨的 2-3-4 堆,銷燬掉輸入的堆。
\stopigBase

\startANSWER
\TODO{略。}
\stopANSWER

\stopPROBLEM

\stopsubject%Problems

\stopchapter
\stopcomponent
