\startsection[
  title={Linked lists},
]

%e10.2-1
\startEXERCISE
如何在單向鏈表上實現動態集合操作 \ALGO{INSERT}?要求時間爲 \m{O(1)}。 \ALGO{DELETE} 呢?
\stopEXERCISE

\startANSWER
在單項鏈表頭部進行插入操作。
而刪除操作則無法在常數時間內完成,除非傳遞的參數是要刪除節點的前驅,或者可以修改節點的 \m{key} 和衛星數據。
\stopANSWER

%e10.2-2
\startEXERCISE
用單向鏈表 \m{L} 實現棧。要求 \ALGO{PUSH} 和 \ALGO{POP} 的時間仍爲 \m{O(1)}。
\stopEXERCISE

\startANSWER
插入和刪除全部在頭部進行即可。
\stopANSWER

%e10.2-3
\startEXERCISE
用單向鏈表 \m{L} 實現隊列。要求 \ALGO{ENQUEUE} 和 \ALGO{DEQUEUE} 的時間仍爲 \m{O(1)}。
\stopEXERCISE

\startANSWER
入隊在頭部進行,出隊在尾部進行,只是需要哨兵跟蹤鏈表中最後一個元素。
\stopANSWER

%e10.2-4
\startEXERCISE
過程 \ALGO{LIST-SEARCH'} 中的每次循環中都要做兩個測試:
一個是 \m{x \ne L.nil},另一個是 \m{x.key \ne k}。
如何消除對前者的測試?
\stopEXERCISE

\startANSWER
\CLRSH{LIST-SEARCH'(L, k)}
\startCLRS
x = L.nil.next
L.nil.key = k
while x.key ≠ k
	x = x.next
return x
\stopCLRS
\stopANSWER

%e10.2-5
\startEXERCISE
用單向循環鏈表實現字典操作 \ALGO{INSERT}、 \ALGO{DELETE} 和 \ALGO{SEARCH}。
運行時間如何?
\stopEXERCISE

\startANSWER
字典操作參數均爲 \m{key}。
利用哨兵,插入操作時間爲 \m{O(1)};刪除和查找操作時間均爲 \m{O(n)};
\stopANSWER

%e10.2-6
\startEXERCISE
動態集合操作 \m{UNION} 以兩個不相交的集合 \m{S_1} 和 \m{S_2} 作爲輸入,
返回的集合 \m{S=S_1\cup S_2} 含有 \m{S_1} 和 \m{S_2} 中的所有元素。
通常這個操作會銷毀 \m{S_1} 和 \m{S_2}。
如何用鏈表數據結構來實現 \ALGO{UNION}?要求時間爲 \m{O(1)}。
\stopEXERCISE

\startANSWER
用雙向鏈表,只需將 \m{S_2} 的第一個元素作爲 \m{S_1} 的最後一個元素的後繼節點即可。
如果用了哨兵,則需要銷毀 \m{S_2}。
\stopANSWER

%e10.2-7
\startEXERCISE
如何以 \m{\Theta(n)} 時間將含有 \m{n} 個元素的單向鏈表反序,要求不能使用遞迴。
要求除鏈表本身外,只能使用固定大小的存儲空間。
\stopEXERCISE

\startANSWER
\CLRSH{REVERSE(list)}
\startCLRS
prev = list.NIL
cur = list.NIL.next

while cur != list.NIL
	next = cur.next
	cur.next = prev
	prev = cur
	cur = next
\stopCLRS
\stopANSWER

%e10.2-8
\startEXERCISE\DIFFICULT
實現雙向鏈表,要求每個元素只用一個指針 \m{x.np},而不是兩個(\m{next} 和 \m{prev})。
假設所有指針都可視爲 \m{k} 位整數,
同時令 \m{x.np = x.next XOR x.prev},即 \m{x.prev} 和 \m{x.next} 的 \m{k} 位“異或”。
(NIL 的值爲 0。)
要訪問鏈表頭需要哪些信息?
如何實現 \ALGO{SEARCH}、 \ALGO{INSERT} 和 \ALGO{DELETE}?
如何在 \m{O(1)} 時間內反序?
\stopEXERCISE

\startANSWER
如果是通過前驅節點訪問的當前節點,則已經直到了 \m{prev} 的值,這時通過 \m{prev XOR np} 即可得 \m{next} 的值。

如果是通過後繼節點訪問的當前節點,則已經直到了 \m{next} 的值,這時通過 \m{next XOR np} 即可得 \m{prev} 的值。

\ALGO{SEARCH}、 \ALGO{INSERT} 和 \ALGO{DELETE} 的實現與普通雙向鏈表沒什麼區別,
只是增加了計算 \m{prev}、 \m{next} 和 \m{np} 的步驟。

要訪問鏈表只需直到第一個元素或者最後一個元素的指針即可。

反序只需要調換以下 \m{head} 和 \m{tail} 的值即可。
\stopANSWER

\stopsection
