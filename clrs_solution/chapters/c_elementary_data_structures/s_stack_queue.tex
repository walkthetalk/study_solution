\startsection[
  reference=section:stack_and_queue,
  title={Stacks and queues},
]

%e10.1-1
\startEXERCISE
仿照圖 10-1,畫圖表示依次執行操作 \ALGO{PUSH(S, 4)}、 \ALGO{PUSH(S, 1)}、 \ALGO{PUSH(S, 3)}、
 \ALGO{POP(S)}、  \ALGO{PUSH(S, 8)} 和 \ALGO{POP(S)} 每一步的結果,
棧 \m{S} 初始爲空,存儲於數列 \m{S[1..6]} 中。
\stopEXERCISE

\startANSWER
\startcombination[4*2]
{\externalfigure[output/e10_1_1-1]}{a}
{\externalfigure[output/e10_1_1-2]}{b}
{\externalfigure[output/e10_1_1-3]}{c}
{\externalfigure[output/e10_1_1-4]}{d}
{\externalfigure[output/e10_1_1-5]}{e}
{\externalfigure[output/e10_1_1-6]}{f}
{\externalfigure[output/e10_1_1-7]}{g}
\stopcombination
\stopANSWER

%e10.1-2
\startEXERCISE
請說明如何讓兩個棧共用一個數列 \m{A[1..n]},
僅當兩個棧的所有元素數目爲 \m{n} 時才會發生上溢。
 \ALGO{PUSH} 和 \ALGO{POP} 操作要在 \m{O(1)} 時間內完成。
\stopEXERCISE

\startANSWER
其中一個棧從 \m{A[1]} 向 \m{A[n]} 方向增長,
另一個從 \m{A[n]} 向 \m{A[1]} 方向增長。
當兩個棧的棧頂相鄰時才會發生上溢。
\stopANSWER

\startEXERCISE
仿照圖 10-2,畫圖表示在初始爲空的隊列 \m{Q} 上依次執行 \ALGO{ENQUEUE(Q, 4)}、
 \ALGO{ENQUEUE(Q, 1)}、 \ALGO{ENQUEUE(Q, 3)}、 \ALGO{DEQUEUE(Q)}、
 \ALGO{ENQUEUE(Q, 8)} 和 \ALGO{DEQUEUE(Q)} 每步的結果,隊列存儲在數列 \m{Q[1..6]} 中。
\stopEXERCISE

\startANSWER
\startcombination[3*3]
{\externalfigure[output/e10_1_3-1]}{a}
{\externalfigure[output/e10_1_3-2]}{b}
{\externalfigure[output/e10_1_3-3]}{c}
{\externalfigure[output/e10_1_3-4]}{d}
{\externalfigure[output/e10_1_3-5]}{e}
{\externalfigure[output/e10_1_3-6]}{f}
{\externalfigure[output/e10_1_3-7]}{g}
{}{}{}{}
\stopcombination
\stopANSWER

%e10.1-4
\startEXERCISE
重寫 \ALGO{ENQUEUE} 和 \ALGO{DEQUEUE} 以處理上溢和下溢的情況。
\stopEXERCISE

\startANSWER
輔助過程:

\CLRSH{NEXT(Q, i)}
\startCLRS
if i == Q.length
	return 1
else
	return i + 1
\stopCLRS

\CLRSH{EMPTY(Q)}
\startCLRS
if Q.head == Q.tail
	return true
else
	return false
\stopCLRS

\CLRSH{FULL(Q)}
\startCLRS
if NEXT(Q, Q.tail) == Q.head
	return true
else
	return false
\stopCLRS

實現:

\CLRSH{ENQUEUE(Q, x)}
\startCLRS
if FULL(Q)
	error "Queue overflow"
Q[Q.tail] = x
Q.tail = NEXT(Q, Q.tail)
\stopCLRS

\CLRSH{DEQUEUE(Q)}
\startCLRS
if EMPTY(Q)
	error "Queue underflow"
x = Q[Q.head]
Q.head = NEXT(Q, Q.head)
return x
\stopCLRS
\stopANSWER

%e10.1-5
\startEXERCISE
棧只能在同一端插入、刪除元素,隊列則可在一端插入、另一端刪除,
而{\EMP 雙端隊列(deque)}兩端均可插入、刪除元素。
以數列實現雙端隊列,寫出四個過程在雙端隊列的兩端插入、刪除元素,時間均爲 \m{O(1)}。
\stopEXERCISE

\startANSWER
輔助過程:

\CLRSH{PREV(Q, i)}
\startCLRS
if i == 1
	return Q.length
else
	return i - 1
\stopCLRS

實現:

\CLRSH{ENQUEUE_TAIL(Q, x)}
\startCLRS
if FULL(Q)
	error "Queue overflow"
Q[Q.tail] = x
Q.tail = NEXT(Q, Q.tail)
\stopCLRS

\CLRSH{ENQUEUE_HEAD(Q, x)}
\startCLRS
if FULL(Q)
	error "Queue overflow"
Q[Q.head] = x
Q.head = PREV(Q, Q.head)
\stopCLRS

\CLRSH{DEQUEUE_HEAD(Q)}
\startCLRS
if EMPTY(Q)
	error "Queue underflow"
x = Q[Q.head]
Q.head = NEXT(Q, Q.head)
return x
\stopCLRS

\CLRSH{DEQUEUE_TAIL(Q)}
\startCLRS
if EMPTY(Q)
	error "Queue underflow"
x = Q[Q.tail]
Q.tail = PREV(Q, Q.tail)
return x
\stopCLRS
\stopANSWER

%e10.1-6
\startEXERCISE[exercise:two_stack_to_queue]
如何用兩個棧實現隊列?分析隊列操作的運行時間?
\stopEXERCISE

\startANSWER
令兩個棧爲 \m{A} 和 \m{B}, \m{B} 用來入隊, \m{A} 用來出隊。
當 \m{A} 爲空時,將 \m{B} 中所有元素轉移到 \m{A} 中。
入隊時間爲 \m{O(1)};當 \m{A} 不空時,出隊時間爲 \m{O(1)},
否則爲 \m{\Theta(n)},其中 \m{n} 爲 \m{B} 中元素的數目。
\stopANSWER

%e10.1-7
\startEXERCISE
如何用兩個隊列實現棧?分析棧操作的運行時間。
\stopEXERCISE

\startANSWER
兩個隊列,角色不同,一爲 ACTIVE,一爲 INVACTIVE, \ALGO{PUSH} 到 ACTIVE 隊列中,
時間爲 \m{\Theta(1)}。
而 \ALGO{POP} 則需將 ACTIVE 隊列中元素 \ALGO{DEQUEUE} 並 \m{QUEUE} 到 INACTIVE 隊列中,
 ACTIVE 隊列中只留一個元素,最後這個元素 \ALGO{DEQUEUE} 後返回;然後交換兩個隊列的角色。
 \ALGO{POP} 時間爲 \m{\Theta(n)},其中 \m{n} 爲 隊列中元素數目。
\stopANSWER

\stopsection
