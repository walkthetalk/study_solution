\startsubject[
  title={Problems},
]

%p27-1
\startPROBLEM
(Implementing parallel loops using nested parallelism)
考慮下面多線程算法,將兩個 \m{n} 元素數列 \m{A[1..n]} 和 \m{B[1..n]} 相加,
並將結果存放在 \m{C[1..n]} 中:

\CLRSH{SUM-ARRAYS(A,B,C)}
\startCLRS
parallel for i = 1 to A.length
	C[i] = A[i] + B[i]
\stopCLRS

\startigBase[a]\startitem
按照 \ALGO{MAT-VEC-MAIN-LOOP} 的樣式,
使用嵌套並行(spawn 和 sync)改寫 \ALGO{SUM-ARRAYS} 中的並行循環。
分析你的實現的並行度。
\stopitem\stopigBase

\startANSWER
\CLRSH{MAT-VEC-MAIN-LOOP(A,B,C,l,r)}
\startCLRS
if l == r
	c[l] = A[l] + B[l]
mid = (l + r) / 2
spawn MAT-VEC-MAIN-LOOP(A,B,C, l, mid)
MAT-VEC-MAIN-LOOP(A,B,C, mid+1, r)
sync
\stopCLRS

\CLRSH{SUM-ARRAYS(A,B,C)}
\startCLRS
len = A.length
MAT-VEC-MAIN-LOOP(A,B,C, 1, len)
\stopCLRS
\stopANSWER

考慮並行循環的另一種實現方式,其中包含了一個指定的 grainsize:

\CLRSH{SUM-ARRAYS'(A,B,C)}
\startCLRS
n = A.length
grainsize = ?		// to be determined
r = /BTEX \m{\left\lceil n / grainsize \right\rceil} /ETEX
for k = 0 to r-1
	spawn /BTEX \CLRSH{ADD-SUBARRAY(A,B,C,k\cdot grainsize + 1, \min((k+1)\cdot grainsize, n))} /ETEX
sync
\stopCLRS

\CLRSH{ADD-SUBARRAY(A,B,C,i,j)}
\startCLRS
for k = i to j
	C[k] = A[k] + B[k]
\stopCLRS

\startigBase[continue]\startitem
假定置 \m{grainsize = 1},上述實現的並行度是多少?
\stopitem\stopigBase

\startANSWER
\m{T_1 = \Theta(n), T_{\infty} = \Theta(n), T_1/T_{\infty} = 1}。
\stopANSWER

\startigBase[continue]\startitem
給出 \ALGO{SUM-ARRAYS'} 的持續時間公式,用 \m{n} 和 \m{grainsize} 表示,
求解 \m{grainsize},使得並行度最大。
\stopitem\stopigBase

\startANSWER
\m{grainsize = \sqrt{n}, T_{\infty} = n/grainsize + grainsize = 2\sqrt{n}}。
\stopANSWER

\stopPROBLEM

%p27-2
\startPROBLEM
(Saving temporary space in matrix multiplication)
 \ALGO{P-MATRIX-MULTIPLY-RECURSIVE} 的缺點是需要分配一個 \m{n\times n} 的臨時矩陣 \m{T},
不利於 \m{\Theta} 記號中的常數因子。
然而 \ALGO{P-MATRIX-MULTIPLY-RECURSIVE} 有很高的並行度。
例如,如果胡略符號 \m{\Theta} 中的常數因子,
對於 \m{1000\times 1000} 的矩陣相乘,
其並行度接近 \m{1000^3 / 10^2 = 10^7},
因爲 \m{\lg 1000 \approx 10}。
絕大多數並行計算機的處理器數目都遠小於 1000 萬。

\startigBase[a]\startitem
描述一個多線程算法,該算法不需要臨時矩陣 \m{T} 且持續時間以 \m{\Theta(n)} 增長。
(\hint 根據 \ALGO{P-MATRIX-MULTIPLY-RECURSIVE} 中的方法,
計算 \m{C=C+AB},但可以並行初始化 \m{C} 並且,
注意在合適的地方插入 sync 語句。)
\stopitem\stopigBase

\startANSWER
並行初始化 \m{C=0},用時 \m{\Theta(\lg n)},
在第 4 個 spawn \m{c_{11} = c_{11} + a_{11}\cdot b_{11}} 後插入 sync,
 \m{T_{\infty}(n) = 2T_{\infty}(n/2) + \Theta(\lg n) = \Theta(n)}。
\stopANSWER

\startigBase[continue]\startitem
給出該算法的工作量和持續時間的遞歸式,並求解。
\stopitem\stopigBase

\startANSWER
\m{T_1 = \Theta(n^3), T_{\infty} = \Theta(n)}。
\stopANSWER

\startigBase[continue]\startitem
分析該算法的並行度。
忽略符號 \m{\Theta} 中的常數因子,
估算 \m{1000\times 1000} 矩陣上的並行度。
並與 \ALGO{P-MATRIX-MULTIPLY-RECURSIVE} 的並行度進行比較。
\stopitem\stopigBase

\startANSWER
並行度: \m{T_1/T_{\infty} = \Theta(n^2) = 1000^2 = 10^6}。

絕大多數計算機的處理器個數遠小於 100 萬。
\stopANSWER
\stopPROBLEM

%p27-3
\startPROBLEM
(Multithreaded matrix algorithms)
\startigBase[a]\startitem
參考\refsection{solve_system_of_linear_equation} 中的 \ALGO{LU-DECOMPOSITION},
給出其多線程實現的僞碼。使其儘可能並行,分析其工作量、持續時間和並行度。
\stopitem\stopigBase

\startANSWER
\TODO{略。}
\stopANSWER

\startigBase[continue]\startitem
對\refsection{solve_system_of_linear_equation} 中的 \ALGO{LUP-DECOMPOSITION} 做同樣處理。
\stopitem\stopigBase
\stopPROBLEM

\startANSWER
\TODO{略。}
\stopANSWER

\startigBase[continue]\startitem
對\refsection{solve_system_of_linear_equation} 中的 \ALGO{LUP-SOLVE} 做同樣處理。
\stopitem\stopigBase

\startANSWER
\TODO{略。}
\stopANSWER

\startigBase[continue]\startitem
參考等式 28.13 對稱正定矩陣求逆,對其做同樣處理。
\stopitem\stopigBase

\startANSWER
\TODO{略。}
\stopANSWER
\stopPROBLEM

%p27-4
\startPROBLEM
(Multithreading reductions and prefix computations eduction)
數列 \m{x[1..n]} 的 \m{\otimes} {\EMP 歸約}(\m{\otimes}-reduction)
就是 \m{y=x[1]\otimes x[2]\otimes\cdots x[n]} 的值,
其中 \m{\otimes} 滿足結合律。

下面程序以串行方式計算子數列 \m{x[i..j]} 的 \m{\otimes} 歸約:

\CLRSH{REDUCE(x,i,j)}
\startCLRS
y=x[i]
for k = i+1 to j
	y = y /BTEX \m{\otimes}/ETEX x[k]
return y
\stopCLRS

\startigBase[a]\startitem
應用嵌套並行實現一個多線程算法 \ALGO{P-REDUCE},
工作量爲 \m{\Theta(n)},要求持續時間爲 \m{\Theta(\lg n)}。
分析該算法。
\stopitem\stopigBase

\startANSWER
\CLRSH{P-REDUCE(x,i,j)}
\startCLRS
if i == j
	return x[i]
else if i + 1 == j
	return x[i] /BTEX \m{\otimes}/ETEX x[j]
mid = (i + j) / 2
spawn y1 = P-REDUCE(x, i, mid)
y2 = P-REDUCE(x, mid + 1, j)
sync
return y1 /BTEX \m{\otimes}/ETEX y2
\stopCLRS
\stopANSWER

另一個相關問題,在數列 \m{x[1..n]} 上求解 \m{\otimes} {\EMP 前綴計算}
(\m{\otimes}-prefix computation),
有時也稱 \m{\otimes} {\EMP 掃描}(\m{\otimes}-scan),
其中 \m{\otimes} 滿足結合律。
 \m{\otimes} 掃描產生數列 \m{y[1..n]}:
\startformula\startmathalignment
\NC y[1] \NC = x[1] \NR
\NC y[2] \NC = x[1] \otimes x[2] \NR
\NC y[3] \NC = x[1] \otimes x[2] \otimes x[3] \NR
\NC      \NC \vdots \NR
\NC y[n] \NC = x[1] \otimes x[2] \otimes x[3] \otimes \cdots \otimes x[n] \NR
\stopmathalignment\stopformula
也就是說,通過運算法 \m{\otimes} 計算出 \m{x} 所有前綴的”和“。
下面串行 SCAN 過程會計算出一組 \m{\otimes} 前綴:

\CLRSH{SCAN(x)}
\startCLRS
n = x.length
let y[1..n] be a new array
y[1] = x[1]
for i = 2 to n
	y[i] = y[i-1] /BTEX \m{\otimes} /ETEX x[i]
return y
\stopCLRS

遺憾地是,無法直接進行多線程 SCAN。
例如,將 for 循環改爲 parallel for 循環會產生競爭,
因爲循環體的每一步迭代都依賴前一個迭代。
下面的 \ALGO{P-SCAN-1} 可以並行計算 \m{\otimes} 前綴,
儘管十分低效:

\CLRSH{P-SCAN-1(x)}
\startCLRS
n = x.length
let y[1..n] be a new array
P-SCAN-1-AUX(x, y, 1, n)
return y
\stopCLRS

\CLRSH{P-SCAN-1-AUX(x, y, i, j)}
\startCLRS
parallel for l = i to j
	y[l] = P-REDUCE(x, 1, l)
\stopCLRS

\startigBase[continue]\startitem
分析 \ALGO{P-SCAN-1} 的工作量、持續時間和並行度。
\stopitem\stopigBase

\startANSWER
\startformula\startmathalignment
\NC T_1 \NC = \Theta(n^2) \NR
\NC T_{\infty} \NC = \Theta(\lg n) + \Theta(\lg n) = \Theta(\lg n) \NR
\NC T_1 / T_{\infty} \NC = \Theta(n^2/\lg n) \NR
\stopmathalignment\stopformula
\stopANSWER

\stopPROBLEM

\stopsubject%Problems
