\subsection{編譯和鏈接的分離}

OpenCL 1.2 對如何編譯和鏈接\cnglo{program}做了擴展,從而支持:
\startigBase
\item 編譯階段、鏈接階段的分離。
編譯階段可以將\cnglo{program}源碼編譯成二進制目標碼,
而鏈接階段可以將其與其他目標碼鏈接成\cnglo{program}執行體。

\item 內嵌頭檔。
在 OpenCL 1.0 和 1.1 中,對於\cnglo{program}源碼所包含的頭檔而言,
構建選項 \ccmm{-I} 可以用來指定其搜索路徑。
而 OpenCL 1.2 對其進行了擴展,允許目標碼中內嵌頭檔的源碼。

\item 庫。
可以使用鏈接器將目標碼和庫鏈接成\cnglo{program}執行體,或者創建一個新庫。
\stopigBase

\topclfunc{clCompileProgram}

\startCLFUNC
cl_int clCompileProgram(
		cl_program program,
		cl_uint num_devices,
		const cl_device_id *device_list,
		const char *options,
		cl_uint num_input_headers,
		const cl_program *input_headers,
		const char **header_include_names,
		void (CL_CALLBACK *pfn_notify)(
				cl_program program,
				void *user_data),
		void *user_data)
\stopCLFUNC

此函式可以編譯\cnglo{program}源碼。
編譯前會先運行預處理器。
編譯工作的目標\cnglo{device}是
\carg{program} 關聯的所有\cnglo{device}或指定的那些\cnglo{device}。
對於編譯後的二進制目標碼,
可以使用 \capi{clGetProgramInfo}(\carg{program}, \cenum{CL_PROGRAM_BINARIES}, ...) 對其進行查詢,
也可以由 \capi{clCreateProgramWithBinary} 用來創建新的\cnglo{programobj}。

\carg{program} 是一個\cnglo{programobj},即編譯的目標。

\carg{device_list} 指向 \carg{program} 所關聯的\cnglo{device}。
如果 \carg{device_list} 是 \cmacro{NULL},
則編譯動作的目標\cnglo{device}為 \carg{program} 關聯的所有\cnglo{device}。
否則,僅為此清單中的\cnglo{device}實施編譯。

\carg{num_devices} 即 \carg{device_list} 中\cnglo{device}的數目。

\carg{options} 指向一個以 null 終止的字串,用來描述編譯選項。
所支持的選項請參閱\insection[compilerOption]。

\carg{num_input_headers} 即陣列 \carg{input_headers} 中\cnglo{program}的數目。

\carg{input_headers} 是一個陣列,元素為內嵌頭檔的\cnglo{program},
這些\cnglo{program}由 \capi{clCreateProgramWithSource} 創建。

\carg{header_include_names} 也是一個陣列,並與 \carg{input_headers} 一一對應。
每一項就是 \carg{program} 源碼中一個內嵌頭檔的名字。
\carg{input_headers} 中對應項即為包含此頭檔源碼的\cnglo{programobj}。
搜索頭檔時,
內嵌頭檔比編譯選項 \ccmm{-I} 所列目錄(\insection[preprocessorOption])有更高優先級。
如果 \carg{header_include_names} 中有多項名字相同的頭檔,則使用最先搜索到的。

例如,假設有如下\cnglo{program}源碼:
\startclc[indentnext=no]
#include <foo.h>
#include <mydir/myinc.h>

__kernel void image_filter (int n, int m,
			__constant float *filter_weights,
			__read_only image2d_t src_image,
			__write_only image2d_t dst_image)
{
	...
}
\stopclc
這個\cnglo{kernel}包含了兩個頭檔 foo.h 和 mydir/myinc.h。
下面來看怎樣將其傳遞並內嵌到\cnglo{programobj}中:
\startclc
cl_program foo_pg = clCreateProgramWithSource(context,
				1, &foo_header_src, NULL, &err);
cl_program myinc_pg = clCreateProgramWithSource(context,
				1, &myinc_header_src, NULL, &err);

// let’s assume the program source described above is given
// by program_A and is loaded via clCreateProgramWithSource

cl_program input_headers[2] = { foo_pg, myinc_pg };
char * input_header_names[2] = { “foo.h”, “mydir/myinc.h” };
clCompileProgram(program_A,
		0, NULL,	// num_devices & device_list
		NULL,		// compile_options
		2,		// num_input_headers
		input_headers,
		input_header_names,
		NULL, NULL);	// pfn_notify & user_data
\stopclc

\carg{pfn_notify} 是\cnglo{app}所註冊的一個回調函式。
在編譯完成後(無論成功還是失敗)會被調用。
如果 \carg{pfn_notify} 不是 \cmacro{NULL},一旦可以開始編譯,
\capi{clCompileProgram} 就會立刻返回,而不必等待編譯完成。
如果\cnglo{context}、所要編譯的\cnglo{program}源碼、\cnglo{device}清單、
所輸入的頭檔以及相應的\cnglo{program},還有構建選項都是有效的,
並且實施編譯所需的\cnglo{host}和\cnglo{device}資源都可用,那麼就可以開始進行編譯了。
如果 \carg{pfn_notify} 是 \cmacro{NULL},
直到編譯完畢, \capi{clCompileProgram} 才會返回。
對此函式的調用可能是異步的。
\cnglo{app}需要保證此函式是線程安全的。

\carg{user_data} 在調用 \carg{pfn_notify} 時作為引數傳入,可以是 \cmacro{NULL}。

如果執行成功,則 \capi{clCompileProgram} 會返回 \cenum{CL_SUCCESS}。
否則,返回下列錯誤碼之一:
\startigBase
\item \cenum{CL_INVALID_PROGRAM},如果 \carg{program} 無效。

\item \cenum{CL_INVALID_VALUE},
如果 \carg{device_list} 是 \cmacro{NULL} 而 \carg{num_devices} 大於零,
或者 \carg{device_list} 不是 \cmacro{NULL} 而 \carg{num_devices} 等於零。

\item \cenum{CL_INVALID_VALUE},
如果 \carg{num_input_headers} 是零,
而 \carg{header_include_names} 或 \carg{input_headers} 不是 \cmacro{NULL};
或者 \carg{num_input_headers} 不是零,
而 \carg{header_include_names} 或 \carg{input_headers} 是 \cmacro{NULL}。

\item \cenum{CL_INVALID_VALUE},
如果 \carg{pfn_notify} 是 \cmacro{NULL},而 \carg{user_data} 不是 \cmacro{NULL}。

\item \cenum{CL_INVALID_DEVICE},
如果 \carg{device_list} 中的 OpenCL \cnglo{device}不是 \carg{program} 所關聯\cnglo{device}。

\item \cenum{CL_INVALID_COMPILER_OPTIONS},
如果 \carg{options} 所指定的編譯選項無效。

\item \cenum{CL_INVALID_OPERATION},
如果對於 \carg{device_list} 中任一\cnglo{device}而言,
之前調用 \capi{clCompileProgram} 或 \capi{clBuildProgram} 為 \carg{program} 編譯或構建
他所對應的\cnglo{program}執行體的動作還未完成。

\item \cenum{CL_COMPILER_NOT_AVAILABLE},
如果沒有可用的編譯器,即 \cenum{CL_DEVICE_COMPILER_AVAILABLE} 是 \cenum{CL_FALSE},
參見\reftab{cldevquery}。

\item \cenum{CL_COMPILE_PROGRAM_FAILURE},如果編譯\cnglo{program}源碼失敗。
如果 \capi{clCompileProgram} 沒有將此錯誤返回,則在編譯結束後會將其返回。

\item \cenum{CL_INVALID_OPERATION},
如果有附着到 \carg{program} 上的\cnglo{kernelobj}。

\item \cenum{CL_INVALID_OPERATION},如果沒有 \carg{program} 的源碼,
即 \carg{program} 不是由 \capi{clCreateProgramWithSource} 創建的。

\item \cenum{CL_OUT_OF_RESOURCES},如果\scdevfailres。

\item \cenum{CL_OUT_OF_HOST_MEMORY},如果\schostfailres。
\stopigBase

\topclfunc{clLinkProgram}

\startCLFUNC
cl_program clLinkProgram (cl_context context,
			cl_uint num_devices,
			const cl_device_id *device_list,
			const char *options,
			cl_uint num_input_programs,
			const cl_program *input_programs,
			void (CL_CALLBACK *pfn_notify)(
						cl_program program,
						void *user_data),
			void *user_data,
			cl_int *errcode_ret)
\stopCLFUNC

此函式可以將一組目標碼和庫鏈接成執行體,
並創建一個包含這個執行體的\cnglo{programobj}。
對於這個執行體,
可以使用 \capi{clGetProgramInfo}(\carg{program}, \cenum{CL_PROGRAM_BINARIES}, ...) 對其進行查詢,
也可以由 \capi{clCreateProgramWithBinary} 用來創建新的\cnglo{programobj}。

如果 \carg{device_list} 不是 \cmacro{NULL},
則所返回的\cnglo{programobj}關聯的就是 \carg{device_list} 中的\cnglo{device};
否則,關聯的就是 \carg{context} 中的\cnglo{device}。

\carg{context} 是一個 OpenCL \cnglo{context}。

\carg{device_list} 指向 \carg{context} 中的\cnglo{device}。
如果 \carg{device_list} 是 \cmacro{NULL},
則鏈接動作的目標\cnglo{device}是 \carg{context} 中的所有\cnglo{device}(仅限具有编译目标可用的那些)。
否則,僅為此清單中的\cnglo{device}實施鏈接(仅限具有编译目标可用的那些)。

\carg{num_devices} 即 \carg{device_list} 中\cnglo{device}的數目。

\carg{options} 指向一個以 null 終止的字串,用來描述鏈接選項。
所支持的選項請參閱\insection[linkOption]。

\carg{num_input_programs} 即 \carg{input_programs} 中\cnglo{program}的數目。

\carg{input_programs} 是一個陣列,每一項都是一個\cnglo{programobj},
他們是編譯過的二元碼或庫,將被鏈接成\cnglo{program}執行體。
對於目標設備而言,有以下幾種情況:
\startigBase
\item \carg{input_programs} 中的所有\cnglo{program}都包含針對此\cnglo{device}的二元碼或庫。
這種情況下,就為此\cnglo{device}實施鏈接來生成\cnglo{program}執行體。

\item 所有\cnglo{program}都不包含針對此\cnglo{device}的二元碼或庫。
這種情況下,不會為此\cnglo{device}實施鏈接,也不會生成對應的\cnglo{program}執行體。

\item 所有其他情況都會返回錯誤 \cenum{CL_INVALID_OPERATION}。
\stopigBase

\carg{pfn_notify} 是\cnglo{app}所註冊的一個回調函式。
在鏈接完成後(無論成功還是失敗)會被調用。

如果 \carg{pfn_notify} 不是 \cmacro{NULL},一旦可以開始鏈接,
\capi{clLinkProgram} 就會立刻返回,而不必等待鏈接完成。
一旦鏈接完成, \carg{pfn_notify} 就會被調用,
並返回一個\cnglo{programobj}(跟 \capi{clLinkProgram} 返回的一樣)。
\cnglo{app}可以查詢鏈接的狀態以及日誌。
此函式可能會被異步調用。
\cnglo{app}需要保證此函式是線程安全的。

如果 \carg{pfn_notify} 是 \cmacro{NULL},
直到鏈接完畢, \capi{clLinkProgram} 才會返回。

\carg{user_data} 在調用 \carg{pfn_notify} 時作為引數傳入,可以是 \cmacro{NULL}。

如果\cnglo{context}、\cnglo{device}、輸入的\cnglo{program}以及鏈接選項都是有效的,
並且實施鏈接所需的\cnglo{host}和\cnglo{device}資源都可用,那麼就可以開始進行鏈接了。
如果可以開始鏈接, \capi{clLinkProgram} 會返回一個非零的\cnglo{programobj}。

如果 \carg{pfn_notify} 是 \cmacro{NULL},
鏈接成功會將 \carg{errcode_ret} 置為 \cenum{CL_SUCCESS},
鏈接失敗會將其置為 \cenum{CL_LINK_FAILURE}。

如果 \carg{pfn_notify} 不是 \cmacro{NULL}, \capi{clLinkProgram} 不必等到鏈接完成,
一旦可以開始鏈接,就可以將 \carg{errcode_ret} 置為 \cenum{CL_SUCCESS} 並返回。
\carg{pfn_notify} 會返回 \cenum{CL_SUCCESS} 或 \cenum{CL_LINK_FAILURE} 以表示鏈接成功與否。

否則 \capi{clLinkProgram} 會返回 \cmacro{NULL},
並在 \carg{errcode_ret} 中返回相應的錯誤碼。
\cnglo{app}應當查詢此\cnglo{programobj}的鏈接狀態,以判定鏈接是否成功。
返回的錯誤碼可以是:
\startigBase
\item \cenum{CL_INVALID_CONTEXT},如果 \carg{context} 無效。

\item \cenum{CL_INVALID_VALUE},
如果 \carg{device_list} 是 \cmacro{NULL} 而 \carg{num_devices} 大於零,
或者 \carg{device_list} 不是 \cmacro{NULL} 而 \carg{num_devices} 等於零。

\item \cenum{CL_INVALID_VALUE},
如果 \carg{num_input_programs} 是零,且 \carg{input_programs} 是 \cmacro{NULL};
或者 \carg{num_input_programs} 是零,而 \carg{input_programs} 不是 \cmacro{NULL};
或者 \carg{num_input_programs} 不是零,而 \carg{input_programs} 是 \cmacro{NULL}。

\item \cenum{CL_INVALID_PROGRAM},
如果 \carg{input_programs} 中的\cnglo{programobj}無效。

\item \cenum{CL_INVALID_VALUE},
如果 \carg{pfn_notify} 是 \cmacro{NULL},而 \carg{user_data} 不是 \cmacro{NULL}。

\item \cenum{CL_INVALID_DEVICE},
如果 \carg{device_list} 中的任一\cnglo{device}不屬於 \carg{context}。

\item \cenum{CL_INVALID_LINKER_OPTIONS},如果 \carg{options} 中的連接器選項無效。

\item \cenum{CL_INVALID_OPERATION},
如果沒有遵守上面對參數 \carg{input_programs} 的介紹中所列的規則。

\item \cenum{CL_LINKER_NOT_AVAILABLE},如果沒有可用的連接器,
即 \cenum{CL_DEVICE_LINKER_AVAILABLE} 是 \cenum{CL_FALSE},參見\reftab{cldevquery}。

\item \cenum{CL_LINK_PROGRAM_FAILURE},如果鏈接失敗。

\item \cenum{CL_OUT_OF_RESOURCES},如果\scdevfailres。

\item \cenum{CL_OUT_OF_HOST_MEMORY},如果\schostfailres。
\stopigBase

