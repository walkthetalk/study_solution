\startcomponent c_methods_of_analysis
\startchapter[
  title={Methods of Analysis},
]

% 3.1
\startQ[Q:3.1]
Using \reffig{3.50}, design a problem to help other
students better understand nodal analysis.
\placefigq{3.50}{3.1}
\stopQ

\startA
...
\stopA

% 3.2
\startQ[Q:3.2]
For the circuit in \reffig{3.51}, obtain $v_1$ and $v_2$.
\placefigq{3.51}{3.2}
\stopQ

\startA
\startformula\startmathalignment
\NC \frac{v_2-v_1}{2} - \frac{v_1}{10} - \frac{v_1}{5} = 6 \NC \NR
\NC \frac{v_2-v_1}{2} + \frac{v_2}{4} = 6 + 3 \NC \NR
\stopmathalignment\stopformula

$v_1 = \sunit{0 volt}$, $v_2 = \sunit{12 volt}$.
\stopA

% 3.3
\startQ[Q:3.3]
Find the current $I_1$ through $I_4$ and the voltage $v_o$
in the circuit of \reffig{3.52}.
\placefigq{3.52}{3.3}
\stopQ

\startA
$I_1 = \sunit{6 ampere}$\\
$I_2 = \sunit{3 ampere}$\\
$I_3 = \sunit{2 ampere}$\\
$I_4 = \sunit{1 ampere}$\\
\stopA

% 3.4
\startQ[Q:3.4]
Given the circuit in \reffig{3.53},
caculate the currents $i_1$ through $i_4$.
\placefigq{3.53}{3.4}
\stopQ

\startA
$I_1 = \sunit{3 ampere}$\\
$I_2 = \sunit{6 ampere}$\\
$I_3 = \sunit{-0.5 ampere}$\\
$I_4 = \sunit{-0.5 ampere}$\\
\stopA

% 3.5
\startQ[Q:3.5]
Obtain $v_o$ in the circuit of \reffig{3.54}.
\placefigq{3.54}{3.5}
\stopQ

\startA
\startformula
\frac{v_o}{120k} + \frac{v_o + 120}{120k} + \frac{v_o+120-60}{30k} = 0
\stopformula
$v_o = \sunit{-60 volt}$.
\stopA

% 3.6
\startQ[Q:3.6]
Solve for $V_1$ in the circuit of \reffig{3.55}
using nodal analysis.
\placefigq{3.55}{3.6}
\stopQ

\startA
$\frac{V_1}{10} = \frac{20-V_1}{4} + \frac{10-V_1}{10/3}$\\
$V_1 = \frac{160}{13}\approx \sunit{12.308 volt}$.
\stopA

% 3.7
\startQ[Q:3.7]
Apply nodal analysis to solve for $V_x$ in the circuit of \reffig{3.56}.
\placefigq{3.56}{3.7}
\stopQ

\startA
$\frac{V_x}{60} + \frac{V_x}{30} + 0.05 V_x = 2$\\
$V_x = \sunit{20 volt}$.
\stopA

\stopchapter
\stopcomponent
