\startsection[
  title={Basic operations on B-trees},
]

%e18.2-1
\startEXERCISE
請給出關鍵字 \m{F}、 \m{S}、 \m{Q}、 \m{K}、 \m{C}、 \m{L}、 \m{H}、 \m{T}、 \m{V}、 \m{W}、
 \m{M}、 \m{R}、 \m{N}、 \m{P}、 \m{A}、 \m{B}、 \m{X}、 \m{Y}、 \m{D}、 \m{Z}、 \m{E} 依序
插入一棵最小度數爲 2 的空 B 樹的結果。
只要畫出在某些節點分裂之前的結構以及最終的結構。
\stopEXERCISE

\startANSWER
\externalfigure[output/e18_2_1-1]
\stopANSWER

%e18.2-2
\startEXERCISE
請解釋在什麼情況下(如果有的話),
在調用 \ALGO{B-TREE-INSERT} 的過程中,
會執行冗餘的 \ALGO{DISK-READ} 或 \ALGO{DISK-WRITE} 操作。
(所謂冗餘的 \ALGO{DISK-READ},是指對已經在主存中的某頁做 \ALGO{DISK-READ}。
冗餘的 \ALGO{DISK-WRITE} 是指將已經存在於磁盤上某頁又完全相同的重寫一遍)
\stopEXERCISE

\startANSWER
若所要插入的節點是滿的,但其父節點未滿,我們需要做如下操作:
\startigBase[2]
\item \ALGO{DISK-READ}:放置關鍵字;
\item \ALGO{DISK-WRITE}:節點裂變;
\item \ALGO{DISK-READ}:讀取父節點;
\item \ALGO{DISK-READ}:填充父節點。
\stopigBase

如果父節點也是滿的,則會重複第二和第三步。也就是說不會出現冗餘。
\stopANSWER

%e18.2-3
\startEXERCISE
請說明如何在一棵 B 樹中找出最小關鍵字,
以及如何找出某一給定關鍵字的前驅。
\stopEXERCISE

\startANSWER
找最小關鍵字的過程與二叉搜索樹類似。最小關鍵字即最左葉子節點的第一個關鍵字。

找前驅要首先找到給定的關鍵字,假設爲 \m{x.key_i}。
\startigBase[2]
\item 如果 \m{x} 不是葉子節點,則要找到以 \m{x.c_i} 爲根的子樹的最大關鍵字。
\item 如果 \m{x} 是葉子節點,且 \m{i>1},則返回 \m{x.key_{i-1}}。
\item 如果 \m{x} 是葉子節點,且 \m{i=1},則自底向上搜索,
找到最後一個節點 \m{y},使得 \m{x.key_i} 是以 \m{y.c_j} 爲根的最左關鍵字(\m{j>0})。
如果 \m{j=1},則返回 \ALGO{NIL},這時 \m{x.key_i} 是整個 B 樹的最小關鍵字,沒有前驅;
否則,返回 \m{y.key_{j-1}}。
\stopigBase
\stopANSWER

%e18.2-4
\startEXERCISE
假設將關鍵字 \m{\{1,2,\ldots,n\}} 插入一棵最小度數爲 2 的空 B 樹中,
那麼最終的 B 樹有多少個節點?
\stopEXERCISE

\startANSWER
\m{O(n)}。
\stopANSWER

%e18.2-5
\startEXERCISE
因爲葉子節點無需指向孩子節點的指針,
那麼對於同樣大小的磁盤頁面,可選用一個與內部節點不同的(更大的) \m{t} 值。
請說明如何修改 B 樹的創建和插入過程來處理這個變化。
\stopEXERCISE

\startANSWER
最小度數爲 \m{t} 的 B 樹,一個節點最多可存儲 \m{2t-1} 個關鍵字以及 \m{2t} 個指針。
而葉子節點沒有子節點,因此 \m{2t} 個指針可以用來存儲關鍵字。
若將新的 \m{t} 值記爲 \m{t'},總空間要足夠容納 \m{2t'-1} 個關鍵字,
即根據關鍵字和指針各佔空間多少,即可求出 \m{t'} 的最大值。

B 樹的創建過程當無需修改(空節點)。
插入過程需要修改。
葉子節點分裂出來的節點仍是葉子節點,且仍只分裂出一個葉子節點,
不會出現一個葉子節點一次分裂出多個葉子節點的情況。
插入過程需要修改兩處:1)葉子節點和內部節點的分裂條件不同。
2)分裂後葉子節點和內部節點需要複製的關鍵字數目不同。
\stopANSWER

%e18.2-6
\startEXERCISE
假設 \ALGO{B-TREE-SEARCH} 的實現是在每個節點內採用二分查找,而不是線性查找。
證明:
無論怎樣選擇 \m{t} (\m{t} 爲 \m{n} 的函數),
這種實現所需的 CPU 時間都爲 \m{O(\lg n)}。
\stopEXERCISE

\startANSWER
令 B 樹高爲 \m{h},所含關鍵字數目爲 \m{n},
則 \m{O(h)=O(\log_{t}n)},且 \m{h\le \log_{t}\frac{n+1}{2}}。
每個節點中關鍵字數目不會大於 \m{2t-1},二分查找用時 \m{O(\lg t)}。
因此總時間爲:
\startformula
O(\lg t \times \log_{t}n) = O(\lg t \times \frac{\lg n}{\lg t}) = O(\lg n)
\stopformula
\stopANSWER

%e18.2-7
\startEXERCISE
假設磁盤硬件允許我們任意選擇磁盤頁面的大小,
但讀取磁盤頁面的時間是 \m{a+bt},
其中 \m{a} 和 \m{b} 爲規定的常數, \m{t} 爲確定磁盤頁大小後的 B 樹的最小度數。
請描述如何選擇 \m{t} 以(近似的)最小化 B 樹的查找時間。
對 \m{a=5ms} 和 \m{b=10ms},請給出 \m{t} 的一個最優值。
\stopEXERCISE

\startANSWER
令 B 樹高爲 \m{h},關鍵字數目爲 \m{n},搜索過程最多需要讀取次數爲 \m{h},
則讀頁面總用時爲 \m{(a+bt)\times h},對 \m{t} 求導數。
\stopANSWER

\stopsection
