\startsection[
  title={Summation formulas and properties},
]

%A.1-1
\startEXERCISE
求 \m{\sum_{k=1}^{n}(2k-1)} 的簡化形式。
\stopEXERCISE

\startANSWER
\startformula\startmathalignment
\NC \sum_{k=1}^{n}(2k-1) \NC = 2\sum_{k=1}^{n}k - \sum_{k=1}^{n}1 \NR
\NC \NC = n(n+1) - n \NR
\NC \NC = n^2 \NR
\stopmathalignment\stopformula
\stopANSWER

%eA.1-2
\startEXERCISE\DIFFICULT
利用調諧級數證明: \m{\sum_{k=1}^{n} 1/(2k-1) = \ln(\sqrt{n}) + O(1)}。
\stopEXERCISE

\startANSWER
\startformula\startmathalignment
\NC \sum_{k=1}^{n}1/(2k-1) \NC = 1 + \frac{1}{3} + \frac{1}{5} + \ldots + \frac{1}{2n-1} \NR
\NC \NC = (1 + \frac{1}{2} + \frac{1}{3} + \ldots + \frac{1}{2n})
          - \frac{1}{2}(1 + \frac{1}{2} + \frac{1}{3} + \ldots + \frac{1}{n}) \NR
\NC \NC = \sum_{k=1}^{2n}\frac{1}{k} - \frac{1}{2}\sum_{k=1}^{n}\frac{1}{k} \NR
\NC \NC = \ln 2n + O(1) - \frac{1}{2}(\ln n + O(1)) \NR
\NC \NC = \ln 2 + \ln n + O(1) - \frac{1}{2}\ln n - \frac{1}{2}O(1) \NR
\NC \NC = \frac{1}{2}\ln n + O(1) \NR
\NC \NC = \ln\sqrt{n} + O(1) \NR
\stopmathalignment\stopformula
\stopANSWER

%eA.1-3
\startEXERCISE
證明: \m{\sum_{k=0}^{\infty}k^2 x^k = x(1+x)/(1-x)^3} 在 \m{|x| < 1} 條件下成立。
\stopEXERCISE

\startANSWER
\startformula\startmathalignment[n=3,align={left,right,left}]
\NC \NC \sum_{k=0}^{\infty} x^k \NC = \frac{1}{1-x} \NR
\NC \text{求導}\qquad \NC \sum_{k=0}^{\infty} k x^{k-1} \NC = \frac{1}{(1-x)^2}  \NR
\NC \text{同乘 \m{x}}\qquad \NC \sum_{k=0}^{\infty} kx^k \NC = \frac{x}{(1-x)^2} \NR
\NC \text{求導}\qquad \NC \sum_{k=0}^{\infty} k^2 x^{k-1} \NC = \frac{1+x}{(1-x)^3}  \NR
\NC \text{同乘 \m{x}}\qquad \NC \sum_{k=0}^{\infty} k^2 x^{k} \NC = \frac{x(1+x)}{(1-x)^3}  \NR
\stopmathalignment\stopformula
\stopANSWER

%eA.1-4
\startEXERCISE\DIFFICULT
證明: \m{\sum_{k=0}^{\infty}(k-1)/2^k = 0}。
\stopEXERCISE

\startANSWER
根據:
\startformula
\sum_{k=0}^{\infty}x^k = \frac{1}{1-x} \qquad A.6
\stopformula
\startformula
\sum_{k=0}^{\infty}kx^k = \frac{x}{(1-x)^2} \qquad A.8
\stopformula

令 \m{x=1/2},則:
\startformula\startmathalignment
\NC \sum_{k=0}^{\infty} \frac{k-1}{2^k} \NC = \sum_{k=0}^{\infty}((k-1)x^k) \NR
\NC \NC = \sum_{k=0}^{\infty}kx^k - \sum_{k=0}^{\infty}x^k \NR
\NC \NC = \frac{x}{(1-x)^2} - \frac{1}{1-x} \NR
\NC \NC = \frac{2x-1}{(1-x)^2} \NR
\NC \NC = 0 \NR
\stopmathalignment\stopformula
\stopANSWER

%eA.1-5
\startEXERCISE\DIFFICULT
對於 \m{|x|<1},計算 \m{\sum_{k=1}^{\infty}(2k+1)x^{2k}} 的值。
\stopEXERCISE

\startANSWER
\startformula\startmathalignment
\NC \sum_{k=0}^{\infty} x^{2k+1} \NC = x^3 + x^5 + x^7 + \ldots \NR
\NC \NC = x + x^3 + x^5 + x^7 + \ldots - x \NR
\NC \NC = x(1+x^2+x^4+x^6+\ldots) - x \NR
\NC \NC = x\sum_{k=0}^{\infty}(x^2)^k - x \NR
\NC \NC = \frac{x}{1-x^2} - x \NR
\NC \NC = \frac{x^3}{1-x^2} \NR
\stopmathalignment\stopformula
等式兩邊進行微分:
\startformula\startmathalignment
\NC \sum_{k=0}^{\infty} (2k+1)x^{2k} \NC = \frac{d}{dx}\frac{x^3}{1-x^2} \NR
\NC \NC = \frac{x^2(3-x^2)}{(1-x^2)^2} \NR
\stopmathalignment\stopformula
\stopANSWER

%eA.1-6
\startEXERCISE
用和的線性性質證明: \m{\sum_{k=1}^{n}O(f_k(i)) = O(\sum_{k=1}^{n}f_k(i))}。
\stopEXERCISE

\startANSWER
\TODO{略。}
\stopANSWER

%eA.1-7
\startEXERCISE
計算乘積 \m{\prod_{k=1}^{n}2\cdot 4^k}。
\stopEXERCISE

\startANSWER
令 \m{P=\prod_{k=1}^{n}2\cdot 4^k},則:
\startformula\startmathalignment
\NC \lg P \NC = \lg(\prod_{k=1}^{n}2\cdot 4^k)  \NR
\NC \NC = \sum_{k=1}^{n}\lg(2\cdot 4^k) \NR
\NC \NC = \sum_{k=1}^{n}\lg 2^{2k+1} \NR
\NC \NC = \sum_{k=1}^{n}(2k+1) \NR
\NC \NC = 2\sum_{k=1}^{n}k + \sum_{k=1}^{n}1 \NR
\NC \NC = 2\frac{n(n+1)}{2} + n \NR
\NC \NC = n(n+2) \NR
\stopmathalignment\stopformula
因此 \m{P = 2^{n(n+2)}}。
\stopANSWER

%eA.1-8
\startEXERCISE\DIFFICULT
計算乘積 \m{\prod_{k=2}^{n}(1-1/k^2)}。
\stopEXERCISE

\startANSWER
\startformula
1 - \frac{1}{k^2} = \frac{k^2-1}{k^2} = \frac{(k-1)(k+1)}{k\cdot k}
\stopformula

\startformula\startmathalignment
\NC     \NC \prod_{k=2}^{n}(1-\frac{1}{k^2}) \NR
\NC =\NC \sum_{k=2}^{n}\frac{(k-1)(k+1)}{k\cdot k} \NR
\NC =\NC \frac{1\cdot 3}{2\cdot 2} \cdot \frac{2\cdot 4}{3\cdot 3}
         \cdot \frac{3\cdot 5}{4\cdot 4}
         \cdots \frac{(n-2)\cdot n}{(n-1)\cdot (n-1)}
         \cdot \frac{(n-1)\cdot (n+1)}{n\cdot n} \NR
\NC =\NC \frac{k+1}{2k} \NR
\stopmathalignment\stopformula
\stopANSWER

\stopsection
