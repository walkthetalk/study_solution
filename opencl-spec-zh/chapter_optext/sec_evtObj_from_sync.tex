\section[section:clEvtObjFromGlSync]{由 GL 同步對象創建 CL 事件對象}

% Overview
\subsection{簡介}

本擴展允許由 OpenGL 隔柵同步對象創建 OpenCL \cnglo{evtobj},
從而潛在地提高在兩種 API 間共享圖像和緩衝的效率。
對應的擴展 \clext{GL_ARB_cl_event} 可以由 OpenCL \cnglo{evtboj}創建 OpenGL 同步對象,
他可以與本擴展互補。

另外,本擴展修改了 \clapi{clEnqueueAcquireGLObjects} 和 \clapi{clEnqueueReleaseGLObjects} 的行為,
如果 OpenGL \cnglo{context} 與 OpenCL \cnglo{context} 綁定到了同一個線程中,
則他可以隱式保證其同步。

如果實作支持本擴展,
則 \cenum{CL_PLATFORM_EXTENSIONS} 或 \cenum{CL_DEVICE_EXTENSIONS} 中
應該有字串 \clext{cl_khr_gl_event},
參見\reftab{cldevquery}。

% New Procedures and Functions
\subsection{新例程和新函式}

\startCLFUNC
cl_event clCreateEventFromGLsyncKHR (
			cl_context context,
			GLsync sync,
			cl_int *errcode_ret);
\stopCLFUNC

% New Tokens
\subsection{新的符記}

調用 \clapi{clGetEventInfo} 時
如果 \carg{param_name} 是 \cenum{CL_EVENT_COMMAND_TYPE},則會返回:
\startclc
CL_COMMAND_GL_FENCE_SYNC_OBJECT_KHR	0x200D
\stopclc

% Additions to Chapter 5 of the OpenCL 1.2 Specification
\subsection{對第五章的補充}

將下面內容添加到\insection[evtObj]的第四段中(\clapi{clCreateUserEvent} 的描述之前):
\startreplacepar
\cnglo{evtobj}也可用來反映 OpenGL 同步對象的狀態。
同步對象指的是 OpenGL 命令流中所執行的隔柵(fence)命令。
這為在 OpenGL 和 OpenCL 間共享緩衝和圖像提供了一種新方法(參見\insection[syncCLGL])。
\stopreplacepar

在\reftab{clGetEventInfo}關於 \clapi{clGetEventInfo} 的描述中,
\carg{param_name} 為 \cenum{CL_EVENT_COMMAND_TYPE} 時,
所返回的 \carg{param_value} 的值增加一項: \cenum{CL_COMMAND_GL_FENCE_SYNC_OBJECT_KHR}。

新添{\ftRef{節 5.9.1}} {\ftEmp{將\cnglo{evtobj}鏈接到 OpenGL 同步對象上}}:
\startreplacepar
可以通過鏈接 OpenGL {\ftEmp{同步對象}}來創建\cnglo{evtobj}。
這種\cnglo{evtobj}的完成就相當於等待與所鏈接 GL 同步對象相關聯隔柵命令的完成。

\topclfunc{clCreateEventFromGLsyncKHR}

\startCLFUNC
cl_event clCreateEventFromGLsyncKHR (
			cl_context context,
			GLsync sync,
			cl_int *errcode_ret)
\stopCLFUNC

此函式會創建一個帶鏈接的\cnglo{evtobj}。

\carg{context} 是一個利用擴展 \clext{cl_khr_gl_sharing},
由 OpenGL \cnglo{context}或共享組創建的的 OpenCL \cnglo{context}。

\carg{sync} 是 \carg{context} 所關聯 GL 共享組中同步對象的名字。

如果成功創建了\cnglo{evtobj},則 \clapi{clCreateEventFromGLsyncKHR} 會將其返回,
並將 \carg{errcode_ret} 置為 \cenum{CL_SUCCESS}。
否則,返回 \cmacro{NULL},並將 \carg{errcode_ret} 置為下列錯誤碼之一:
\startigBase
\item \cenum{CL_INVALID_CONTEXT},
如果 \carg{context} 無效,或者不是由 GL \cnglo{context}創建的。

\item \cenum{CL_INVALID_GL_OBJECT},
如果 \carg{sync} 不是 \carg{context} 所關聯 GL 共享組中同步對象的名字。
\stopigBase

對於這種\cnglo{evtobj}調用 \clapi{clGetEventInfo} 時會返回下列值:
\startigBase
\item 如果查詢的是 \cenum{CL_EVENT_COMMAND_QUEUE},則結果為 \cmacro{NULL},
因為此事件沒有與任何 OpenCL \cnglo{cmdq}關聯。

\item 如果查詢的是 \cenum{CL_EVENT_COMMAND_TYPE},
則結果是 \cenum{CL_COMMAND_GL_FENCE_SYNC_OBJECT_KHR},
表明此事件關聯的是 GL 同步對象,而不是 OpenCL \cnglo{cmd}。

\item 如果查詢的是 \cenum{CL_EVENT_COMMAND_EXECUTION_STATUS},
則結果要麼是 \cenum{CL_SUBMITTED},表明同步對象所關聯的隔柵\cnglo{cmd}還未完成,
要麼是 \cenum{CL_COMPLETE},表明隔柵\cnglo{cmd}完成了。
\stopigBase

\clapi{clCreateEventFromGLsyncKHR} 會在返回的\cnglo{evtobj}上實施隱式的 \clapi{clRetainEvent}。
創建這種帶鏈接的\cnglo{evtobj}時也會在所鏈接的 GL 同步對象上放置一個引用。
當這種\cnglo{evtobj}被刪除時,這個引用也會被移除。

\clapi{clCreateEventFromGLsyncKHR} 所返回的事件可能
只能由 \clapi{clEnqueueAcquireGLObjects} 使用。
將這種事件傳遞給其他 CL API 會生成錯誤 \cenum{CL_INVALID_EVENT}。
\stopreplacepar % stop replace par

% Additions to Chapter 9 of the OpenCL 1.2 Specification
\subsection{對第九章的補充}

對 \clapi{clEnqueueAcquireGLObjects} 的參數 \carg{event} 增加以下描述:
\startreplacepar
如果一個 OpenGL \cnglo{context}綁定到了當前線程上,
那麼同時滿足下列兩個條件的 OpenGL \cnglo{cmd}會在執行
緊跟 \clapi{clEnqueueAcquireGLObjects} 的所有 OpenCL \cnglo{cmd}
(這些 OpenCL \cnglo{cmd}會影響或存取 \carg{mem_objects} 中的\cnglo{memobj})前完成:
\startigNum[indentnext=no]
\item 會影響或存取 \carg{mem_objects} 中\cnglo{memobj}的內容;
\item 是在調用 \clapi{clEnqueueAcquireGLObjects} 之前在此 OpenGL \cnglo{context}上發起的。
\stopigNum
如果所返回的\cnglo{evtobj}不是 \cmacro{NULL},
則只有當這樣的 OpenGL \cnglo{cmd}完成後,他才會報告完成。
\stopreplacepar

對 \clapi{clEnqueueReleaseGLObjects} 的參數 \carg{event} 增加以下描述:
\startreplacepar
如果一個 OpenGL \cnglo{context}綁定到了當前線程上,
那麼只有當 \clapi{clEnqueueReleaseGLObjects} 之前的所有 OpenCL \cnglo{cmd}
(這些 OpenCL \cnglo{cmd}會影響或存取 \carg{mem_objects} 中的\cnglo{memobj})
執行完畢後,
同時滿足下列兩個條件的 OpenGL \cnglo{cmd}才會執行:
\startigNum[indentnext=no]
\item 會影響或存取 \carg{mem_objects} 中\cnglo{memobj}的內容;
\item 是在調用 \clapi{clEnqueueReleaseGLObjects} 之後在此 OpenGL \cnglo{context}上發起的。
\stopigNum
如果所返回的\cnglo{evtobj}不是 \cmacro{NULL},
則只有當他報告完成後,才會執行這些 OpenGL \cnglo{cmd}。
\stopreplacepar

用下列內容取代\insection[syncCLGL]的第二段:
\startreplacepar
調用 \clapi{clEnqueueAcquireGLObjects} 之前,
\cnglo{app}必須確保會存取 \carg{mem_objects} 中對象並且處於擱置狀態的 OpenGL 操作全部完成。

如果支持擴展 \clext{cl_khr_gl_event},
並且 OpenGL \cnglo{context}與 OpenCL \cnglo{context}綁定到了同一線程上,
則 OpenCL 實作會確保(針對此 OpenGL \cnglo{context})這種擱置的 OpenGL 操作全部完成。
這也叫{\ftRef{隱式同步}}。

如果支持擴展 \clext{cl_khr_gl_event},
並且所談及的 OpenGL \cnglo{context}支持隔柵同步對象,
要想確定 OpenGL \cnglo{cmd}完成了,
可以用 \capi{glFenceSync} 在那些\cnglo{cmd}後面放置一個 GL 隔柵\cnglo{cmd},
然後用 \clapi{clCreateEventFromGLsyncKHR} 由所產生的 GL 同步對象創建一個事件,
並通過 \clapi{clEnqueueAcquireGLObjects} 來確定這個\cnglo{evtobj}完成了。
這種方法可能比 \clapi{glFinish} 更加高效,
稱作{\ftRef{顯式同步}}。
當 OpenGL \cnglo{context}綁定的是存取\cnglo{memobj}的另一個線程時,
顯式同步最有用。

如果支持擴展 \clext{cl_khr_gl_event},
要想確定 OpenGL \cnglo{cmd}完成了,
可以在所有帶有對這些對象的擱置引用的 OpenGL \cnglo{context}上
發起並等待\cnglo{cmd} \clapi{glFinish} 的完成。
一些實作可能提供其他有效的同步方法。
如果存在這樣的方法,會在特定\cnglo{platform}的文檔中對其進行描述。

注意,對於所有 OpenGL 實作以及所有 OpenCL 實作,
只有 \clapi{glFinish} 才是可移植的,其他方法都不是。
鑒於 \clapi{glFinish} 是一種代價高昂的操作,
如果在某個\cnglo{platform}上支持擴展 \clext{cl_khr_gl_event},
應盡量避免使用 \clapi{glFinish}。
\stopreplacepar

% Issues
\subsection{問題}

\startQUESTION
如何處理 CL 事件和 GL 同步的相互引用?
\stopQUESTION
\startANSWER
已有提案:帶鏈接的 CL 事件會在 GL 同步對象上放置一個引用。
刪除 CL 事件時會移除這個引用。
還有一些代價更高的方案可以通過 GL 同步來反映 CL 事件\cnglo{refcnt}的變化。
\stopANSWER

\startQUESTION
在其他 API 中如何處理到同步基元的鏈接?
\stopQUESTION
\startANSWER
還未解決。
我們想至少要有一種方式可以將事件鏈接到 EGL 同步對象上。
可能沒有模擬 DX 的概念。
對於所鏈接的每種同步基元應當都有一個入口點,
如 \clapi{clCreateEventFromEGLSyncKHR}。

另一種方案就是通用的 \clapi{clCreateEventFromExternalEvent},可以接受一個特性列。
這個特性列中可以包含外部基元的類型以及其附屬資訊
(GL 同步對象句柄、 EGL display 以及同步對象句柄,等等)。
這樣就可以重用同一個入口點。

這些可能會作為一個獨立的擴展。
\stopANSWER

\startQUESTION
\cenum{CL_EVENT_COMMAND_TYPE} 對應的是\cnglo{cmd}(隔柵)的類型還是
所鏈接同步對象的類型?
\stopQUESTION
\startANSWER
已有提案:所鏈接同步對象的類型。
\stopANSWER

\startQUESTION
是否要同時支持顯式同步和隱式同步?
\stopQUESTION
\startANSWER
已有提案:是的。
隱式同步適用於 GL 和 CL 在同一\cnglo{app}線程中執行的情況。
而顯式同步則適用於在不同線程中執行、但 \capi{glFinish} 代價又太高的情況。
\stopANSWER

\startQUESTION
本擴展是\cnglo{platform}擴展還是\cnglo{device}擴展?
\stopQUESTION
\startANSWER
已有提案:\cnglo{platform}擴展。
這樣的話,要想僅僅使用公開的 GL API 來實現 sync->event 語義需要做大量的工作;
但是,同一運行時中可能會有多個對 GL 支持層級不同的驅動和\cnglo{device}同時存在。
\stopANSWER

\startQUESTION
什麼地方才能使用由 GL 同步對象生成的事件?
\stopQUESTION
\startANSWER
已有提案:僅當調用 \clapi{clEnqueueAcquireGLObjects} 時才能使用,
其他任何地方使用這種事件都會產生錯誤。
其他地方也沒有明確的用例,而且要想支持他還有一個成本問題,
在所有其他將其作為參數的\cnglo{cmd}中檢查事件源都要有相應的成本,
經過權衡,採用了目前的方案。
\stopANSWER

