\section{存儲類別限定符}

所支持的存儲類別限定符有 \cqlf{typedef}、 \cqlf{extern} 和 \cqlf{static}。
而 \cqlf{auto} 和 \cqlf{register} 則不在支持之列。

存儲類別限定符 \cqlf{extern} 只可用在
函式(包括\cnglo{kernel}函式和非\cnglo{kernel}函式)、
在\cnglo{program}作用域內聲明的全局變量
以及函式(包括\cnglo{kernel}函式和非\cnglo{kernel}函式)內部聲明的變量上。
而存儲類別限定符 \cqlf{static} 只可用在非\cnglo{kernel}函式和
在\cnglo{program}作用域內聲明的全局變量上。

例:
\startclc
extern constant float4 noise_table[256];
static constant float4 color_table[256];

extern kernel void my_foo(image2d_t img);
extern void my_bar(global float *a);

kernel void my_func(image2d_t img, global float *a)
{
	extern constant float4 a;
	static constant float4 b;	// error.
	static float c;			// error.

	...
	my_foo(img);
	...
	my_bar(a);
	...
}
\stopclc
