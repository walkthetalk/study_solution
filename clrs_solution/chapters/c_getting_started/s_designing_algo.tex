\startsection[
  title={Designing algorithms},
]

\startEXERCISE
參照圖 2.4,對數列 $A=\langle 3,41,52,26,38,57,9,49\rangle$ 進行歸並排序。
\stopEXERCISE
\startANSWER
\externalfigure[output/e2_3_1-1]
\stopANSWER

\startEXERCISE
\ALGO{MERGE-SORT} 例程中,
第一行在測試時用的是“\If $p\ge r$” 而不是“\If $p\ne r$”。
如果在調用 \ALGO{MERGE-SORT} 時, $p>r$,
則子數列 $A[p:r]$ 爲空。
只要第一次調用 \ALGO{MMERGE-SORT(A,1,n)} 時 $n\ge 1$,
測試“\If $p\ne r$”就能保證遞歸調用時不會出現 $p>r$ 的情況。
\stopEXERCISE
\startANSWER
第一次調用時 $n\ge 1$ 即 $p\le r$,滿足要求;
如果測試條件為 $p\ne r$,
則 $p\le q$ 且 $q+1\le r$,
即滿足遞歸調用時 $p\le r$;
因此只要保證第一次調用時 $n\ge 1$,
測試 $p\ne r$ 就能保證遞歸調用時不會出現 $p>r$ 的情況。
\stopANSWER

\startEXERCISE
分析例程 \ALGO{MERGE-SORT} 中第 12-18 行 \While 循环中的循環不變性。
根據這個性質,與 \While 循環中第 20-23 行和 24-27 行一起,
證明例程 \ALGO{MERGE-SORT} 的正確性。

\startCLRSCODE
\CLRSH{MERGE(A,p,q,r)}
$n_L = q - p + 1$	// length of $A[p:q]$
$n_R = r - q$		// length of $A[q + 1 : r]$
\stopCLRSCODE
\stopEXERCISE
\startANSWER
\stopANSWER

\startEXERCISE
不使用哨兵,重寫 MERGE 例程。
\stopEXERCISE
\startANSWER
\startCLRS
n1 = q - p + 1
n2 = r - q
let L[1..n1+1] and R[1..n2+1] be new arrays
for i = 1 to n1
	L[i] = A[p + i - 1]
for j = 1 to n2
	R[j] = A[q + j]
i = 1
j = 1
for k = p to r
	if i > n1
		while j <= n2
			A[k] = R[j]
			k = k + 1
			j = j + 1
		return
	if j > n2
		while i <= n1
			A[k] = L[i]
			k = k + 1
			i = i + 1
		return
	if L[i] <= R[j]
		A[k] = L[i]
		i = i + 1
	else A[k] = R[j]
		j = j + 1
\stopCLRS
\stopANSWER

\startEXERCISE
用數學歸納法證明 \m{T(n) = n \lg(n)}:
\startformula
T(n) = \startmathcases
\NC 2		\NC 若 \m{n = 2}, \NR
\NC 2T(n/2) + n	\NC 若 \m{n = 2^k}, \m{k > 1}。 \NR
\stopmathcases
\stopformula
\stopEXERCISE
\startANSWER
如果 \m{T(n) = n \lg(n)},那麼:
\startformula\startalign
\NC T(2n)	\NC = 2n \lg(2n) \NR
\NC		\NC = 2n (\lg(n) + 1) \NR
\NC		\NC = 2n \lg(n) + 2n \NR
\NC		\NC = 2T(n) + 2n \NR
\stopalign\stopformula
\stopANSWER

\startEXERCISE
用遞迴方式重寫插入排序。
\stopEXERCISE
\startANSWER
\CLRSH{INSERTION(A, n)}
\startCLRS
if n < 2
	return
INSERTION-SORT(A, n-1)
key = A[n]
i = n - 1
while i > 0 and A[i] > key
	A[i+1] = A[i]
	i = i - 1
A[i + 1] = key
\stopCLRS

\startformula
T(n) = \startmathcases
\NC \Theta(1)		\NC 若 $n = 1$, \NR
\NC T(n-1) + C(n-1)	\NC 否則。 \NR
\stopmathcases
\stopformula
\stopANSWER

\startEXERCISE[exercise:bin_search]
二分查找。
\stopEXERCISE
\startANSWER
\CLRSH{BINARY-SEARCH(A, v)}
\startCLRS
low = 1
high = A.length
while low <= high
	m = (low + high) / 2
	if A[m] == v
		return m
	if A[m] < v
		low = m + 1
	else
		high = m - 1
return NIL
\stopCLRS

\startformula
T(n+1) = T(n/2) + c
\stopformula
\stopANSWER

\startEXERCISE
插入排序時用二分法查找要插入的位置,
以改進插入排序最壞情況的總運行時間至 $\Theta(nlgn)$?
\stopEXERCISE
\startANSWER
不行,二分查找只可能減少比較的次數,而無法減少數據搬移的次數。
\stopANSWER

%e2.3-7
\startEXERCISE
設計一個所用時間爲 $\Theta(nlgn)$ 的算法,
用來在集合 $S$ 中查找是否有兩個數的和正好是 $x$。
\stopEXERCISE
\startANSWER
\CLRSH{PAIR-EXISTS(S, x)}
\startCLRS
A = MERGE-SORT(S)

for i = 1 to S.length
	if BINARY-SEARCH(A, x - S[i]) != NIL
		return true

return false
\stopCLRS
\stopANSWER

\stopsection
