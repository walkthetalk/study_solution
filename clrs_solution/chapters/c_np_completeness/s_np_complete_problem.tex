\startsection[
  title={NP-complete problems},
]

%e34.5-1
\startEXERCISE
{\EMP 子圖同構問題}:取兩個無向圖 \m{G_1} 和 \m{G_2},
回答 \m{G_1} 是否與 \m{G_2} 的一個子圖同構。
證明:子圖同構問題是 NP 完全的。
\stopEXERCISE

\startANSWER
\TODO{略。}
\stopANSWER

%e34.5-2
\startEXERCISE[exercise:34.5-2]
給定一個 \m{m\times n} 的整數矩陣 \m{A} 和一個整型的 \m{m} 維向量 \m{b},
 {\EMP 0-1 整數規劃問題}即研究是否有一個整型的 \m{n} 維向量 \m{x},
其元素取自集合 \m{\{0,1\}},滿足 \m{Ax\le b}。
證明: 0-1 整數規劃問題是 NP 完全的。
(\hint 由 \ALGO{3-CNF-SAT} 問題進行歸約。)
\stopEXERCISE

\startANSWER
\TODO{略。}
\stopANSWER

%e34.5-3
\startEXERCISE[exercise:34.5-3]
{\EMP 整數線性規劃問題}與\inexercise[34.5-2] 中給出的 0-1 整數規劃十分相似,
區別僅在於向量 \m{x} 的值可以取任何整數,而不僅是 0 或 1。
假定 0-1 整數規劃問題是 NP 難度的,
證明:整數線性規劃問題是 NP 完全的。
\stopEXERCISE

\startANSWER
\TODO{略。}
\stopANSWER

%e34.5-4
\startEXERCISE
證明:如果目標值 \m{t} 表示成一元形式,
試說明如何在多項式時間內求解子集的和。
\stopEXERCISE

\startANSWER
\TODO{略。}
\stopANSWER

%e34.5-5
\startEXERCISE
{\EMP 集合劃分問題}的輸入爲一個數字集合 \m{S}。
問題是:這些數字是否能被劃分成兩個集合 \m{A} 和 \m{\overbar{A}=S-A},
使得 \m{\sum_{x\in A}x = \sum_{x\in\overbar{A}}x}。
證明:集合劃分問題是 NP 完全的。
\stopEXERCISE

\startANSWER
\TODO{略。}
\stopANSWER

%e34.5-6
\startEXERCISE
證明:哈密頓路徑問題是 NP 完全的。
\stopEXERCISE

\startANSWER
\TODO{略。}
\stopANSWER

%e34.5-7
\startEXERCISE
{\EMP 最長簡單迴路問題}是在一個圖中,
找出一個具有最大長度的簡單迴路(無重複的頂點)。
證明:這個問題是 NP 完全的。
\stopEXERCISE

\startANSWER
\TODO{略。}
\stopANSWER

%e34.5-8
\startEXERCISE
在{\EMP 半 \ALGO{3-CNF} 可滿足性}中,
給定一個 \ALGO{3-CNF} 形式的公式 \m{\phi},
他包含 \m{n} 個變量和 \m{m} 個子句,
其中 \m{m} 是偶數。
我們希望確定是否存在對 \m{\phi} 中變量的一個真值賦值,
使得 \m{\phi} 中恰有一半的子句爲 0,
同時恰有另一半的子句爲 1。
證明:半 \ALGO{3-CNF} 可滿足性問題是 NP 完全的。
\stopEXERCISE

\startANSWER
\TODO{略。}
\stopANSWER

\stopsection
