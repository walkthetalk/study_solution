\startsection[
  title={Asymptotic notation: formal definitions},
]

%e3.2-1
\startEXERCISE
令 $f(n)$ 和 $g(n)$ 爲漸進非負函數。
利用記號 $\Theta$ 的基本定義,
證明 \m{\max(f(n),g(n)) = \Theta(f(n)+g(n))}。
\stopEXERCISE
\startANSWER
由於單調遞增,則:
\startformula\startalign[n=3]
\NC \exists n_1, n_2: \NC f(n) \geq 0 \NC \quad\text{若 \m{n > n_1};} \NR
\NC                   \NC g(n) \geq 0 \NC \quad\text{若 \m{n > n_2}。} \NR
\stopalign\stopformula

設 \m{n_0 = \max(n_1,n_2)},對於 \m{n > n_0}:
\startformula\startalign
\NC f(n) \NC \leq \max(f(n), g(n)) \NR
\NC g(n) \NC \leq \max(f(n), g(n)) \NR
\NC (f(n) + g(n))/2 \NC \leq \max(f(n),g(n)) \NR
\NC \max(f(n), g(n)) \NC \leq f(n) + g(n) \NR
\stopalign\stopformula

對於最後兩個不等式,有:
\startformula
0 \leq \frac{1}{2}(f(n)+g(n)) \leq \max(f(n),g(n)) \leq f(n) + g(n)\text{,若 \m{n > n_0}。}
\stopformula

這與 \m{\Theta(f(n)+g(n))} 的定義一致,其中 \m{c_1 = 1/2}, \m{c_2=1}。
\stopANSWER

%e3.2-2
\startEXERCISE
解釋一下爲什麼說“算法 \m{A} 的運行時間至少是 \m{O(n^2)}”沒有任何意義。
\stopEXERCISE
\startANSWER
$O$ 是指上界,“至少”是指下界。
\stopANSWER

%e3.2-3
\startEXERCISE
$2^{n+1} = O(2^n)$ 是否成立?
$2^{2n} = O(2^n)$ 是否成立?
\stopEXERCISE
\startANSWER
$2^{n+1} = O(2^n)$ 成立。

$2^{2n} = O(2^n)$ 不成立,
因爲對於任意常數 $c$,
都不存在 $n_0$,當 $n>n_0$ 時,
使得 $2^n \cdot 2^n \leq c 2^n$。
\stopANSWER

%e3.2-4
\startEXERCISE
證明定理 3.1。

當且僅當 \m{f(n) = O(g(n))} 並且 \m{f(n) = \Omega(g(n))} 時,才有 \m{f(n) = \Theta(g(n))}。
\stopEXERCISE
\startANSWER
如果 $f(n) = \Theta(g(n))$,則:
\startformula
0 \le c_1g(n) \le f(n) \le c_2g(n) \quad \text{若} n > n_0
\stopformula
將兩個常數 $c_1$ 和 $c_2$ 代入 $O$ 和 $\Omega$ 的定義中即得:
\startformula\startalign
\NC f(n) \NC = O(g(n)) \NR
\NC f(n) \NC = \Omega(g(n)) \NR
\stopalign\stopformula

反之,如果 $f(n) = O(g(n))$ 並且 $f(n) = \Omega(g(n))$,則:
\startformula\startalign
\NC 0 \le c_3 g(n) \le f(n) \NC \quad \text{若} n \ge n_1 \NR
\NC 0 \le f(n) \le c_4 g(n) \NC \quad \text{若} n \ge n_2 \NR
\stopalign\stopformula
設 \m{n_3=\max(n_1,n_2)},合並兩個不等式,得:
\startformula
0 \leq c_3g(n) \leq f(n) \leq c_4g(n) \quad \text{若} n > n_3
\stopformula
\stopANSWER

%e3.2-5
\startEXERCISE
證明當且僅當算法的最壞情況運行時間爲 \m{O(g(n))},且最好情況運行時間爲 \m{\Omega(g(n))} 時,
其運行時間才是 \m{\Theta(g(n))}。
\stopEXERCISE
\startANSWER
如果最壞情況運行時間爲 \m{O(g(n))},且最好情況運行時間爲 \m{\Omega(g(n))},
設最壞情況運行時間爲 \m{T_w},最好情況運行時間爲 \m{T_b},則:
\startformula\startalign
\NC 0 \leq c_1g(n) \leq T_b(n) \NC \quad \text{若} n > n_b \NR
\NC 0 \leq T_w(n) \leq c_2g(n) \NC \quad \text{若} n > n_w \NR
\stopalign\stopformula
結合兩式,有:
\startformula
0 \leq c_1g(n) \leq T_b(n) \leq T_w(n) \leq c_2g(n)
   \quad \text{若} n > \max(n_b, n_w)
\stopformula
因此,運行時間爲 $\Theta(g(n))$。

反之,證明略。
\stopANSWER

\startEXERCISE
證明集合 \m{o(g(n)) \cap \omega(g(n))} 爲空。
\stopEXERCISE
\startANSWER
對於常量 \m{c > 0}:
\startformula\startalign
\NC \exists n_1 > 0 : \NC 0 \leq f(n) < cg(n) \NR
\NC \exists n_2 > 0 : \NC 0 \leq cg(n) < f(n) \NR
\stopalign\stopformula

令 $n_0 = \max(n_1,n_2)$,若 $n>n_0$,則:
\startformula
f(n) < cg(n) < f(n)
\stopformula
顯然不成立,不存在這樣的函式 $f(n)$。命題得證。
\stopANSWER

\startEXERCISE
將參數 $n$ 推廣爲兩個參數 $m$ 和 $n$,
且這兩個參數相互獨立,以不同的速率趨向於 $\inf$。
給定函式 $g(n,m)$,
存在正常數 $c$、 $n_0$ 和 $m_0$,
使得對於所有的 $n\ge n_0$ 或 $m\ge m_0$,
均滿足 $0 \le f(n,m) \le cg(n,m)$,
這樣的函式 $f(n,m)$ 構成的集合用 $O(g(n,m))$ 表示,
即 $O(g(n,m))=\{f(n,m)\}$。

給出 $\Omega(g(n,m))$ 和 $\Theta(g(n,m))$ 的定義。
\stopEXERCISE
\startANSWER
存在正常數 $c$、 $n_0$ 和 $m_0$,
使得對於所有的 $n\ge n_0$ 或 $m\ge m_0$,
有 $0 \le cg(n,m) \le f(n,m)$,
即 $\Omega(g(n,m)) = \{f(n,m)\}$。

存在正常數 $c_1$、 $c_2$、 $n_0$ 和 $m_0$,
使得對於所有的 $n\ge n_0$ 或 $m\ge m_0$,
有 $0 \le c_1 g(n,m) \le f(n,m) \le c_2 g(n,m)$,
即 $\Theta(g(n,m)) = \{f(n,m)\}$。
\stopANSWER

\stopsection
