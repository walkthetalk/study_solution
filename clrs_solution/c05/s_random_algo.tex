\startsection[
  title={Randomized algorithms},
]

%5.3-1
\startEXERCISE
Marceau 教授不同意引理 5.4 證明中所用的循環不變式。
第一次迭代之前循環不變式是否成立?他對此提出了質疑。
他的理由是,我們也可以宣稱一個空數列不包含 0-排列,
從而在第一次迭代之前使得循環不變式失效。
請重寫過程 \ALGO{RANDOMLY-PERMUTE},
使得在第一次迭代之前子數列非空的情况下,
循環不變式依舊適用。
並据此证明引理 5.4。

\CLRSH{RANDOMLY-PERMUTE(A,n)}
\startCLRSCODE
for i = 1 to n
	// swap A[i] with A[RANDOM(i,n)]
\stopCLRSCODE
\stopEXERCISE
\startANSWER
變通一下,其實我們可以在循環之前,隨機選擇一個元素來替代第一個。
這樣不變式對於 1-排列是有效的。
\stopANSWER

%5.3-2
\startEXERCISE
Kelp 教授決定寫一個過程來隨機產生任意排列,
\emph{恆等排列}(identity permutation)除外。
所謂恆等排列是指所有元素保持原位不動。
他給出了如下過程:

\CLRSH{PERMUTE-WITHOUT-IDENTITY(A)}
\startCLRSCODE
n = A.length
for i = 1 to n - 1
	// 交換 $A[i]$ 與 $A[\ALGO{RANDOM(i+1, n)}]$
\stopCLRSCODE
這段代碼能得到 Kelp 教授想要的結果嗎?
\stopEXERCISE
\startANSWER
不能。
比如有數列 $1,2,3$,無法得到 $3,2,1$。
要想得到 $3,2,1$,第一步只能是 $1$ 和 $3$ 進行交換,
得到 $3,2,1$,第二步只能是 $2$ 和 $1$ 交換,
得到的是 $3,1,2$,無法得到 $3,2,1$。
\stopANSWER

%5.3-3
\startEXERCISE
假設我們不是將元素 \m{A[i]} 與子數列 \m{A[i..n]} 中的一個隨機元素交換,
而是將他與數列任意位置上的隨機元素交換:

\CLRSH{PERMUTE-WITH-ALL(A)}
\startCLRSCODE
n = A.length
for i = 1 to n
	// 交換 $A[i]$ 與 $A[\ALGO{RANDOM(1, n)}]$
\stopCLRSCODE
這段代碼會產生一個均勻隨機排列嗎?爲什麼?
\stopEXERCISE
\startANSWER
不會。有 \m{n^n} 種執行過程,但只有 \m{n!} 種排列。
由於 \m{n!} 無法整除 \m{n^n},不可能是均勻分布。
(爲什麼無法整除?對於 \m{n>2}, \m{n-1} 可以整除 \m{n!},但卻不能整除 \m{n^n})。
參見 \simpleurl{http://blog.codinghorror.com/the-danger-of-naivete/}。
\stopANSWER

%5.3-4
\startEXERCISE
Knievel 教授建議用下面的過程來生成均勻隨機排列:

\CLRSH{PERMUTE-BY-CYCLE(A)}
\startCLRSCODE
n = A.length
// 令 B[1..n] 爲一個新數列
offset = \ALGO{RANDOM(1, n)}
for i = 1 to n
	dest = i + offset
	if dest > n
		dest = dest - n
	B[dest] = A[i]
return B
\stopCLRSCODE

請說明每個元素 \m{A[i]} 出現在 \m{B} 中任何特定位置的概率是 \m{1/n}。
然後通過說明排列結果不是均勻隨機排列,證明 Armstrong 教授錯了。
\stopEXERCISE

\startANSWER
如果 \m{j \equiv \text{offset} + i \pmod{n}},則 \m{A[i]} 就會出現在 \m{B[j]} 處,其概率爲 \m{1/n}。
但他不能生成所有排列,他所生成的排列都是數列 \m{A} 循環右移得到。
\stopANSWER

%5.3-5
\startEXERCISE
Gallup 教授希望創建集合 \m{\{1,2,3,\ldots,n\}} 的一個\emph{隨機樣本(random sample)},
即一個具有 \m{m} 個元素的子集 \m{S},其中 \m{0\le m\le n},
使得以同樣的概率創建每個 \m{m}-子集。
一種方法是對 \m{i = 1,2,\ldots,n} 令 \m{A[i] = i},
調用 \ALGO{RANDOMLY-PERMUTE(A)},
然後取最前面的 \m{m} 個元素。
這種方法會調用 \m{n} 次 \ALGO{RANDOM}。
在 Gallup 教授的應用中, \m{n} 比 \m{m} 大的多,
所以他想調用 \ALGO{RANDOM} 更少的次數,就能創建一個隨機樣本。
請說明下面遞迴過程返回 \m{\{1,2,3,\ldots,n\}} 的一個隨機 \m{m}-子集 \m{S},
每種 \m{m}-子集的概率相等,
且只需調用 \m{m} 次 \ALGO{RANDOM}。

\CLRSH{RANDOMLY-SAMPLE(m,n)}
\startCLRSCODE
S = ∅
for k = n-m+1 to n	// 迭代 $m$ 次
	i = RANDOM(1, k)
	if i ∈ S
		S = S ∪ {n}
	else S = S ∪ {i}
	return S
\stopCLRSCODE
\stopEXERCISE

\startANSWER
每種組合的概率應爲 \m{1/\binom{n}{m}}。
我們來證明下面的命題:

\ALGO{RANDOM-SAMPLE(m,n)} 返回均勻分布的組合。

在 \m{m} 上進行歸納。 \m{m} 爲 \m{1} 或 \m{0} 時顯然是成立的。
現在假定 \m{m-1} 時命題成立,看看 \m{m} 會發生什麼。

遞迴調用 \m{m - 1} 返回的樣本時均勻分布的。
然後有兩種情況, $i ∈ S$ 是否成立。

如果 $i ∈ S$ 成立(概率: \m{m/n}),則其概率爲:
\startformula
\frac{m}{n}\frac{1}{\binom{n-1}{m-1}} = 1/\binom{n}{m}
\stopformula

如果 $i ∈ S$ 不成立(概率: \m{(n-m)/n}),則其概率爲:
\startformula
\frac{n-m}{n}\frac{1}{\binom{n-1}{m}} = 1/\binom{n}{m}
\stopformula

即對於 $m,n$,所返回的每種結果的概率均爲 $1/binom{n}{m}$。
歸納完成,命題得證。

另外, $k$ 一共迭代了 $n-(n-m+1) + 1$ 共 $m$ 次。
\stopANSWER

\stopsection
